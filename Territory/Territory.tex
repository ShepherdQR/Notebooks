%%============
%%  ** Author: Shepherd Qirong
%%  ** Date: 2020-04-02 13:25:14
%%  ** Github: https://github.com/ShepherdQR
%%  ** LastEditors: Shepherd Qirong
%%  ** LastEditTime: 2022-06-11 22:37:37
%%  ** Copyright (c) 2019--20xx Shepherd Qirong. All rights reserved.
%%============

\documentclass[UTF8]{book}
\usepackage{ctex}
\usepackage{multirow,booktabs}
\usepackage{amsmath,amsthm,amsfonts,amssymb,bm,mathrsfs,upgreek} 
\usepackage[paper=a4paper,top=3.5cm,bottom=2.5cm,
left=2.7cm,right=2.7cm,
headheight=1.0cm,footskip=0.7cm]{geometry}
\usepackage{color,graphicx,verbatim}
\RequirePackage{setspace}
\setstretch{1.523}

\begin{document}


\chapter{Territory}


    

\chapter{People}


\section{Pontryagin【1908-1988】}

苏联盲人数学家。
老师Alexandrov是苏联科学院院士,研究点集拓扑,与Hopf合著的《拓扑学》影响了一代拓扑学家。

Pontryagin在19岁提出对偶定理,27岁博士毕业后称为莫斯科大学教授。30岁出版《连续群》。提出一种示性类。在控制论领域,1956年提出Pontryagin最大值原理,贝尔曼动态规划方法(在一定条件下整体最优则局部最优)和LQ理论与卡尔曼滤波(给出非平稳随机系统状态的递推最小方案估计)。被称为是现代控制论的三大里程碑。

著作《最优过程的数学理论》《常微分方程》。

\end{document}