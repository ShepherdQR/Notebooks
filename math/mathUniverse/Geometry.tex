%%============
%%  ** Author: Shepherd Qirong
%%  ** Date: 2022-04-09 22:31:44
%%  ** Github: https://github.com/ShepherdQR
%%  ** LastEditors: Shepherd Qirong
%%  ** LastEditTime: 2022-04-10 00:22:11
%%  ** Copyright (c) 2019--20xx Shepherd Qirong. All rights reserved.
%%============


\chapter{Geometry}


\section{Differential Geometry}
\subsection{摘要}
以梁灿彬课程为主。


\subsection{Topological Space}


$f: \mathbb{R} \to \mathbb{R} $ is $C^0$ (读作c nought,$C^k$意思是k阶导函数存在且连续 ), if Y得到任意开区间的“逆像”($f^{-1}[B] := \{x \in X | f(x) \in B\}$)是x的开区间之并,open set 之并。

$x \hookrightarrow y$, 一般用于表示 inclusion 或 embedding(嵌入)。在这里通常表示这个map具有2个性质,1)injective,1个y只对应1个x;2)structure-preserving,不同x之间的关系和对应y之间的关系保持,如$X1 < X2$,映射过去后$y1 < y2$.

$f: \mathbb{R}^n \to \mathbb{R}^m $, m个n元函数。

X的所有子集记为$\mathscr P$

$\mathscr T$,X的一些开集的集合,称为X的一个拓扑。选拓扑是指定集合中的哪些子集的开的。先问set的拓扑是什么,再问开不开。

\begin{equation}
    \begin{split}
    &X, \varnothing \in \mathscr T\\
    &O_i \in\mathscr T, i=1, \cdots, n \Longrightarrow \bigcap _{i=1}^n O_i  \in\mathscr T, 即有限个的交也开 \\
    &O_\alpha \in\mathscr T, \forall \alpha, \Longrightarrow \bigcup  _\alpha  O_\alpha \in\mathscr T,  即无限个的并也开 \\
    \end{split}
  \end{equation}


$\mathscr T = \{ \cdots \}$, 离散拓扑,开集最多;
$\mathscr T = \{ X, \varnothing \}$, 凝聚拓扑,开集最少。

$\mathbb{R}^1$, open interval;$\mathbb{R}^2$, open disk;$\mathbb{R}^n$, open ball;

$B(X_0, r) := \{ x_i \in \mathbb{R}^n |   |x_i - x_0| \leq r\}$

usual topology: $\mathscr T_u := \{ \mbox{可表为开球之并的集合}\}$, 一般认为$\mathbb{R}^n \in \mathscr T_u $。$\in$属于。

$(X, \mathscr T)$, 拓扑空间X;
$A \subset X, (A, \mathscr S)$, 拓扑子空间。$ \subset $含于。

其中$\mathscr S := \{  V \subset A | \exists O \in \mathscr T, s.t. O \cap A = V\}$, $\mathscr S$ 由$\mathscr T$诱导出。诱导拓扑的定义是,可由父集合中的拓扑定义的开集,交子集得到的集合。

open subset, 能写成开区间之并的subset 


\subsection{Homeomorphism, 同胚}

有拓扑结构的空间,映射map的连续性有意义。

$f:X \to Y$ is $C^0$ if $ O \in \mathscr S \rightarrow f^{-1}[O] \in \mathscr T$。即X中由X的拓扑定义的开集$O'$,映射到Y上后,变成了Y上的X的拓扑的诱导拓扑下的开集。即映射到Y上后,得到的这个集合O,这个集合O的子集可以由X的拓扑定义的开集,交上集合O得到。

$C^0$ , continus; 


无附加structure, 只能说one one,onto, 有了附加structure可以讨论continus。

同胚:存在一个one one,onto, $f$和$f^{-1}$均$C^0$。

微分同胚:one one,onto, $f$和$f^{-1}$均$C^\infty$。即 one one,onto, 正反光滑。

$f: \mathbb{R} \to \mathbb{R} $ is $C^0$ at $x\in  \mathbb{R}$, if $\forall \varepsilon >0, \exists \delta >0, s.t. |x' - x| < \delta \rightarrow |f(x') - f(x)| <   \delta$, 这样定义 $C^0$需要用到距离,附加的structure

