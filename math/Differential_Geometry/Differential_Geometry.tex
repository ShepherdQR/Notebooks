%************
%  * @Author: Shepherd Qirong
%  * @Date: 2020-02-19 18:46:38
%  * @Github: https://github.com/ShepherdQR
%  * @LastEditors: Shepherd Qirong
%  * @LastEditTime: 2020-02-19 18:56:04
%  * @Copyright (c) 2019--20xx Shepherd Qirong. All rights reserved.
%************


\documentclass[UTF8]{article}
\usepackage{ctex}
\usepackage{multirow,booktabs}
\usepackage{amsmath,amsthm,amsfonts,amssymb,bm,mathrsfs,upgreek} 
\usepackage[paper=a4paper,top=3.5cm,bottom=2.5cm,
left=2.7cm,right=2.7cm,
headheight=1.0cm,footskip=0.7cm]{geometry}
\usepackage{color, graphicx, verbatim}
\RequirePackage{setspace}%%行间距
\setstretch{1.523}

\DeclareMathOperator{\rank}{rank}

\DeclareMathOperator{\sgn}{sgn}

\begin{document}

\section{Background}
\subsection{Euclid空间}
有序的n元组的全体称为n维Euclid空间,记为$\mathbb R^n$,称$\boldsymbol p=(p_i)_{i=1}^n \in \mathbb R^n$是$\mathbb R^n$的一个点。\\
为便于研究,本论文以$ \mathbb R^3$为背景空间,所涉及的函数默认为可微实值函数。如果实函数$f$的任意阶偏导数存在且连续,则称函数是可微的(或无限可微的,或光滑的,或$C^\infty$的)。\\
由于微分运算是函数的局部运算,限制所讨论函数的定义域在$ \mathbb R^3$中的任意开集,所讨论的结论仍然成立。\\
自然坐标函数:定义在$\mathbb R^n$上的实值函数$x_i: \mathbb R^n \to  \mathbb R$,使得$\boldsymbol p=(p_i)_{i=1}^n = \left( x_i(\boldsymbol p) \right)_{i=1}^n   $\\
切向量:由$\mathbb R^n$ 中的二元组构成,$\boldsymbol v_{\boldsymbol p}=(\boldsymbol p,\boldsymbol v)$,其中$\boldsymbol p$是作用点,$\boldsymbol v$是向量部分\\
切空间$T_p  \mathbb R^n$: 作用点$\boldsymbol p \in \mathbb R^n$的所有切向量的集合。利用向量加法与数量乘法使某点的切空间称为向量空间,与背景空间存在非平凡同构。\\
向量场$\boldsymbol V$:作用于空间点的向量函数,$\boldsymbol V(\boldsymbol p)\in T_p  \mathbb R^n $\\
逐点化原理:$(\boldsymbol V+\boldsymbol W)(\boldsymbol p)=\boldsymbol V(\boldsymbol p)+\boldsymbol W(\boldsymbol p),\ (f \boldsymbol V)(\boldsymbol p)= f(\boldsymbol p)\boldsymbol V (\boldsymbol p)$\\
自然标架场:定义$\boldsymbol U_i=(\delta _j^i)_{j=1}^n$,按Einstein求和约定,有$\boldsymbol V(\boldsymbol p)=v^i(\boldsymbol p)\boldsymbol U_i(\boldsymbol p)$,称$v^i$为场的Euclid坐标函数,其中Kronecker $\delta$函数定义为:
\begin{equation}
\label{Kronecker_delta}
\delta _i^j=\left\{ 
    \begin{aligned}
    1,\  & i =j\\
    0,\  & i \neq j\\
    \end{aligned}
     \right.
\end{equation}
 













\end{document}