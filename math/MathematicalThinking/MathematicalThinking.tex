%%============
%%  ** Author: Shepherd Qirong
%%  ** Date: 2019-06-20 20:04:18
%%  ** Github: https://github.com/ShepherdQR
%%  ** LastEditors: Shepherd Qirong
%%  ** LastEditTime: 2021-11-03 21:20:22
%%  ** Copyright (c) 2019--20xx Shepherd Qirong. All rights reserved.
%%============
\documentclass[UTF8]{article}
\usepackage{ctex}
\usepackage{multirow,booktabs}
\usepackage{amsmath,amsthm,amsfonts,amssymb,bm,mathrsfs,upgreek} 
\usepackage[paper=a4paper,top=3.5cm,bottom=2.5cm,
left=2.7cm,right=2.7cm,
headheight=1.0cm,footskip=0.7cm]{geometry}
\usepackage{color}
\RequirePackage{setspace}%%行间距
\setstretch{1.523}

\begin{document}
数学的思维方式与创新-84-北大(丘维声)

6,1039. 

\section{数学史上的重大创新}
\subsection{分析:微积分的创立和完备化}
观察现象主要特征,抽象出概念。探索。猜测。证明。\\
如求瞬时速度, $s=at^2$,$\frac{\Delta s}{\Delta t}=2at+\Delta t$,牛顿忽略$\Delta t$,叫做留数,留下来的数。\\
如何解决不等于零又等于零的矛盾?\\
delta t 趋近于0,无限,柯西引入极限的概念:函数在x0附近有定义,在x0可以没有定义,如果存在c使得x趋近于x0但不等于x0时,$|f(x)-c|$可以无限小,称c是x趋近于x0时f(x)的极限。\\
$\forall \varepsilon >0,\ \exists \delta >0$, that when $0<|x-x_0|<\delta$, we have $|f(x)-c|<\varepsilon$
\subsection{几何:欧几里得几何到非欧几里得几何}
从平直空间到弯曲空间。\\
从定义和公理,推导和推演。平行公设。高斯和波约,罗巴切夫斯基(1829年),平行公设只是假设。现实世界如何实现非欧几何的用处。高斯想法把球面本身看做一个空间。后来黎曼发展了。弯曲空间的几何是黎曼几何,如球面上的直线定义为大圆的一部分,这样发现过已知直线外一点不存在其平行线。在双曲几何模型下可以实现罗巴切夫斯基几何。\\
\subsection{代数学中}
伽瓦罗,代数学从研究方程的根,到研究代数系统的结构和保持运算的映射。

\section{集合的划分}
交空并全的划分方法:模n同余是$\mathbb Z$的一个二元关系。两个集合的笛卡尔积\\
a与b模n同余:$(a,b) \in \bigcup _{i=0}^{n-1} H_i \times H_i \subseteq \mathbb Z \times \mathbb Z$.抽象:非空集合s,$S\times S$的子集W是S是上的二元关系,有关系的记为aWb\\

\subsection{等价关系}
反身性,对称性,传递性,$a \sim b$. $a \sim b \Leftrightarrow b \sim a$. $a \sim b , b \sim c \Rightarrow a \sim c$\\
$\bar a$ 是a确定的等价类,$\left\{ x \in S |x \sim a \right\}$。易有
$\bar x = \bar y \Leftrightarrow x \sim y$ 。\\


定理1:集合S上等价关系$\sim$给出的等价类的集合是S的一个划分。\\
证明思路:需要证明并全,交空。交空比较难,需要研究等价类的性质。等价类的代表不唯一。\\
Step1)  It is obvious that $\cup _{a \in S} {\bar a} \subseteq S$,and for any $b \in S$, we have $b \in \bar b \in \cup _{a \in S} {\bar a}$, this means $S \subseteq \cup _{a \in S} {\bar a}  $, so $ \cup _{a \in S} {\bar a} =S $.\\
Step2) To prove $\bar x \neq \bar y \Rightarrow \bar x \bigcap \bar y = \varnothing  $, we prove the contrapositive $ \bar x \bigcap \bar y \neq \varnothing \Rightarrow  \bar x = \bar y$, and this is easy to prove.\\








\end{document}