

\chapter{Philosophy}
哲学是爱智慧。哲学伴随着哲学史。

人参与社会,需要对话,语言的对话功能,需要有共同的对话平台,平台包括2方面:逻辑和共同的前提。最早的对话平台是一堆人构成的集团之间的神话。古希腊城邦,有一系列神话。为了城邦联合对抗波斯人,需要更为通用的对话平台,产生了哲学。哲学把人的理性作为共同的前提。而神话的共同前提设定是最高的神秘存在,就是神,内涵是人的一系列基本价值。

实用理性,不做终极追溯,有用就行。纯粹理性的最高价值是普世真理。

不同人的感性经验不同,抽象出共同的东西,一层层抽象,最后得到超验的东西,达到了神话的程度,自在之物,但不是神话,因为神话是凭空设定的。

哲学是把人类包括在内的,对终极存在的不停的追问、思考。

\section{Knowledge}

\subsection{Materials}

\subsubsection{ 陈宣良 哲学十讲}
陈宣良,1947年生于北京。1978年入武汉大学哲学系读硕士学位,1984年入中国人民大学哲学系攻读博士学位,1987年到中国青年政治学院任教。1989年到法国定居至今。著有《法国本体论哲学的演进》《理性主义》《死与道德》等。



神话,集团中的对话平台。

\section{命题}

\subsection{我思故我在}

笛卡尔论证上帝存在,因为经院哲学是主流的对话平台,启蒙思想作为新生的,从生存角度还是想融合进主流的。通过理性追溯到信仰,达到理性和信仰的和解。

理性无限,连上帝存在都可以证明了。人的理性建立在个人有限的不准确的感觉基础上,只能拉低理性的水平。尽量去掉感性,纯粹的理性更多一些来提高理性的水平。

我思故我在。思,所以是。思是思。作为人自身存在的是思想。思想是独立存在的,是它自己。自在的存在,自在之物,自己就是自己的东西。“存在”也叫作“实体”,它本身不是什么,可以什么都是也可以什么都不是,还没有本质的。

哲学追溯最原始的存在,建立一个终极平台。

上帝是当时哲学的终点。古希腊是无数神话,基督教是1个神话,基督教分裂后,陷入危机。每个人的上帝是不一样的,都想证明上帝的存在。当需要证明上帝存在的时候,就不是最高权威了,其实是人的理性来代替上帝的理性。笛卡尔与经院哲学的显著区别,就在于证明上帝存在的方式,脱离了神话本身。基督教的上帝时时在场,笛卡尔的上帝创世之后,世界就自己运转了。用逻辑的方式确立了一个最高的存在,这个存在其实就是人的思想,思想是自己存在的,思想就是思想。

引出了二元论的问题,目前无解。

理性主义最核心原则:怀疑。
首先回忆,然后自我意识的提升的第一步是怀疑。没有怀疑过的东西都不是真理。首先要确定1个不可怀疑的东西,一个起点,这个起点就是“我在怀疑”这件事。

科学不是理性主义,只有在反省科学的时候,才进入理性主义。

\subsection{存在就是被感知}

理性限制在有限范围内,信仰在理性之外,这样达到理性和信仰的和解。

洛克的白板说,通过不断获得对象的观念组合形成逻辑,产生对世界的解释。

因果关系是人的联想。休谟的观念,习惯是人生的伟大指南。如果直接关注未来是事情,目前的经验不能准确知道未来的事情,只能占卜了。

经验主义,按照观察到的事件之间的联系,人为创造现象发生的条件,如果获得了预测到的结论,建立了可实践的科学。

“认识”本身就是对对象的干扰。如果要设定是白板,所观察的对象就要是主动的发送观念,有问题。主体和对象对立之后,主观和客观对立起来之后,是无法解决的问题。

\subsection{人为自然界立法}



真理:主观认识和客观存在一致,且逻辑推演无矛盾。

康德想要找到和现实一致的且合乎逻辑的哲学,首先批判人的理性,找到理性的极限。康德认为,逻辑先天就有,先天的认识框架是时间和空间,时间和空间不是客观存在的(与牛顿的时间和空间是客观存在的不一致),内容从经验中来。人用时间和空间把散乱的现象、经验组织起来形成知识。纯粹理性指的是个人的纯粹的时间和空间概念是无用的。

如何判断出先天的清楚的又有内容的命题呢?例如1=1这样的命题收集起来没有意义。用时空框架把握感知内涵。

用时空框架看到世界时,怎么断定看到的世界就是那个世界本身呢?没法断定。看到的是现象。没法突破现象界的真理。或者是杂乱无章的观念,或者是纯粹的没有内涵的逻辑。

现象之后是什么呢?是自在之物,是存在本身,要说出它是什么的时候,它已经是什么而不是它了,它是什么呢,它是它自己。人没有可能认识那个东西。

人本身也是现象界的存在。掌握现象界的真理就够了。

存在是个问题,自在之物的设定,或对于存在存而不论,相当于上帝的设定,是用不着的,是多余的。


唯物主义,断定的本原的存在是物质;唯心主义任务是精神性的。自然界在精神内还是精神外,有问题。

P4 20:05

