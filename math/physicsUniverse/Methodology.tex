%%============
%%  ** Author: Shepherd Qirong
%%  ** Date: 2019-06-20 20:04:18
%%  ** Github: https://github.com/ShepherdQR
%%  ** LastEditors: Shepherd Qirong
%%  ** LastEditTime: 2021-12-21 22:38:08
%%  ** Copyright (c) 2019--20xx Shepherd Qirong. All rights reserved.
%%============



\chapter{Methodology}

\section{Introduction}
Today is 20211211, and I deciede to note down all of my knowledge about physics in this notebook. Actually we think for a while whether to seperatre the knowledge into different documents.

\section{Preference}
\subsection{Volabulary}
orthogonal matrix, 正交矩阵


\section{History}



\section{观点}

\subsection{需要再确认的观点}
\subsubsection{行星和卫星公转轨道为什么是椭圆?}

一个焦点位于原点的圆锥曲线
$\frac{1}{r}=C\left[1+e\cos(\theta-\theta^{\prime})\right]$
$f=-\frac{k}{r^{2}}~,\quad V=-\frac{k}{r}$
只考虑2体,角动量守恒求出轨道方程,角动量l与E看做常数,
$$\frac{1}{r}=\frac{mk}{l^{2}}\left(1+\sqrt{1+\frac{2El^{2}}{mk^{2}}}\cos(\theta-\theta^{\prime})\right)$$
离心率e<1椭圆,等于1是抛物线,大于1是双曲线。$e=\sqrt{1+\frac{2El^{2}}{mk^{2}}}$
