%%============
%%  ** Author: Shepherd Qirong
%%  ** Date: 2019-06-20 20:04:18
%%  ** Github: https://github.com/ShepherdQR
%%  ** LastEditors: Shepherd Qirong
%%  ** LastEditTime: 2021-12-25 23:25:00
%%  ** Copyright (c) 2019--20xx Shepherd Qirong. All rights reserved.
%%============



\chapter{Methodology}

\section{Introduction}
Today is 20211211, and I deciede to note down all of my knowledge about physics in this notebook. Actually we think for a while whether to seperatre the knowledge into different documents.

\section{Preference}
\subsection{Volabulary}
orthogonal matrix, 正交矩阵


\section{History}



\section{观点}

\subsection{Videos}

牛顿经典力学,场论定域论,最小作用量原理,都可以解释从A到B的路径,是等效的。
So I made the hypothesis often that the laws are going to turn out to be, in the end, simple like the checkerboard, and that all the complexity is from size.

If you will not say that it is true in a region that you have not looked at, you do not know anything.

We always must make statements about the regions that we have not seen.

The mass of an object changes when it moves.

\subsection{需要再确认的观点}
\subsubsection{行星和卫星公转轨道为什么是椭圆?}

一个焦点位于原点的圆锥曲线
$\frac{1}{r}=C\left[1+e\cos(\theta-\theta^{\prime})\right]$
$f=-\frac{k}{r^{2}}~,\quad V=-\frac{k}{r}$
只考虑2体,角动量守恒求出轨道方程,角动量l与E看做常数,
$$\frac{1}{r}=\frac{mk}{l^{2}}\left(1+\sqrt{1+\frac{2El^{2}}{mk^{2}}}\cos(\theta-\theta^{\prime})\right)$$
离心率e<1椭圆,等于1是抛物线,大于1是双曲线。$e=\sqrt{1+\frac{2El^{2}}{mk^{2}}}$
