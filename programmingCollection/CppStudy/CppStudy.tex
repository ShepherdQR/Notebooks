%%============
%%  ** Author: Shepherd Qirong
%%  ** Date: 2021-06-12 18:33:02
%%  ** Github: https://github.com/ShepherdQR
%%  ** LastEditors: Shepherd Qirong
%%  ** LastEditTime: 2021-08-18 23:33:58
%%  ** Copyright (c) 2019--20xx Shepherd Qirong. All rights reserved.
%%============
\documentclass[UTF8]{article}
\usepackage{ctex}
\usepackage{amsmath,amsthm,amsfonts,amssymb,bm,mathrsfs,upgreek} 
\usepackage[paper=a4paper,top=3.5cm,bottom=2.5cm,
left=2.7cm,right=2.7cm,
headheight=1.0cm,footskip=0.7cm]{geometry}
\usepackage{color,xcolor}
\RequirePackage{setspace}
\setstretch{1.523}

\usepackage{listings}
\lstset{breaklines}%这条命令可以让LaTeX自动将长的代码行换行排版
\lstset{extendedchars=false}%这一条命令可以解决代码跨页时,章节标题

\definecolor{mygreen}{rgb}{0,0.6,0}
\definecolor{mygray}{rgb}{0.5,0.5,0.5}
\definecolor{mymauve}{rgb}{0.58,0,0.82}
\lstset{ %
backgroundcolor=\color{white},   % choose the background color
basicstyle=\footnotesize\ttfamily,        % size of fonts used for the code
columns=fullflexible,
breaklines=true,                 % automatic line breaking only at whitespace
captionpos=b,                    % sets the caption-position to bottom
tabsize=4,
commentstyle=\color{mygreen},    % comment style
escapeinside={\%*}{*)},          % if you want to add LaTeX within your code
keywordstyle=\color{blue},       % keyword style
stringstyle=\color{mymauve}\ttfamily,     % string literal style
frame=single,
rulesepcolor=\color{red!20!green!20!blue!20},
% identifierstyle=\color{red},
language=c++,
}

\begin{document}
\title{CPP Study Note}
 


\section{Basic Grammer}

\subsection{Books}

c++ primer v5, the author is the creator of the first c++ compiler.

the c++ programming language v4, the author is the creator of the c++.

the c++ standard library: a tutorial and reference.



\subsection{Reference}
The reference is ofen used as parameters of a function. 
rreference is a const pointer

\begin{lstlisting}
    // const T& or const T cannot be passed to T&
    %* \dots *)
    int var(1);
    int &ref = var; //ref is always the second name of var.
    %*
        \dots
    *)
    int& func(const double& iVal,...){...}
    func() = 3; // We can use the function just as a variable
\end{lstlisting}

 
\subsection{Const}
\begin{lstlisting}
    // const T* can be transformed to T* using (T*)
    %* \dots *)
    int var(1);
    const int* pCanNotModifyVar = &var; //const T* p CANNOT modify the variable it points to.
\end{lstlisting}
const object function can only be called by const object.\\

\begin{lstlisting}
    void testFuncAdd20210718_2()
{
    const int a = 10;
	int* pConstModifier = (int*)&a;// compiler finds and allocates memory
	const int  *q = &a;
	{
		*pConstModifier = 20;
	}
	std::cout <<a <<std::endl;//10
	std::cout <<*pConstModifier <<std::endl;//20
	std::cout << (&a == pConstModifier ) <<std::endl;//1
	std::cout << (q == pConstModifier ) <<std::endl;//1
}
\end{lstlisting}

\subsection{Memory}
%%\paragraph{Memory} \quad\\
Dynamic
\begin{lstlisting}
    T* p = new T;
    if(p){
        delete p;
        p = nullptr;
    }
    int arraySize(6);
    T* p = new T[arraySize];
    if(p){
        delete[] p;
        p = nullptr;
    }
\end{lstlisting}


\subsection{IO}

dec,hex, oct





\section{Function}
函数定义中可以为最右边连续若干参数有省却值,可以用于扩充函数参数时减少对原有代码的修改。
\paragraph{Inline} \quad\\
We use Inline to tell Compiler to do copy-paste. The function is short and called many times.调用时间少了,可执行文件体积增大

完成形参的构造(如调用类的拷贝构造等)之后,再进入函数体内。
\begin{lstlisting}


    
\end{lstlisting}
\paragraph{Oveload} \quad\\
重载。Functions with the same name but the variables are different.We donot care the returned type.




\section{Object}
结构化程序设计没有封装和隐藏的概念,导致数据结构与函数之间相互的关系、函数之间的调用关系不明确。
把数据结构和对其进行的操作方法放在一起。

类的实现也可以直接在.h中。
\subsection{Basic}
给出了构造函数,编译器不默认提供无参构造函数。
复制构造函数,只有1个
\begin{lstlisting}
    C(const C& iC);
\end{lstlisting}
返回值是类A的实例时,函数返回时调用A的复制构造函数

声明时可给出变量的初始值,后面可在构造函数中重新赋值。

\begin{lstlisting}
    //类型转换构造函数
    C(int iIn)
    C c1 = 12, c2;
    c2 = 9;//error
    c2 = C(9); // ok
\end{lstlisting}

参数对象消亡时调用析构函数;函数返回时是生成临时对象返回,在临时对象调用的那条语句之后,临时对象消亡,调用析构。

When we do something in constructor, we need to make sure the function works well in copy constructor.

在构造函数的初始化列表中对类的成员类进行初始化,,顺序是类中成员声明顺序。

\subsection{Decoration}
没有使用成员变量的方法,可以通过空的类指针调用,相当于翻译后传了一个空的this指针。static方法也可以空类指针调用,static方法参数中没有this指针 \\
static和global在main结束后按入栈的顺序析构。\\
sizeof doesn't contain static variables.\\
friend function, a class must declare WhichClass::WhichFunction is its friend, so that the specific function can access its private members. Also the same as friend class.\\
friend relationship cannot be passed, or be inherited.\\
每个对象的空间中都有this指针。(x)\\

\begin{lstlisting}
  提示编译器不生成默认复制构造函数:A(CONST A&)=delete;
  提示编译器提供默认构造函数 A()=default;
  默认情况一旦写了构造函数,编译器就不生成无参的默认构造函数了。
  委托构造函数 A():A(0,0,0){}
\end{lstlisting}



\section{UML}
united modeling language. 


\section{Design Pattern}
\subsection{Principles}
Open for extension, closed for modification. 对扩展开放,对修改封闭。类的改动通过增加而不是修改代码实现。
Single Responsibility. 类对外提供一种功能。模块只能有一种被修改的理由。
Dependence inversion. 依赖接口。业务层与实现层都向协议层靠拢。
interface segegation. 一个接口提供一种功能。
Liskov substitution. 抽象类出现的地方都可以用其实现类替换。
composite or aggregate reuse. 优先使用组合,因为继承时父类的修改会影响子类,对象组合会降低依赖关系。
law of demeter一个对象应对其他对象尽量少的了解。



\section{Others}




\begin{lstlisting}
   http://api.open-notify.org/astros.json

   https://openweathermap.org/current

\end{lstlisting}


建议把xy push到局部vector中,把局部vector push到vec中

Notepad++里Run可以设置用.bat来编译运行,bat里面要先cd到其目录

penetration test, using attack tools.

进程因为创建而就绪,因为调度而运行,因为IO事件而阻塞,IO事件完毕或时间片用完回到就绪状态,运行完成后消亡。

c++看gcc-cp目录

它是个编译器,也可以画图写别的pythonjava之类的

我都没有研究过,我需要找点资料好好学习一下,个人觉着c用起来更像是地址的别名,fortran的更像是地址的别名的别名,两个的比较应该找一些开源的转换库看看,

\begin{lstlisting}
    using namespace std;在局部使用挺好的
   using FP = void (*) (int, const std::string&); //Replace using typedef
\end{lstlisting}


伸缩型数组,在c中比存指针成员的好处是1)不占1个地址的存储空间,2)最适合制作动态buffer,因为可以直接就把buffer的结构体和缓冲区一起分配。
\begin{lstlisting}
   struct MyStruct
   {
       int			i;
       float		j;
       double		arr[0];		//伸缩型数组成员
       //double	arr[];		//这样声明也可以
   };
   pf = malloc(sizeof(MyStruct) + N*sizeof(double));/*只分配一块内存*/
\end{lstlisting}
c\#生态还是局限


假如一种语言不允许运行期动态分配内存的话,确实只要有stack就够了,内存的数据区只划分成两大块,一块用于独立于过程的数据(编译、连接时已知大小),剩下的全部做为一个stack。C、C++这样的允许动态分配内存的语言,当然还是要有一个heap的,这样内存就划分成三个区域了。在引入线程概念之前,确实如你所想的,除了运行时动态确定大小的数据必须放在heap里面,其他都可以放在stack里,我们可以把stack设置的足够大。然而,后来线程出现了。每个线程有自己独立的stack,一个程序可以创建成百上千个线程,那么stack的大小也就只能是一个很小的值,内存才不会被用光。在windows系统里,每个线程的stack默认只有1MB,LINUX默认8MB,这样,stack就成了紧缺资源,必须非常有节制地使用,体量比较大的数据都要由程序员手工分配到heap里去并手工管理其生命期。

java 半编译半解释

反码:符号位不变,其余位翻转。
正整数的补码:是原码;
负整数的补码:是反码+1;
注意运算可能溢出。

\begin{lstlisting}
   a = 3*5,a*4;//60
   typedef int MyINT;
   using MyINT = int;

   decltype(i) j = 2;//the type of j is the same as i.
initializer_list


 \end{lstlisting}

 每一个练习封装成一个函数收集在一个.hpp里面,在一个main里面调用相应的练习函数

 我认为核心还是数据,设计类的方法要围绕着数据来,公开出去的方法要维护好隐藏起来的成员变量

 量子加密RSA

 原来stl的模板采用的那叫静态多态,编译期间翻译,所以模板的声明实现分离在头文件里实例一份就行了
\end{document}