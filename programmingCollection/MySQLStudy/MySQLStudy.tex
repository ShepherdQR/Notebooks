\documentclass[UTF8]{article}
\usepackage{ctex}
\usepackage{multirow,booktabs}
\usepackage{amsmath,amsthm,amsfonts,amssymb,bm,mathrsfs,upgreek,chemarrow} 
\usepackage[paper=a4paper,top=3.5cm,bottom=2.5cm,
left=2.7cm,right=2.7cm,
headheight=1.0cm,footskip=0.7cm]{geometry}
\usepackage{color, graphicx, verbatim}
\RequirePackage{setspace}%%行间距
\setstretch{1.523}

\DeclareMathOperator{\rank}{rank}

\DeclareMathOperator{\sgn}{sgn}

\begin{document}

xmind 记笔记

\section{基本概念}
数据库,database,按照数据结构来组织、存储和管理数据的仓库。
MySQL,需要熟练。关系型数据库。
程序+数据=软件。
应用程序是操作和查询数据库服务器,数据库服务器响应和提供数据。
数据库:增删改查。
数据库管理系统,dbms。
MySQL免费,小项目,一百万以下的数据。\\
甲骨文ORACLE,可以提供支持,大项目。\\
老的:sybase,银行、电信系统。db2。\\
分布式:HBASE,mongoDB。\\
嵌入式SQLite
SQL Server
Postgre SQL
数据库系统(Database System)包括数据库(Database),数据库管理系统(Database Management System),应用开发工具,管理员及用户。

\section{MySQL基本操作}
\subsection{数据库基础操作}
\begin{comment}
    创建的root密码是123456
    登录:cmd:mysql -h主机名空格-u用户名空格-p密码
    cmd mysql -uroot -p回车
    mysql -hlocalhost -uroot -p -P3306
    登录同时打开数据库:
    mysql -uroot -p -D 数据库名字

    修改密码:mysqladmin -uroot -p旧密码 password 新密码\\
    
    
    退出: exit
    quit
    \q
    ctrl+c
    
    修改命令提示符的名称:如修改为shepherd,每次关闭mysql后失效
    mysql -uroot -p --prompt='shepherd>'
    mysql -uroot -p --prompt='\h~\a~\d~\D'
    h是host,a是用户名,d是选择的数据库,D是日期

    查询版本号:mysql -V
    mysql --version

    ;和\g 是执行改行命令
    \c 不执行改行
    help,\h或者 ? 加上关键字可以查找帮助

    支持折行写,单引号不要折行;
    MySQL关键字大写。
    库、表、字段名称要避开关键字,否则需要反引号以区分。


    批命令:
    select user();
    select version();
    select now();
    select current_date();
    select current_time();
    select current_timestamp();
    以上几行保存,在cmd中输入以下:\\
    \. E:\Notes\Notes_Latex\MySQL_study\Codes\t001.sql  



    数据库的基本操作

    DATABASE 和 SCHEMA 一样
    show databases; -- 查看数据库
    create database t001; -- 注意双减号注释,前后都需要空格
    show databases;
    select database(); -- a查看当前选定的库
    use t001; -- 打开数据库
    select database();
    DROP DATABASE IF EXISTS t001; -- c删除库

    CREATE DATABASE IF NOT EXISTS t007 DEFAULT CHARACTER SET 'GBK'; -- 指定编码
    SHOW WARNINGS;
    SHOW CREATE DATABASE t007;
    ALTER DATABASE t007 DEFAULT CHARACTER SET 'UTF8'; -- 修改编码
    SHOW CREATE DATABASE t007;


\end{comment}


\subsection{表}
表由行row和列column组成
关系模型(二维表):不同的属性叫字段;每行数据是一个记录;每个记录都不相同的字段叫主键。\\
做表,先确定字段,确定每个字段的类型。

\begin{comment}
    CREATE TABLE IF NOT EXISTS table_name(
        字段名称 字段类型 [可选:完整性约束条件]

    ) EIGEN=存储引擎, charset=编码方式

    数值类型:
        整形
        TINYINT (-128, 127) or (0, 255),即8次幂,1字节
        SMALLINT 16次幂,2字节
        MEDIUMINT 24次幂,3字节
        INT (-2^31, 2^31-1) or (0, 2^24-1),32次幂,4字节
        BIGINT 64次幂,8字节
        BOOL, BOOLEAN, 等价于TINYINT(1),1字节

        浮点型
        FLOAT[(M,D)],M是总位数,D是小数位数,单精度浮点数精度大约7位小数。不设置时根据硬件条件设置
        取值范围(-3.4e38, -1.17e-38) & 0 & (1.175e-38, 3.4e38), 4字节
        DOUBLE[(M,D)],M是总位数,D是小数位数,取值范围(-1.79e308, -2.22e-308) & 0 & (2.22e-308, 1.79e308), 8字节
        DECIMAL[(M,D)],M是总位数,D是小数位数,类似DOUBLE,内部以字符串形式存储,字节M+2

        字符串类型
        CHAR(M), 定长串, M字节, M=[0,255]
        VAARCHAR(M), 变长串, 所占字节为L+1   小于M,M=[0,2^16-1],65536-1
        TINYTEXT, L < 2^8,L+1字节
        TEXT, L < 2^16,L+2字节
        MEDIUMTEXT, L < 2^24,L+3字节
        LONGTEXT, L < 2^32,L+4字节
        ENUM('VALUE1', 'VALUE2',...), 枚举,1或2字节,取决于枚举个数,最多2^16-1个值
        SET('VALUE1', 'VALUE2',...), 集合,1或2或3或4或8字节,取决于set成员个数,最多64个值

        日期类型:
        DATE, 3字节
        TIME, 3字节
        YEAR, 1字节
        DATETIME, 8字节
        TIMESTAMP 4字节

        二进制类型:
        不常用。
    

\end{comment}




\subsection{表间联系}
表a是学生表,有学生的学号姓名班级等信息,表b是课程表,是所有开课课程的序号、课名称、老师等信息,表a和b没有公共元素,表c是选课表,有学号和所选课程号、成绩等信息。表c联系起表a和b。
需要建5个键,3个表要有3个主键,同时表c要有2个外键投射到表a和表b\\
一对一的表,常用在登录表和详细表。方便查询。

\subsection{SQL Language}
Structured Query Language,结构化查询语言。包括数据定义语言(DDL),数据操作语言(DML)。

\subsubsection{DDL数据定义语言}
    
    \begin{comment}
    CREATE TABLE 创建表
    DROP TABLE 删除表
    ALTER TABLE 修改表+( ADD 字段名 类型, MODIFY 字段名 新字段名 新字段类型, DROP 待删除字段, RENAME TO 表的新名字)
    TRUNCATE TABLE 清空表
    DESCRIBE 查看表结构
    建立主键、外键等约束(PRIMARY KEY, NOT NULL, UNIQUE, DEFAULT '男', AUTO_INCREMENT)
    UNIQUE 意思是不能有重复的。
    \end{comment}

\subsubsection{DML数据操作语言}
增删改

INSERT INTO 表名(字段列表) VALUSES ('XX','XX'),...,('XX','XX'); 日期和字符串要加引号,int格式不需要引号。注意顺序,数值,数量对应好。

UPDATE 表名 SET 字段名=字段值 WHERE 条件
DELETE FROM 表名 WHERE 条件

数据查询最重要:
(1)单表查询
SELECT DISTINCT 字段列表 FROM 表名 WHERE 条件
ORDER BY 排序依据
GROUP BY 分组依据
HAVING 分组筛选条件

limit语句

单表查询

特殊查询

\subsubsection{DQL数据查询语言}
数据查询

\subsubsection{DCL数据控制语言}
控制数据的使用权限



\subsubsection{视图}
SQL-> 外模式,包括各个视图 -> 模式,包括各个基本表, -> 内模式,包括各个存储文件

create view 名称 as select 选择的项目




\subsubsection{索引}
普通索引

唯一索引:可以空
 
主键索引:不能空

组合索引

create [unique] index 名称 on 表名(字段名)
几十万行记录能感觉到几毫秒的速度提高。
加入索引会提高查询速度,增删的速度会降低。
大表,表中数据经常查询,建索引。
like 查询不支持索引


\subsection{数据库设计}
需求分析:分析业务和数据处理要求

概要设计:设计数据库和E-R模型图
详细设计:E-R图转换为多张表,逻辑设计,
代码编写:



subsubsection{概念模型}
实体,客观世界中的对象
实体的属性
实体间的联系:n对m的


subsubsection{E-R图}
entity-relationship
E-R图:矩形是实体,属性是圆形,联系是菱形
比如,教师-学生-课程;
比如,论坛用户-发帖-跟帖-版块
比如,商店-顾客

subsubsection{e-R图到关系范式的转换}
1对1的关系:部门-经理,部门这个表中加入经理编号这个自造的字段,把经理表中的主键加入。
1对多:科室-医生,把医生表加入科室编号。
m对n:学生-课程,构造3个表,中间的联系做一个表,转化成1对多
顾客-商品,构建新表(订购)

subsubsection{关系范式}
评估表的质量,评估规范标准。
1NF,2NF, 3NF, BCNF,  4NF, 5NF。范式越高,质量越高,代码难度越难写。
项目快就范式低,项目人力资源充足,就提高范式
通常3范式就可以。
1NF,目标是确保每列的原子性,每一列不可分割。不满足1NF的数据库不是关系数据库。
2NF,实体属性完全依赖主键,不能只依赖主键的一部分。如选课表(学号,姓名,课程号,成绩,系名,系主任),复合主键(学号+课程号),这不符合2NF,因为这里如姓名,姓名之和主键的一部分(学号)有关系,不是和整个主键有关系。更好的做法是分成两个表选课表(学号,课程号,成绩),学生表(学号,姓名,系名,系主任)
3NF,在2NF基础上,任意非主属性都不传递依赖于主关键字。
如系主任之和系名有关,系名和学号有关。相当于把一些部件放在子装配体里面。应该把学生表分成两个表,学生表(学号,姓名,系代号),系别表(系代号,系名,系主任);


冗余设计:
在业务频繁的时候增加冗余。

水平分表与垂直分表:
数据大时。
水平分表,把数据放在不同表中,如按照学号奇偶划分,大表可以提高查询速度,表中数据有独立性时如记录数据中的不同时期等的数据。缺点是会增加复杂度。
垂直分表,把不常用的信息放在另外表中。管理有冗余,查询所有信息时需要join命令。

读写分离
数据库备份三份,1个master库用于更新insert,update,delete,2个slave库用于查询select,read,





\end{document}