%%============
%%  ** Author: Shepherd Qirong
%%  ** Date: 2022-06-04 22:55:43
%%  ** Github: https://github.com/ShepherdQR
%%  ** LastEditors: Shepherd Qirong
%%  ** LastEditTime: 2022-06-04 22:56:09
%%  ** Copyright (c) 2019--20xx Shepherd Qirong. All rights reserved.
%%============




\chapter{Design Pattern}

\section{UML}
united modeling language. 
\section{Design Pattern}

\subsection{Principles}
Open for extension, closed for modification. 对扩展开放,对修改封闭。类的改动通过增加而不是修改代码实现。
Single Responsibility. 类对外提供一种功能。模块只能有一种被修改的理由。
Dependence inversion. 依赖接口。业务层与实现层都向协议层靠拢。
interface segegation. 一个接口提供一种功能。
Liskov substitution. 抽象类出现的地方都可以用其实现类替换。
composite or aggregate reuse. 优先使用组合,因为继承时父类的修改会影响子类,对象组合会降低依赖关系。
law of demeter一个对象应对其他对象尽量少的了解。



\section{Others}




\begin{lstlisting}
   http://api.open-notify.org/astros.json

   https://openweathermap.org/current

\end{lstlisting}


建议把xy push到局部vector中,把局部vector push到vec中

Notepad++里Run可以设置用.bat来编译运行,bat里面要先cd到其目录

penetration test, using attack tools.

进程因为创建而就绪,因为调度而运行,因为IO事件而阻塞,IO事件完毕或时间片用完回到就绪状态,运行完成后消亡。

c++看gcc-cp目录

它是个编译器,也可以画图写别的pythonjava之类的

我都没有研究过,我需要找点资料好好学习一下,个人觉着c用起来更像是地址的别名,fortran的更像是地址的别名的别名,两个的比较应该找一些开源的转换库看看,

\begin{lstlisting}
    using namespace std;在局部使用挺好的
   using FP = void (*) (int, const std::string&); //Replace using typedef
\end{lstlisting}


伸缩型数组,在c中比存指针成员的好处是1)不占1个地址的存储空间,2)最适合制作动态buffer,因为可以直接就把buffer的结构体和缓冲区一起分配。
\begin{lstlisting}
   struct MyStruct
   {
       int			i;
       float		j;
       double		arr[0];		//伸缩型数组成员
       //double	arr[];		//这样声明也可以
   };
   pf = malloc(sizeof(MyStruct) + N*sizeof(double));/*只分配一块内存*/
\end{lstlisting}
c\#生态还是局限


假如一种语言不允许运行期动态分配内存的话,确实只要有stack就够了,内存的数据区只划分成两大块,一块用于独立于过程的数据(编译、连接时已知大小),剩下的全部做为一个stack。C、C++这样的允许动态分配内存的语言,当然还是要有一个heap的,这样内存就划分成三个区域了。在引入线程概念之前,确实如你所想的,除了运行时动态确定大小的数据必须放在heap里面,其他都可以放在stack里,我们可以把stack设置的足够大。然而,后来线程出现了。每个线程有自己独立的stack,一个程序可以创建成百上千个线程,那么stack的大小也就只能是一个很小的值,内存才不会被用光。在windows系统里,每个线程的stack默认只有1MB,LINUX默认8MB,这样,stack就成了紧缺资源,必须非常有节制地使用,体量比较大的数据都要由程序员手工分配到heap里去并手工管理其生命期。

java 半编译半解释

反码:符号位不变,其余位翻转。
正整数的补码:是原码;
负整数的补码:是反码+1;
注意运算可能溢出。

\begin{lstlisting}
   a = 3*5,a*4;//60
   typedef int MyINT;
   using MyINT = int;

   decltype(i) j = 2;//the type of j is the same as i.
initializer_list


 \end{lstlisting}

 每一个练习封装成一个函数收集在一个.hpp里面,在一个main里面调用相应的练习函数

 我认为核心还是数据,设计类的方法要围绕着数据来,公开出去的方法要维护好隐藏起来的成员变量

 量子加密RSA

 原来stl的模板采用的那叫静态多态,编译期间翻译,所以模板的声明实现分离在头文件里实例一份就行了

