%%============
%%  ** Author: Shepherd Qirong
%%  ** Date: 2020-04-02 13:25:14
%%  ** Github: https://github.com/ShepherdQR
%%  ** LastEditors: Shepherd Qirong
%%  ** LastEditTime: 2023-04-15 22:24:33
%%  ** Copyright (c) 2019--20xx Shepherd Qirong. All rights reserved.
%%============

\documentclass[UTF8]{book}
\usepackage{ctex}
\usepackage{multirow,booktabs}
\usepackage{amsmath,amsthm,amsfonts,amssymb,bm,mathrsfs,upgreek} 
\usepackage[paper=a4paper,top=3.5cm,bottom=2.5cm,
left=2.7cm,right=2.7cm,
headheight=1.0cm,footskip=0.7cm]{geometry}
\usepackage{color,graphicx,verbatim}
\RequirePackage{setspace}
\setstretch{1.523}

\begin{document}


\chapter{Introduction}

记日记是确有必要的。

\chapter{Year2023}
\section{Month2}
\paragraph{20220208}

盘坐在荷叶上,滑动,滚落。


\chapter{Year2022}


\section{Month11}
\paragraph{20221102}
十八世纪的饭菜滋味。二十一世纪的饭菜滋味。

\section{Month10}

\paragraph{20221025}
昨天,我从楼上一跃而下,是13楼还是0.6楼,我看不清了。也可能是今天,或者明天。

\paragraph{20221024}
今天是1024,圣诞节快乐。我准备尽快创作一篇小说,一篇散文,一篇诗歌。

\paragraph{20221023}
我想起了当我小的时候,我在快速划过田野时瞥见的一些泥土和杂草,或许我没有想起。

\paragraph{20221022}
2022ian10月22日,压死骆驼的最后一根稻草。

\paragraph{20221015}
昨天想了八个字“盲人观花,我是我们”。
中午,家人烧炕。奶奶,大姑妈,三姑妈,姥姥,姨,家人都在,一起吃饭。我想聚聚是好的,人多热闹,吃饭有劲。


\chapter{Year2020}
\section{Month6}
\paragraph{20220610}


\chapter{Year2011}
\section{Month10}
\paragraph{20111025}
下3:体育课没读书,也有些疲倦。大家好好着。\\
下4:“所待”即关联,精神世界没有必然联系,内部有规律,规律是被完全把握了的。\\
我和谁都不争,和谁争我都不屑。\\
晚:吃饭时究竟看了杂志,高中的美好的友谊,没有我的份的倾向。\\
晚2:上节课,做了一道题,我用了麻烦的方法,自以为不好。唉,这估计是进步吧。这节语文,懊恼数课上的不抓紧。

\paragraph{20111026}
早上是一定冷了,不愿意起床。\\
今天是周三,我于是得第三次带着牵挂午饭了。

\paragraph{20111027}
早:大人物只能被搬出来,或抬出来。\\
哪怕全世界都推翻,全世界都混乱,全世界都将其遗忘。\\

上一:此时,记住所有好的东西,知识经验;此刻,催促回忆好的一切,构想。\\
到北京上学,可以买,可以逛。我要去,我想去够。\\

上三:我曾经哲学是抽象总括事物,现在改为细化感情了。\\

上4:做题做顺腿了。\\

中午:So attractive Beijing is that I'd like to go there.\\
初二的研究已达到很高水平了。\\
日子改变着我们的颜色。\\
脆弱:土壤贫瘠,养分在植物中,破坏难恢复。

下午:体育课跑了2圈又1000米,跑操4圈,果然我被误会了。我这样是好的不专注其它,可这在控制下就被干扰了。

晚2:数学自习又发了半小时多呆。每天发呆很长时间,我也很抱歉。

晚3:今日起,改掉所有小动作,除非不做专注的事。

\paragraph{20111028}
上午还算好吧。

中午: 路上,一只苹果的被吃。我因为人们的丰富多姿,千味百味的。苹果核儿没人去管它,人们的虚情假意,这那的。

此时,此地,此身。

没见过面的人称为“朋友”,观众朋友,读者朋友,小孩子本身就是小朋友。

下2:物理课也比较轻松,闲。

下4:你以为你能呢。莫大的奚落。


\paragraph{20111029}

早:人也便是这样,总不免成奴隶而骄傲。

早1:难道就该把老朋友遗忘,还有那旧日的时光?身边感到冷,眼前突然辽阔。

中午:吃苹果桃,读东西,睡觉,我觉得不坏。

晚1:我认为一个人都可以在主观上做到极致,至少我能,这样小心地想着。

\paragraph{20111030}
下一:中午,我在想如何处理那么多的书与文章,又一想,似乎并不多。以晚饭时的惬意为界限,划分出一段段的高兴。努力着,在路上。听从直觉和心灵的启示,不要在他人的思想下生活。

晚4:不落魄,晚上回家吃药,吃点东西,喝点水,饮料是没有了。有长远的计划,既然将来挺长一段时间往返于学校、家,就趁机多计划点长久的坚持。

\paragraph{20111031}
今天是十月最后一天,好好弄。

早,快结束了,也有些困倦,总归是积极的。

上三:我讨厌一切没来由的声响。

下4:我是上,又是上中,两起两领。

晚2:希望你懂得,人们是希望看到事物堕落下去的。

晚4:感觉今天很累,怎么也高兴不起来。


\paragraph{20111101}
早:真个是不行,今天实验昨天下午的“淡化生命、非生命界限”。

上1:活着,就会有改变的可能。

上二:也是睡呀睡,课间迷迷地睡着。

下一:使劲背英语。

下三:体育课上,我又要拍子有要球的,自我感觉挺不好的。

晚2:多么繁忙的一天呀,是太累了。


\paragraph{20111102}
早:昨天跑步时,“还是行家懂”,关于专业吸粪。

上3:虚,梦,过去事,未来事。

下4:今天起,一定做题一个个的。


\paragraph{20111103}
上3:昨晚睡得早,今上午累死我了。

晚2:你要调整自己的心情。

晚3:我可是得好好搞,弄出信心来!


\paragraph{20111104}
上2:数,写顺了腿。

上4:Wait until the rain slows down.

时间谁也夺不走,但你不夺它自己就走了。————英语老师

下1:知道就行了,不知道也没关系。————史师


\paragraph{20111106}
来了,确不想来。

将要吃饭,其实我也认为自己虚伪着。

我就多多地写,每天写。我用最少的时间用完4瓶墨水。

也多是无聊,看些基础,主动记忆。

晚4:一地残阳,漫天忧伤。

下3:其实,一点点积累才是真生活。


\paragraph{20111107}
晚2:半小壶温咖啡,我当水喝了个半饱。


\paragraph{20111108}
早:珍惜呀!昨晚是矛盾着学习游玩。

上一:人是会后悔的,但还得向前去。

下3:体育课上,生了许多感想。


\paragraph{20111109}
上一:困倦的兽醒来,我睡,要。

上3:明白起来吧,现在不是呼呼睡的时候。我现在挺难受的,脖子头,又想睡,还淡淡恶心。

下一:我盘算着今后的生活:
1. 做学案写上日期与心情,不要太急,慢慢来。\\
2. 清晨,复习昨天内容。中午复习上午,吃晚饭复习下午,晚上复习一天。\\
3. 学案一点是一点的,用心做,因为那是一大本,厚厚的。\\
4. 我不害怕一切,因为我已经做到好处了。\\
5. 咱想不到高级的思想,真?\\
6. 每天背。有大量的进,才能出。\\
7. 控制,别一弄就开小差。

下2:有了大量的自由思想的时间,心灵上不自由,又睡恹恹的。每天要积极生活。

晚1:晚风吹来,我却觉得糟糕了,因为今天要结束了。细致认真规范扎实。这个心情真好。看大方的你。如今,平常训练,十分重视。要进入考试状态,调整到考试兴奋点。适度紧张,节奏感,一系列连续性。积极面对下一科。基础性考试。

因为我们好过,所以我们必须好下去。
别审错题,第一位的是认真领会。

还有你,我熟悉的忧伤,你依然珍藏,或许我早该知道。【主体3变】

激动的对无法理解自己的人谈论着自己高兴的话题。今下午跑操,不是冷,是寒。闭上眼,四面八方的风碎裂整个世界,昏暗的阳光斑驳掉整世的黄土。重重叠叠交织着苍凉的情致。

晚3:结束了,我创了一种模式,又想着,以后一张学案一总结,每周总结概括,把握核心内容,干劲十足!练字的问题,我得好好写。

晚4:以前有一个白【bo,3声】子,一个黄【bo,3声】子,貌似我喜欢用白【bo,3声】子?或者用黄【bo,3声】子。泡面的时候一正一反,很方便。可是早没了吧......


\end{document}