%%============
%%  ** Author: Shepherd Qirong
%%  ** Date: 2020-04-02 13:25:14
%%  ** Github: https://github.com/ShepherdQR
%%  ** LastEditors: Shepherd Qirong
%%  ** LastEditTime: 2023-06-18 18:28:17
%%  ** Copyright (c) 2019--20xx Shepherd Qirong. All rights reserved.
%%============


\documentclass[UTF8]{Diaries}
\begin{document}

\title{MyDiaries}
\date{Created on 20200402.\\   Last modified on \today.}
\maketitle
\tableofcontents




\chapter{Introduction}

记日记是确有必要的。

\chapter{Year2023}
\section{Month2}
\paragraph{20220208}

盘坐在荷叶上,滑动,滚落。


\section{Month6}
\paragraph{每月演讲}
\begin{lstlisting}


    【0 引
    阿弥陀经中描写的极乐国土,“黄金为地。昼夜六时,天雨曼陀罗华。”
    废物和尸体应该是看不到的。(安妮·艾尔诺

    旅人并光共游,
    透明空寂盲目飞行,(安东尼奥·马查多)
    降落在中原腹地,中央戊己土
    在这里温和的步入极乐(狄兰托马斯

    【1 东 春
    东方甲乙木

    朝霞照耀(吉皮乌斯诗选)
    在春之菜园
    高于人类和时间6000英尺。(尼采


    一架架蜻蜓围绕着三叶草(米沃什
    肥胖的黄蜂伏在菜花上(鲁迅

   
    白玉兰,虞美人,风信子,郁金香,丁香与海棠,芍药与牡丹
    
    一切都已醒来,无根之树飞向开花的斧头(策兰,王阳明
    有把雨伞在屋里(毛姆
    

    

    【2 南,夏
    南方丙丁火;


    黄茉莉,夹竹桃,飞燕草,鸡冠花(谢明洲的「时光露滴」。


    不渴而饮

    
    
    
    大海把一个水手吞到深处(卡瓦菲斯
    
    
    一下干枯,又翠绿鲜活。
    
    
    如同黑夜,又像光明,广大浩漫(波德莱尔
    犹如喷泉。(莫言

    
    
    【西,秋
    西方庚辛金;

    绕水的土路上,
    弥漫着干枯的气味(希梅内斯

    木芙蓉,迷迭香,万寿菊,彩叶草,桂花与睡莲,牵牛与月季

    云高,
    like autumn leaves(泰戈尔
    一朵云不能把另一朵云怎么样(库切

    【北,冬
    北方壬癸水
    南天竹,君子兰,蜡梅与水仙

    满地梨花雪(周邦彦
    

    一盏在风中熄灭的灯(海德格尔

    睡意昏沉(叶芝
    老人正梦见狮子。(海明威

    【结, 边界





    【材料
    一花一世界,一草一天国。
    朋友们跟着我游览着极乐世界。
    光线中的光线,照破万朵河山,引燃百万飞鸟;柔浪中的柔浪,(伊凡·哥尔,毕肖普
    
  
    万物伤悲,化成一块木头,哭成一个风中的字,凝成缕缕思念。(张棋荣:哭成一块木头,哭成一个风中的字,哭成一阵思念。

    我只是一声拂动云雾的叹息,我将在永恒中飘散、消失。(普吕多姆

    炼月烹天(张岱
    
    所有这尘土,所有这些在复活和风化过程中的遗骸,所有这爱,所有这些石头,所有这悲哀。(阿米亥
    
    有的人走向幸福,有的人开始悲苦。(卡布斯教诲录
    
    仇恨的喊叫声(加缪
    
    玫瑰林中的月光
    忘川的水边,苍鹭
    
    
    
    泪眼问花花不语,乱红飞过秋千去(冯延巳
    
    斜阳暮(秦观
    采菊(陶潜
    春意闹(宋祁
    花弄影(张先
    水长东(后主
    
    独上高楼,蓦然回首
    
    一池萍碎(苏轼
    
    冷月无声(姜夔
    
    
    一番洗清秋(柳永
    
    
    被孩子丢弃的破碎玩具(里尔克
    
    如果和花闹了别扭,(小王子
    可以手指摘下来喂给他们。
    
    两条腿更好!(乔治奥威尔
    
    一些纠缠不清的、关系的蛛网在寻找形式。(卡尔维诺
    
    最后的真理是不会出错的(伽森狄,
    
    
    我梦想似玫瑰凋谢般柔和的诗句(撒曼
    
    再见月亮玫瑰,你好月亮玫瑰(阿拉贡、艾吕雅
    
    只当观众,不当演员。(笛卡尔
    
    重合关系(石里克
    
    
    
    
    



    
\end{lstlisting}



\section{Month5}

\paragraph{20230525}
长颈鹿,北极熊,猪八戒和孙悟空。

\paragraph{20230531}
5月最后一天。







\chapter{Year2022}


\section{Month11}
\paragraph{20221102}
十八世纪的饭菜滋味。二十一世纪的饭菜滋味。

\section{Month10}

\paragraph{20221025}
昨天,我从楼上一跃而下,是13楼还是0.6楼,我看不清了。也可能是今天,或者明天。

\paragraph{20221024}
今天是1024,圣诞节快乐。我准备尽快创作一篇小说,一篇散文,一篇诗歌。

\paragraph{20221023}
我想起了当我小的时候,我在快速划过田野时瞥见的一些泥土和杂草,或许我没有想起。

\paragraph{20221022}
2022ian10月22日,压死骆驼的最后一根稻草。

\paragraph{20221015}
昨天想了八个字“盲人观花,我是我们”。
中午,家人烧炕。奶奶,大姑妈,三姑妈,姥姥,姨,家人都在,一起吃饭。我想聚聚是好的,人多热闹,吃饭有劲。


\chapter{Year2020}
\section{Month6}
\paragraph{20220610}


\chapter{Year2011}

\section{Month4}
\paragraph{20110407之后20110411之前}

春天的衣服漫天飞舞,勤劳的牧羊少年早已把困倦的劳累的五彩缤纷的满含睡意的大白羊赶进了草场。绿绿得一塌糊涂,牧羊少年吹着口哨,蓝天被晃得东倒西歪。

一只梨子,或随地而生的芽草,桌上一块橘皮,开遍小花的粉树,或者空气中的和暖,脑中的血管,都洋溢着春天的欢笑,睡意沉寂。

\paragraph{20110423}
明暗变化吸引人,早晨的时光,滴露的浆果,带着雪碧的清凉甜意,蹒跚疲惫的魂灵。

美好的夜晚,这惊心动魄的契机,正是酝酿奇迹的关键时刻,而我,这牧羊少年,他在干什么?他该不会找不到迷路的羔羊,一只,了吧?他该不会把羊卖掉把钱丢掉把自由束缚在玻璃店而丢掉梦想,了吧?他该不会枯竭感知预兆的心,遗忘王给的宝石,了吧?

向前看,一片牛羊。只个好饱,牧羊少年上车上学,耳朵接受传递着车人的喊叫,牧羊少年没法笑,人间毒药。

飘来,牧羊少年热闹着,说;
飘去,牧羊少年沉静的,心。


\paragraph{20110529}

不必追溯上下了,现在是黑的吧,这夜,我真想多多地舞。

游戏的现实,假的梦,美的人儿。

我想像睁开刚闭上的眼的刹那,我成了一个新的我,一个完完全全是拼接了太多金的丝亮的线的我。舞蹈释放速度太慢了,我这会儿又是乐又是跳,心底的————提到心,又是在,多讨厌的俩字————讨厌的事物不去想它,有吗?那些我随时想杀死的人们。我当权,先屠戮,杀掉那些我认为不如我的人,留下那好的真的人。或者连他们也不要,因为根本没有我。

于是,王子和公主过上了幸福的生活......

爱我的多了去了,我也觉得她们可爱不是吗?是不过一个个失了时了。于是,最终————又一个讨厌的词,我欲发不可收拾,狂躁掀翻这......什么也掀不动。所以说,就是先证明投入的可靠性,既而全身心以退。

一段记忆的总结,往往是本儿在人心飞,在场艰难起来,破旧的灯光照着,泥土了双眼。昏旧厚重的感觉,来自,来自————这是更讨厌了。

其实文字的预言极不可靠,为此,用糊涂的笔调写内心,表现出入世的为难,借此应该能得到不行无力的结果。其实不是。


\paragraph{20110529至2011110904之间}

象征没有什么实利的价值却可积蓄和释放巨大的精神能量。革命中的象征性事件有画龙点睛之功,以一种有声有色的行动为革命造型和成像,以一种历史创造力的爆发聚焦着大众的理想和激情。生活本身就成为了艺术。......

现代社会的行为无所不有而一切行为都不能让人动心,一个行为与艺术相融并已失去边界的时代,是行为与艺术都在挣扎焦灼的时代,人类行为的闪光和雷电能否出现还?


 





\section{Month10}
\paragraph{20111025}
下3:体育课没读书,也有些疲倦。大家好好着。\\
下4:“所待”即关联,精神世界没有必然联系,内部有规律,规律是被完全把握了的。\\
我和谁都不争,和谁争我都不屑。\\
晚:吃饭时究竟看了杂志,高中的美好的友谊,没有我的份的倾向。\\
晚2:上节课,做了一道题,我用了麻烦的方法,自以为不好。唉,这估计是进步吧。这节语文,懊恼数课上的不抓紧。

\paragraph{20111026}
早上是一定冷了,不愿意起床。\\
今天是周三,我于是得第三次带着牵挂午饭了。

\paragraph{20111027}
早:大人物只能被搬出来,或抬出来。\\
哪怕全世界都推翻,全世界都混乱,全世界都将其遗忘。\\

上一:此时,记住所有好的东西,知识经验;此刻,催促回忆好的一切,构想。\\
到北京上学,可以买,可以逛。我要去,我想去够。\\

上三:我曾经哲学是抽象总括事物,现在改为细化感情了。\\

上4:做题做顺腿了。\\

中午:So attractive Beijing is that I'd like to go there.\\
初二的研究已达到很高水平了。\\
日子改变着我们的颜色。\\
脆弱:土壤贫瘠,养分在植物中,破坏难恢复。

下午:体育课跑了2圈又1000米,跑操4圈,果然我被误会了。我这样是好的不专注其它,可这在控制下就被干扰了。

晚2:数学自习又发了半小时多呆。每天发呆很长时间,我也很抱歉。

晚3:今日起,改掉所有小动作,除非不做专注的事。

\paragraph{20111028}
上午还算好吧。

中午: 路上,一只苹果的被吃。我因为人们的丰富多姿,千味百味的。苹果核儿没人去管它,人们的虚情假意,这那的。

此时,此地,此身。

没见过面的人称为“朋友”,观众朋友,读者朋友,小孩子本身就是小朋友。

下2:物理课也比较轻松,闲。

下4:你以为你能呢。莫大的奚落。


\paragraph{20111029}

早:人也便是这样,总不免成奴隶而骄傲。

早1:难道就该把老朋友遗忘,还有那旧日的时光?身边感到冷,眼前突然辽阔。

中午:吃苹果桃,读东西,睡觉,我觉得不坏。

晚1:我认为一个人都可以在主观上做到极致,至少我能,这样小心地想着。

\paragraph{20111030}
下一:中午,我在想如何处理那么多的书与文章,又一想,似乎并不多。以晚饭时的惬意为界限,划分出一段段的高兴。努力着,在路上。听从直觉和心灵的启示,不要在他人的思想下生活。

晚4:不落魄,晚上回家吃药,吃点东西,喝点水,饮料是没有了。有长远的计划,既然将来挺长一段时间往返于学校、家,就趁机多计划点长久的坚持。

\paragraph{20111031}
今天是十月最后一天,好好弄。

早,快结束了,也有些困倦,总归是积极的。

上三:我讨厌一切没来由的声响。

下4:我是上,又是上中,两起两领。

晚2:希望你懂得,人们是希望看到事物堕落下去的。

晚4:感觉今天很累,怎么也高兴不起来。


\section{Month11}

\paragraph{20111101}
早:真个是不行,今天实验昨天下午的“淡化生命、非生命界限”。

上1:活着,就会有改变的可能。

上二:也是睡呀睡,课间迷迷地睡着。

下一:使劲背英语。

下三:体育课上,我又要拍子有要球的,自我感觉挺不好的。

晚2:多么繁忙的一天呀,是太累了。


\paragraph{20111102}
早:昨天跑步时,“还是行家懂”,关于专业吸粪。

上3:虚,梦,过去事,未来事。

下4:今天起,一定做题一个个的。


\paragraph{20111103}
上3:昨晚睡得早,今上午累死我了。

晚2:你要调整自己的心情。

晚3:我可是得好好搞,弄出信心来!


\paragraph{20111104}
上2:数,写顺了腿。

上4:Wait until the rain slows down.

时间谁也夺不走,但你不夺它自己就走了。————英语老师

下1:知道就行了,不知道也没关系。————史师


\paragraph{20111106}
来了,确不想来。

将要吃饭,其实我也认为自己虚伪着。

我就多多地写,每天写。我用最少的时间用完4瓶墨水。

也多是无聊,看些基础,主动记忆。

晚4:一地残阳,漫天忧伤。

下3:其实,一点点积累才是真生活。


\paragraph{20111107}
晚2:半小壶温咖啡,我当水喝了个半饱。


\paragraph{20111108}
早:珍惜呀!昨晚是矛盾着学习游玩。

上一:人是会后悔的,但还得向前去。

下3:体育课上,生了许多感想。


\paragraph{20111109}
上一:困倦的兽醒来,我睡,要。

上3:明白起来吧,现在不是呼呼睡的时候。我现在挺难受的,脖子头,又想睡,还淡淡恶心。

下一:我盘算着今后的生活:
1. 做学案写上日期与心情,不要太急,慢慢来。\\
2. 清晨,复习昨天内容。中午复习上午,吃晚饭复习下午,晚上复习一天。\\
3. 学案一点是一点的,用心做,因为那是一大本,厚厚的。\\
4. 我不害怕一切,因为我已经做到好处了。\\
5. 咱想不到高级的思想,真?\\
6. 每天背。有大量的进,才能出。\\
7. 控制,别一弄就开小差。

下2:有了大量的自由思想的时间,心灵上不自由,又睡恹恹的。每天要积极生活。

晚1:晚风吹来,我却觉得糟糕了,因为今天要结束了。细致认真规范扎实。这个心情真好。看大方的你。如今,平常训练,十分重视。要进入考试状态,调整到考试兴奋点。适度紧张,节奏感,一系列连续性。积极面对下一科。基础性考试。

因为我们好过,所以我们必须好下去。
别审错题,第一位的是认真领会。

还有你,我熟悉的忧伤,你依然珍藏,或许我早该知道。【主体3变】

激动的对无法理解自己的人谈论着自己高兴的话题。今下午跑操,不是冷,是寒。闭上眼,四面八方的风碎裂整个世界,昏暗的阳光斑驳掉整世的黄土。重重叠叠交织着苍凉的情致。

晚3:结束了,我创了一种模式,又想着,以后一张学案一总结,每周总结概括,把握核心内容,干劲十足!练字的问题,我得好好写。

晚4:以前有一个白【bo,3声】子,一个黄【bo,3声】子,貌似我喜欢用白【bo,3声】子?或者用黄【bo,3声】子。泡面的时候一正一反,很方便。可是早没了吧......


\paragraph{20111110}

早:考试了,适度紧张,我要百分百自信。

10:55,原来是这样,我认为该回家吃饭了。

1:37,中午睡了挺长时间,今儿起,克制了我要!咱们得努力了!

第一次月考时,一女生化卷满是写画,着实令人吃惊,另有半袋零食等。

快6点了,要嘚瑟,一有地位,二有领域。

晚一:长大了有钱买些精细的动手的玩具,实体木模型、变形金刚,又大又好,晓的戒指也是要买的。

晚2:一个人的无聊,是因为他没有一个可以进行下去的话题。

晚3:To make the best use of the 596 days left, I want to read 3 pages of words daily.

我得搞个每天N问。


\paragraph{20111113}

上,自习着,也忙也闲。考试,也好也坏。昨天今天,日子太悠闲了。

10:15,“有大量东西需背诵,当奋当奋!”

2009年4月21日周二晚9:50,“不行,以后必须快,时间太快了!没时间做别的了!”

上午,即将放学,正在放学,要写东西呢,正。

只有自己忘记在哪里,才算真自在。

下午3,人也真是有智慧,做这那的事。

彩虹,是看得见的那种远,就像记忆,或许早已在思念中悄然沉淀。

晚2,如果要胡言乱语,我估计是最后到下的。

晚4,自信,努力向前。


\paragraph{20111114}

这种抗敌奋战的能力。

看看我们的正道能否感化邪道。

大肠杆菌被用完就死了,人活完了,经历够了就死去了。

在一个角落里默默地哭泣。肯定了那么,这东西连问不用问。


\paragraph{20111115}

晚1,陷入生活太深了,陷入数学太浅了。


\paragraph{20111116}

早:昨晚得1点半了吧,睡不大好。一种画,看起来看画的人在看画,又在画中,估计是埃舍尔的画吧,找找。

我越来越不照套了,思想越来越少了,中午,学会了不对富贵谄媚,人是容易向权势卑躬屈膝的,另外,不要诋毁部分人。金钱与骗子无关。

晚4,所谓英语竞赛,周日上午大约没休息的可能,我因而失望分心,加上晚饭没怎么吃,又因为昨晚睡晚刚又喝冷咖啡,数习不在状态,因而学了一节课没有主动思考,这种状态实在是不好。


\paragraph{20111117}

代正玲一月2次。

早遇伯阳,说见我左右晃,最后说“中了昂”。

下一:凌汛,你提到了凌汛。

晚一:时间计划上,汤水不漏。


\paragraph{20111118}

内含子被舍去不表达,表达的是外显子。

对立统一,是非,并列共存。


\paragraph{20111120}

每时每刻都在思考,以摆脱他人的思想。自己的心寂寞地出来,又寂寞地消散。

自己可爱着不是大心痛,那叫欢乐的忧伤。


\paragraph{20111121}

早,是愿睡不愿醒来的早上,黑着的天。

诗,穷而后工。

下2:交作业,全称命题,不是存在性命题。

人生容易与困难。

别人痛苦着你的痛苦。尽最大努力满足,做人。

情商。一直见一个人,貌似没那么苍老,一抬头,过去了半个小时。

晚9:35,灵验了。我先是发出声响,让凳子而已,而后做个手势没事无状况,而后看到了“右手螺旋定则”与“电流”之类云云。


\paragraph{20111122}

我虽在某时感到一种神奇,但此时又睡了自己。

M2,又是我想到疾苦,节约;有时我又大花钱要。

N2,什么时候才能有足够宁静的思考。迎着风,就像烛火,悠悠天明。


\paragraph{20111124}

下4,完全一样只不过有点微小差别。————物师。


\paragraph{20111125}

9:16,灵验了,数学课上。


\paragraph{20111126}

大艺术家能支配一个幻想世界。


\paragraph{20111127}

春夏秋冬多少人会走,春夏秋冬多少人会留。

少年,不一样的颜色。

上2,某日体育场,某高三,别人答英语靠语感,我凭手感。

丈量信仰的神圣。

永不放弃青春的承诺。

滂沱雨,无底涧,涉激流,登彼岸。

忆君清泪如铅水。



\paragraph{20111128}

上1:我们就是研究。

早自习做生物题没做够,另妈回来了。

上3,古代的忠君,比爱国更变态。



\section{零散收集}

终于可以停下来,喘口气。

窗外草坪损绿如肤,高大的柏树壁立的黛色隔离了喧嚣的市声。鸟翅哑划过。隔着窗玻璃,我知道它一定发出了自己的声音,我宁愿以心听,听自己内心的风景,比窗外景物更真切。

感到内心深处对衰老的抗拒,时间在某个柔软若幻的地方停止了。

他的走就像是被迫一样;而我的到来也没有任何充足的依据。

我为这个世界因为某个念头就被改变而感受了一种游戏的快乐。

一站到白天的立场,晚上是梦。时间就是无形的水,让人飘浮,是它分隔出一个个世界,新的旧的交织成一体。世界与世界重叠、互换,熟悉与陌生交互,我从一个世界进入另一世界,再也无法坚守一个清晰不变的自我。

《中国散文双年展》【祝勇】
纵情之痛,冷冰川。

天真的人就是让雨落在自己熟悉的地方。冰冻的人冒着泥土钻出的大风刮来的热。我仅仅喜欢简朴的东西,我无法对此作出解释,我为此感到高兴。

人不应该不如作品的自由更自由。

纵横歪倒贵天真。思想的狐狸。

每一次创造就是一次生命,创造性是原朴的天真的诗性直觉的觉醒。

激情是逃亡,逃亡是自我约束。

艺术因其满是猜疑,才满着魅力。

我的幸福在于我拥有的那些趣味和无法无天的勇气,我的才情独一无二。

天才是特殊需要的时候出现的那种人,有时会令人生厌。

我无法像我想像的那样成功,我的手永远无法准确地表达我自己。

天空躺在一片湛蓝的梦里,身边那些云朵是天空飘渺的梦幻,我们在虚空的世界里找不到栖身的地方。亮晶晶的星却永远被镶在蓝色夜幕上,它们将人世间的荣辱一一铭记。我们也会有在不经意间删除的记忆,我们想尽量拥抱生活,却不断有狂风雨浇熄我们的热情,其实早就学会了不向生活过多要什么,只怪弄云各形态的风把我们吹新了形状吹向一段新的开始。

今天中午下些雨,吃完饭的时候,现在于是很不热,凉着的也有的风在,都盼着日子,一天天快了,两个三,小日子慢慢的,仿佛一周了快,其实一半大约不到。游戏中东西便宜了,买来钱有四百多万了,之前九百多,汗。没有任何想法的感觉很充实。因为热考虑多的是怎么在这里学下去,继续;课怎么进行,下去还。

20110904,9:00,至于新的目的,风沙太大了,说出来就给吹散掉了。

20110917,因为大家都搞成了八爪乱足章鱼,爬来游去的我们,我依旧看到了那些,我不以为真的头脑。

20110925,回不来了的,还有那个大手电筒的巨人,踩着远方踩走了。......所谓糊涂只能是装的,事件的意义与本体意志无关,价值是影响的合集。

20111006,好怀念那些思想因为自己认为着可爱。

20111016,因为有一种感觉:高古,不见了。是真的不见了,有回不来的可能。......睡着也累着,昏昏酸的。

起来,就不会走路了。回想着左一右一,歪歪扭扭着来,晃晃笑笑,这个黄昏的世界,不叫混沌,男女着可爱。可不知什么时候又消失了。

20111025,吃素饺子吃榨菜吃咸菜。雨天有泥水,可是少灰尘。跑操时又想起了时空转移,那卡卡西的术。

20111028,路上,一只苹果的被吃。我没有能力转换主体,中心散了,我聚拢不起来,成不了个儿。

20111029,我一直在想,如何不挣扎,如何一下子便跳脱出去。反复着反复,那些大智慧,你的心。回不去的只是我。

主题游离,心归一,一直追求高密度的东西,书的我,睡,笑,SS的我,嗖嗖的我,被怕的我,厕所的我,我的我,的我。

每一次成长,每一次仰望。

衰老的身躯拖垮思维的节奏,思维拉长,旋转,扭成面,一堆堆。用水丢水,痛了谁?去承认一种刺激。

我的周遭是简洁明快的线条。


\begin{lstlisting}

    关于《生日歌》
    一个人的生日,大家的歌。
    幻想也无妨,就算逃亡。飞翔。
    岁月的锄头太重了,拿不起放不下。
    声音都是从天外来的,一只小钟。
    瞬间的美是消失的美。
    理想根本不存在,只有想像中的自己。
    我有一个木书房,金银珠宝全是硬的。
    闭上眼睛便不知她是否眼睛了。
    引入一个意向,鸡腿散发的不是幽香。
    感冒了,北风一着急,摇落散香的花。
    影子转身,谁惊了谁?
    为了花落,不得不打开花苞。
    抗争的结果,是顺应自然。
    一路走,坏了鞋,成就了脚。
    沉重的声音不能飞翔,移动成遨游。
    雪与风无关,自始至终雪都是无声地落。
    文字与感觉的距离,长长长长成我的思念。
    艺术是对自我的审判,是痛苦的自我复制。

    20111118,我是车轮我左转右转我是飞。
    着急了我,一直忙着,睡觉,喝中药似的咖啡,被叫。
    下大雨,我是呀呀咿。我是被笑,如果可以。我们是舞台,希望被掌。我是许多咖啡,我是瞌睡仍然。
    一世烟尘一世风,几宵飘雪几宵情。雪飞风刀冷,太阳出来,软了世界,一片晴空。某浓度,我发酵。我是浓的,我无有精神,我大睡不醒。
    一个站不住的人,经起了怎样的风雨,我飘摇着走,梦中游。
    能解决很多事情,我没有时间,回忆都是假的,没有橡皮。
    我笑了,哭成一块木头,哭成一个风中的字,哭成一阵思念。

    20111203,《在一个冬天,怎么睡着》
    冬天是红的。冬天是火炉,是白骨。白骨是红的,罗宾汉走着想着。罗宾汉给梦踩死了,他也变红了。红的山,红的人,红的冬天。
    冬天是绿的。大地给风吹走了,行人们走没了腿脚,忽忽地乘着风,凭一双眼睛,绿的。明晃晃的坟,鬼们热烈地歌唱,一场盛宴。诡怪的萤火,绿绿地闪。
    
\end{lstlisting}
 



\chapter{Year2010}

You are highly reagred!

要说!



\section{Month4}

\paragraph{20100424}

Money: Red pen, 8 yuan. Black pen, 15 yuan...


\paragraph{20100428}

Windy, strong wind, sunny, cloudy. 
Well, every afteroon, there are many students waiting for entrying around the gate, smiling and talking. I am among them. So is XMY.
在路上,有对自己说话。


\paragraph{20100429}

Experiments. Why do I speak to others? The only person in the whole world that can understand me quite well is me! Try to say good sentences to myself.


\paragraph{20100430}

Tommorrow is Labour Day.世路如今已惯,此心到处悠然。Get ready for something a long time before you want to do. In such fun wether, I will make so much great progess. What if I became an English-speaking boy? Amazing!


\section{Month5}
\paragraph{20100501}

做作,装傻,骄傲。


\paragraph{20100515}

我在考虑是否随身携带一面小镜,随时让饱满的春光反射到我心灵的谷底。

\section{Month6}
\paragraph{20100604}

该丢的我及时丢,因为死时什么也带不走,于是说只为曾经拥有。

Our friendship bagan with an unexpected beginning.


\paragraph{20100605}

I will not forgive you, only if there is a good reason.


\paragraph{20100606}

Day by day in every way, nobody can be a tree.


\paragraph{20100607}

What others think of me has nothing to do with me.


\paragraph{20100608}

Long time no see. How have you been?


\paragraph{20100609}

凡是写下的,都值得研究。

同样是水,有的进入我身体,成为一部分的我;有的进入下水道,脏了自己,令我伤心。我脏了水,洁了衣物。


\section{Month8}
\paragraph{20100831}

我英语:变化多端,高深莫测,千疮百孔,千言万语,跌宕起伏,宛转悠扬,抑扬顿挫,栩栩如生,优美动听。 





\chapter{Year2009}

探索史也是失败史。

\section{Month1}


\paragraph{20090123周五,晴:一篇整的文章}

《关于“创造性”的联想》
【文章粗糙,评述在此:从歌德的话引起,到毕加索,到演员,想到武林外传。】

“想到那些演员们,他们的想像力蛮丰富的,把自己想像成一个又一个人物,在一个又一个世界中生了一次又一次,喜怒哀乐亦随剧中人物,真是好玩啊。”

“我一定会构思一个精彩的剧本。”

“2009年1月25日大年三十晚5:40前写完了。”

\paragraph{20090123周五,晴:一篇整的文章}

【写了2篇文章,大年三十】

《三全“食”美》【摘录】
今天晚上吃饭时,我忽然间发现我与爸爸妈妈的关系是靠食物联系起来的。联想到一患绝症男孩以一颗纽扣自喻,比喻父母的关系靠自己维持后,不禁哑然失笑,真的吗?

平时总是妈妈掌勺给我做饭,一天也就三次见妈妈的面,即吃饭时。

春节已经来了,而我却没感到春节的气氛。早饭面条加鸡蛋,菜一点剩的也没有,有的菜肴就是香肠,硬邦邦的,没口味。

鬼才知这个春节怎么过!早饭后兀自兴叹起来。但中午却大大不同了。老妈又买了超多的水果瓜子,仅此而已。但这着实令我吃惊。

更吃惊的还在下午。老爸从老家回来了,带回了一大堆乱七八糟东西,我知道他一准没带书,因为他没对我讲话,唉,我最爱的书没带回来。但食品也不错。老妈老爸早在那议论开了:“这个挺好的”,“那奶什么牌的。”“啊,还有蒜苗呢”......

最丰盛的是晚饭。谁叫今天年三十呢,与早餐面条午餐泡面相比,晚餐太丰盛了,什么鸡鸭(有吗?)鱼肉,海参鲍鱼(没有的!),这令我着实眼花缭乱了:“哪儿刨活来的?”水饺是重头戏,这时才想起来老妈忘记放硬币了!没关系,我们都未发觉,都还挺高兴的。

为什么高兴?谁问的问题?太俗了!今儿什么日子?大年三十。吃饱了,往沙发一靠,吃着干果水果,喝着观世音(特铁的那种),以特舒服的心情攀看春晚,谁会拥有悲伤凄凉的心境?谁心里都得装一个字“乐”!

\section{Month9}


\paragraph{20090906:一篇整的文章}
2009090611:30,周日,天大晴。

《初三了,我想对您说......(最新的思维世界观与方法论)》

“您”者,敬称也。“老师”一职业,诚然有大有小,但如“世界归根结底是我们的”一样,到底还是伟大的。

对教诲心灵者,我想说:您的举止言谈,小子虽小心,但终究会领会用意的(终究会影响到我的)。若当时未表或信息或惭疚态,请原谅之————当然这是废话。若当时心理活动未被觉察,请原谅之————当然这也近乎废话。

对传授知识者,我想说:思路广阔,因材施教,勤思生智,看来与学知识的态度相当,这是自然————学成熟后授业,虽小于等于学者科学者,但大大大于学生。

我偏好思维世界,故想对大脑说点什么。首先要勤思,其次要多思,当然应该精思。我的大脑啊,你看看你,怎么这么迂腐!人家爱迪生几年前就知道了“择书而阅,择业而从”,人家侯宝林比你小时就比你勤奋,人家......突然想到了读书读到用冷水吸头的范仲淹,若记载为真,我便释然了,————批评到此结束。

初三了,自然想到了高三————小学三年级时想到小学毕业,感觉还长呢,于是终止了;来古城后不久读到《海燕》,想到初二才学,想得美好起来了;如今是早已感到自己的差劲了。

初三了,我对你说,别骄傲啊!看列宁一年读完了四年的课程;也别自卑啊!现在与华罗庚在同一起跑线上。初三了,我的大脑啊,该醒了。

哈哈哈,我还活着,不是吗?(12:00)

【END】

【老师评:今后作文要向正统,拿出方向训练。毕竟咱们的学习要通过中考这一条路,你明白了?20090908】

【自评:20090912晚9:15。我明白了。我在好好写字。我在好好修正思路。明白了难懂的难看,明白了平深的深度很大。我认为困难,于是思考得弱了,我认为平实,其实无聊;我认为很有思想,其实很浪费时间。毕竟走错了立即(坚持了一会)回来,是不太坏的。】


\paragraph{20090911:一篇整的文章}
20090911,晚8:45

《俭朴容易勤劳难(一开始的“精”字后的回顾与认识强化)》

出生在一个平凡的家庭,我只想这样说。可“平凡”不等于“简单”,我们家的现有矛盾是由俭朴造成的。

俭朴,确是我们的长项。我爷爷不多言语,节约成癖(无贬义);到我爸爸,更是头脑简单到能省则省。现在是明白了,这是一个大陋习。首先因为爷爷奶奶爸爸妈妈等都是平常的好人,也说明他们不勤劳。勤劳是努力,是奋斗。俭朴乃穷人的财富,正是他们没把俭朴当成一种处世方法,而当成了必然的人生中的一部分,这就远离了“俭朴是富人的智慧”。于是明白了我家的总体不富裕与眼光短。(爸爸的兄弟除外。)

由于出生及所处的年代,我们的爷爷奶奶辈(可戏称“奶奶辈”,略含嘲讽,惭愧)会自觉不自觉地认“(贪污和)浪费是极大的犯罪”(毛泽东语)为金科玉律。我认为即使有些片面,“人生极处是体验”也是值得借鉴的。

随认识能力的发展,我可能(或一定?)会认同奶奶辈,但他们的“不勤劳”我是定要改正的。我是说我要努力使自己更灿烂些,毕竟人生是有限的,多重复一次就少一次意外的可能。况且我不认为司马光的“由奢入俭难”适合我,因为我相信目前我受周围环境影响程度不浅。

综上所述,现阶段奋斗目标是实现由“俭”到“勤”的转变,即难度颇高的思维定式的突破(我认为即使一次头脑中的化学反应,只要条件合适,就会被重复进行数次)。

【END】

【老师评:观点挺新,字要好好写!20090914】


\paragraph{20090919:一篇整的文章}

20090919,16:35-17:25

《傍晚,真实的小城复活(对偶与通感,比较不同时段的小城)》

我所在的小城市,一天的不同时刻给你的感受是不同的。傍晚时的小城,我认为是最真实的。

太阳没落山之前,小城中的人们就加快了步伐,或许他们在回家的路上。经受了一天的阳光照射,小城也该休息了。首先在你视野中暗下去的是小楼与小楼间的小土路,是大树与小草,是建筑工地上的红砖黑泥黄土。自然,你会看到满街乱蹿的黑色白色轿车,挤满眼帘的膨胀中的参差的街道,以及与你若即若离的尘粉。总之你会看到凝结的白天的小城。

再等一会儿,当路灯照出你的影子时,小城复活了。大街是琥珀色两侧镶有无数珍珠的绸带,不急不慢地随车灯的亮灭起伏;车灯,外酥里嫩的肉丸似的光亮飘来,飘来模糊的开车人闲适的目光,飘来水沸时的四下里的吵嚷。寂静处更寂静,但介于白日与夜里之间,是肥皂般的白滑之地;房屋里外更有趣味,介于白日与夜里之间,是诠释一天与谋划一天的最佳时间地点。

傍晚了,小城,摘下了给太阳看的面具,在戴上给黑夜看的面具前,她得到了片刻休息。白日里工作的人黑夜休息,还有一种人相反。但傍晚多是休息的时段,所以这时的浮华淡了。

真实的小城是美的,不真实的似乎更美。入夜,有梦者随小城睡去,多梦者或无梦者则继续制造涟漪。

但白日里与入夜后的小城是充实的,所以真实的小城只在傍晚复活。

【END】

【老师评:比喻很恰当,好!】



\section{Month10}

\paragraph{20091004:一篇整的文章}

20091004晚6:10

《仪式仪式什么都不是(表达得不好。没否定“仪式”,想着美女,有整体感,思维有中断。)》

【原文每段起始为带圈序号,此处以带括号的序号代替。20230523注】

(1)有一种句子,你会不自觉地以为好,虽然不知作者,但你也会因它们的更新之快而叫妙。它们讲究韵脚与内容,后者常常是不雅的。

(2)有一个节日,你会看月吃饼,你会觉得那天得团圆一下。当然也有人搞小小的祭祀,也会看对自我的小小犒赏与回顾,顺便吃点鲜食美味。

(3)对那“谣句”(姑且这么叫吧)的记忆淡淡淡了,今重提一是今上午听到了,一是曾编制过几则。而对中秋节,因为昨天才过,所以谈谈认识吧。

(4)我有一个漂亮的小妹妹,中秋节的下午我们手拉手,走啊走。渐渐地,我高兴了————我们拉着手呢。看看路人,都平庸着,于是更高兴了————我与漂亮的妹妹并肩走呢!可是,当我们从“那个地方”(我不说是哪里)出来后,我便不满足了————大街上牵手算什么呢————可又无处抱怨。

(5)晚上,我自己了。中秋节耶,我又有委屈了。但在不高兴的下一瞬我又糊涂了,再下一瞬又高兴极了————我看到对楼一老奶奶在烧香拜着什么,而且似乎觉得带给我的震撼不够,在我眼皮底下忽地跪下了。这令我有点哭笑不得:活了这些年了,你怎么......既而一想我不是不会那样做吗,便又高兴地看星星了。

【原注:写(6)以后时感觉不好(包括6)】

(6)结婚确实是件大事。大约三年级前曾规划着结婚用999辆而不是99辆,大约五六年级时设想结婚仪式上的我们,大约一个月前想象过闹洞房时的我和她......当然虚荣心现在还有,也会因它变成“实荣心”概率的不大而沮丧。

(7)每个人都希望被了解。我不认为以后若做好鸳鸯时会不讲究“排场”,但比今天所见过的新郎要好是一定的。想着想着又皱起了眉头:为什么能预见未来的我法力在降低?为什么法力高时也为看得太远?想想我大概什么也不是。

(8)最后说一下人心的善变与善隐。我的妹妹吧,昨天我们那么好,可今天我就看到了她的浮躁,可能是昨天的心情太好了的缘由,也可能是出书店后的沉思忘了细节的缘由,我回忆不出昨天她的不好。鲁迅的长妈妈爱“曹曹切切”,可背后说人话的人谁都是。因为没人能感知到你的想法,于是随你爱恨吧。

(9)若“谣句”是重复立异者的仪式,牵手是小小幸福的仪式,婚礼是不甘被人忽视的仪式,沉默是思想杂乱着的仪式,那么可见仪式需要标新了。

【END】

【老师评:内容好!20091013】


\paragraph{20091012:一篇整的文章}
20091012晚7:10

《续写无钱于勒回家门(续写《我的叔叔于勒》,外貌神态描写弱,不善于用多个人物)》

我是早发现我的叔叔曾看我了,只不过没与任何人讲罢了。那天他找到了我,我把他看作陌生人接待,他对我的举止感到诧异,我不奇怪,因为我肯定他是在确定了我是他侄子才找我的。说完“不认识我了”之后,他的第三局话令我意外。“我会还我哥钱的,我知道过去不对过,但现在我在努力赚钱。”我瞅了一眼那一折的裤子,答应与我父母商量一下。

在我与姐的努力下,父亲列了一大张条例。我递给于勒,发现他灰黄的眼更灰黄了。我认为他会同意的,可他说,“我没有那么多钱,真的没有......”

以后我再也没见过我的于勒叔叔。

【END】

【老师评:内容好!20091013】


\paragraph{20091018:一篇整的文章}

20091018,中午

《大根大器(没有大根大器,活着就是活着。写了许久前的思想,以后要更新。)》

有个禅宗公案,大致是说:佛祖要讲大乘佛法时,五百罗汉退去了,接着佛祖解释说他们不能一口气吞下长江水。说清楚一点就是不是所有人都能受得了同样的东西。但这没有什么可抱怨的,因为我们在修炼着,修炼着我们的大根大器。

“你认为你有大根大器,你就有大根大器”固然是唯心主义观点,但它能帮助我们认识自己。

人最终会死,那么我们活着为什么呢?改造这个世界。有这样想法的人是幸运的,他们会不断前进。结果大多是空虚,但若一超越平凡便觉得自己不安分了,那就太悲哀了。

书籍的出现可以使我们死得更有价值一点————死时你的想法仍在更新中。鲁迅死时说“倘能生存,我仍然要学习”,萧伯纳写墓志铭说“我早就知道,无论我活多久,这种事情总会发生”,他们之所以仍能感动我们,在于他们大根大器了吗?如果答案是肯定的,那么我的墓志铭应为“我也不愿意这样”。

我还在学习,书对我还有用,因此现在我不悲哀;也因此将来我会因为这句话悲哀。因为这种种想法,所以羡慕那些“书虫子们”,尤其是那种刚出生的或饿了好几天的,那时的欲望是很大的。

说大根大器的,想到了虫子。其实虫子与一条江河没什么不同,吞不吞的下与读不读书没什么不同。因为我还没死呢,我还可以小根小器呢!

【END】

【老师评:将自己的作文思路向正规中学生作文方向靠拢!20091019】


\paragraph{20091024:一篇整的文章}

《百分之百的精彩(有段韵脚较和谐,句子不连贯,有一点思想,无谋略)》

我们这些“乳虎”“朝阳”,绝不是循规蹈矩、彷徨无为的一代,我们渴望潇洒地走这一遭,希望做事百分之百精彩。

我们渴望被肯定。是谁背后说过我们太多坏话?是谁认为我们迷迷茫茫的,很傻?马上改变你的看法。我们有很大追求,我们能一点点地追求,我们不怕失败但求无悔是希望————百分之百精彩。

我们渴望个性张扬。每个人都认为自己很有思想,我们只是想不断实践不断探索。使自己更睿智的道路有千万条,我们只会走自己的。在你眼中我们的一切可能都不完备,但我们认为够精彩。

我们知道绝对的精彩永远不存在,可对它的追求永不会更改。即使同行者都倒下,没人理睬;甚至是方向错误违反规则被淘汰,我们仍会高抬头大步向前迈————我们选择的道路百分百精彩。

我们在课上“逍遥游”,因为一个光头的童话作家说这样才有想像力;我们疯狂地一天读上百篇英语文章,因为李阳的一句“我疯狂,我成功”。你说我们渴望成功渴望快乐渴望体验一切幸福,可我们不够现实不够成熟不够精彩。

我说我们有短促的专注没有长久的坚持,有十足的干劲无十足的本领,有一大堆精彩的点子而逐渐被遗失。你搔首摇头疑惑不解。我们说不明白就跺脚,大笑并拍着大腿跳,这样的时刻百分百精彩。

每天我们都是崭新的“朝阳”,努力使着每一天更加精彩。

【END】

【老师评:好!20091026】



\chapter{Year2008}



\section{Month9}
20080918周四晚,让一本本子作为作文本。八年级的一个寒假,大约是20090123.

妈妈不再会变老,我不再长大。

爱使人富有。

我骄傲!

把忧伤变为唱歌,痛苦伤心时大笑,身心都要柔韧。



\chapter{End}

\end{document}
