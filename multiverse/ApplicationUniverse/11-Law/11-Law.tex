%%============
%%  ** Author: Qirong ZHANG
%%  ** Date: 2024-12-17 23:29:47
%%  ** Github: https://github.com/ShepherdQR
%%  ** LastEditors: Qirong ZHANG
%%  ** LastEditTime: 2024-12-17 23:48:15
%%  ** Copyright (c) 2019 Qirong ZHANG. All rights reserved.
%%  ** SPDX-License-Identifier: LGPL-3.0-or-later.
%%============



\documentclass[UTF8]{../ApplicationUniverse}
\begin{document}

\title{11-Law}
\date{Created on 20241217.\\   Last modified on \today.}
\maketitle
\tableofcontents


\chapter{Introduction}
法律

\section{法律普及读物}

% \subsection{11}
% \subsubsection{22}
%     \paragraph{222}


  
\chapter{法的理论 (法学 )}

\section{立法理论}
\section{法制与民主}
\section{法的起源与本质}
\section{法的历史类型}
    \subsection{奴隶制国家的法}
    \subsection{封建制国家的法}
    \subsection{资本主义国家的法}
    \subsection{社会主义国家的法}
        \subsubsection{本质与作用}
        \subsubsection{制定与实施}

\section{比较法学}

\section{法学史、法律思想史}
    \subsubsection{世界}
    \subsubsection{中国}

\section{法制史}
    \subsubsection{世界}
    \subsubsection{中国}

\section{法学与其他学科的关系}
        \subsubsection{法律逻辑学}
        \subsubsection{法律社会学}
        \subsubsection{法伦理学}
        \subsubsection{司法心理学}
        \subsubsection{法律语言学}
        \subsubsection{其他}





\chapter{法学各部门}
\section{各国法律综合汇编}
\section{国家法、宪法}
        \subsubsection{理论}
        \subsubsection{法的历史}
        \subsubsection{学习、研究}
        \subsubsection{解释、案例}
        \subsubsection{法律汇编}
\section{主要法}
        \subsubsection{行政法}
        \subsubsection{财政法}
            \paragraph{金融法}
            \paragraph{经济法}
        \subsubsection{土地法}
        \subsubsection{农业经济管理法}
        \subsubsection{劳动法}
        \subsubsection{自然资源与环境保护法}
        \subsubsection{青少年法}
        \subsubsection{军法}

\section{民法}
        \subsubsection{婚姻法}
        \subsubsection{商法 (总论 )}
\section{刑法}
\section{诉讼法}
        \subsection{诉讼制度}
        \subsection{当事人}
        \subsection{证据制度}
        \subsection{调解制度}
        \subsection{回避与辩护制度}
        \subsection{诉讼程序}
            \subsubsection{起诉}
            \subsubsection{审判程序}
            \subsubsection{执行程序}
            \subsubsection{特别程序}
        \subsection{民事诉讼法}
        \subsection{刑事诉讼法}
        \subsection{行政诉讼法}
        \subsection{仲裁法}


\section{司法制度}
        \subsubsection{司法行政}
        \subsubsection{法院}
        \subsubsection{检察机关}
        \subsubsection{律师制度}
        \subsubsection{公证制度}
        \subsubsection{监狱制度、劳动改造制度}
        \subsubsection{劳动教养制度}

\section{犯罪学}
        \subsubsection{犯罪原因}
        \subsubsection{犯罪心理学}
        \subsubsection{犯罪社会学}
        \subsubsection{犯罪预防与治理}
        \subsubsection{其他}

\section{刑事侦查学 (犯罪对策学、犯罪侦查学 )}
        \subsection{犯罪同一认定}
        \subsection{侦查技术与方法}
        \subsection{现场勘查}
        \subsection{预审学}
        \subsection{司法鉴定学}
            \subsubsection{痕迹学}
            \subsubsection{文书检验}
            \subsubsection{司法化学检验}
            \subsubsection{司法会计学}

\section{法医学}
        \subsubsection{法医基础科学}
        \subsubsection{法医物证检验学}
        \subsubsection{司法精神医学}
        \subsubsection{法医鉴定学}
        \subsubsection{妇婴法医学}
        \subsubsection{法医人类学}
        







\chapter{中国法律}
\section{理论}
        \subsubsection{方针、政策及其阐述}
        \subsubsection{学习、研究}
        \subsubsection{解释、案例}
        \subsubsection{法律汇编}

\section{国家法、宪法}
    \subsection{国家机构组织法}
        \subsubsection{国家权力机关组织法}
        \subsubsection{国家行政机关组织法}
        \subsubsection{国家军事机关组织法}
        \subsubsection{审判机关、检察机关组织法}

    \subsection{选举法}
    \subsection{国籍法}
    \subsection{地方自治法}
    \subsection{特别行政区基本法}
    \subsection{行政法}
        \subsubsection{行政管理法令}
        \subsubsection{国防军事管理法令}
        \subsubsection{外事管理法令}
        \subsubsection{公安管理法令}
        \subsubsection{华侨、民族事务管理法令}
        \subsubsection{文教、卫生管理法令}
        \subsubsection{科学技术管理法令}
        \subsubsection{公用事业管理法令}
        \subsubsection{民政和社会保障事业管理法令}
        \subsubsection{民政事业管理法令}
        \subsubsection{社会保障法令}
        \subsubsection{青少年法}
        \subsubsection{其他法令}


\section{财政法}
    \subsection{预算法、决算法}
    \subsection{税法}
        \subsubsection{流转税法}
        \subsubsection{所得税法}
        \subsubsection{财产税法}
        \subsubsection{其他税法}

    \subsection{证券法}
    \subsection{财务管理和会计法}
    \subsection{审计法}
    \subsection{金融法}
        \subsubsection{银行法}
        \subsubsection{信托法、信贷法}
        \subsubsection{保险法}
        \subsubsection{货币管理法令}
        \subsubsection{外汇管理法令}
        \subsubsection{证券管理法令}

\section{经济法}
    \subsection{国民经济与社会发展法令}
        \subsubsection{企业法、公司法}
        \subsubsection{破产法}
    \subsection{工业企业经济管理法}
    \subsection{农业经济管理法令}
    \subsection{商业经济管理法令}
    \subsection{涉外经济管理法令}
    \subsection{交通运输经济和邮电经济管理法令}
    \subsection{基本建设管理法令}
    \subsection{经济合同法}
    \subsection{经济特区经济法令}

\section{土地法}
        \subsubsection{土地改革法}
        \subsubsection{农业土地法}
        \subsubsection{国有土地管理及使用法}
        \subsubsection{城市及城郊用地法令}
        \subsubsection{建筑用地法令}
        \subsubsection{特殊用途土地法令}
        \subsubsection{森林用地法令}
        \subsubsection{水利用地法令}
        \subsubsection{国家建设征用土地法}

\section{农业经济管理法令}

\section{劳动法}
        \subsubsection{劳动报酬法}
        \subsubsection{劳动合同法}
        \subsubsection{劳动纪律、劳动奖励法令}
        \subsubsection{劳动保护法令}
        \subsubsection{劳动保险法令}
        \subsubsection{工会与职工民主管理法令}
        \subsubsection{职工培训法令}
        \subsubsection{社会福利法令}
        \subsubsection{劳动争议处理法令}
        \subsubsection{劳动监督与检查法令}
        \subsubsection{其他}

\section{自然资源与环境保护法}
        \subsubsection{土地法}
        \subsubsection{矿产法}
        \subsubsection{森林法}
        \subsubsection{草原法}
        \subsubsection{水产法}
        \subsubsection{水法}
        \subsubsection{能源法}
        \subsubsection{环境保护法}
        \subsubsection{其他}

\section{民法}
    \subsection{民法通则}
    \subsection{物权}
    \subsection{债权}
    \subsection{知识产权}
        \subsubsection{著作权法}
        \subsubsection{专利法}
        \subsubsection{商标法}
        \subsubsection{其他}
    \subsubsection{继承法}
    \subsubsection{合同法}
    \subsubsection{民事其他法权}
    \subsubsection{婚姻法}
    \subsubsection{商法 (总论 )}

\section{刑法}
    \subsection{总则}
        \subsubsection{犯罪}
        \subsubsection{刑罚的种类}
        \subsubsection{刑罚的运用}
    \subsection{分则}
        \subsubsection{危害国家安全罪}
        \subsubsection{危害公共安全罪}
        \subsubsection{破坏社会主义市场经济秩序罪}
        \subsubsection{侵犯公民人身权利、民主权利罪}
        \subsubsection{侵犯财产罪}
        \subsubsection{妨害社会管理秩序罪}
        \subsubsection{危害国防利益罪}
        \subsubsection{贪污贿赂罪}
        \subsubsection{渎职罪}
        \subsubsection{军人违反职责罪}
        \subsubsection{其他}

        
\section{诉讼法}
        \subsubsection{民事诉讼法}
        \subsubsection{刑事诉讼法}
        \subsubsection{行政诉讼法}
        \subsubsection{仲裁法}

\section{司法制度}
    \subsection{司法行政}
        \subsubsection{司法行政机构}
        \subsubsection{司法行政工作}
        \subsubsection{司法统计}
        \subsubsection{司法教育}
        \subsubsection{司法人员}
    \subsection{法院}
        \subsubsection{最高人民法院}
        \subsubsection{地方各级人民法院}
        \subsubsection{专门法院}
            \paragraph{军事法院}

    \subsection{检察院}
        \subsubsection{最高检察院}
        \subsubsection{地方各级人民检察院}
        \subsubsection{专门人民检察院}
        \subsubsection{司法监督}

    \subsection{律师制度}
    \subsection{公证制度}
    \subsection{监狱制度与劳动改造制度}
    \subsection{劳动教养制度}

\section{地方法制}
\section{中国法制史}









\chapter{世界各国法律}

\subsubsection{亚洲各国法律}
\subsubsection{非洲各国法律}
\subsubsection{欧洲各国法律}
\subsubsection{大洋洲各国法律}
\subsubsection{美洲各国法律}








\chapter{国际法}
\section{国际法理论}
\section{国家}
\section{领土}
    \subsubsection{领土问题}
    \subsubsection{国界和边境}
    \subsubsection{领水}
    \subsubsection{领空、航空法}
    \subsubsection{海洋法}
    \subsubsection{外交领事法}
    \subsubsection{条约法}
    \subsubsection{国际组织法}

\section{平时国际法}
\section{战时国际法 (战争法 )}
    \subsubsection{中立}
    
\section{国际经济法}
    \subsubsection{国际商法 (国际贸易法 )}
    \subsubsection{海商法}
    \subsubsection{国际财政金融法}
    \subsubsection{国际税法}
    \subsubsection{国际投资法}
    \subsubsection{国际技术转让法}
    \subsubsection{国际环境保护法}

\section{国际私法}
    \subsubsection{国际民法}
    \subsubsection{国际民事诉讼程序法}
    \subsubsection{国际商事仲裁与国际海事仲裁法}
    \subsubsection{国际刑法}

\section{国籍法}
    \subsubsection{外国人的法律地位}
    \subsubsection{人权的国际保护}
    \subsubsection{移民法}
    \subsubsection{各国国籍法}

\section{外层空间法 (宇宙法 )}

\section{核法}






\end{document}

