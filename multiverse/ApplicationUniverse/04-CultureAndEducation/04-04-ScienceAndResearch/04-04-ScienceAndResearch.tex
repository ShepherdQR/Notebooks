%%============
%%  ** Author: Qirong ZHANG
%%  ** Date: 2024-12-22 23:36:05
%%  ** Github: https://github.com/ShepherdQR
%%  ** LastEditors: Qirong ZHANG
%%  ** LastEditTime: 2024-12-27 00:37:11
%%  ** Copyright (c) 2019 Qirong ZHANG. All rights reserved.
%%  ** SPDX-License-Identifier: LGPL-3.0-or-later.
%%============




\documentclass[UTF8]{../../ApplicationUniverse}
\begin{document}

\title{04-04-ScienceAndResearch}
\date{Created on 20241222.\\   Last modified on \today.}
\maketitle
\tableofcontents


\chapter{Introduction}

科学、科学研究



\chapter{理论}
\section{科学学}
\section{知识学}
\section{未来学}
\section{科学研究的方法论}
\section{科学发明、发现研究 (创造学 )}
\section{专利研究}
    \subsubsection{理论方法}
    \subsubsection{制度与管理}
    \subsubsection{编写与出版}
    \subsubsection{各国专利文献概况}
    \subsubsection{专利综合汇编}
\section{技术标准研究}




\chapter{科学研究工作}
\subsubsection{组织和管理}
\subsubsection{工作方法}
\subsubsection{群众性科学研究工作}
\subsubsection{科学工作者}


\chapter{世界各国科学研究事业}
\section{世界}
    \subsubsection{国际组织}
    \subsubsection{交流和合作}
    \subsubsection{会议、会谈}
    \subsubsection{协定}
    \subsubsection{科学事业史}
\section{中国}
    \subsection{方针政策及其阐述}
    \subsection{规划、计划}
    \subsection{机构和团体}
        \subsubsection{中国科学院}
        \subsubsection{中国社会科学院}
        \subsubsection{地方科学研究机构}
        \subsubsection{专业科学研究机构}
        \subsubsection{科学团体、协会、学会、学社}
    \subsection{对外科学研究工作的交流与合作}
    \subsection{地方科学研究事业}
    \subsection{科学事业史}


\chapter{情报学、情报工作}
\section{情报学}
    \subsubsection{情报工作自动化、网络化}
\section{情报工作体制、组织}
    \subsubsection{组织和管理}
    \subsubsection{规章制度}
    \subsubsection{情报工作者}
    \subsubsection{情报机构的建筑、设备}
\section{情报资料的搜集、保管}
    \subsubsection{情报源的调查研究和选择}
    \subsubsection{搜集、采购}
    \subsubsection{登记整理}
    \subsubsection{质量鉴定}
    \subsubsection{保管、典藏}
\section{情报资料的处理}
    \subsection{情报资料的分析和研究}
        \subsubsection{综述}
        \subsubsection{科学技术总结}
    \subsection{情报编译报道}
        \subsubsection{题录、索引}
        \subsubsection{简介}
        \subsubsection{文摘}
        \subsubsection{快报}

\section{情报检索}
    \subsection{情报检索中心}
    \subsection{情报检索方法和工具}
        \subsubsection{分类目录与分类法}
        \subsubsection{主题目录与主题法}
        \subsubsection{著者目录}
        \subsubsection{机构目录}
        \subsubsection{情报咨询工具}
        \subsubsection{其他}
    \subsection{半机械化、机械化检索系统}
    \subsection{计算机情报检索系统}
        \subsubsection{数值情报检索系统、事实情报检索系统}
        \subsubsection{书目情报检索系统}
        \subsubsection{全文情报检索系统}
        \subsubsection{多媒体情报检索系统}
        \subsubsection{其他情报检索系统}

\section{机器翻译}
\section{情报过程自动化的方法和设备}
    \subsubsection{文献库的方法和设备}
    \subsubsection{情报的存贮和检索设备}
    \subsubsection{情报载体}
    \subsubsection{机械化、自动化编索引}
    \subsubsection{自动化作文摘}
    \subsubsection{情报逻辑加工、情报逻辑系统}
\section{文献复制方法和设备}
    \subsubsection{照相复制}
    \subsubsection{热敏复制}
    \subsubsection{重氮复制}
    \subsubsection{静电复制}
\section{情报资料的利用}
\section{世界各国情报事业}
    \subsection{世界}
        \subsubsection{参考工具书}
        \subsubsection{统计资料}
        \subsubsection{国际组织与活动}
        \subsubsection{情报事业史}
    \subsection{中国}
        \subsubsection{方针政策及其阐述}
        \subsubsection{情报事业的建设和发展}
        \subsubsection{情报工作组织和活动}
            \paragraph{国家情报中心}
            \paragraph{地方情报机构}
            \paragraph{专业情报机构}
            \paragraph{基层情报机构}
            \paragraph{情报工作会议}
            \paragraph{情报学会}
        \subsubsection{情报工作的协调和合作、情报网}
        \subsubsection{对外情报工作交流}
        \subsubsection{地方情报事业}
        \subsubsection{情报事业史}


\chapter{END}




\end{document}

