%%============
%%  ** Author: Qirong ZHANG
%%  ** Date: 2024-12-22 23:36:05
%%  ** Github: https://github.com/ShepherdQR
%%  ** LastEditors: Qirong ZHANG
%%  ** LastEditTime: 2024-12-27 00:23:00
%%  ** Copyright (c) 2019 Qirong ZHANG. All rights reserved.
%%  ** SPDX-License-Identifier: LGPL-3.0-or-later.
%%============




\documentclass[UTF8]{../../ApplicationUniverse}
\begin{document}

\title{04-03-Information}
\date{Created on 20241227.\\   Last modified on \today.}
\maketitle
\tableofcontents


\chapter{Introduction}

信息与知识传播



\chapter{信息与传播理论}
\section{信息理论}
\section{信息处理技术}
\section{信息资源及其管理}
\section{传播理论}
    \subsubsection{传播媒介}
    \subsubsection{大众传播}



\chapter{新闻学、新闻事业}
\section{新闻学}
    \subsubsection{新闻工作自动化、网络化}
    \subsubsection{新闻学史}
\section{组织和管理}
\section{新闻采访和报道}
    \subsubsection{新闻采访}
    \subsubsection{新闻写作}
\section{编辑工作}
\section{新闻工作者}
    \subsubsection{编辑}
    \subsubsection{记者}
    \subsubsection{通讯员}
\section{报纸的出版发行}
\section{各类型报纸}
    \subsubsection{中央报纸 (全国性 )}
    \subsubsection{地方报纸}
    \subsubsection{专业报纸}
\section{新闻摄影}
\section{世界各国新闻事业}
    \subsection{世界}
        \subsubsection{国际组织与活动}
        \subsubsection{通讯社、报社}
        \subsubsection{互助合作与交流}
        \subsubsection{会议}
        \subsubsection{协定}
        \subsubsection{新闻事业史}
    \subsection{中国}
        \subsubsection{方针政策及其阐述}
        \subsubsection{新闻事业组织}
        \subsubsection{通讯社}
        \subsubsection{中央报社}
        \subsubsection{地方报社}
        \subsubsection{专业报社}
        \subsubsection{对外新闻工作交流}
        \subsubsection{地方新闻事业}
        \subsubsection{新闻事业史}
 



\chapter{广播、电视事业}
\section{广播、电视工作理论}
    \subsubsection{广播、电视工作自动化}
\section{组织和管理}
\section{编辑、写作和播送业务}
    \subsubsection{采访、编辑、写作}
    \subsubsection{播送业务}
    \subsubsection{节目制作}
\section{广播、电视宣传和群众工作}
\section{广播、电视工作者}
\section{世界各国广播、电视事业}
    \subsection{世界}
        \subsubsection{国际组织与活动}
        \subsubsection{互助合作与交流}
        \subsubsection{会议}
        \subsubsection{协定}
        \subsubsection{广播、电视事业史}
    \subsection{中国}
        \subsubsection{方针政策及其阐述}
        \subsubsection{广播、电视事业组织}
        \subsubsection{广播电台、电视台}
        \subsubsection{对外广播、电视工作交流}
        \subsubsection{地方广播、电视事业}
        \subsubsection{广播、电视事业史}
   





\chapter{出版事业}
\section{出版工作理论}
    \subsubsection{出版工作自动化}
\section{组织和管理}
\section{编辑工作}
    \subsubsection{选题、组稿}
    \subsubsection{编辑、校对}
    \subsubsection{装帧设计}
\section{印刷工作}
\section{发行工作}
\section{书刊宣传、评介}
    \section{各类型出版物编辑出版}
    \subsubsection{工具书编辑出版}
    \subsubsection{期刊编辑出版}
    \subsubsection{电子出版物编辑出版}

\section{出版工作者}
\section{世界各国出版事业}
    \subsection{世界}
        \subsubsection{国际组织与活动}
        \subsubsection{互助合作与交流}
        \subsubsection{会议}
        \subsubsection{协定}
        \subsubsection{出版事业史}
    \subsection{中国}
        \subsubsection{方针政策及其阐述}
        \subsubsection{出版事业组织与活动}
        \subsubsection{出版社}
        \subsubsection{发行机构}
        \subsubsection{对外出版工作交流}
        \subsubsection{地方出版事业}
        \subsubsection{出版事业史}






\chapter{群众文化事业}
\section{群众文化工作理论}
\section{工作方法}
    \subsubsection{宣传工作}
    \subsubsection{辅导工作}
    \subsubsection{群众文艺活动、娱乐活动}
    \subsubsection{报告会、座谈会、故事会}
\section{文化馆 (站 )、文化宫}
\section{俱乐部}
\section{青年宫、少年宫、少年之家}
\section{展览馆、展览会}
\section{公园}
\section{群众文化活动}
\section{游乐场、歌舞厅}

\section{世界各国群众文化事业}

    包括世界和中国
    \subsubsection{方针政策及其阐述}
    \subsubsection{群众文化事业组织}
    \subsubsection{对外群众文化工作交流}
    \subsubsection{地方群众文化事业}
    \subsubsection{群众文化事业史}







\chapter{图书馆学、图书馆事业}
\section{图书馆学}
    \subsection{图书馆自动化、网络化}
        \subsubsection{图书馆管理集成系统}
        \subsubsection{网络化}
        \subsubsection{网络资源开发与利用}
        \subsubsection{数据库建设}
        \subsubsection{电子图书馆、数字图书馆}
    \subsection{图书馆学史}


\section{图书馆管理}
    \subsubsection{组织机构}
    \subsubsection{规章制度}
    \subsubsection{图书馆统计}
    \subsubsection{图书馆业务研究与辅导}
    \subsubsection{图书馆工作者}

\section{读者工作}
    \subsection{图书宣传}
        \subsubsection{图书展览}
        \subsubsection{报告会}
        \subsubsection{读者座谈会}
        \subsubsection{阅读辅导}
    \subsection{馆内阅览}
    \subsection{外借}
    \subsection{馆际借书}
    \subsection{流通站、流通书车}
    \subsection{参考咨询}
        \subsubsection{咨询解答工作}
        \subsubsection{定题服务}
        \subsubsection{书目工作}
        \subsubsection{图书馆利用法}
    \subsection{文献检索}

\section{藏书建设和藏书组织}
    \subsubsection{图书补充}
    \subsubsection{图书交换}
    \subsubsection{图书登记}
    \subsubsection{藏书组织}
    \subsubsection{图书保护}

\section{文献标引与编目}
    \subsection{文献工作研究方法、工作方法}
    \subsection{文献检索语言 (总论 )}
    \subsection{分类法}
        \subsubsection{分类理论与方法}
        \subsubsection{分类表}
            \paragraph{综合性分类表}
            \paragraph{专业分类表}
            \paragraph{专类文献分类表}
        \subsubsection{同类书排列法}
            \paragraph{著者号码表}
                \subparagraph{中文著者号码表}
                \subparagraph{外文著者号码表}
            \paragraph{著者号码拼号法}
            \paragraph{种次号编号法}
            \paragraph{其他编号法}
    \subsection{主题法}
        \subsubsection{主题法理论与方法 (总论 )}
        \subsubsection{标题法与标题表}
            \paragraph{理论与方法}
            \paragraph{综合性标题表}
            \paragraph{专业性标题表}
        \subsubsection{关键词法}
        \subsubsection{叙词法与叙词表}
            \paragraph{理论与方法}
            \paragraph{综合性叙词表}
            \paragraph{专业性叙词表}
    \subsection{文献编目}
        \subsubsection{文献著录}
            \subsubsection{编目过程}
        \subsubsection{目录体系}
        \subsubsection{各种目录组织法}
            \paragraph{字顺目录 (字典式目录 )}
            \paragraph{书名目录}
            \paragraph{著者目录}
            \paragraph{主题目录}
            \paragraph{分类目录}
            \paragraph{专题目录}
        \subsubsection{各种编目方式}
            \paragraph{集中编目、合作编目}
            \paragraph{在版编目 (CIP )}
            \paragraph{重新编目}
        \subsubsection{文献标引与编目自动化}
            \paragraph{自动标引}
            \paragraph{计算机编目}
            \paragraph{机读目录}
    \subsection{图书馆目录的使用}







\section{各种文献工作}
\subsection{善本、线装古籍}
\subsection{期刊}
\subsection{报纸}
\subsection{地图}
    \subsubsection{科技情报资料}
    \subsubsection{专利}
    \subsubsection{标准}
    \subsubsection{样本}
\subsection{乐谱}
\subsection{试听}
    \subsubsection{图片、照片}
    \subsubsection{缩微资料}
    \subsubsection{视听资料}
    \subsubsection{电子出版物}
\subsection{盲文图书}


\section{文献学}
\subsection{图书学}
\subsection{版本学}
    \subsubsection{中国版本}
    \subsubsection{外国版本}
    \subsubsection{书影}
\subsection{校勘学}
\subsection{题跋、书评}


\section{目录学}
\subsection{书目编制法}
    \subsubsection{普通书目}
    \subsubsection{联合目录}
    \subsubsection{专题书目}
\subsection{专科目录学}
    \subsubsection{马克思主义目录学}
    \subsubsection{社会科学目录学}
    \subsubsection{文艺目录学}
    \subsubsection{史学目录学}
    \subsubsection{科学、技术目录学}
    \subsubsection{生物科学目录学}
    \subsubsection{医学目录学}
    \subsubsection{农业目录学}
\subsection{文摘、索引}


\section{各类型图书馆}
\subsection{国家图书馆}
\subsection{公共图书馆}
    \subsubsection{直辖市、省、自治区图书馆}
    \subsubsection{市、地、县、区图书馆}
    \subsubsection{乡镇图书馆 (室 )}
    \subsubsection{街道图书馆 (室 )}
\subsection{厂矿企业图书馆}
    \subsubsection{政府机关图书馆}
    \subsubsection{部队图书馆}
    \subsubsection{工会图书馆}
\subsection{科学研究机构图书馆、专业图书馆}
\subsection{高等学校、中等专业学校图书馆}
\subsection{中、小学图书馆}
\subsection{儿童图书馆}
    \subsubsection{残疾人图书馆}
    \subsubsection{私人图书馆}
    \subsubsection{版本图书馆}
    \subsubsection{其他图书馆}
\subsection{图书馆建筑、设备}
    \subsubsection{图书馆建筑}
    \subsubsection{图书馆设备}
    \subsubsection{图书馆工作的机械化和自动化设备}
    \subsubsection{图书馆用品}

\subsection{世界各国图书馆事业}
    \subsubsection{参考工具书}
    \subsubsection{统计资料}
    \subsubsection{国际组织和活动}
    \subsubsection{世界图书馆事业史}

\subsection{中国图书馆事业}
    \subsubsection{方针政策及其阐述}
    \subsubsection{图书馆的组织与活动}
    \subsubsection{图书馆网、图书馆工作的协调和合作}
    \subsubsection{图书馆业务辅导}
    \subsubsection{各类型图书馆}
    \subsubsection{对外图书馆工作交流}
    \subsubsection{地方图书馆事业}
    \subsubsection{图书馆事业史}










\chapter{博物馆学、博物馆事业}
\section{博物馆学}
\section{组织和管理}
\section{藏品的采集、征集、鉴定}
\section{文物复制}
\section{藏品整理和保管}
    \subsubsection{分类、编目、登记}
    \subsubsection{保管}
    \subsubsection{修复}
\section{陈列、展览工作}
\section{群众宣传工作}
\section{建筑和设备}
\section{各类型展览馆、博物馆}
    \subsubsection{革命博物馆、纪念馆}
    \subsubsection{历史博物馆}
    \subsubsection{地志博物馆}
    \subsubsection{民族史志博物馆}
    \subsubsection{私人博物馆}


\section{世界各国博物馆事业}
包括中国和世界其他国家
\subsubsection{方针政策及其阐述}
\subsubsection{博物馆事业组织与活动}
\subsubsection{博物馆工作的协调和合作}
\subsubsection{各类型展览馆、博物馆}
\subsubsection{地方博物馆事业}
\subsubsection{博物馆事业史}
 






\chapter{档案学、档案事业}
\section{档案学}
    \subsubsection{档案工作自动化}
    \subsubsection{档案学史}
\section{档案管理}
    \subsubsection{组织机构}
    \subsubsection{规章制度}
    \subsubsection{统计方法}
    \subsubsection{档案工作者}
\section{收集和整理}
    \subsubsection{收集}
    \subsubsection{鉴定}
    \subsubsection{修复与整理}
    \subsubsection{分类法与主题法}
    \subsubsection{编目}
\section{保管和利用}
    \subsubsection{保管与典藏}
    \subsubsection{保护}
    \subsubsection{流通、利用}
\section{公布、出版}
\section{各种类型档案工作}
    \subsubsection{历史档案}
    \subsubsection{文书档案}
    \subsubsection{技术档案}
    \subsubsection{其他档案}
\section{特种档案工作}
\section{文书工作}
\section{建筑和设备}

\section{世界各国档案事业}
包括中国和世界其他国家
\subsubsection{方针政策及其阐述}
\subsubsection{档案事业组织与活动}
\subsubsection{地方档案事业}
\subsubsection{档案事业史}
 


\chapter{END}




\end{document}

