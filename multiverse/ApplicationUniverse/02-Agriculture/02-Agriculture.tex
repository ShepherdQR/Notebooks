%%============
%%  ** Author: Qirong ZHANG
%%  ** Date: 2022-06-05 21:15:52
%%  ** Github: https://github.com/ShepherdQR
%%  ** LastEditors: Qirong ZHANG
%%  ** LastEditTime: 2025-01-01 16:25:01
%%  ** Copyright (c) 2019 Qirong ZHANG. All rights reserved.
%%  ** SPDX-License-Identifier: LGPL-3.0-or-later.
%%============



\documentclass[UTF8]{../ApplicationUniverse}
\begin{document}

\title{02-Agriculture}
\date{Created on 20220605.\\   Last modified on \today.}
\maketitle
\tableofcontents


\chapter{Introduction}

农业:农业基础科学、农业工程、农学 (农艺学 )、植物保护、农作物、、、、

\chapter{农业史}


\chapter{农业科学研究方针、政策及其阐述}







\chapter{农业基础科学}
\section{农业数学}
\section{农业物理学}
\section{农业化学}
\section{农业地理学}
\section{农业生物学}
\section{肥料学}
\section{土壤学}
    \subsection{土壤物理}
    \subsection{土壤化学}
    \subsection{土壤地理}
    \subsection{土壤生物}
    \subsection{土壤生态}
    \subsection{土壤耕作}
    \subsection{土壤改良}
    \subsection{土壤肥料}
    \subsection{土壤分类}
    \subsection{土壤调查与评价}


\section{农业生产环境保护}
\section{气象学与气候学}
\section{生态学}
\section{植物学}
\section{微生物学}
\section{营养学}




\chapter{农业工程}
\section{农业动力、农村能源}
\section{农业机械及农具}
\section{农业机械化}
\section{农业电气化与自动化}

\section{农田水利(包括灌溉工程、排水工程)}
\section{农业航空}
\section{农业建筑}
\section{农田基本建设、农垦}
\section{农业工程勘测、土地测量}

\section{ELSE}
\subsection{水土保持学}
\subsection{农田测量}
\subsection{农业环保工程}
\subsection{农业区划}
\subsection{农业系统工程}









\chapter{农学 (农艺学 )}
\section{作物生物学原理、栽培技术与方法}
\subsection{作物形态学}
\subsection{作物生理学}
\subsection{作物遗传学}
\subsection{作物生态学}
\subsection{种子学}

\section{作物品种与品种资源}
作物种质资源学
\section{作物遗传育种与良种繁育}
\section{耕作学与有机农业}
\section{播种}
\section{栽植}
    \subsection{气培法}
    \subsection{水培}
    \subsection{雾培}
\section{田间管理}
作物耕作
\section{农产品收获、加工及贮藏}
\section{农产品的综合利用}
\section{农产副业技术}












\subsubsection{植物保护}





\chapter{植物保护}
\section{植物检疫}
\section{植物免疫}
\section{植物病理}
\section{植物药理}
\section{农林昆虫}
\section{植物病毒}
\section{农药}
\section{植物病虫害测报}

\section{各种防治方法}
    \subsection{农药防治 (化学防治 )}
    \subsection{植物保护机械}
    \subsection{有害生物化学防治}
    \subsection{有害生物生物防治}
    \subsection{有害生物综合防治}


\section{灾害及其防治}
    \subsection{气象灾害及其防治}
    \subsection{病虫害及其防治}
        \subsubsection{抗病虫害育种}

    \subsection{鸟兽害及其防治}
        \subsubsection{鸟兽、鼠害防治}
    \subsection{有害植物及其清除}
        \subsubsection{杂草防治}



\chapter{农作物}
\subsubsection{禾谷类作物}
\subsubsection{豆类作物}
\subsubsection{薯类作物}
\subsubsection{饲料作物、牧草}
\subsubsection{绿肥作物}
\subsubsection{经济作物}
\subsubsection{野生植物}
\subsubsection{野生植物}
\subsubsection{热带、亚热带作物}

\chapter{园艺学}
\section{一般性问题}
\section{苗圃学}
\section{温室园艺 (保护地栽培 )}
\section{观赏园艺 (花卉和观赏树木 )}

\section{果树园艺}
\section{瓜果园艺}
   \subsection{葡萄栽培}
\section{蔬菜园艺}
\section{果蔬贮藏与加工}
\section{茶}










\chapter{林业}
\subsubsection{林业基础科学}
\subsubsection{造林学、林木育种及造林技术}
\subsubsection{绿化建设}
\subsubsection{森林经营学、森林计测学 (测树学 )、森林经理学}
\subsubsection{森林保护学}
\subsubsection{森林工程、林业机械}
\subsubsection{森林采运与利用}
\subsubsection{森林树种}



\chapter{畜牧、动物医学、狩猎、蚕、蜂}
\subsubsection{普通畜牧学}
\subsubsection{家畜}
\subsubsection{家禽}
\subsubsection{动物医学 (兽医学 )}
\subsubsection{狩猎、野生动物驯养}
\subsubsection{畜禽产品的综合利用}
\subsubsection{蚕桑}
\subsubsection{养蜂、益虫饲养}







\chapter{水产、渔业}
\subsubsection{水产基础科学}
\subsubsection{水产地区分布、水产志}
\subsubsection{水产资源}
\subsubsection{水产保护学}
\subsubsection{水产工程}
\subsubsection{水产养殖技术}
\subsubsection{水产捕捞}
\subsubsection{水产物运输、保鲜、贮藏、加工、包装}







\chapter{其他}
\section{农业经济学}
\section{水文学}
\section{酿酒学}
\section{畜牧业(动物科学)}
\section{养蜂(养蜂)}
\section{生物系统工程}
\section{食品工程}
\section{水产养殖}
\section{鱼菜共生}
\section{食物科学}
\section{烹饪艺术}
\section{纯化}



\end{document}


