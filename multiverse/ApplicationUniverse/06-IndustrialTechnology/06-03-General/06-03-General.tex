%%============
%%  ** Author: Qirong ZHANG
%%  ** Date: 2025-01-01 16:23:55
%%  ** Github: https://github.com/ShepherdQR
%%  ** LastEditors: Qirong ZHANG
%%  ** LastEditTime: 2025-01-01 17:23:17
%%  ** Copyright (c) 2019 Qirong ZHANG. All rights reserved.
%%  ** SPDX-License-Identifier: LGPL-3.0-or-later.
%%============



\documentclass[UTF8]{../../ApplicationUniverse}
\begin{document}

\title{06-03-General}
\date{Created on 20250101.\\   Last modified on \today.}
\maketitle
\tableofcontents


\chapter{Introduction}

一般工业技术:

工程基础科学、工程设计与测绘、工程材料学、工业通用技术与设备、声学工程、制冷工程、真空技术、摄影技术、计量学



\chapter{工程基础科学}
\section{工程数学}
    \subsection{数论与代数的应用}
    \subsection{数学分析与函数的应用}
    \subsection{几何的应用}
    \subsection{概率论、数理统计的应用}
        \subsubsection{运筹学的应用}
        \subsubsection{工程控制论}
        \subsubsection{可靠性理论}
    \subsection{计算数学的应用}

\section{工程力学}
    \subsubsection{工程静力学}
    \subsubsection{工程动力学}
    \subsubsection{工程振动学}
    \subsubsection{变形体工程力学}
    \subsubsection{工程塑性力学、工程弹性力学}
    \subsubsection{工程流体力学}
\section{工程物理学}
    \subsubsection{工程热力学}
    \subsubsection{工程声学}
    \subsubsection{工程光学}


\section{工程化学}
\section{工程天文学}
\section{工程地质学}
\section{工程仿生学}
\section{人体工程学}









\chapter{工程设计与测绘}
\section{工程设计}
\section{工程测量}
\section{工程制图}
    \subsubsection{制图数学}
    \subsubsection{绘图法、描图法}
    \subsubsection{复制法、晒图法}
    \subsubsection{计算机辅助工程制图}
\section{工程模拟}









\chapter{工程材料学}
\section{理论}
    \subsection{工程材料力学 (材料强弱学 )}
    \subsection{工程材料试验}
        \subsubsection{物理试验法}
        \subsubsection{化学试验法}
        \subsubsection{机械试验法}
        \subsubsection{加工性试验法}
        \subsubsection{组织检查法、非破坏性试验法}
        \subsubsection{简易识别法}
    \subsection{材料结构及物理性质}
    \subsection{材料腐蚀与保护}
    \subsection{材料重量计算}

\section{金属材料}
\section{非金属材料}
    \subsubsection{无机质材料}
    \subsubsection{有机质材料}
    \subsubsection{高分子材料}
\section{复合材料}
    \subsubsection{金属复合材料}
    \subsubsection{非金属复合材料}
    \subsubsection{金属-非金属复合材料}
\section{功能材料}
\section{耐高温材料、耐低温材料}
\section{耐腐蚀材料}
    \subsubsection{智能材料}
    \subsubsection{特种结构材料}










\chapter{工业通用技术与设备}
\section{爆破技术}
\section{密封技术}
\section{薄膜技术}
\section{粉末技术}
\section{工业设计}
    \subsubsection{产品设计}
    \subsubsection{产品模型制作}
\section{包装工程}
    \subsection{包装设计}
        \subsubsection{装潢设计}
        \subsubsection{结构设计}
    \subsection{包装材料}
        \subsubsection{纸、纸板}
        \subsubsection{木、木材}
        \subsubsection{塑料}
        \subsubsection{金属}
        \subsubsection{玻璃、陶瓷}
        \subsubsection{其他}
    \subsection{包装类型}
        \subsubsection{缓冲包装}
        \subsubsection{充气包装}
        \subsubsection{运输包装}
        \subsubsection{防锈包装}
        \subsubsection{防潮包装}
        \subsubsection{其他}
    \subsection{包装机械设备}
        \subsubsection{单机}
        \subsubsection{组合机}
        \subsubsection{自动控制机}
    \subsection{包装技术检测}
    \subsection{包装工厂}
    \subsection{各类产品包装}
\section{工厂、车间}
    \subsubsection{规划与设计}
    \subsubsection{设备安装与运行}
    \subsubsection{力能供应与节能}
    \subsubsection{空调与照明}
    \subsubsection{给水、排水}
    \subsubsection{安全与卫生}
    \subsubsection{技术管理}
    \subsubsection{贮运}
    \subsubsection{工业三废处理与综合利用}









\chapter{声学工程}
\section{声学仪器}
    \subsubsection{声振荡器}
    \subsubsection{辐射器和接收器}
    \subsubsection{液声仪}
    \subsubsection{流体测位仪}
    \subsubsection{声音发讯仪}
    \subsubsection{声波分析器}
    \subsubsection{超声波仪器}
    \subsubsection{语音测验仪器}
\section{声学测量}
    \subsubsection{互易原理和声学校准}
    \subsubsection{声压的测量}
    \subsubsection{振动与冲击的测量}
    \subsubsection{声功率的测量}
    \subsubsection{声场的测量}
    \subsubsection{频谱分析}
    \subsubsection{声阻抗的测量}
    \subsubsection{声学仪器校准}
    \subsubsection{计算技术在声学测量中的应用}
\section{振动、噪声及其控制}
    \subsection{振动体的振动与辐射}
    \subsection{振动与噪声的发生}
        \subsubsection{机器振动与噪声}
        \subsubsection{交通运输工具的振动与噪声}
        \subsubsection{高航速的振动与噪声}
        \subsubsection{城市噪声}
    \subsection{噪声发生器与振动发生器}
        \subsubsection{噪声发生器及其分析}
        \subsubsection{振动发生器、振动台及其分析}
        \subsubsection{材料机件的耐振试验、振动疲劳及声疲劳试验}
    \subsection{振动和噪声的控制及其利用}
        \subsubsection{隔振、减振材料与结构}
        \subsubsection{消声器、滤波器及其测试}
        \subsubsection{噪声的利用}
\section{电声工程}
\section{超声工程}
    \subsubsection{超声测量}
    \subsubsection{超声换能器}
    \subsubsection{超声控制与检测}
    \subsubsection{超声的应用}
\section{水声工程}
    \subsection{水下声源}
    \subsection{水声材料}
    \subsection{水声仪器与设备}
        \subsubsection{水声换能器、水听器}
        \subsubsection{水声探测设备}
        \subsubsection{发射与接收设备}
        \subsubsection{显示、记录与数据处理设备}
        \subsubsection{水池、水槽}
    \subsection{水声探测}
    \subsection{水下通信 (声纳通信 )}
    \subsection{水声导航}
\section{光声工程}









\chapter{制冷工程}
\section{制冷理论}
    \subsection{制冷的热力学、传热学、传质学、流体力学}
    \subsection{制冷剂与载冷剂的物化性能}
    \subsection{深冷工质物化性质}
    \subsection{空气制冷循环}
    \subsection{蒸汽压缩式制冷循环}
    \subsection{吸收式制冷循环}
    \subsection{蒸汽喷射式制冷循环}
    \subsection{回热式气体制冷循环}
        \subsubsection{涡流管制冷循环}
        \subsubsection{温差电制冷循环}
        \subsubsection{深冷气体循环}
        \subsubsection{绝热去磁}
\section{制冷材料}
\section{制冷机械和设备}
    \subsection{制冷机}
    \subsection{压缩机}
    \subsection{膨胀机}
    \subsection{液体泵}
    \subsection{低温泵}
    \subsection{制冷设备}
        \subsubsection{冷藏库与制冰设备}
        \subsubsection{空调器}
        \subsubsection{低温试验箱}
        \subsubsection{冰箱}
        \subsubsection{热交换及其设备}
        \subsubsection{精馏及其设备}
        \subsubsection{气体分离设备}
        \subsubsection{气体液化设备}
        \subsubsection{附属设备}
    \subsection{贮运设备}
\section{制冷技术}
    \subsubsection{超低温技术}
    \subsubsection{气体纯化技术}
    \subsubsection{实验测量及自动化技术}
    \subsubsection{安全技术}
\section{制冷应用}













\chapter{真空技术}
\section{真空技术基础理论}
    \subsubsection{气体动力学}
    \subsubsection{流体动力学}
    \subsubsection{热动力学}
    \subsubsection{气体与固体}
    \subsubsection{辐射}
    \subsubsection{真空物理学}
    \subsubsection{气体电子学}
\section{真空材料}
    \subsubsection{金属材料}
    \subsubsection{非金属材料}
    \subsubsection{密封材料}
\section{真空获得技术及设备}
    \subsection{真空获得技术}
    \subsection{真空泵}
        \subsubsection{水银旋转及泰浦勒真空泵}
        \subsubsection{机械真空泵}
            \paragraph{往复真空泵}
            \paragraph{水环真空泵}
            \paragraph{旋片真空泵}
            \paragraph{离心真空泵}
            \paragraph{油旋转机械真空泵}
            \paragraph{机械增压泵 (罗茨真空泵 )}
            \paragraph{分子真空泵}
        \subsubsection{液体喷射真空泵}
        \subsubsection{蒸汽流泵}
            \paragraph{油增压泵}
            \paragraph{油扩散泵}
                \subparagraph{高真空油扩散泵}
                \subparagraph{超高真空油扩散泵}
                \subparagraph{汞扩散泵}
        \subsubsection{物理化学真空泵}
        \subsubsection{吸附泵}
        \subsubsection{电离泵}
        \subsubsection{低温泵}
    \subsection{真空系统 (机组 )}
        \subsubsection{低真空系统及机组}
        \subsubsection{高真空系统及机组}
        \subsubsection{超高真空系统及机组}
        \subsubsection{其他真空系统及机组}
    \subsection{真空元件}
        \subsubsection{真空阀}
        \subsubsection{真空继电器}
        \subsubsection{真空冷阱}
        \subsubsection{其他}
    \subsection{真空设备的制造工艺}
\section{真空测试及仪器}
\section{真空测试技术}
    \subsection{真空计 (全压测量 )}
        \subsubsection{压缩式真空计}
        \subsubsection{电阻式真空计}
        \subsubsection{电离式真空计}
        \subsubsection{复合式真空计}
    \subsection{真空质谱仪}
    \subsection{真空检漏与仪器}
        \subsubsection{高频火花检漏仪}
        \subsubsection{卤素检漏仪}
        \subsubsection{质谱仪检漏器}
    \subsection{真空自动记录仪}
\section{真空技术的应用}













\chapter{摄影技术}
\section{摄影理论}
    \subsection{摄影光学}
        \subsubsection{光线}
            \paragraph{天然光线}
            \paragraph{人工光线}
            \paragraph{光的测定}
        \subsubsection{针孔成像}
        \subsubsection{透镜成像}
        \subsubsection{光圈与景深}
        \subsubsection{色调与滤色镜}
    \subsection{摄影化学}
        \subsubsection{感光原理}
        \subsubsection{彩色胶片感光原理}
        \subsubsection{显影原理}
        \subsubsection{定影原理}
        \subsubsection{调色原理}
\section{拍摄技术}
\section{感光材料}
\section{摄影机具与设备}
    \subsection{光学镜头、滤光器}
        \subsubsection{摄影镜头}
        \subsubsection{印片光学镜头}
        \subsubsection{放映镜头}
        \subsubsection{宽银幕镜头}
        \subsubsection{校正镜头}
        \subsubsection{艺术效果镜头}
        \subsubsection{滤光器}
        \subsubsection{其他镜头}
    \subsection{照相设备与复制设备}
        \subsubsection{照相机}
        \subsubsection{图书资料复制设备}
            \paragraph{银盐复制机}
            \paragraph{重氮复制机}
            \paragraph{红外复制机}
            \paragraph{紫外复制机}
            \paragraph{静电复制机}
            \paragraph{缩微复制机}
        \subsubsection{阅读器}
    \subsection{摄影设备}
        \subsubsection{摄影机}
            \paragraph{新闻摄影机、小型摄影机}
            \paragraph{大中型摄影机}
            \paragraph{特技摄影机}
            \paragraph{立体摄影机}
            \paragraph{全景摄影机}
            \paragraph{字幕动画摄影机}
            \paragraph{高速摄影机}
            \paragraph{水下摄影机}
            \paragraph{航空摄影机}
                \subparagraph{x射线、紫外线摄影机}
                \subparagraph{激光全息摄影装置}
                \subparagraph{声全息摄影装置}
                \subparagraph{其他摄影装置}
        \subsubsection{特技摄影装置}
    \subsection{暗房设备}
        \subsubsection{显影机}
        \subsubsection{放大机}
        \subsubsection{印相机}
        \subsubsection{冲洗机}
        \subsubsection{烘干机}
        \subsubsection{其他设备}
    \subsection{洗印设备}
        \subsubsection{洗片机}
        \subsubsection{印片机}
        \subsubsection{缩放印片机}
        \subsubsection{清片机}
        \subsubsection{配光台}
        \subsubsection{其他设备}
    \subsection{剪接设备}
        \subsubsection{声画编辑机}
        \subsubsection{套片机}
        \subsubsection{接片机}
        \subsubsection{裁片机}
        \subsubsection{倒片机}
    \subsection{录音设备、还音设备}
    \subsection{放映设备}
        \subsubsection{放映机}
        \subsubsection{幻灯}
        \subsubsection{发电机}
        \subsubsection{扩大机}
        \subsubsection{银幕}
    \subsection{光源设备、照明设备}
        \subsubsection{光源、灯具}
        \subsubsection{采光、反光系统}
\section{各种摄影技术}
    \subsection{彩色摄影}
    \subsection{立体摄影}
    \subsection{全景摄影 (摇镜头摄影 )}
    \subsection{红外线摄影、紫外线摄影}
    \subsection{放射线摄影}
    \subsection{水下摄影}
    \subsection{空中摄影}
    \subsection{卫星摄影}
    \subsection{高速摄影}
    \subsection{显微摄影}
    \subsection{光电微光摄影}
    \subsection{传真摄影}
    \subsection{全息摄影}
        \subsubsection{光全息摄影}
        \subsubsection{声全息摄影}
        \subsubsection{微波全息摄影}
    \subsection{电影摄影}
\section{洗印技术}
    \subsubsection{负片过程}
    \subsubsection{正片过程}
    \subsubsection{翻摄与复制}
    \subsubsection{相片修整}
    \subsubsection{特种材料底基的印相法}
    \subsubsection{电影洗印}
\section{拍摄技术的应用}









\chapter{计量学}
\section{计量单位与单位制}
    \subsection{计量单位与单位制参考工具书}
        \subsubsection{计量单位手册}
        \subsubsection{度量衡换算法和换算表}
    \subsection{公制 (米制 )}
    \subsection{中制 (市制 )}
    \subsection{各国单位制}
\section{几何量计量}
    \subsubsection{长度计量}
    \subsubsection{角度计量}
\section{力学计量}
    \subsubsection{力值计量}
    \subsubsection{质量计量}
    \subsubsection{密度与粘度计量}
    \subsubsection{速度与加速度计量}
    \subsubsection{压力与真空计量}
    \subsubsection{冲击与振动计量}
    \subsubsection{流量与流速计量}
    \subsubsection{液面与物位计量}
    \subsubsection{硬度计量}
    \subsubsection{硬度计量}
    \subsubsection{时间与频率计量}
\section{热学计量}
    \subsubsection{热量计量}
    \subsubsection{温度计量}
    \subsubsection{湿度计量}
\section{声学计量}
\section{光学计量}
\section{电磁学与无线电计量}
    \subsubsection{电学计量}
    \subsubsection{磁学计量}
    \subsubsection{无线电计量}
\section{电离辐射和放射性计量}
\section{物理化学计量}

\chapter{END}






\end{document}


