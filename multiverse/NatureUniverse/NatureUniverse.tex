%%============
%%  ** Author: Shepherd Qirong
%%  ** Date: 2021-12-04 21:11:37
%%  ** Github: https://github.com/ShepherdQR
%%  ** LastEditors: Qirong ZHANG
%%  ** LastEditTime: 2024-12-29 19:25:04
%%  ** Copyright (c) 2019--20xx Shepherd Qirong. All rights reserved.
%%============


\documentclass[UTF8]{NatureUniverse}

\includeonly{
    01-Introduction
}

\begin{document}
%%============
%%  ** Author: Shepherd Qirong
%%  ** Date: 2022-06-04 23:40:10
%%  ** Github: https://github.com/ShepherdQR
%%  ** LastEditors: Shepherd Qirong
%%  ** LastEditTime: 2022-06-04 23:40:55
%%  ** Copyright (c) 2019--20xx Shepherd Qirong. All rights reserved.
%%============

\chapter{Introduction}

20220604,进入计算的世界。





\chapter{生物学}    %01 Biology
Biology:普通生物学、细胞生物学、遗传学、生理学、生物化学、生物物理学、分子生物学





\chapter{化学}    %03

无机化学、有机化学、高分子化学、物理化学、分析化学、应用化学


\chapter{Crystallography}    %04  Crystallography




\chapter{地球科学}    %04 Geoscience

地球科学:天文学、测绘学、地球物理学、大气科学 (气象学 )、地质学、海洋学、自然地理学





\chapter{空间科学}    %05
\section{天体生物学}
\section{观测天文学}
% \subsection{伽马射线天文学}
% \subsection{红外天文学}
% \subsection{微波天文学}
% \subsection{光学天文学}
% \subsection{射电天文学}
% \subsection{紫外线天文学}
% \subsection{X射线天文学}
\section{天体物理学}
% \subsection{引力天文学}
% \subsection{黑洞}
\section{星际介质}
\section{数值模拟}
% \subsection{天体物理等离子体}
% \subsection{星系形成与演化}
% \subsection{高能天体物理学}
% \subsection{流体力学}
% \subsection{磁流体动力学}
% \subsection{恒星形成}
\section{物理宇宙学}
\section{恒星天体物理学}
% \subsection{日震学}
% \subsection{恒星演化}
% \subsection{恒星核合成}
\section{行星科学}



\chapter{物理学}    %06








\chapter{计算机科学}


\section{理论}
% \subsection{计算理论}
%   \subsubsection{自动机理论(形式语言)}
%   \subsubsection{可计算性理论}
%   \subsubsection{计算复杂性理论}
%   \subsubsection{并发理论}
% \subsection{计算机架构}
%   \subsubsection{量子计算}
% \subsection{分布式计算}
% \subsubsection{网格计算}
% \subsection{并行计算}
% \subsubsection{高性能计算}

\section{算法}
% \subsection{数据结构}
% \subsection{计算几何}
% \subsection{分布式算法}
% \subsection{并行算法}
% \subsection{随机算法}

\section{编程语言}
%\subsection{编程语言的逻辑基础}
%   \subsubsection{形式化方法(形式验证)}
%   \subsubsection{语义理论(自然语言、程序语言)}
%   \subsubsection{类型理论}
%   \subsubsection{多值逻辑}
%   \subsubsection{模糊逻辑}
%   \subsubsection{推理技术}

% \subsection{编译原理}
% \subsection{编程范式}
%     \subsubsection{并发编程}
%     \subsubsection{函数式编程}
%     \subsubsection{命令式编程}
%     \subsubsection{逻辑编程}
%     \subsubsection{面向对象编程}
% \subsection{程序设计语言的设计与实现}


\section{软件}
% \subsection{系统软件}
%   \subsubsection{操作系统}
% \subsection{应用软件}


\section{硬件}
% \subsection{芯片设计}
% \subsection{超大规模集成电路设计}
% \subsection{嵌入式}


\section{计算机通信(网络)}
% \subsection{云计算}
% \subsection{信息论}
% \subsection{高速互连网络}
% \subsection{普适计算}
% \subsection{无线计算(移动计算)}
% \subsection{网络信息安全}

\section{计算机安全性和可靠性}
% \subsection{密码学}
% \subsection{信息安全技术}
% \subsection{容错计算}



\section{人工智能}
% \subsection{认知科学}
%     \subsubsection{自动推理}
%     \subsubsection{计算机视觉}
%     \subsubsection{机器学习}
%     \subsubsection{人工神经网络}
%     \subsubsection{自然语言处理(计算语言学)}
% \subsection{专家系统}
% \subsection{机器人}


\section{与其他领域结合}
% %%在计算数学,自然科学,工程和医药:
% \subsection{代数(符号)计算}
% \subsection{计算生物学(生物信息学)}
% \subsection{计算化学}
% \subsection{计算数学}
% \subsection{计算神经科学}
% \subsection{计算数论}
% \subsection{计算物理}
% \subsection{计算机辅助工程}
% \subsection{计算流体动力学}
% \subsection{有限元分析}
% \subsection{数值分析}
% \subsection{科学计算(计算科学)}

% %%社会科学、艺术、人文和专业领域的计算:
% \subsection{社区信息学}
% \subsection{计算经济学}
% \subsection{计算金融}
% \subsection{计算社会学}
% \subsection{数字人文(人文计算)}
% \subsection{计算机硬件的历史}
% \subsection{计算机科学史}
% \subsection{人文信息学}

% \subsection{数据库}
%     \subsubsection{分布式数据库}
%     \subsubsection{对象数据库}
%     \subsubsection{关系数据库}
% \subsection{数据管理}
% \subsection{数据挖掘}
% \subsection{信息架构}
% \subsection{信息管理}
% \subsection{信息检索}
% \subsection{知识管理}
% \subsection{多媒体,超媒体}
%     \subsubsection{声音和音乐计算}
% \subsection{人机交互}
% \subsection{图像处理与科学可视化}





\chapter{系统理论}
\section{1.计算机系统与应用}
\section{2.应用数学}
\section{3.管理工程}
\section{4.控制理论}


\chapter{数学}
\section{主条目:数学和数学大纲}
\section{主条目:数学学科分类}
\section{纯数学}
\section{数理逻辑与数学基础}
\section{直觉逻辑}
\section{模态逻辑}
\section{模型论}
\section{证明论}
\section{递归论}
\section{集合论}
\section{代数(大纲)}
\section{结合代数}
\section{范畴论}
\section{拓扑理论}
\section{微分代数}
\section{场论}
\section{群论}
\section{集团代表}
\section{同调代数}
\section{K理论}
\section{晶格理论(秩序理论)}
\section{李代数}
\section{线性代数(向量空间)}
\section{多重线性代数}
\section{非结合代数}
\section{表征理论}
\section{环论}
\section{交换代数}
\section{非交换代数}
\section{通用代数}
\section{分析}
\section{复杂分析}
\section{功能分析}
\section{算子理论}
\section{谐波分析}
\section{傅立叶分析}
\section{非标分析}
\section{常微分方程}
\section{p进分析}
\section{偏微分方程}
\section{真实分析}
\section{微积分(大纲)}
\section{概率论}
\section{遍历理论}
\section{测度论}
\section{积分几何}
\section{随机过程}
\section{几何(轮廓)和拓扑}
\section{仿射几何}
\section{代数几何}
\section{代数拓扑}
\section{凸几何}
\section{差分拓扑}
\section{离散几何}
\section{有限几何}
\section{伽罗华几何}
\section{一般拓扑}
\section{几何拓扑}
\section{积分几何}
\section{非交换几何}
\section{非欧几何}
\section{射影几何}
\section{数论}
\section{代数数论}
\section{解析数论}
\section{算术组合}
\section{几何数论}
\section{应用数学}
\section{近似理论}
\section{组合数学(大纲)}
\section{编码理论}
\section{密码学}
\section{动力系统}
\section{混沌理论}
\section{分形几何}
\section{博弈论}
\section{图论}
\section{信息论}
\section{数学物理}
\section{量子场论}
\section{量子引力}
\section{弦理论}
\section{量子力学}
\section{统计力学}
\section{数值分析}
\section{行动调查}
\section{分配问题}
\section{决策分析}
\section{动态规划}
\section{库存理论}
\section{线性规划}
\section{数学优化}
\section{最佳维护}
\section{实物期权分析}
\section{调度}
\section{随机过程}
\section{系统分析}
\section{统计(大纲)}
\section{精算学}
\section{人口统计学}
\section{计量经济学}
\section{数理统计}
\section{数据可视化}
\section{计算理论}
\section{计算复杂性理论 [1] }





\end{document}
