%%============
%%  ** Author: Shepherd Qirong
%%  ** Date: 2019-06-20 20:04:18
%%  ** Github: https://github.com/ShepherdQR
%%  ** LastEditors: Shepherd Qirong
%%  ** LastEditTime: 2022-06-05 17:49:02
%%  ** Copyright (c) 2019--20xx Shepherd Qirong. All rights reserved.
%%============







\chapter{06-01-Methodology}


\section{Methodology}


\section{理论物理}%TheoreticalPhysics
% \subsection{数学物理}
% \subsection{电磁场理论}
% \subsection{经典场论}
% \subsection{相对论和引力场}
% \subsection{量子力学} % Quantum
% \subsection{统计物理学} % StatisticalPhysics
%    \subsubsection{统计力学}
% \subsection{其他}

\section{力学} % Mechanics
% \subsection{牛顿力学}
% \subsection{固体力学}
% \subsection{空气动力学}
% \subsection{流体动力学}


\section{声学} % Acoustics
% \subsection{物理声学}
% \subsection{非线性声学}
% \subsection{量子声学}
% \subsection{超声学}
% \subsection{水声学}
% \subsection{应用声学}

\section{光学} % Optics
% \subsection{几何光学}
% \subsection{物理光学}
% \subsection{非线性光学}
% \subsection{光谱学}
% \subsection{量子光学}
% \subsection{信息光学}
% \subsection{导波光学}
% \subsection{发光学}
% \subsection{红外物理}
% \subsection{激光物理}
% \subsection{应用光学}
% \subsection{原子、分子和光学物理}


\section{热学}
% \subsection{热力学}
% \subsection{热物性学}
% \subsection{传热学}


\section{电磁学} % Electromagnetics
% \subsection{电学}
% \subsection{静电学}
% \subsection{静磁学}
% \subsection{电动力学} % Electrodynamics
% \subsection{电磁波}
%   \subsubsection{微波}
% \subsection{无线电}
%   \subsubsection{量子无线电}
%   \subsubsection{超高频无线电}
%   \subsubsection{统计无线电}


\section{粒子物理} % ParticlePhysics
% \subsection{电子物理}
%   \subsubsection{量子电子}
%   \subsubsection{电子电离与真空物理}
%   \subsubsection{带电粒子光学}
% \subsection{原子分子物理} % AtomicAndMolecular
%   \subsubsection{原子与分子理论}
%   \subsubsection{原子光谱}
%   \subsubsection{分子光谱}
%   \subsubsection{波谱学}
%   \subsubsection{原子与分子碰撞过程}


\section{凝聚态物理学}
% \subsection{凝聚态理论}
% \subsection{金属物理学}
% \subsection{半导体}
% \subsection{电介质}
% \subsection{晶体}
% \subsection{非晶态}
% \subsection{液晶}
% \subsection{薄膜}
% \subsection{低维物理}
% \subsection{表面与界面}
% \subsection{固体发光}
% \subsection{磁学}
% \subsection{超导}
% \subsection{低温} % Cryophysics
% \subsection{高压}


\section{等离子体物理}
% \subsection{热核聚变等离子体}
% \subsection{低温等离子体}
% \subsection{等离子体光谱学}
% \subsection{凝聚态等离子体}
% \subsection{非中性等离子体}


\section{核物理} % Nuclear
% \subsection{核结构}
% \subsection{核能谱}
% \subsection{低能核反应}
% \subsection{中子物理学}
% \subsection{裂变}
% \subsection{聚变}
% \subsection{轻粒子核物理}
% \subsection{重离子核物理}
% \subsection{中高能核物理}

\section{高能物理}
% \subsection{基本粒子物理学}
% \subsection{宇宙线}
% \subsection{粒子加速器}
% \subsection{高能物理实验}

\section{其他}
% \subsection{应用物理学}
% \subsection{天体物理学}
% \subsection{生物物理学}
% \subsection{实验物理}
% \subsection{地球物理学}
% \subsection{医学物理学}
% \subsection{固态物理学} % SolidState





















\chapter{Introduction}
Today is 20211211, and I deciede to note down all of my knowledge about physics in this notebook. Actually we think for a while whether to seperatre the knowledge into different documents.

\chapter{Preference}



\section{Volabulary}
orthogonal matrix, 正交矩阵


\chapter{History}



\chapter{观点}

\section{Videos}

牛顿经典力学,场论定域论,最小作用量原理,都可以解释从A到B的路径,是等效的。
So I made the hypothesis often that the laws are going to turn out to be, in the end, simple like the checkerboard, and that all the complexity is from size.

If you will not say that it is true in a region that you have not looked at, you do not know anything.

We always must make statements about the regions that we have not seen.

The mass of an object changes when it moves.

\subsection{需要再确认的观点}
\subsubsection{行星和卫星公转轨道为什么是椭圆?}

一个焦点位于原点的圆锥曲线
$\frac{1}{r}=C\left[1+e\cos(\theta-\theta^{\prime})\right]$
$f=-\frac{k}{r^{2}}~,\quad V=-\frac{k}{r}$
只考虑2体,角动量守恒求出轨道方程,角动量l与E看做常数,
$$\frac{1}{r}=\frac{mk}{l^{2}}\left(1+\sqrt{1+\frac{2El^{2}}{mk^{2}}}\cos(\theta-\theta^{\prime})\right)$$
离心率e<1椭圆,等于1是抛物线,大于1是双曲线。$e=\sqrt{1+\frac{2El^{2}}{mk^{2}}}$



