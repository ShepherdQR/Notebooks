%%============
%%  ** Author: Shepherd Qirong
%%  ** Date: 2019-06-20 20:04:18
%%  ** Github: https://github.com/ShepherdQR
%%  ** LastEditors: Qirong ZHANG
%%  ** LastEditTime: 2024-12-28 22:51:40
%%  ** Copyright (c) 2019--20xx Shepherd Qirong. All rights reserved.
%%============


\chapter{06-01-Methodology}


\section{Methodology}


开尔文的2朵乌云:迈克尔逊-莫雷实验,发展出相对论;黑体辐射,发展出量子力学。

\section{理论物理}%TheoreticalPhysics-2
数学物理、电磁场理论、经典场论、相对论、量子论、统计物理学、非线性物理学


\section{力学} % Mechanics  3



\section{声学} % Acoustics  4



\section{光学} % Optics



\section{热学} % thermology  6
热学:基础热学、热力学、物质分子运动论

\section{电磁学} % Electromagnetics
电学,磁学,电动力学,电磁学,无线电,真空无线电,凝聚态物理学



\section{Atomic、Nuclear、HighEnergy} % AtomicAndNuclearAndHighEnergy

粒子物理(分子、原子、电子)、核物理、高能物理、波谱学、能谱学


粒子物理:分子、原子、电子

原子核物理学:核结构、核衰变、核反应、中子、重离子

高能物理学:宇宙线、粒子

波谱学

能谱学


\section{固体} % SolidState 



\section{半导体} % Semiconductor


\section{特殊态物理} % SpecialState

低温、高压、高温、等离子体




\section{其他} 
% \subsection{应用物理学}
% \subsection{天体物理学}
% \subsection{生物物理学}
% \subsection{实验物理}
% \subsection{地球物理学}
% \subsection{医学物理学}


\section{Questions} %14



















\chapter{Introduction}
Today is 20211211, and I deciede to note down all of my knowledge about physics in this notebook. Actually we think for a while whether to seperatre the knowledge into different documents.

\chapter{Preference}



\section{Volabulary}
orthogonal matrix, 正交矩阵



\chapter{物理学研究方法}
    \section{物理学实验方法与设备}
    \section{物理测量}
    \section{电子计算机在物理学中的应用}
    \section{物理学词典}




\chapter{History}





\chapter{观点}

\section{Videos}

牛顿经典力学,场论定域论,最小作用量原理,都可以解释从A到B的路径,是等效的。
So I made the hypothesis often that the laws are going to turn out to be, in the end, simple like the checkerboard, and that all the complexity is from size.

If you will not say that it is true in a region that you have not looked at, you do not know anything.

We always must make statements about the regions that we have not seen.

The mass of an object changes when it moves.

\subsection{需要再确认的观点}
\subsubsection{行星和卫星公转轨道为什么是椭圆?}

一个焦点位于原点的圆锥曲线
$\frac{1}{r}=C\left[1+e\cos(\theta-\theta^{\prime})\right]$
$f=-\frac{k}{r^{2}}~,\quad V=-\frac{k}{r}$
只考虑2体,角动量守恒求出轨道方程,角动量l与E看做常数,
$$\frac{1}{r}=\frac{mk}{l^{2}}\left(1+\sqrt{1+\frac{2El^{2}}{mk^{2}}}\cos(\theta-\theta^{\prime})\right)$$
离心率e<1椭圆,等于1是抛物线,大于1是双曲线。$e=\sqrt{1+\frac{2El^{2}}{mk^{2}}}$



