%%============
%%  ** Author: Shepherd Qirong
%%  ** Date: 2021-12-21 22:01:55
%%  ** Github: https://github.com/ShepherdQR
%%  ** LastEditors: Qirong ZHANG
%%  ** LastEditTime: 2024-12-02 23:33:46
%%  ** Copyright (c) 2019--20xx Shepherd Qirong. All rights reserved.
%%============


\documentclass[UTF8]{../06-Physics}
\begin{document}

\title{06-04-Acoustics}
\date{Created on 20241202.\\   Last modified on \today.}
\maketitle
\tableofcontents




\chapter{Introduction}
声学

\chapter{声的原理}
    \section{基本理论}
    \section{振动体 (声源 )}
    \section{振动的发生方法}
    \section{机电类比}
    \section{固体中振动的传播}
    \section{声与物质的相互作用}

\chapter{声的传播}
    \subsubsection{声速}
    \subsubsection{声场}
    \subsubsection{声的反射与折射}
    \subsubsection{声的吸收与衰减}
    \subsubsection{声的干涉、衍射和散射}
    \subsubsection{声的共振与声的辐射}
    \subsubsection{大振幅声波、非线性效应}
    \subsubsection{噪音}


\chapter{声的合成与分析}
\chapter{物理声学}
\chapter{次声学}
    \section{次声的发生}
    \section{次声在大气中的传播}
    \section{大气中的次声源}



\chapter{超声学}
    \subsubsection{超声的发生}
    \subsubsection{超声的传播}
    \subsubsection{声光作用}
    \subsubsection{超声效应}
    \subsubsection{微波超声、声子}
    \subsubsection{声能学}
    \subsubsection{超声应用}



\chapter{水声学}
    \subsubsection{水声传播}
    \subsubsection{水中声波的散射和混响}
    \subsubsection{水中声起伏}
    \subsubsection{气泡、空化、湍流、尾流的声源特性}
    \subsubsection{水下噪声}
    \subsubsection{水声的应用}

    
\subsubsection{生理声学}
\subsubsection{应用声学}




\end{document}

