%%============
%%  ** Author: Qirong ZHANG
%%  ** Date: 2021-12-21 22:01:55
%%  ** Github: https://github.com/ShepherdQR
%%  ** LastEditors: Qirong ZHANG
%%  ** LastEditTime: 2024-12-28 23:15:58
%%  ** Copyright (c) 2019 Qirong ZHANG. All rights reserved.
%%  ** SPDX-License-Identifier: LGPL-3.0-or-later.
%%============



\documentclass[UTF8]{../06-Physics}
\begin{document}

\title{06-12-AtomicAndNuclearAndHighEnergy}
\date{Created on 20241228.\\   Last modified on \today.}
\maketitle
\tableofcontents





\chapter{Introduction}



AtomicAndNuclearAndHighEnergy

粒子物理:分子、原子、电子

原子核物理学:核结构、核衰变、核反应、中子、重离子

高能物理学:宇宙线、粒子

波谱学

能谱学

\chapter{Atomic}
粒子物理:分子、原子、电子

\section{分子物理学}
    \subsubsection{分子结构}
    \subsubsection{分子的性质及其测定}
    \subsubsection{分子光谱}
    \subsubsection{分子间的作用、激发与离解}
    \subsubsection{碰撞与散射}
\section{原子物理学}
    \subsection{原子的结构}
    \subsection{原子的性质及其测定}
    \subsection{原子光谱学}
        \subsubsection{谱线结构}
        \subsubsection{光谱线在电场及磁场中的分裂}
    \subsection{原子间的作用、激发与电离}
    \subsection{碰撞与散射}
        \subsubsection{原子与分子碰撞过程}
    \subsection{同位素}
\section{介分子与μ分子}
\section{介原子与μ原子}
\section{电子偶素与μ子素}

\section{电子物理}
  \subsubsection{量子电子}
  \subsubsection{电子电离与真空物理}
  \subsubsection{带电粒子光学}




\chapter{Nuclear}
原子核物理学:核结构、核衰变、核反应、中子、重离子

\section{原子核物理实验}
\section{原子核的结构与性质}
    \subsection{结构}
        \subsubsection{结构模型}
    \subsection{性质及其测定}
    \subsection{核谱学}
    核能谱

    \subsection{受激态}
    \subsection{核力}
        \subsubsection{理论}
        \subsubsection{性质与实验研究}


\section{放射性原子核衰变}
    \subsection{各种射线及其衰变}
        \subsubsection{α射线及α衰变}
        \subsubsection{β射线及β衰变}
        \subsubsection{γ射线及γ衰变}
        \subsubsection{同质异能素}
    \subsection{射线与物质的相互作用}
    \subsection{人工放射性}





\section{原子核反应}
低能核反应、高能核反应
    \subsection{受激嬗变}
        \subsubsection{一般理论}
        \subsubsection{共振理论、R-矩阵理论}
        \subsubsection{截面、能量关系}
        \subsubsection{角分布、角关联}
        \subsubsection{核反应的统计模型}
        \subsubsection{核反应的光学模型}
        \subsubsection{直接相互作用理论}
        \subsubsection{散射、极化}
    \subsection{各种类型的核反应}
        \subsubsection{中子引起的核反应}
        \subsubsection{质子、氘核、氚核、α粒子引起的核反应}
        \subsubsection{原子核引起的核反应}
        \subsubsection{电子和光子引起的核反应}
        \subsubsection{介子和超子引起的核反应}
        \subsubsection{其他粒子引起的核反应}
    \subsection{裂变}
        \subsubsection{理论与机制}
        \subsubsection{截面}
        \subsubsection{平均中子数}
        \subsubsection{角分布及质量分布}
        \subsubsection{能谱}
        \subsubsection{辐射}
        \subsubsection{链式反应、循环反应}
    \subsection{聚变}




\section{中子物理}
    \subsection{中子的基本性质}
    \subsection{中子类型}
        \subsubsection{慢中子}
        \subsubsection{中能中子}
        \subsubsection{快中子}
    \subsection{中子源和中子探测器}
    \subsection{中子能谱}
    \subsection{中子截面}
        \subsubsection{吸收截面}
        \subsubsection{俘获截面}
        \subsubsection{散射截面}
        \subsubsection{总截面}
    \subsection{中子衍射及其应用}

\section{重离子核物理}
轻粒子核物理、重离子核物理








\chapter{HighEnergy}

高能物理学:宇宙线、粒子


中高能核物理、高能物理

\section{宇宙线}
    \subsection{物理性质及探测}
    \subsection{初级宇宙线 (原始宇宙线 )}
        \subsubsection{簇射}
        \subsubsection{原子核星裂}
    \subsection{宇宙线的起源和传播}
    \subsection{宇宙线的应用}



\section{粒子物理学}
基本粒子物理学
    \subsection{实验与测定}
        \subsubsection{高能加速器}
        \subsubsection{探测器与探测法}
        \subsubsection{测量和数据处理设备}
        \subsubsection{对撞机}
    \subsection{对称性质与守恒定理}
    \subsection{相互作用}
        \subsubsection{弱相互作用}
        \subsubsection{电磁相互作用}
        \subsubsection{强相互作用}
        \subsubsection{超强相互作用}
        \subsubsection{引力相互作用}
        \subsubsection{引力相互作用}
    \subsection{结构模型}



\section{粒子类型}

    \subsection{光子与规范粒子交子、引力子入此。}
    \subsection{轻子}
        \subsubsection{中微子及其反粒子}
        \subsubsection{电子及其反粒子}
        \subsubsection{μ子及其反粒子}
        \subsubsection{τ子及其反粒子}
    \subsection{介子}
        \subsubsection{π介子及其反粒子}
        \subsubsection{κ介子及其反粒子}
        \subsubsection{其他介子}
    \subsection{重子}
        \subsubsection{质子及其反粒子}
        \subsubsection{中子及其反粒子}
        \subsubsection{λ超子及其反粒子}
        \subsubsection{ε超子}
        \subsubsection{反ε超子}
        \subsubsection{ξ超子}
        \subsubsection{反ξ超子}
    \subsection{简单核 (原子序数或电荷小于3的核 )}
        \subsubsection{氘核}
        \subsubsection{氚核}
        \subsubsection{α粒子}
        \subsubsection{氦3}
        \subsubsection{氦5}



\chapter{波谱学}
\chapter{能谱学}

\chapter{END}






\end{document}