%%============
%%  ** Author: Qirong ZHANG
%%  ** Date: 2024-12-02 23:02:57
%%  ** Github: https://github.com/ShepherdQR
%%  ** LastEditors: Qirong ZHANG
%%  ** LastEditTime: 2024-12-02 23:06:16
%%  ** Copyright (c) 2019 Qirong ZHANG. All rights reserved.
%%  ** SPDX-License-Identifier: LGPL-3.0-or-later.
%%============



\documentclass[UTF8]{../../06-Physics}
\begin{document}

\title{06-03-05-SolidMechanics}
\date{Created on 20241202.\\   Last modified on \today.}
\maketitle
\tableofcontents


\chapter{Introduction}

固体力学



\chapter{材料力学}
\chapter{结构力学}

\chapter{弹性力学}
    \subsection{二维问题 (平面问题 )}
    \subsection{三维问题 (空间问题 )}
    \subsection{接触问题}
    \subsection{应力集中问题}
    \subsection{非线性弹性力学}
    \subsection{热弹性力学 (热应力 )}
    \subsection{非均匀介质弹性力学}
    \subsection{各向异性弹性力学}
    \subsection{弹性稳定性问题}

\chapter{塑性力学}
    \subsection{塑性力学基本理论}
    \subsection{理想塑性力学}
    \subsection{弹塑性力学}
    \subsection{塑性流动问题}
    \subsection{极限分析}
    \subsection{蠕变理论}
    \subsection{弹塑性稳定性问题}

\chapter{粘弹塑性介质力学}
\chapter{强度理论}
    \subsection{断裂理论}
        \subsubsection{脆性断裂}
        \subsubsection{韧性断裂}
        \subsubsection{碎裂 (反射碎裂 )}

    \subsection{疲劳理论}
        \subsubsection{腐蚀疲劳}
        \subsubsection{应力腐蚀}
        \subsubsection{各种因素对疲劳的影响}

    \subsection{强度理论的原子学说及微观机理}
    \subsection{强度理论的实验}
    \subsection{损伤理论}



\chapter{变形固体动力学}
    \subsection{动载荷}
    \subsection{动力稳定性}
    \subsection{冲击载荷下的材料强度}
    \subsection{应力波}
        \subsubsection{弹性波}
        \subsubsection{热弹性波}
        \subsubsection{不完全弹性波}
        \subsubsection{分层介质中的波}
    \subsection{冲击波}
        \subsubsection{热冲击波}
    \subsection{转子动力学}
    \subsection{散体力学}




\chapter{实验应力分析}
    \subsection{光测法}
        \subsubsection{激光测试}
        \subsubsection{全息法}
    \subsection{电测法}
    \subsection{机械测定法}
    \subsection{涂盖法 (脆膜法 )}
    \subsection{高温变形测试技术}
    \subsection{X射线法}
    \subsection{比拟法、模拟理论}
    \subsection{声学方法}
    \subsection{其他}






\end{document}

