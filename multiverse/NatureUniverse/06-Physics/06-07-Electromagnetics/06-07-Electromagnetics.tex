%%============
%%  ** Author: Qirong ZHANG
%%  ** Date: 2021-12-21 22:01:55
%%  ** Github: https://github.com/ShepherdQR
%%  ** LastEditors: Qirong ZHANG
%%  ** LastEditTime: 2024-12-28 22:06:41
%%  ** Copyright (c) 2019 Qirong ZHANG. All rights reserved.
%%  ** SPDX-License-Identifier: LGPL-3.0-or-later.
%%============



\documentclass[UTF8]{../06-Physics}
\begin{document}

\title{06-07-Electromagnetics}
\date{Created on 20220605.\\   Last modified on \today.}
\maketitle
\tableofcontents





\chapter{Introduction}


\section{Basic}

电学,磁学,电动力学,电磁学,无线电,真空无线电,凝聚态物理学


\section{Preference}
注重发现的过程,物理思想。研究对象变化导致需要新的概念、手段。

The energy is flowing through the space around the conductor.

赵凯华,电磁学,高教出版社;
陈秉乾,电磁学专题研究,高教。












\chapter{电学}

\section{静电场}

\subsection{库仑定律和场强叠加原理}

观察现象,提出问题,猜测,实验,规律,新物理量,公式(定量表述),成立条件,适用范围,精度,理论地位,发展,应用。 

物理定律的内涵和外延。

电力,非接触,和引力、电磁力一样非接触,最好先研究点电荷。猜测电力和距离平方成反比,因为均匀球壳中不受力,就像均匀球壳中心不受引力。1785库伦扭秤实验测电斥力。力弱,漏电。测电引力,类比单摆周期和到质心距离成正比,做单摆实验。
物理规律有层次,空间对称性层次高。

从特称判断到全称判断,从特殊到一般。

库伦定理,真空(可以不真空),静止(必须,可以是相对于静止电荷对运动电核的作用,)运动电荷产生的场和速度有关,静电荷和动电荷之间的相互作用不遵循牛顿第三定律,牛三本质是动量守恒,说明除了两个电荷,还有第三者参与了且第三者的动量也变化了,这个第三者是场。在某个参考系中可能2电荷相对静止,只有库仑力,换个参考系2电荷运动,产生磁场,点作用+磁作用。电磁现象联系紧密。

成立的尺度范围:原子级别,原子核内部可能有问题。大尺度到太阳星系范围可能也没有问题。

库伦定理是迄今为止物理学中最精确的实验定律之一,电磁力和距离的$-2 \pm 10^{-16}$次方成正比(1971年测量数据)。因为这个比例系数和光子的静止质量有关,电磁波没有静止质量。

\begin{displaymath}
    \boldsymbol F_{ab} = \frac{1}{4 \pi \varepsilon _0} \frac{q_a \cdot q_b}{r^2} \boldsymbol {\hat{r} } 
\end{displaymath}

电荷性质:守恒性,量子性,非相对论性。电荷有正负,引力只有正,电荷可以屏蔽,引力不能。

电场强度 $\boldsymbol E = \boldsymbol F / q_0$

用试探电荷来描绘要测量的某点电荷的场强分布,因而试探电荷电量要小来减少对测量电荷场强分布的影响,尺寸要小来描绘精确。

场强线性叠加 

$$\boldsymbol E =  \int d \boldsymbol E
= \frac{1}{4 \pi \varepsilon _0} \int \frac{\boldsymbol {\hat{r} } }{r^2} dq $$

\subsection{高斯定理,环路定理,电势}

非接触力的机制,这种作用是怎么进行的,有媒介物吗?有传递时间吗?

先研究静态的力场,然后研究动态的力场,然后研究力场与其中电荷的相互作用。把场作为研究的对象。

场:一定空间中连续分布的物体。如矢量场,标量场。

静电场的几何描述:电场线,切线方向表示电场方向。如何从整体上描述一个场,或较容易一下子区分不同的场?因为电场线画出来眼花缭乱的。最早是麦克斯韦做的,用类比的方法。看看流体力学怎么做的。不可压缩流体的恒定流动可以画出流速场,流体力学抓住了何处有源、汇,何时有旋,并且发现可以用通量的概念描述这种特征,包围源的通量大于0,包围汇的通量小于0,其余情况等于0。

$ \vec{A} \cdot d \vec{s} = A \cos \theta ds $,$\vec{A}$可取$\vec{v},\vec{E},\vec{B}$ 

\begin{equation}
\begin{split}
    &\oiint _S  \vec{E}d \vec{s}  = \frac{1}{\varepsilon} \sum_{S_{in}} q_i,
    \mbox{高斯定理, 有源场,通量不等于0}\\
    &\oint _l  \vec{E}d \vec{l}  = 0,
    \mbox{无旋场,闭合环路积分等于0}\\
\end{split}
\end{equation}

定义电势:$U_p = \int _p ^ \infty \vec{E} \cdot d \vec{l} $

场强和电势都可叠加,即$\vec{E} = \int d \vec{E}$, $U = \int dU$

电场线和等势面的关系:处处正交,$\vec{E}$指向U减小的方向,两者的关系:$\vec{E} = -\nabla U = - (\frac{\partial U }{\partial x},\frac{\partial U }{\partial y},\frac{\partial U }{\partial z})$





小结:
\begin{equation}
    \mbox{计算场强的3种方法:}
    \left\{ 
        \begin{aligned}
            &\oiint _S  \vec{E}d \vec{s}  = \frac{1}{\varepsilon} \sum_{S_{in}} q_i\\
            &\boldsymbol E =  \int d \boldsymbol E
            = \frac{1}{4 \pi \varepsilon _0} \int \frac{\boldsymbol {\hat{r} } }{r^2} dq\\
            &\vec{E} = -\nabla U = - (\frac{\partial U }{\partial x},\frac{\partial U }{\partial y},\frac{\partial U }{\partial z})
        \end{aligned}
    \right.
\end{equation}

\begin{equation}
    \mbox{计算电势的2种方法:}
    \left\{ 
        \begin{aligned}
            &U_p = \int _p ^ \infty \vec{E} \cdot d \vec{l}\\
            &U = \int dU = \frac{1}{4 \pi \varepsilon _0} \int \frac{dq}{r}
        \end{aligned}
    \right.
\end{equation}


\subsection{导体和电容}

场与物质的相互作用。导体和绝缘体解释,提出自由电荷、束缚(极化)电荷。


\chapter{磁学}
    \subsubsection{磁学}
    \subsubsection{电磁感应}
    \subsubsection{电磁波与电磁场}
    \subsubsection{电磁测量}
    \subsubsection{物质的电磁性质}


\chapter{电动力学} % Electrodynamics

田光善,p2;
参考书:
场论,Landau,Lifschitz;
classical electordynamics, J D. Jackson

A history of the throries of autheer and electricity. E T Whittacher.



\chapter{电磁波}
  \section{微波}
  



  

\chapter{无线电物理学}
\section{电磁波传播理论}
\section{超高频无线电物理}
\section{无线电线路理论}
\section{统计无线电物理}
\section{量子无线电物理}
    \subsection{量子振荡器理论及频率标准}
    \subsection{量子放大器理论}
    \subsection{量子调制器与检波器理论}
    \subsection{无线电波段中的量子起伏理论}
\section{无线电波谱学}






\chapter{真空无线电子学 (电子物理学 )}
\section{气体放电 (气体导电 )}
\section{基本物理过程}
\section{各类型放电}
    \subsection{辉光放电}
    \subsection{弧光放电}
    \subsection{火花放电}
    \subsection{高频放电}
    \subsection{脉冲放电}
    \subsection{固体放电}
\section{阴极电子学}
    \subsubsection{热电子发射、热阴极}
    \subsubsection{二次电子发射、二次电子发射阴极}
    \subsubsection{光致发射、光阴极、外光电效应}
    \subsubsection{场致发射、场致发射阴极}
    \subsubsection{离子发射、离子发射阴极}
\section{带电粒子光学}
    \subsubsection{电子光学}
    \subsubsection{离子光学}



\chapter{凝聚态物理学}






\end{document}