%%============
%%  ** Author: Shepherd Qirong
%%  ** Date: 2023-04-05 10:23:55
%%  ** Github: https://github.com/ShepherdQR
%%  ** LastEditors: Shepherd Qirong
%%  ** LastEditTime: 2024-09-15 21:33:30
%%  ** Copyright (c) 2019--20xx Shepherd Qirong. All rights reserved.
%%============


\documentclass[UTF8]{../../09-Mathematics}
\begin{document}



\title{09-04-01-LinearAlgebra}
\date{Created on 20230405.\\   Last modified on \today.}
\maketitle
\tableofcontents


\chapter{Overall}



2条主线:linear space, linear mapping





\chapter{Space}




线性空间(加法、乘法) -> 线性映射 <->矩阵

现实几何空间还有距离和角度的概念,可用内积(双线性函数)来刻画。具有度量的线性空间。分类:欧几里得空间(实数域、有限维、线性空间、内积,内积有交换律)、酉空间(复数域、有限维、线性空间、内积,内积有共轭交换律)。

线性变换:空间A到空间AZ自身的线性映射。

\subsection{Linear Space}

向量加法、标量乘法构成的单位环。


\subsection{Metric Space}

\begin{definition}
    The set X with a distance function d, d satisfies\ref{Defination:Distance_function}. Metric Space is noted as $(X,d)$.
\end{definition}

\begin{definition}
    紧集:$(X,d)$中的子集A,A中任意序列都存在一子列$x_n$,$x_n$收敛到A中某点。
\end{definition}

\begin{definition}
    稠密集:$(X,d)$中的子集A,对于X中的任意点x,A中存在点a,使得$d(x,a)< \varepsilon $
\end{definition}

\begin{definition}
    X可分:$(X,d)$中存在一个可数稠密集。
\end{definition}


\subsubsection{Complete Metric Space}

\begin{definition}
    收敛:sequence $\{ x_n \}$ 收敛到c,means that $\lim_{x \to \infty} d(x_n, c) = 0$, noted as $\lim_{x \to \infty} x_n = c $
\end{definition}

\begin{definition}
    Cauchy基本列:$\lim_{m \to \infty,n \to \infty}d(x_m, x_n)=0  $
\end{definition}

\begin{definition}
    完备距离空间:所有Cauchy基本列收敛于一点
\end{definition}

\begin{definition}
  不完备:对于苹果空间,从宇宙开始到宇宙结束的所有苹果序列,收敛到我,则不完备。
\end{definition}

\subsection{Banach Space}

完备、赋范、线性。


\subsection{Inner Product Space}

\begin{equation}\label{}
\begin{aligned}
  &(\alpha x + \beta y ) \cdot z = \alpha x \cdot z + \beta y \cdot z, &   \mbox{线性}\\
  &x \cdot (\alpha y + \beta z) = \bar{ \alpha}    x \cdot y + \bar{ \beta}    x \cdot z,&   \mbox{共轭线性}\\
  &x \cdot y = \bar{ y \cdot x}, &   \mbox{共轭对称}\\
  &x \cdot x \geqslant 0 , &\mbox{正定} \Rightarrow || x || = \sqrt{x \cdot x} \\
  &|x \cdot y| \leqslant || x || \cdot || y ||, &satisfies  Cauchy-Schwarz \\
\end{aligned}
\end{equation}
 

\subsection{Hilbert Space}

完备,内积。

内积$\Rightarrow$范数$\Rightarrow$完备


\begin{proposition}
  $[0,1]$上的复连续函数空间$ C([0, 1])$,定义内积$f \cdot g = \int_{0}^{1} f(t)g(t) \,dt $, proof that $ C([0, 1])$不是Hilbert Space

  \begin{equation}
    \begin{aligned}
    &f_n(t) =
    \begin{cases}
    &1,\qquad 0 \leqslant t \leqslant \frac{1}{2}\\
    &-2n(t- \frac{1}{2}) + 1,\qquad \frac{1}{2} < t \leqslant \frac{1}{2n} + \frac{1}{2}\\
    &0, \qquad \frac{1}{2n} + \frac{1}{2} < t \leqslant 1\\
    \end{cases}\\
    &|| f_n - f_m|| \leqslant (\frac{1}{n} + \frac{1}{m})^{\frac{1}{2}} \rightarrow 0, is \  Cauchy \  Sequence.\\
    & \lim f_n = 
    \begin{cases}
      &1,  0 \leqslant t \leqslant \frac{1}{2}\\
      &0,  \frac{1}{2} < t \leqslant 1\\
    \end{cases}\\
    & \therefore \lim f_n \notin C([0, 1])
    \end{aligned}
  \end{equation}

\end{proposition}


\subsection{Euclid Space}
有序的n元组的全体称为n维Euclid空间,记为$\mathbb R^n$,称$\boldsymbol p=(p_i)_{i=1}^n \in \mathbb R^n$是$\mathbb R^n$的一个点。\\
为便于研究,本论文以$ \mathbb R^3$为背景空间,所涉及的函数默认为可微实值函数。如果实函数$f$的任意阶偏导数存在且连续,则称函数是可微的(或无限可微的,或光滑的,或$C^\infty$的)。\\
由于微分运算是函数的局部运算,限制所讨论函数的定义域在$ \mathbb R^3$中的任意开集,所讨论的结论仍然成立。\\
自然坐标函数:定义在$\mathbb R^n$上的实值函数$x_i: \mathbb R^n \to  \mathbb R$,使得$\boldsymbol p=(p_i)_{i=1}^n = \left( x_i(\boldsymbol p) \right)_{i=1}^n   $\\
切向量:由$\mathbb R^n$ 中的二元组构成,$\boldsymbol v_{\boldsymbol p}=(\boldsymbol p,\boldsymbol v)$,其中$\boldsymbol p$是作用点,$\boldsymbol v$是向量部分\\
切空间$T_p  \mathbb R^n$: 作用点$\boldsymbol p \in \mathbb R^n$的所有切向量的集合。利用向量加法与数量乘法使某点的切空间称为向量空间,与背景空间存在非平凡同构。\\
向量场$\boldsymbol V$:作用于空间点的向量函数,$\boldsymbol V(\boldsymbol p)\in T_p  \mathbb R^n $\\
逐点化原理:$(\boldsymbol V+\boldsymbol W)(\boldsymbol p)=\boldsymbol V(\boldsymbol p)+\boldsymbol W(\boldsymbol p),\ (f \boldsymbol V)(\boldsymbol p)= f(\boldsymbol p)\boldsymbol V (\boldsymbol p)$\\
自然标架场:定义$\boldsymbol U_i=(\delta _j^i)_{j=1}^n$,按Einstein求和约定,有$\boldsymbol V(\boldsymbol p)=v^i(\boldsymbol p)\boldsymbol U_i(\boldsymbol p)$,称$v^i$为场的Euclid坐标函数,其中Kronecker $\delta$函数定义为:
\begin{equation}
\label{Kronecker_delta}
\delta _i^j=\left\{ 
    \begin{aligned}
    1,\  & i =j\\
    0,\  & i \neq j\\
    \end{aligned}
     \right.
\end{equation}


与度量有关的线性变换:
正交变换
对称变换



\subsection{Unitary Linear Space}

酉空间 


与度量有关的线性变换:
酉变换
Hermite变换



\chapter{linear mapping}


\chapter{Vector}


\section{Basic Defination}

we define the basic element as following, where $ \boldsymbol e_i $ means $x_i = 1, x_j = 0$ for all $j \neq i$. When we say a vector, it means a column vector.

\begin{equation}
\vec{x}  = \boldsymbol x = [x_1, x_2,\dots]^T
= \begin{bmatrix}
    x_1 \\
    \vdots \\
    x_n
\end{bmatrix}
= \Sigma x_i \boldsymbol{e_i} 
\end{equation}

We define Kronecker sign to simply the description of $\boldsymbol e_i \cdot \boldsymbol e_j $.

\begin{equation}
    \begin{split}
    &\delta _{ij}:=
    \begin{cases}
    &1,\qquad i = j\\
    &0,\qquad i \neq j\\
    \end{cases}\\
    \end{split}
\end{equation}

The set of bases $\{ \boldsymbol e_i  \}  \xrightarrow{apply} \boldsymbol{x} \longrightarrow    \{ x_i \}   $.
%%\stackrel{apply}
%%\xrightarrow[under]{up} 





\section{Operation}




\subsection{Dot Product}

We define in algebra, $ \boldsymbol{x} \cdot \boldsymbol{y} := \sum{x_iy_i \delta _{ij}} = \boldsymbol{x}^T \cdot \boldsymbol{y}$.

Then the defination is restricted to the choose of the coordinate system. 


\subsection{Cross Product}

$a, b \in \mathbb F^m, a \wedge b = c \in \mathbb F^n $, if m = n, we have m = 0, 1, 3, 7. Therefore, we define cross product in 3d.

\begin{equation}
  u \times  v = 
  \begin{vmatrix}
     i & j & k\\
     u_1 & u_2 & u_3\\
     v_1 & v_2 & v_3\\
  \end{vmatrix}
\end{equation}


\begin{proposition}
  外积对于u、v双线性。从定义易知。
\end{proposition}

\begin{proposition}
  $(a \times b) \times c =  (c \cdot a) b -  (b \cdot c) a   $

  证明:
  \begin{equation}
    \begin{vmatrix}
       i & j & k\\
       23 & 31 & 12\\
       1 & 2 & 3\\
    \end{vmatrix}
  \end{equation}
\end{proposition}
例如对i分量,有$31 \cdot 3-12\cdot2$,形式上ijk一样,因而证明i即可。展开后,按正负号分类,we have $(313+212) - (133+122)$, 两部分都加上111即得。b和-a的线性组合。


\begin{proposition}

  混合积$(u, v, w) = (u \times  v ) \cdot w$,其具有轮换对称性。

  证明:
  for $\cdot w$, we have $23 \cdot 1 + 31 \cdot 2 + 12 \cdot 3$

  $231-321, 312-132, 123-213$

  对$\cdot v$,即中间元素按1,2,3顺序组合,易有wu;同样对$\cdot u$,易有vw。即证。
  

  另外,从展开后的分量对应上,易有
  \begin{equation}
    (u, v, w)  =
    \begin{vmatrix}
      u_1 & u_2 & u_3\\
      v_1 & v_2 & v_3\\
      w_1 & w_2 & w_3\\
    \end{vmatrix}
  \end{equation}

\end{proposition}



\subsection{Add}

\begin{equation}
    \begin{split}
    & \boldsymbol x + \boldsymbol y := \sum (x_i + y_i)\boldsymbol e_i\\
    & k \cdot \boldsymbol x := \sum kx_i\boldsymbol e_i\\
\end{split}
\end{equation}

Law $\boldsymbol{x} + \boldsymbol{y} = \boldsymbol{y} + \boldsymbol{x}$,
law $ (\boldsymbol{x} + \boldsymbol{y} )+ \boldsymbol{z} = \boldsymbol{x} +( \boldsymbol{y} + \boldsymbol{z})$ is not obvious in the view of Set Theory.


\subsection{geometry Properties}

\subsubsection{Length and Angle}

\begin{equation}
    \begin{split}
    &\parallel \boldsymbol{x} \parallel := \sqrt{\boldsymbol x \cdot \boldsymbol x}\\
    &\cos {\theta_{x,y}} : = \frac
    {\boldsymbol x \cdot \boldsymbol x}
    {\parallel \boldsymbol{x} \parallel \cdot \parallel \boldsymbol{y} \parallel}\\
\end{split}
\end{equation}


\subsubsection{Distance}

Distance function satisfies the following:
\begin{equation}\label{Defination:Distance_function}
    \begin{split}
    &d(\boldsymbol x, \boldsymbol y) \geqslant 0\\
    &d(\boldsymbol x, \boldsymbol y) = d(\boldsymbol y, \boldsymbol x)\\
    &d(\boldsymbol x, \boldsymbol y) \leqslant d(\boldsymbol x, \boldsymbol z) + d(\boldsymbol z, \boldsymbol y)\\
\end{split}
\end{equation}

\begin{equation}
    \begin{split}
    &d_p = [\sum |x_i-y_i|^p]^{\frac{1}{p}}, 1\leqslant p < \infty\\
    & d_{\infty} = \max_i|x_i-y_i|\\
\end{split}
\end{equation}



\chapter{Matrix Theory}


\section{Basic Notation}

Normally, we consider vector space over the fields of real or complex numbers.

\subsection{linear equation Ax = B}

\subsubsection{Defination}

linear equation in n variables.$\sum_{i = 1}^{i=n}a_ix^i = b$, which can be written as $\boldsymbol a^T \boldsymbol x = b$. We collect m equations and write like this:

\begin{equation}
  \begin{bmatrix}
    \boldsymbol a^T_1 \\
    \vdots \\
    \boldsymbol a^T_m \\
  \end{bmatrix}
  \cdot
  \begin{bmatrix}
    \boldsymbol x
  \end{bmatrix}
  =
  \begin{bmatrix}
     b_1 \\
    \vdots \\
     b_m \\
  \end{bmatrix}
\end{equation}

Noticed that $x_1$ is only applied toe the first column of the left matrix, we can say that $ \boldsymbol x$ is one point, or a specific composition, of the space spanned by the column vector of the matrix. Then it is easy to see that this equation has the solution, only if the vector $ \boldsymbol b$ is in the space spanned by the column vector of the matrix.


Or we can write like this:
\begin{equation}
  \begin{bmatrix}
     a_{11} & \cdots & a_{1n}\\
     & \ddots & \\
     a_{m1} & \cdots & a_{mn} \\
  \end{bmatrix}
  \cdot
  \begin{bmatrix}
    x_1 \\
    \vdots \\
    x_n \\
  \end{bmatrix}
  =
  \begin{bmatrix}
    b_1 \\
    \vdots \\
    b_m \\
  \end{bmatrix}
\end{equation}

The equation $\boldsymbol A \boldsymbol x = \boldsymbol b$ has solution, means y可由A的列向量线性表出。

If $\boldsymbol b = \boldsymbol 0$, called homogeneous linear equations, homogeneous because 所有非0项是1次的。if $\boldsymbol b \neq \boldsymbol 0$, it is inhomogeneous. 显然0向量(zero solution, or trivial soltution) 是一个解. A的列向量正交,只有零解;若A的列向量线性相关,有多解,即可按多种方式回到原点。



\subsubsection{Complex Matrices}

\paragraph{Conjugate Transposition}

for matrix $ \boldsymbol  C \in \mathbb{C}^{m \times n}$, we mark the Conjugate Transposition as $C^H$,
where $c^T_{ji} = \overline{c}_{ij}$

in representation, $ \boldsymbol  C = \boldsymbol A + i \boldsymbol B$. Usually 4 real matrix multiplications are needed to calculate $(C+iD)(E+ i F)$, actually 3  multiplications  are enough. $(C+iD)(E+ i F) =  (C+D)(E-F)+CF-DE + i(DE+CF)$




\subsubsection{Number of solution}

\paragraph{非齐次线性}

n元线性方程组解的个数等解集结构的研究,期待在不求解的情况下有所了解,就需要研究系数矩阵表示的n维向量空间的性质。

构造增广矩阵$[\boldsymbol A,\boldsymbol b ]$后,初等行变换化为阶梯型,如\ref{fig:number_of_solution}所示,解的个数讨论。

\begin{figure}[h]
  \centering
    \begin{tikzpicture}
      \draw (0,0) rectangle (6,8);
      \draw (6,0) rectangle (7,8);
      \draw (0,6)--(2,6)--(4,2)--(6,2);
      \draw (7.5,0)--(8,0);
      \draw (7.5,2)--(8,2);
      \draw (7.8,1) node {R};

      \draw (4,1)--(4,1.5);
      \draw (5,1.3) node {d};
    \end{tikzpicture}
  \caption{【number of solution】}\label{fig:number_of_solution}
\end{figure}

总共有n+1列,下面r行都是0.

(1)$d = 0$,即最后一个个主元在第n+1列,即存在方程0=1,无解, no solution。\\
(2)$d = 1$,即最后一个个主元在第n列,唯一解, one solution, $tr \boldsymbol A_{mn} = m$。\\
(3)$d > 1$,即最后一个个主元在第t列,$t < n$。高度R所在的行号记为r。有无穷个解。解可以这样写出,共R行,即R个主元,每个主元都用所在行的常数项d和n-r个自由元表示出来。

根据主元的构造过程,t的列号一定大于等于r。


当$A_{ii}$都是主元的时候,$d \neq 1$, $tr \boldsymbol A_{mn} < m$,inifinity solution,最后一行是解的超平面方程, 图中 d 是解的维度,$d =  n -tr \boldsymbol A_{mn}$, 如d为3,有3列独立的,即解空间是三维的。齐次方程组的未知数个数大于方程个数,有无数解。



$\det \boldsymbol A = 0$, no solution, or infinite solution.
$\det \boldsymbol A \neq 0$, one solution.




\paragraph{齐次线性}

一定有0解,因而当有非0解时,有无穷个解。n列时,系数矩阵的秩$r < n$。

方程个数$s < n$时,由于$r  \leqslant s < n$,易知有无穷个解。



\subsection{Matrix}


\subsubsection{Defination}

If we have a serious of $\boldsymbol x$, we have a serious of $b$, like this:
\begin{equation}
  A_{mn} \cdot X_{nt} = B_{mt}
  \Longrightarrow
  \begin{bmatrix}
     a_{11} & \cdots & a_{1n}\\
     & \ddots & \\
     a_{m1} & \cdots & a_{mn} \\
  \end{bmatrix}
  \cdot
  \left[
    \begin{array}{c:c:c}
      x_{11} & \cdots & x_{1t}\\
     \vdots&  & \vdots \\
     x_{n1} & \cdots & x_{nt} \\
    \end{array}
 \right] 
  =
 \left[
    \begin{array}{c:c:c}
      b_{11} & \cdots & b_{1t}\\
      \vdots& \cdots & \vdots \\
      b_{m1} & \cdots & b_{mt} \\
    \end{array}
 \right] 
\end{equation}

The normal definition of the product of two matrix is as above. 



\subsubsection{Multiplication}


There are 6 views sorting with the loop order, we fully understand that. for example, we can think the order jki(j is the outer, i is the inner) as follows
\begin{equation}
    \begin{split}
    &i: | \ \ \cdot = | \\
    &k: [\ | \ ]  \ \ | =  \sum|  \\
    &j:  [\ | \ ]  \ \  [\ | \ ]=  [\ | \ ]\\
\end{split}
\end{equation}


We collect the 6 vews into one table as fallows.

\begin{table}[htbp]
    \newcommand{\tabincell}[2]{\begin{tabular}{@{}#1@{}}#2\end
    {tabular}}
    \centering
    \begin{threeparttable}
    \caption{$\boldsymbol A_{ik} \boldsymbol X_{kj} = \boldsymbol B_{ij}$ }
    \label{tab:label}
    \begin{tabular}{cccccc}
        \toprule
        Order & innerLoop & MiddleLoop & dataAccess & view & comment\\
        \midrule
        ijk & S-S:dot & rowV-M 
        & $\boldsymbol A_{\alpha:},[\boldsymbol X_{:\beta}], \boldsymbol B_{\alpha:} $ 
        & $ [\overline{\ \ } ] \cdot [|||] =  [\overline{\ \ } ]$
        & dot view $\rightharpoonup \downharpoonright$ \\

        jik & S-S:dot & M-columnV
        & $[\boldsymbol A_{\alpha:}],\boldsymbol X_{:\beta}, \boldsymbol B_{:\beta} $ 
        & $ [\equiv ] \cdot [|]=  [|]$
        & dot view $ \downharpoonright \rightharpoonup$\\

        ikj & S-rowV:saxpy & rowV-M:gaxpy
        & $\boldsymbol A_{\alpha:},[\boldsymbol X_{\beta:}], \boldsymbol B_{\alpha:} $ 
        & $[\overline{\ \ } ] gaxpy [\equiv ] = [\overline{\ \ } ]$
        & useOfA $\rightharpoonup \downharpoonright$\\

        jki & colV-S:saxpy & M-colV:gaxpy
        & $[\boldsymbol A_{:\alpha}],\boldsymbol X_{:\beta}, \boldsymbol B_{:\alpha} $ 
        & $[||| ] gaxpy [| ] = [| ]$
        & useOfB $ \downharpoonright \rightharpoonup$\\

        kij& S-rowV:saxpy & colV-rowV:outP
        & $\boldsymbol A_{:\alpha},\boldsymbol X_{\beta:},\sum \boldsymbol B_{row} $ 
        & $\sum[| ] outProd [\overline{\ \ } ] =\sum[\equiv ]$
        & $ on \  A \downharpoonright outProd \rightharpoonup$\\

        kji& colV-S:saxpy & colV-rowV:outP
        & $\boldsymbol A_{:\alpha},\boldsymbol X_{\beta:},\sum \boldsymbol B_{col} $ 
        & $\sum[| ] outProd [\overline{\ \ } ] =\sum[||| ]$
        & $on \  X  \downharpoonright outProd \rightharpoonup$\\
        \bottomrule
    \end{tabular}
    \begin{tablenotes}
        \item[1] S for scalar, V for vector, M for matrix;colV for column vector;outP for out product.
        \item[2] $[\overline{\ \ } ] gaxpy [\equiv ] = [\overline{\ \ } ]$ is 
$\sum[\cdot] gaxpy [\overline{\ \ } ] = \sum[\overline{\ \ }]$.
        \item[3] $[||| ] gaxpy [| ] = [| ]$ is $\sum[| ] gaxpy [\cdot] = \sum[| ]$.
      \end{tablenotes}
    \end{threeparttable}
\end{table}



\begin{table}[htbp]
    \newcommand{\tabincell}[2]{\begin{tabular}{@{}#1@{}}#2\end
    {tabular}}
    \centering
    \begin{threeparttable}
    \caption{$\boldsymbol A_{ik} \boldsymbol X_{kj} = \boldsymbol B_{ij}$ }
    \label{tab:label}
    \begin{tabular}{cccc}
        \toprule
        Order & InnerLoop & MiddleLoop & OuterLoop \\
        \midrule
        ijk & $(rowV, colV) = S$ & $(rowV, [colV]) = rowV$ & collection \\
        jik & $(rowV, colV) = S$ & $([rowV], colV) = rowV$ & collection \\

        ikj & $(S, colV) = colV$ & $(rowV, [colV]) = \sum colV$ & collection \\
        jki& $(colV, S) = colV$ & $([colV], colV) = \sum colV$ & collection \\

        kij& $(S, rowV) = rowV$ & $(colV, rowV) = [rowV] $ & collection and $\sum [rowV] $ \\
        kji& $(colV, S) = colV$ & $(colV, rowV) = [colV] $ & collection and $\sum [colV] $ \\
        \bottomrule
    \end{tabular}
    \begin{tablenotes}
        \item[1] $\sum$ comes with k.
      \end{tablenotes}
    \end{threeparttable}
\end{table}




%[\equiv ]


\subsubsection{Transposition}


Defination: $a_{ij}^T = a_{ji}$

\begin{proposition}
    $(\boldsymbol {AB})^T = \boldsymbol B ^T \boldsymbol A^T$

    Proof: $L = (a_{ik}b_{kj})^T = c_{ij}^T = c_{ji} = b_{jk}a_{ki} = R \ \Box$
\end{proposition}

\begin{proposition}
    We take a look a the product with reflect $T : \boldsymbol x \rightarrow  \boldsymbol{T} \cdot \boldsymbol{x}$.
    $(\boldsymbol {Tx})^T \boldsymbol {Ty} = \boldsymbol x^T(\boldsymbol T^T \boldsymbol T) \boldsymbol y = [(\boldsymbol T \boldsymbol T^T) \boldsymbol x]^T \boldsymbol y$. $0 \leqslant \|\boldsymbol T \boldsymbol T^T\| < 1$, $\boldsymbol T$ is a contractive mapping.
\end{proposition}





\subsection{operation}



\subsubsection{dot procuct AX = B}

Focus on each element of B.

\paragraph{vector vector}

for vector, $\boldsymbol x  . * \boldsymbol y  = \boldsymbol x^T \boldsymbol y$, 






\paragraph{matrix matrix}

for matrix, this is the definition of the multiplication of the matrix, $\boldsymbol A_{mn}  . * \boldsymbol B_{mn}  = [a_{ij} \cdot b_{ij} ]_{mn}$



\subsubsection{outer procuct AX = B}

Focus on each element of X, with X is seperated as row by row.

\paragraph{vector vector}

$\boldsymbol x \boldsymbol y^T := [x_i]_{m1} \cdot  [y_j]_{1n} =[x_i y_j]_{mn}   $

In row view, we have $i \rightarrow : \boldsymbol A _{i:} = x_i \cdot \boldsymbol y^T$, this notation means that for each i, we do the follows. And $ \boldsymbol A _{i:}$ means the ith row of the row seperation of $ \boldsymbol A$

In column view, we have $j \rightarrow : \boldsymbol A _{:j} = \boldsymbol x \cdot  y_j$

\paragraph{matrix matrix}


$ [|||]  outerProduct [\overline{\ \ } ] =  [ \ \ ]$, we just sum each matrix $\boldsymbol M$, where $\boldsymbol M =[|]  outerProduct [\overline{\ \ } ]  $.

$\boldsymbol X_{mk} \cdot \boldsymbol Y_{kn} = k \rightarrow : outerProduct \ of (\boldsymbol X_{:k}, \boldsymbol  Y_{k:})$

We carefully focus on the use of each element of the matrix $\boldsymbol Y$, like  $A_{11}, A_{12},A_{13}, \cdots$, we can see it is true.


\paragraph{question }

\question{power function 001 }
solve $(\boldsymbol x \boldsymbol y^T)^k$. If k=1, easy. if k>1, ans = $ (\boldsymbol y^T \boldsymbol x )^{k-1}  \boldsymbol x \boldsymbol y^T$

\question{power function 002}

solve $(\boldsymbol X \boldsymbol Y^T)^k, X,Y \in \mathbb{R}^{n \times 2}$. Same trick like power function 001.


\subsubsection{saxpi}

\paragraph{scalar scalar}
$y = ax + y$

\paragraph{scalar vector}
$ \boldsymbol y = a \cdot  \boldsymbol  x + \boldsymbol y$


\paragraph{matrix vector}
$ \boldsymbol y = \boldsymbol A \cdot  \boldsymbol  x + \boldsymbol y$



% $ [\overline{\ \ } ] \cdot [|||] =  [\overline{\ \ } ]$


\subparagraph{view row: $ [\overline{\ \ } ] \cdot | =  [\overline{\ \ } ]$}

This is the basic view of the dot product of the matrix.

in view row first, we have:

\begin{algorithm}[H]
    \caption{saxpyMatrixVectorRowAlgo1}\label{algo:saxpyMatrixVectorRowAlgo1}
    \SetAlgoLined
    \KwIn{ $\boldsymbol A_{mn},\boldsymbol x,\boldsymbol y $}
    \KwOut{ $\boldsymbol y $}
    Initialization:$i=0,j=0$\;
    \For{$i \leftarrow 0$ \KwTo $m-1$}{
        \For{$j \leftarrow 0$ \KwTo $n-1$}{
            $ y_i \leftarrow A_{ij}x_j + y_i $
        }
    }
    \KwRet $\boldsymbol y$\;
\end{algorithm}


We seperate $  \boldsymbol A $ as row, $  \boldsymbol A_{mn} = [\boldsymbol r_{i}^T,...] ^T$, the j range can be shinked, the algorithm is as follows. This means that, we operate each row at a time, and think each row is one whole object.

\begin{algorithm}[H]
    \caption{saxpyMatrixVectorRowAlgo2}\label{algo:saxpyMatrixVectorRowAlgo2}
    \SetAlgoLined
    \KwIn{ $\boldsymbol A_{mn}= [\boldsymbol r_{i}^T,...] ^T,\boldsymbol x,\boldsymbol y $}
    \KwOut{ $\boldsymbol y $}
    Initialization:$i=0,j=0$\;
    \For{$i \leftarrow 0$ \KwTo $m-1$}{
        $ y_i \leftarrow \boldsymbol r_{i}^T \cdot \boldsymbol x + y_i $
    }
    \KwRet $\boldsymbol y$\;
\end{algorithm}




\subparagraph{view column: $ [|||]  outerProduct [\overline{\ \ } ] =  [ \ \ ]$}



$\boldsymbol A_{mn} \boldsymbol x = \boldsymbol y $, we seperate A column by column, x row by row, use outer product , focus on the use of x.


in column view, we add each column of $\boldsymbol A$ to the same output column to get the new $\boldsymbol y$, and the weight of each column comes from each row of  $\boldsymbol x$

\begin{algorithm}[H]
    \caption{saxpyMatrixVectorColumnAlgo1}\label{algo:saxpyMatrixVectorColumnAlgo1}
    \SetAlgoLined
    \KwIn{ $\boldsymbol A_{mn},\boldsymbol x,\boldsymbol y $}
    \KwOut{ $\boldsymbol y $}
    Initialization:$i=0,j=0$\;
    \For{$j \leftarrow 0$ \KwTo $n-1$}{
        \For{$i \leftarrow 0$ \KwTo $m-1$}{
            $ y_i \leftarrow A_{ij}x_j + y_i $
        }
    }
    \KwRet $\boldsymbol y$\;
\end{algorithm}

Also with column seperation of $\boldsymbol A_{mn} = [\boldsymbol c_i, ...]$, we have the vector view algorithm:
\begin{algorithm}[H]
    \caption{saxpyMatrixVectorColumnAlgo2}\label{algo:saxpyMatrixVectorColumnAlgo2}
    \SetAlgoLined
    \KwIn{ $\boldsymbol A_{mn}= [\boldsymbol c_i, ...],\boldsymbol x,\boldsymbol y $}
    \KwOut{ $\boldsymbol y $}
    Initialization:$i=0,j=0$\;
    \For{$j \leftarrow 0$ \KwTo $n-1$}{
        $\boldsymbol y \leftarrow \boldsymbol c_i \cdot  x_j + \boldsymbol y $
    }
    \KwRet $\boldsymbol y$\;
\end{algorithm}





\subsection{solve equation}

$\boldsymbol A_{mn} \boldsymbol x  =  \boldsymbol y$
求解方法,如消元法、迭代法等。

\subsubsection{elimination 消元法}


\paragraph{Gaussian Elimination}

基础步骤的$O(n)$的,但是最终组合起来就是$O(n^3)$的。


利用初等变换化(同解变换)为“阶梯形(或称上三角形)”,从下往上回代。

阶梯型:1)0行在下方;2)每行首个非0元的列号随行号增大而严格增大。

简化阶梯型:1)阶梯型;2)主元是1;3)主元所在列其他元素是0.



简化阶梯型后,可直接写出一般解,如下方程,其中主变量是$x_1, x_3$, 其余是自由未知量。
\begin{equation}
    \begin{split}
    & \begin{bmatrix}
            1 & -1 & 0\\
            0 & 0 & 1\\
            0 & 0 & 0\\
        \end{bmatrix}
        \cdot
        \begin{bmatrix}
            x_1 \\
            x_2 \\
            x_3\\
        \end{bmatrix}
        =
        \begin{bmatrix}
        2 \\
        -1 \\
        0 \\
        \end{bmatrix} \\
        &Ans: x_1 = x_2 + 2; x_3 = -1\\
    \end{split}
\end{equation}

Augmented matrix



\section{determinant}
行列式,定义、性质、展开、Gramer法则等


\subsection{排列}

\paragraph{偶排列}

如 2431,顺序对有${24,23}$,逆序对有${21,43,41,31}$,逆序数是4,记为$\tau(2431) = 4 $,是偶数则为偶排列。

\begin{lemma}
    对换改变奇偶性,如2431是偶排列,对换4和1后得到的2134是奇排列。

    证明:对换ab,\\
    若ab相邻:偏序函数原来查询(ab),记为$P(a,b)$, 对换后改为$P(b,a)$,反号,而b更后面的元素相关的查询不受影响,因而改变符号;\\
    若ab不相邻:记为$a x_1 \cdots x_t b$,经过t次对换变为$ x_1 \cdots x_ta b$,经过t+1次对换变为$ bx_1 \cdots x_ta $,即改变符号。
    若ab不相邻,还可以这样考虑:对换前后,与a和b有关的查询为$(a,[x_i, b]), (x_i,b)$,对换后即将其中a和b互换,影响的查询共有2t+1个,即改变符号。
    即证。
\end{lemma}




\section{polynomial}
因式分解定理,多项式的根,多元多项式。


\section{operation}
初等变换、代数运算、分块运算、乘法、秩

\section{Transformation}

线性变换、坐标变换、像与核、特征向量、特征子空间、商空间

正交变换
规范变换

酉相似

\subsection{Elementary Transformation}
初等变换。

1) 交换两行: $ \boldsymbol A \xrightarrow {(i,j)} \boldsymbol B$ \\

2) 某行乘以不为0的数: $ \boldsymbol A \xrightarrow {\lambda(i)} \boldsymbol B$ \\

3) 某行乘以不为0的数加到另一行上: $ \boldsymbol A \xrightarrow {\lambda(i) + (j)} \boldsymbol B$

初等矩阵: 单位矩阵执行一系列初等变换得到的矩阵. 

初等变换作用于矩阵$\boldsymbol A$, 等于初等变换作用于单位阵之后得到的初等矩阵$\boldsymbol E$再作用于$\boldsymbol A$. 



\section{Form}

\subsection{Jordan}

Jordan型、根子空间分解、循环子空间、多项式矩阵相抵不变量、特征方阵与相似标准型

\subsection{二次}
配方法构造、对称方阵的相合、相合不变量




\end{document}