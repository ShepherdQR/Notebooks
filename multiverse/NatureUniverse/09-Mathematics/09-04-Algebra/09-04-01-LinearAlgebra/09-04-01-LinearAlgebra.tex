%%============
%%  ** Author: Shepherd Qirong
%%  ** Date: 2023-04-05 10:23:55
%%  ** Github: https://github.com/ShepherdQR
%%  ** LastEditors: Shepherd Qirong
%%  ** LastEditTime: 2023-05-01 18:53:24
%%  ** Copyright (c) 2019--20xx Shepherd Qirong. All rights reserved.
%%============


\documentclass[UTF8]{../../09-Mathematics}
\begin{document}

\title{09-04-01-LinearAlgebra}
\date{Created on 20230405.\\   Last modified on \today.}
\maketitle
\tableofcontents




\chapter{Vector}


\section{Basic Defination}

we define the basic element as following, where $ \boldsymbol e_i $ means $x_i = 1, x_j = 0$ for all $j \neq i$. When we say a vector, it means a column vector.

\begin{equation}
\vec{x}  = \boldsymbol x = [x_1, x_2,\dots]^T
= \begin{bmatrix}
    x_1 \\
    \vdots \\
    x_n
\end{bmatrix}
= \Sigma x_i \boldsymbol{e_i} 
\end{equation}

We define Kronecker sign to simply the description of $\boldsymbol e_i \cdot \boldsymbol e_j $.

\begin{equation}
    \begin{split}
    &\delta _{ij}:=
    \begin{cases}
    &1,\qquad i = j\\
    &0,\qquad i \neq j\\
    \end{cases}\\
    \end{split}
\end{equation}

The set of bases $\{ \boldsymbol e_i  \}  \xrightarrow{apply} \boldsymbol{x} \longrightarrow    \{ x_i \}   $.
%%\stackrel{apply}
%%\xrightarrow[under]{up} 





\section{Operation}




\subsection{Dot Product}

We define in algebra, $ \boldsymbol{x} \cdot \boldsymbol{y} := \sum{x_iy_i \delta _{ij}} = \boldsymbol{x}^T \cdot \boldsymbol{y}$.

Then the defination is restricted to the choose of the coordinate system. 



\subsection{Add}

\begin{equation}
    \begin{split}
    & \boldsymbol x + \boldsymbol y := \sum (x_i + y_i)\boldsymbol e_i\\
    & k \cdot \boldsymbol x := \sum kx_i\boldsymbol e_i\\
\end{split}
\end{equation}

Law $\boldsymbol{x} + \boldsymbol{y} = \boldsymbol{y} + \boldsymbol{x}$,
law $ (\boldsymbol{x} + \boldsymbol{y} )+ \boldsymbol{z} = \boldsymbol{x} +( \boldsymbol{y} + \boldsymbol{z})$ is not obvious in the view of Set Theory.


\subsection{geometry Properties}

\subsubsection{Length and Angle}

\begin{equation}
    \begin{split}
    &\parallel \boldsymbol{x} \parallel := \sqrt{\boldsymbol x \cdot \boldsymbol x}\\
    &\cos {\theta_{x,y}} : = \frac
    {\boldsymbol x \cdot \boldsymbol x}
    {\parallel \boldsymbol{x} \parallel \cdot \parallel \boldsymbol{y} \parallel}\\
\end{split}
\end{equation}


\subsubsection{Distance}

Distance function satisfies the following:
\begin{equation}\label{Defination:Distance_function}
    \begin{split}
    &d(\boldsymbol x, \boldsymbol y) \geqslant 0\\
    &d(\boldsymbol x, \boldsymbol y) = d(\boldsymbol y, \boldsymbol x)\\
    &d(\boldsymbol x, \boldsymbol y) \leqslant d(\boldsymbol x, \boldsymbol z) + d(\boldsymbol z, \boldsymbol y)\\
\end{split}
\end{equation}

\begin{equation}
    \begin{split}
    &d_p = [\sum |x_i-y_i|^p]^{\frac{1}{p}}, 1\leqslant p < \infty\\
    & d_{\infty} = \max_i|x_i-y_i|\\
\end{split}
\end{equation}

\chapter{Space}


\subsection{Linear Space}

向量加法、标量乘法构成的单位环。


\subsection{Metric Space}

\begin{definition}
    The set X with a distance function d, d satisfies\ref{Defination:Distance_function}. Metric Space is noted as $(X,d)$.
\end{definition}

\begin{definition}
    紧集:$(X,d)$中的子集A,A中任意序列都存在一子列$x_n$,$x_n$收敛到A中某点。
\end{definition}

\begin{definition}
    稠密集:$(X,d)$中的子集A,对于X中的任意点x,A中存在点a,使得$d(x,a)< \varepsilon $
\end{definition}

\begin{definition}
    X可分:$(X,d)$中存在一个可数稠密集。
\end{definition}


\subsubsection{Complete Metric Space}

\begin{definition}
    收敛:sequence $\{ x_n \}$ 收敛到c,means that $\lim_{x \to \infty} d(x_n, c) = 0$, noted as $\lim_{x \to \infty} x_n = c $
\end{definition}

\begin{definition}
    Cauchy基本列:$\lim_{m \to \infty,n \to \infty}d(x_m, x_n)=0  $
\end{definition}

\begin{definition}
    完备距离空间:所有Cauchy基本列收敛于一点
\end{definition}

\begin{definition}
  不完备:对于苹果空间,从宇宙开始到宇宙结束的所有苹果序列,收敛到我,则不完备。
\end{definition}

\subsection{Banach Space}

完备、赋范、线性。


\subsection{Inner Product Space}

\begin{equation}\label{}
\begin{aligned}
  &(\alpha x + \beta y ) \cdot z = \alpha x \cdot z + \beta y \cdot z, &   \mbox{线性}\\
  &x \cdot (\alpha y + \beta z) = \bar{ \alpha}    x \cdot y + \bar{ \beta}    x \cdot z,&   \mbox{共轭线性}\\
  &x \cdot y = \bar{ y \cdot x}, &   \mbox{共轭对称}\\
  &x \cdot x \geqslant 0 , &\mbox{正定} \Rightarrow || x || = \sqrt{x \cdot x} \\
  &|x \cdot y| \leqslant || x || \cdot || y ||, &satisfies  Cauchy-Schwarz \\
\end{aligned}
\end{equation}
 

\subsection{Hilbert Space}

完备,内积。

内积$\Rightarrow$范数$\Rightarrow$完备


\begin{proposition}
  $[0,1]$上的复连续函数空间$ C([0, 1])$,定义内积$f \cdot g = \int_{0}^{1} f(t)g(t) \,dt $, proof that $ C([0, 1])$不是Hilbert Space

  \begin{equation}
    \begin{aligned}
    &f_n(t) =
    \begin{cases}
    &1,\qquad 0 \leqslant t \leqslant \frac{1}{2}\\
    &-2n(t- \frac{1}{2}) + 1,\qquad \frac{1}{2} < t \leqslant \frac{1}{2n} + \frac{1}{2}\\
    &0, \qquad \frac{1}{2n} + \frac{1}{2} < t \leqslant 1\\
    \end{cases}\\
    &|| f_n - f_m|| \leqslant (\frac{1}{n} + \frac{1}{m})^{\frac{1}{2}} \rightarrow 0, is \  Cauchy \  Sequence.\\
    & \lim f_n = 
    \begin{cases}
      &1,  0 \leqslant t \leqslant \frac{1}{2}\\
      &0,  \frac{1}{2} < t \leqslant 1\\
    \end{cases}\\
    & \therefore \lim f_n \notin C([0, 1])
    \end{aligned}
  \end{equation}

\end{proposition}


\subsection{Euclid Space}
有序的n元组的全体称为n维Euclid空间,记为$\mathbb R^n$,称$\boldsymbol p=(p_i)_{i=1}^n \in \mathbb R^n$是$\mathbb R^n$的一个点。\\
为便于研究,本论文以$ \mathbb R^3$为背景空间,所涉及的函数默认为可微实值函数。如果实函数$f$的任意阶偏导数存在且连续,则称函数是可微的(或无限可微的,或光滑的,或$C^\infty$的)。\\
由于微分运算是函数的局部运算,限制所讨论函数的定义域在$ \mathbb R^3$中的任意开集,所讨论的结论仍然成立。\\
自然坐标函数:定义在$\mathbb R^n$上的实值函数$x_i: \mathbb R^n \to  \mathbb R$,使得$\boldsymbol p=(p_i)_{i=1}^n = \left( x_i(\boldsymbol p) \right)_{i=1}^n   $\\
切向量:由$\mathbb R^n$ 中的二元组构成,$\boldsymbol v_{\boldsymbol p}=(\boldsymbol p,\boldsymbol v)$,其中$\boldsymbol p$是作用点,$\boldsymbol v$是向量部分\\
切空间$T_p  \mathbb R^n$: 作用点$\boldsymbol p \in \mathbb R^n$的所有切向量的集合。利用向量加法与数量乘法使某点的切空间称为向量空间,与背景空间存在非平凡同构。\\
向量场$\boldsymbol V$:作用于空间点的向量函数,$\boldsymbol V(\boldsymbol p)\in T_p  \mathbb R^n $\\
逐点化原理:$(\boldsymbol V+\boldsymbol W)(\boldsymbol p)=\boldsymbol V(\boldsymbol p)+\boldsymbol W(\boldsymbol p),\ (f \boldsymbol V)(\boldsymbol p)= f(\boldsymbol p)\boldsymbol V (\boldsymbol p)$\\
自然标架场:定义$\boldsymbol U_i=(\delta _j^i)_{j=1}^n$,按Einstein求和约定,有$\boldsymbol V(\boldsymbol p)=v^i(\boldsymbol p)\boldsymbol U_i(\boldsymbol p)$,称$v^i$为场的Euclid坐标函数,其中Kronecker $\delta$函数定义为:
\begin{equation}
\label{Kronecker_delta}
\delta _i^j=\left\{ 
    \begin{aligned}
    1,\  & i =j\\
    0,\  & i \neq j\\
    \end{aligned}
     \right.
\end{equation}






\chapter{linear algebra}

\section{linear equation}

Normally, we consider vector space over the fields of real or complex numbers.

\subsection{Ax = B}

\subsubsection{Defination}

linear equation in n variables.$\sum_{i = 1}^{i=n}a_ix^i = b$, which can be written as $\boldsymbol a^T \boldsymbol x = b$. We collect m equations and write like this:

\begin{equation}
  \begin{bmatrix}
    \boldsymbol a^T_1 \\
    \vdots \\
    \boldsymbol a^T_m \\
  \end{bmatrix}
  \cdot
  \begin{bmatrix}
    \boldsymbol x
  \end{bmatrix}
  =
  \begin{bmatrix}
     b_1 \\
    \vdots \\
     b_m \\
  \end{bmatrix}
\end{equation}

Noticed that $x_1$ is only applied toe the first column of the left matrix, we can say that $ \boldsymbol x$ is one point, or a specific composition, of the space spanned by the column vector of the matrix. Then it is easy to see that this equation has the solution, only if the vector $ \boldsymbol b$ is in the space spanned by the column vector of the matrix.


Or we can write like this:
\begin{equation}
  \begin{bmatrix}
     a_{11} & \cdots & a_{1n}\\
     & \ddots & \\
     a_{m1} & \cdots & a_{mn} \\
  \end{bmatrix}
  \cdot
  \begin{bmatrix}
    x_1 \\
    \vdots \\
    x_n \\
  \end{bmatrix}
  =
  \begin{bmatrix}
    b_1 \\
    \vdots \\
    b_m \\
  \end{bmatrix}
\end{equation}

The equation $\boldsymbol A \boldsymbol x = \boldsymbol b$ has solution, means y可由A的列向量线性表出。

If $\boldsymbol b = \boldsymbol 0$, called homogeneous linear equations, homogeneous because 所有非0项是1次的。if $\boldsymbol b \neq \boldsymbol 0$, it is inhomogeneous. 显然0向量(zero solution, or trivial soltution) 是一个解. A的列向量正交,只有零解;若A的列向量线性相关,有多解,即可按多种方式回到原点。

\subsubsection{Number of solution}

构造增广矩阵$[\boldsymbol A,\boldsymbol b ]$后,初等行变换化为阶梯型,如\ref{fig:number_of_solution}所示,解的个数讨论。

\begin{figure}[h]
  \centering
    \begin{tikzpicture}
      \draw (0,0) rectangle (6,8);
      \draw (6,0) rectangle (7,8);
      \draw (0,6)--(2,6)--(4,2)--(6,2);
      \draw (7.5,0)--(8,0);
      \draw (7.5,2)--(8,2);
      \draw (7.8,1) node {r};

      \draw (4,1)--(4,1.5);
      \draw (5,1.3) node {d};
    \end{tikzpicture}
  \caption{【number of solution】}\label{fig:number_of_solution}
\end{figure}

$r \neq 0$, no solution;

$d = 1$, one solution, $tr \boldsymbol A_{mn} = m$,;

$d \neq 1$, $tr \boldsymbol A_{mn} < m$,inifinity solution,最后一行是解的超平面方程, 图中 d 是解的维度,$d =  n -tr \boldsymbol A_{mn}$, 如d为3,有3列独立的,即解空间是三维的。齐次方程组的未知数个数大于方程个数,有无数解。


\subsection{Matrix}


\subsubsection{Defination}

If we have a serious of $\boldsymbol x$, we have a serious of $b$, like this:
\begin{equation}
  A_{mn} \cdot X_{nt} = B_{mt}
  \Longrightarrow
  \begin{bmatrix}
     a_{11} & \cdots & a_{1n}\\
     & \ddots & \\
     a_{m1} & \cdots & a_{mn} \\
  \end{bmatrix}
  \cdot
  \left[
    \begin{array}{c:c:c}
      x_{11} & \cdots & x_{1t}\\
     \vdots&  & \vdots \\
     x_{n1} & \cdots & x_{nt} \\
    \end{array}
 \right] 
  =
 \left[
    \begin{array}{c:c:c}
      b_{11} & \cdots & b_{1t}\\
      \vdots& \cdots & \vdots \\
      b_{m1} & \cdots & b_{mt} \\
    \end{array}
 \right] 
\end{equation}

The normal definition of the product of two matrix is as above. 

\subsubsection{Transposition}
Defination: $a_{ij}^T = a_{ji}$

\begin{proposition}
    $(\boldsymbol {AB})^T = \boldsymbol B ^T \boldsymbol A^T$

    Proof: $L = (a_{ik}b_{kj})^T = c_{ij}^T = c_{ji} = b_{jk}a_{ki} = R \ \Box$
\end{proposition}

\begin{proposition}
    We take a look a the product with reflect $T : \boldsymbol x \rightarrow  \boldsymbol{T} \cdot \boldsymbol{x}$.
    $(\boldsymbol {Tx})^T \boldsymbol {Ty} = \boldsymbol x^T(\boldsymbol T^T \boldsymbol T) \boldsymbol y = [(\boldsymbol T \boldsymbol T^T) \boldsymbol x]^T \boldsymbol y$. $0 \leqslant \|\boldsymbol T \boldsymbol T^T\| < 1$, $\boldsymbol T$ is a contractive mapping.
\end{proposition}


\subsection{solve equation}
求解方法,如消元法、迭代法等。

\subsubsection{消元法}
利用初等变换化为“阶梯形(或称上三角形)”,从下往上回代。


Augmented matrix

\section{determinant}
行列式,定义、性质、展开、Gramer法则等

\section{polynomial}
因式分解定理,多项式的根,多元多项式。


\section{operation}
初等变换、代数运算、分块运算、乘法、秩

\section{Transformation}

线性变换、坐标变换、像与核、特征向量、特征子空间、商空间

正交变换
规范变换

酉相似

\subsection{Elementary Transformation}
初等变换。

1) 交换两行: $ \boldsymbol A \xrightarrow {(i,j)} \boldsymbol B$ \\

2) 某行乘以不为0的数: $ \boldsymbol A \xrightarrow {\lambda(i)} \boldsymbol B$ \\

3) 某行乘以不为0的数加到另一行上: $ \boldsymbol A \xrightarrow {\lambda(i) + (j)} \boldsymbol B$

初等矩阵: 单位矩阵执行一系列初等变换得到的矩阵. 

初等变换作用于矩阵$\boldsymbol A$, 等于初等变换作用于单位阵之后得到的初等矩阵$\boldsymbol E$再作用于$\boldsymbol A$. 



\section{Form}

\subsection{Jordan}

Jordan型、根子空间分解、循环子空间、多项式矩阵相抵不变量、特征方阵与相似标准型

\subsection{二次}
配方法构造、对称方阵的相合、相合不变量




\end{document}