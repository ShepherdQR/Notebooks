%%============
%%  ** Author: Qirong ZHANG
%%  ** Date: 2023-04-05 10:23:55
%%  ** Github: https://github.com/ShepherdQR
%%  ** LastEditors: Qirong ZHANG
%%  ** LastEditTime: 2024-12-09 20:50:15
%%  ** Copyright (c) 2019 Qirong ZHANG. All rights reserved.
%%  ** SPDX-License-Identifier: LGPL-3.0-or-later.
%%============


\documentclass[UTF8]{../../09-Mathematics}
\bibliographystyle{../../GBT7714-2005NLang}
\begin{document}



\title{09-04-05-LinearAlgebra}
\date{Created on 20230405.\\   Last modified on \today.}
\maketitle
\tableofcontents


\chapter{Overall}



2条主线:linear space, linear mapping








\chapter{Vector}


\section{Basic Defination}

we define the basic element as following, where $ \boldsymbol e_i $ means $x_i = 1, x_j = 0$ for all $j \neq i$. When we say a vector, it means a column vector.

\begin{equation}
\vec{x}  = \boldsymbol x = [x_1, x_2,\dots]^T
= \begin{bmatrix}
    x_1 \\
    \vdots \\
    x_n
\end{bmatrix}
= \Sigma x_i \boldsymbol{e_i} 
\end{equation}

We define Kronecker sign to simply the description of $\boldsymbol e_i \cdot \boldsymbol e_j $.

\begin{equation}
    \begin{split}
    &\delta _{ij}:=
    \begin{cases}
    &1,\qquad i = j\\
    &0,\qquad i \neq j\\
    \end{cases}\\
    \end{split}
\end{equation}

The set of bases $\{ \boldsymbol e_i  \}  \xrightarrow{apply} \boldsymbol{x} \longrightarrow    \{ x_i \}   $.
%%\stackrel{apply}
%%\xrightarrow[under]{up} 





\section{Operation}




\subsection{Dot Product}

We define in algebra, $ \boldsymbol{x} \cdot \boldsymbol{y} := \sum{x_iy_i \delta _{ij}} = \boldsymbol{x}^T \cdot \boldsymbol{y}$.

Then the defination is restricted to the choose of the coordinate system. 


\subsection{Cross Product}

$a, b \in \mathbb F^m, a \wedge b = c \in \mathbb F^n $, if m = n, we have m = 0, 1, 3, 7. Therefore, we define cross product in 3d.

\begin{equation}
  u \times  v = 
  \begin{vmatrix}
     i & j & k\\
     u_1 & u_2 & u_3\\
     v_1 & v_2 & v_3\\
  \end{vmatrix}
\end{equation}


\begin{proposition}
  外积对于u、v双线性。从定义易知。
\end{proposition}

\begin{proposition}
  $(a \times b) \times c =  (c \cdot a) b -  (b \cdot c) a   $

  证明:
  \begin{equation}
    \begin{vmatrix}
       i & j & k\\
       23 & 31 & 12\\
       1 & 2 & 3\\
    \end{vmatrix}
  \end{equation}
\end{proposition}
例如对i分量,有$31 \cdot 3-12\cdot2$,形式上ijk一样,因而证明i即可。展开后,按正负号分类,we have $(313+212) - (133+122)$, 两部分都加上111即得。b和-a的线性组合。


\begin{proposition}

  混合积$(u, v, w) = (u \times  v ) \cdot w$,其具有轮换对称性。

  证明:
  for $\cdot w$, we have $23 \cdot 1 + 31 \cdot 2 + 12 \cdot 3$

  $231-321, 312-132, 123-213$

  对$\cdot v$,即中间元素按1,2,3顺序组合,易有wu;同样对$\cdot u$,易有vw。即证。
  

  另外,从展开后的分量对应上,易有
  \begin{equation}
    (u, v, w)  =
    \begin{vmatrix}
      u_1 & u_2 & u_3\\
      v_1 & v_2 & v_3\\
      w_1 & w_2 & w_3\\
    \end{vmatrix}
  \end{equation}

\end{proposition}



\subsection{Add}

\begin{equation}
    \begin{split}
    & \boldsymbol x + \boldsymbol y := \sum (x_i + y_i)\boldsymbol e_i\\
    & k \cdot \boldsymbol x := \sum kx_i\boldsymbol e_i\\
\end{split}
\end{equation}

Law $\boldsymbol{x} + \boldsymbol{y} = \boldsymbol{y} + \boldsymbol{x}$,
law $ (\boldsymbol{x} + \boldsymbol{y} )+ \boldsymbol{z} = \boldsymbol{x} +( \boldsymbol{y} + \boldsymbol{z})$ is not obvious in the view of Set Theory.


\subsection{geometry Properties}

\subsubsection{Length and Angle}

\begin{equation}
    \begin{split}
    &\parallel \boldsymbol{x} \parallel := \sqrt{\boldsymbol x \cdot \boldsymbol x}\\
    &\cos {\theta_{x,y}} : = \frac
    {\boldsymbol x \cdot \boldsymbol x}
    {\parallel \boldsymbol{x} \parallel \cdot \parallel \boldsymbol{y} \parallel}\\
\end{split}
\end{equation}


\subsubsection{Distance}

Distance function is a metric.


\begin{equation}
    \begin{split}
    &d_p = [\sum |x_i-y_i|^p]^{\frac{1}{p}}, 1\leqslant p < \infty\\
    & d_{\infty} = \max_i|x_i-y_i|\\
\end{split}
\end{equation}


\chapter{Matrix Theory} %% empty here



\chapter{determinant}

行列式论
行列式,定义、性质、展开、Gramer法则等








\chapter{Matrix Analysis}
《矩阵理论-陈大新》

\section{矩阵序列}
\section{矩阵幂级数}


\section{矩阵函数}

\subsection{定义}
\subsection{$e^At$}
\subsection{计算}




\chapter{Generalized Inverse}


\section{单边逆}

\section{Moore-Penrose Pseudoniverse}

$$
\begin{aligned}
    \boldsymbol A \boldsymbol G \boldsymbol A &= \boldsymbol A\\
    \boldsymbol G \boldsymbol A \boldsymbol G &= \boldsymbol G\\
    (\boldsymbol G \boldsymbol A )^T &= \boldsymbol G \boldsymbol A\\
    (\boldsymbol A \boldsymbol G )^T &= \boldsymbol A \boldsymbol G\\
\end{aligned}
$$


\begin{lemma}
    $\boldsymbol G$ is unique.
    
    Prove: suppose $\boldsymbol G_1 \neq \boldsymbol G_2$,
    $$
    \begin{aligned}
        \boldsymbol G_2 &= \boldsymbol G_2 \underline{\boldsymbol A}  \boldsymbol G_2\\
        &= \boldsymbol G_2 \underline{\boldsymbol A \boldsymbol G_1} \cdot \underline{\boldsymbol A \boldsymbol G_2} = \boldsymbol G_2 [(\boldsymbol A \boldsymbol G_1)^T \cdot (\boldsymbol A \boldsymbol G_2)^T] = \boldsymbol G_2(\boldsymbol A \boldsymbol G_2\boldsymbol A \boldsymbol G_1)^T= \boldsymbol G_2(\boldsymbol A  \boldsymbol G_1)^T\\
        &= \boldsymbol G_2 \underline{\boldsymbol A}  \boldsymbol G_1\\
        &=\underline{\boldsymbol G_2 \boldsymbol A} \cdot \underline{\boldsymbol G_1 \boldsymbol A } \boldsymbol G_1 =\boldsymbol G_1 \boldsymbol A \cdot \boldsymbol G_2 \boldsymbol A  \cdot \boldsymbol G_1=\boldsymbol G_1 \boldsymbol A \boldsymbol G_1\\
        &= \boldsymbol G_1 \qed
    \end{aligned}
    $$

\end{lemma}



\subsection{Properties}

\subsubsection{solution of Ax=y}
$\boldsymbol A \boldsymbol x = \boldsymbol y$, where$\boldsymbol A_{mn},m>n$


\begin{algorithm}[H]
    \caption{Algorithm LinearEquation:MP}\label{algo:LinearEquation:MP}
    \SetAlgoLined
    \KwIn{ $\boldsymbol A \boldsymbol x = \boldsymbol y$, where $\boldsymbol A_{mn},m>n$, and $ \boldsymbol y$ are unknown.}
    \KwOut{$\boldsymbol x$}
    $\boldsymbol A (\boldsymbol A^T \boldsymbol A^{-T}) \boldsymbol x = \boldsymbol y$\;
    let $\boldsymbol C = \boldsymbol A \boldsymbol A^T$, $\boldsymbol I = \boldsymbol A^{-T} \boldsymbol x$, $\boldsymbol C \boldsymbol I = \boldsymbol y$\;
    solve $\boldsymbol I$\;
    $\boldsymbol x = $\;
    \KwRet $\boldsymbol x = \boldsymbol A^T \boldsymbol I$\;
\end{algorithm}

The algorithm meas $\boldsymbol x = (\boldsymbol A \boldsymbol A^T \boldsymbol A^{-T})^{-1} \boldsymbol y = [(\boldsymbol A \boldsymbol A^T )\boldsymbol A^{-T}]^{-1} \boldsymbol y = \boldsymbol A^T (\boldsymbol A \boldsymbol A^T )^{-1} \boldsymbol y$.

$\boldsymbol A^T (\boldsymbol A \boldsymbol A^T )^{-1}$, and $(\boldsymbol A^T \boldsymbol A )^{-1}\boldsymbol A^T$ are Moore-Penrose Pseudoniverse


\subsection{相容方程的解}
\subsection{反射广义逆}
\subsection{最小范数解}
\subsection{最小二乘解}






\chapter{多线性代数}

\chapter{向量代数、因子代数、代数不变量论}


\chapter{线性不等式}

\chapter{线性代数的应用}




\chapter{参考文献说明}
《矩阵理论-陈大新》\upcite{5ch001shepherdQR2020paper}:好的观点的来源。


\bibliography{reference1}%, reference2

\end{document}
