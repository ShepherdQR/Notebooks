%%============
%%  ** Author: Qirong ZHANG
%%  ** Date: 2024-12-01 22:57:46
%%  ** Github: https://github.com/ShepherdQR
%%  ** LastEditors: Qirong ZHANG
%%  ** LastEditTime: 2025-04-26 22:20:08
%%  ** Copyright (c) 2019 Qirong ZHANG. All rights reserved.
%%  ** SPDX-License-Identifier: LGPL-3.0-or-later.
%%============



\documentclass[UTF8]{../../09-Mathematics}
\bibliographystyle{../../GBT7714-2005NLang}
\begin{document}



\title{09-04-06-01-TriconnectedComponents}
\date{Created on 20250425.\\   Last modified on \today.}
\maketitle
\tableofcontents


\chapter{Defination}
 
 

\section{Notes}

\subsection{basic}


$\mathscr{E} (S) := \{ e =(u,v), u\in S, or\ v \in S\}, S \in V$.

path from v to w is noted as $p:v \stackrel{*}{\Rightarrow} w $.
 
Tree T is directed, rooted. 
An edge within T is noted as $p:v \rightarrow w $, v is the father of w, w is the son of v.
The path within T is noted as $p:v \stackrel{*}{\rightarrow} w $, v is the ancestor of w, w is the descendant of v. $D(v)$ is the set of all descendants of v.

frond edge is noted as $w --> v$, means $v \stackrel{*}{\rightarrow} w $.

Palm tree, a tree with some fronds.


\subsection{structure}

separation point, or articulation point: for a, there exits 2 distinct vertices v,w, and a is on every path  $p:v \stackrel{*}{\rightarrow} w $.

biconnected multifraph: no separation point. Which means, for each triple of distinct vertices v, w, a, there is a path $p:v \stackrel{*}{\rightarrow} w $ that a is not on the path.

biconnected components: use G to generate a set subgraph Gi, 4 properties: \\
(1) Gi is biconnected;\\
(2) No Gi is a prope subgraph of a biconnected subgraph of G   \\ %%todo  
(3) all vertices of {Gi}, non-separation point occurs exactly once, separation point occurs more than onec.
(4) every two Gi and Gj, contains at most one common vertex, which (if any) is a separation point.

biconnected components is unique. %%todo 



\chapter{群的推广}
\chapter{群论的应用}






\chapter{参考文献说明}
《矩阵理论-陈大新》\upcite{5ch001shepherdQR2020paper}:好的观点的来源。


\bibliography{reference1}%, reference2

\end{document}