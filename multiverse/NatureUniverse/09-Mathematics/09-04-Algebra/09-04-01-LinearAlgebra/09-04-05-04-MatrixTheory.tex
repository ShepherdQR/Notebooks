%%============
%%  ** Author: Qirong ZHANG
%%  ** Date: 2023-04-05 10:23:55
%%  ** Github: https://github.com/ShepherdQR
%%  ** LastEditors: Qirong ZHANG
%%  ** LastEditTime: 2024-12-21 12:29:34
%%  ** Copyright (c) 2019 Qirong ZHANG. All rights reserved.
%%  ** SPDX-License-Identifier: LGPL-3.0-or-later.
%%============


\documentclass[UTF8]{../../09-Mathematics}
\bibliographystyle{../../GBT7714-2005NLang}
\begin{document}



\title{09-04-05-04-MatrixTheory}
\date{Created on 20241206.\\   Last modified on \today.}
\maketitle
\tableofcontents


% \begin{lstlisting}

% \end{lstlisting}



\chapter{Overall}

Matrix Theory



Augmented matrix



\chapter{Matrix Space}

\subsection{Matrix}


\subsubsection{Defination}

If we have a serious of $\boldsymbol x$, we have a serious of $b$, like this:
\begin{equation}
  A_{mn} \cdot X_{nt} = B_{mt}
  \Longrightarrow
  \begin{bmatrix}
     a_{11} & \cdots & a_{1n}\\
     & \ddots & \\
     a_{m1} & \cdots & a_{mn} \\
  \end{bmatrix}
  \cdot
  \left[
    \begin{array}{c:c:c}
      x_{11} & \cdots & x_{1t}\\
     \vdots&  & \vdots \\
     x_{n1} & \cdots & x_{nt} \\
    \end{array}
 \right] 
  =
 \left[
    \begin{array}{c:c:c}
      b_{11} & \cdots & b_{1t}\\
      \vdots& \cdots & \vdots \\
      b_{m1} & \cdots & b_{mt} \\
    \end{array}
 \right] 
\end{equation}

The normal definition of the product of two matrix is as above. 

\subsubsection{Complex Matrices}

\paragraph{Conjugate Transposition}

for matrix $ \boldsymbol  C \in \mathbb{C}^{m \times n}$, we mark the Conjugate Transposition as $C^H$,
where $c^T_{ji} = \overline{c}_{ij}$

in representation, $ \boldsymbol  C = \boldsymbol A + i \boldsymbol B$. Usually 4 real matrix multiplications are needed to calculate $(C+iD)(E+ i F)$, actually 3  multiplications  are enough. $(C+iD)(E+ i F) =  (C+D)(E-F)+CF-DE + i(DE+CF)$



\subsubsection{Multiplication}


There are 6 views sorting with the loop order, we fully understand that. for example, we can think the order jki(j is the outer, i is the inner) as follows
\begin{equation}
    \begin{split}
    &i: | \ \ \cdot = | \\
    &k: [\ | \ ]  \ \ | =  \sum|  \\
    &j:  [\ | \ ]  \ \  [\ | \ ]=  [\ | \ ]\\
\end{split}
\end{equation}


We collect the 6 vews into one table as fallows.

\begin{table}[htbp]
    \newcommand{\tabincell}[2]{\begin{tabular}{@{}#1@{}}#2\end
    {tabular}}
    \centering
    \begin{threeparttable}
    \caption{$\boldsymbol A_{ik} \boldsymbol X_{kj} = \boldsymbol B_{ij}$ }
    %\label{tab:label}
    \begin{tabular}{cccccc}
        \toprule
        Order & innerLoop & MiddleLoop & dataAccess & view & comment\\
        \midrule
        ijk & S-S:dot & rowV-M 
        & $\boldsymbol A_{\alpha:},[\boldsymbol X_{:\beta}], \boldsymbol B_{\alpha:} $ 
        & $ [\overline{\ \ } ] \cdot [|||] =  [\overline{\ \ } ]$
        & dot view $\rightharpoonup \downharpoonright$ \\

        jik & S-S:dot & M-columnV
        & $[\boldsymbol A_{\alpha:}],\boldsymbol X_{:\beta}, \boldsymbol B_{:\beta} $ 
        & $ [\equiv ] \cdot [|]=  [|]$
        & dot view $ \downharpoonright \rightharpoonup$\\

        ikj & S-rowV:saxpy & rowV-M:gaxpy
        & $\boldsymbol A_{\alpha:},[\boldsymbol X_{\beta:}], \boldsymbol B_{\alpha:} $ 
        & $[\overline{\ \ } ] gaxpy [\equiv ] = [\overline{\ \ } ]$
        & useOfA $\rightharpoonup \downharpoonright$\\

        jki & colV-S:saxpy & M-colV:gaxpy
        & $[\boldsymbol A_{:\alpha}],\boldsymbol X_{:\beta}, \boldsymbol B_{:\alpha} $ 
        & $[||| ] gaxpy [| ] = [| ]$
        & useOfB $ \downharpoonright \rightharpoonup$\\

        kij& S-rowV:saxpy & colV-rowV:outP
        & $\boldsymbol A_{:\alpha},\boldsymbol X_{\beta:},\sum \boldsymbol B_{row} $ 
        & $\sum[| ] outProd [\overline{\ \ } ] =\sum[\equiv ]$
        & $ on \  A \downharpoonright outProd \rightharpoonup$\\

        kji& colV-S:saxpy & colV-rowV:outP
        & $\boldsymbol A_{:\alpha},\boldsymbol X_{\beta:},\sum \boldsymbol B_{col} $ 
        & $\sum[| ] outProd [\overline{\ \ } ] =\sum[||| ]$
        & $on \  X  \downharpoonright outProd \rightharpoonup$\\
        \bottomrule
    \end{tabular}
    \begin{tablenotes}
        \item[1] S for scalar, V for vector, M for matrix;colV for column vector;outP for out product.
        \item[2] $[\overline{\ \ } ] gaxpy [\equiv ] = [\overline{\ \ } ]$ is 
$\sum[\cdot] gaxpy [\overline{\ \ } ] = \sum[\overline{\ \ }]$.
        \item[3] $[||| ] gaxpy [| ] = [| ]$ is $\sum[| ] gaxpy [\cdot] = \sum[| ]$.
      \end{tablenotes}
    \end{threeparttable}
\end{table}



\begin{table}[htbp]
    \newcommand{\tabincell}[2]{\begin{tabular}{@{}#1@{}}#2\end
    {tabular}}
    \centering
    \begin{threeparttable}
    \caption{$\boldsymbol A_{ik} \boldsymbol X_{kj} = \boldsymbol B_{ij}$ }
    %\label{tab:label}
    \begin{tabular}{cccc}
        \toprule
        Order & InnerLoop & MiddleLoop & OuterLoop \\
        \midrule
        ijk & $(rowV, colV) = S$ & $(rowV, [colV]) = rowV$ & collection \\
        jik & $(rowV, colV) = S$ & $([rowV], colV) = rowV$ & collection \\

        ikj & $(S, colV) = colV$ & $(rowV, [colV]) = \sum colV$ & collection \\
        jki& $(colV, S) = colV$ & $([colV], colV) = \sum colV$ & collection \\

        kij& $(S, rowV) = rowV$ & $(colV, rowV) = [rowV] $ & collection and $\sum [rowV] $ \\
        kji& $(colV, S) = colV$ & $(colV, rowV) = [colV] $ & collection and $\sum [colV] $ \\
        \bottomrule
    \end{tabular}
    \begin{tablenotes}
        \item[1] $\sum$ comes with k.
      \end{tablenotes}
    \end{threeparttable}
\end{table}




%[\equiv ]


\subsubsection{Transposition}


Defination: $a_{ij}^T = a_{ji}$

\begin{proposition}
    $(\boldsymbol {AB})^T = \boldsymbol B ^T \boldsymbol A^T$

    Proof: $L = (a_{ik}b_{kj})^T = c_{ij}^T = c_{ji} = b_{jk}a_{ki} = R \ \Box$
\end{proposition}

\begin{proposition}
    We take a look a the product with reflect $T : \boldsymbol x \rightarrow  \boldsymbol{T} \cdot \boldsymbol{x}$.
    $(\boldsymbol {Tx})^T \boldsymbol {Ty} = \boldsymbol x^T(\boldsymbol T^T \boldsymbol T) \boldsymbol y = [(\boldsymbol T \boldsymbol T^T) \boldsymbol x]^T \boldsymbol y$. $0 \leqslant \|\boldsymbol T \boldsymbol T^T\| < 1$, $\boldsymbol T$ is a contractive mapping.
\end{proposition}





\subsection{operation}


\subsubsection{procuct }

$ \boldsymbol A \boldsymbol x =\boldsymbol y$



\subsubsection{dot procuct AX = B}

Focus on each element of B.

\paragraph{vector vector}

for vector, $\boldsymbol x  . * \boldsymbol y  = \boldsymbol x^T \boldsymbol y$, 



\paragraph{matrix matrix}

for matrix, this is the definition of the multiplication of the matrix, $\boldsymbol A_{mn}  . * \boldsymbol B_{mn}  = [a_{ij} \cdot b_{ij} ]_{mn}$



\subsubsection{outer procuct AX = B}

Focus on each element of X, with X is seperated as row by row.

\paragraph{vector vector AX = B}

$\boldsymbol x \boldsymbol y^T := [x_i]_{m1} \cdot  [y_j]_{1n} =[x_i y_j]_{mn}   $

In row view, we have $i \rightarrow : \boldsymbol A _{i:} = x_i \cdot \boldsymbol y^T$, this notation means that for each i, we do the follows. And $ \boldsymbol A _{i:}$ means the ith row of the row seperation of $ \boldsymbol A$

In column view, we have $j \rightarrow : \boldsymbol A _{:j} = \boldsymbol x \cdot  y_j$

\paragraph{matrix matrix}


$ [|||]  outerProduct [\overline{\ \ } ] =  [ \ \ ]$, we just sum each matrix $\boldsymbol M$, where $\boldsymbol M =[|]  outerProduct [\overline{\ \ } ]  $.

$\boldsymbol X_{mk} \cdot \boldsymbol Y_{kn} = k \rightarrow : outerProduct \ of (\boldsymbol X_{:k}, \boldsymbol  Y_{k:})$

We carefully focus on the use of each element of the matrix $\boldsymbol Y$, like  $A_{11}, A_{12},A_{13}, \cdots$, we can see it is true.


\paragraph{question }

\question{power function 001 }
solve $(\boldsymbol x \boldsymbol y^T)^k$. If k=1, easy. if k>1, ans = $ (\boldsymbol y^T \boldsymbol x )^{k-1}  \boldsymbol x \boldsymbol y^T$

\question{power function 002}

solve $(\boldsymbol X \boldsymbol Y^T)^k, X,Y \in \mathbb{R}^{n \times 2}$. Same trick like power function 001.


\subsubsection{saxpi}

\paragraph{scalar scalar}
$y = ax + y$

\paragraph{scalar vector}
$ \boldsymbol y = a \cdot  \boldsymbol  x + \boldsymbol y$


\paragraph{matrix vector}
$ \boldsymbol y = \boldsymbol A \cdot  \boldsymbol  x + \boldsymbol y$



% $ [\overline{\ \ } ] \cdot [|||] =  [\overline{\ \ } ]$


\subparagraph{view row: $ [\overline{\ \ } ] \cdot | =  [\overline{\ \ } ]$}

This is the basic view of the dot product of the matrix.

in view row first, we have:

\begin{algorithm}[H]
    \caption{saxpyMatrixVectorRowAlgo1}\label{algo:saxpyMatrixVectorRowAlgo1}
    \SetAlgoLined
    \KwIn{ $\boldsymbol A_{mn},\boldsymbol x,\boldsymbol y $}
    \KwOut{ $\boldsymbol y $}
    Initialization:$i=0,j=0$\;
    \For{$i \leftarrow 0$ \KwTo $m-1$}{
        \For{$j \leftarrow 0$ \KwTo $n-1$}{
            $ y_i \leftarrow A_{ij}x_j + y_i $
        }
    }
    \KwRet $\boldsymbol y$\;
\end{algorithm}


We seperate $  \boldsymbol A $ as row, $  \boldsymbol A_{mn} = [\boldsymbol r_{i}^T,...] ^T$, the j range can be shinked, the algorithm is as follows. This means that, we operate each row at a time, and think each row is one whole object.

\begin{algorithm}[H]
    \caption{saxpyMatrixVectorRowAlgo2}\label{algo:saxpyMatrixVectorRowAlgo2}
    \SetAlgoLined
    \KwIn{ $\boldsymbol A_{mn}= [\boldsymbol r_{i}^T,...] ^T,\boldsymbol x,\boldsymbol y $}
    \KwOut{ $\boldsymbol y $}
    Initialization:$i=0,j=0$\;
    \For{$i \leftarrow 0$ \KwTo $m-1$}{
        $ y_i \leftarrow \boldsymbol r_{i}^T \cdot \boldsymbol x + y_i $
    }
    \KwRet $\boldsymbol y$\;
\end{algorithm}




\subparagraph{view column: $ [|||]  outerProduct [\overline{\ \ } ] =  [ \ \ ]$}



$\boldsymbol A_{mn} \boldsymbol x = \boldsymbol y $, we seperate A column by column, x row by row, use outer product , focus on the use of x.


in column view, we add each column of $\boldsymbol A$ to the same output column to get the new $\boldsymbol y$, and the weight of each column comes from each row of  $\boldsymbol x$

\begin{algorithm}[H]
    \caption{saxpyMatrixVectorColumnAlgo1}\label{algo:saxpyMatrixVectorColumnAlgo1}
    \SetAlgoLined
    \KwIn{ $\boldsymbol A_{mn},\boldsymbol x,\boldsymbol y $}
    \KwOut{ $\boldsymbol y $}
    Initialization:$i=0,j=0$\;
    \For{$j \leftarrow 0$ \KwTo $n-1$}{
        \For{$i \leftarrow 0$ \KwTo $m-1$}{
            $ y_i \leftarrow A_{ij}x_j + y_i $
        }
    }
    \KwRet $\boldsymbol y$\;
\end{algorithm}

Also with column seperation of $\boldsymbol A_{mn} = [\boldsymbol c_i, ...]$, we have the vector view algorithm:

\begin{algorithm}[H]
    \caption{saxpyMatrixVectorColumnAlgo2}\label{algo:saxpyMatrixVectorColumnAlgo2}
    \SetAlgoLined
    \KwIn{ $\boldsymbol A_{mn}= [\boldsymbol c_i, ...],\boldsymbol x,\boldsymbol y $}
    \KwOut{ $\boldsymbol y $}
    Initialization:$i=0,j=0$\;
    \For{$j \leftarrow 0$ \KwTo $n-1$}{
        $\boldsymbol y \leftarrow \boldsymbol c_i \cdot  x_j + \boldsymbol y $
    }
    \KwRet $\boldsymbol y$\;
\end{algorithm}




\subsection{properties}


\subsubsection{geometry properties}

\subsubsection{4 subspace}

% \begin{lstlisting}

% \end{lstlisting}

$R(\boldsymbol A)$: A的列空间\\
$N(\boldsymbol A)$: A的右零空间,即满足$\boldsymbol A \boldsymbol x = \boldsymbol 0$的所有$\boldsymbol x$在的空间\\
$R(\boldsymbol A^T)$: A的行空间\\
$N(\boldsymbol A^T)$: A的左零空间\\

\begin{lemma}
$N(\boldsymbol A) = R(\boldsymbol A^T)^\perp$

proof: $ \forall \boldsymbol x \in N(\boldsymbol A)$, $\boldsymbol \beta \in R(\boldsymbol A^T)$, we have $\boldsymbol x \cdot \boldsymbol \beta = 0$, which means that $\boldsymbol x \in R(\boldsymbol A^T)^\perp$,$\qed$.
\end{lemma}


\newpage
\chapter{Linear System}

Normally, we consider vector space over the fields of real or complex numbers.

\subsection{linear equation Ax = B}

\subsubsection{Defination}

linear equation in n variables.$\sum_{i = 1}^{i=n}a_ix^i = b$, which can be written as $\boldsymbol a^T \boldsymbol x = b$. We collect m equations and write like this:

\begin{equation}
  \begin{bmatrix}
    \boldsymbol a^T_1 \\
    \vdots \\
    \boldsymbol a^T_m \\
  \end{bmatrix}
  \cdot
  \begin{bmatrix}
    \boldsymbol x
  \end{bmatrix}
  =
  \begin{bmatrix}
     b_1 \\
    \vdots \\
     b_m \\
  \end{bmatrix}
\end{equation}

Noticed that $x_1$ is only applied toe the first column of the left matrix, we can say that $ \boldsymbol x$ is one point, or a specific composition, of the space spanned by the column vector of the matrix. Then it is easy to see that this equation has the solution, only if the vector $ \boldsymbol b$ is in the space spanned by the column vector of the matrix.


Or we can write like this:
\begin{equation}
  \begin{bmatrix}
     a_{11} & \cdots & a_{1n}\\
     & \ddots & \\
     a_{m1} & \cdots & a_{mn} \\
  \end{bmatrix}
  \cdot
  \begin{bmatrix}
    x_1 \\
    \vdots \\
    x_n \\
  \end{bmatrix}
  =
  \begin{bmatrix}
    b_1 \\
    \vdots \\
    b_m \\
  \end{bmatrix}
\end{equation}

The equation $\boldsymbol A \boldsymbol x = \boldsymbol b$ has solution, means y可由A的列向量线性表出。

If $\boldsymbol b = \boldsymbol 0$, called homogeneous linear equations, homogeneous because 所有非0项是1次的。if $\boldsymbol b \neq \boldsymbol 0$, it is inhomogeneous. 显然0向量(zero solution, or trivial soltution) 是一个解. A的列向量正交,只有零解;若A的列向量线性相关,有多解,即可按多种方式回到原点。






\subsubsection{Number of solution}

\paragraph{非齐次线性}

n元线性方程组解的个数等解集结构的研究,期待在不求解的情况下有所了解,就需要研究系数矩阵表示的n维向量空间的性质。

构造增广矩阵$[\boldsymbol A,\boldsymbol b ]$后,初等行变换化为阶梯型,如\ref{fig:number_of_solution}所示,解的个数讨论。

\begin{figure}[h]
  \centering
    \begin{tikzpicture}
      \draw (0,0) rectangle (6,8);
      \draw (6,0) rectangle (7,8);
      \draw (0,6)--(2,6)--(4,2)--(6,2);
      \draw (7.5,0)--(8,0);
      \draw (7.5,2)--(8,2);
      \draw (7.8,1) node {R};

      \draw (4,1)--(4,1.5);
      \draw (5,1.3) node {d};
    \end{tikzpicture}
  \caption{【number of solution】}\label{fig:number_of_solution}
\end{figure}

总共有n+1列,下面r行都是0.

(1)$d = 0$,即最后一个个主元在第n+1列,即存在方程0=1,无解, no solution。\\
(2)$d = 1$,即最后一个个主元在第n列,唯一解, one solution, $tr \boldsymbol A_{mn} = m$。\\
(3)$d > 1$,即最后一个个主元在第t列,$t < n$。高度R所在的行号记为r。有无穷个解。解可以这样写出,共R行,即R个主元,每个主元都用所在行的常数项d和n-r个自由元表示出来。

根据主元的构造过程,t的列号一定大于等于r。


当$A_{ii}$都是主元的时候,$d \neq 1$, $tr \boldsymbol A_{mn} < m$,inifinity solution,最后一行是解的超平面方程, 图中 d 是解的维度,$d =  n -tr \boldsymbol A_{mn}$, 如d为3,有3列独立的,即解空间是三维的。齐次方程组的未知数个数大于方程个数,有无数解。



$\det \boldsymbol A = 0$, no solution, or infinite solution.
$\det \boldsymbol A \neq 0$, one solution.




\paragraph{齐次线性}

一定有0解,因而当有非0解时,有无穷个解。n列时,系数矩阵的秩$r < n$。

方程个数$s < n$时,由于$r  \leqslant s < n$,易知有无穷个解。


\subsection{solve equation}

$\boldsymbol A_{mn} \boldsymbol x  =  \boldsymbol y$
求解方法,如消元法、迭代法等。

\subsubsection{elimination 消元法}


\paragraph{Gaussian Elimination}

基础步骤的$O(n)$的,但是最终组合起来就是$O(n^3)$的。


利用初等变换化(同解变换)为“阶梯形(或称上三角形)”,从下往上回代。

阶梯型:1)0行在下方;2)每行首个非0元的列号随行号增大而严格增大。

简化阶梯型:1)阶梯型;2)主元是1;3)主元所在列其他元素是0.



简化阶梯型后,可直接写出一般解,如下方程,其中主变量是$x_1, x_3$, 其余是自由未知量。
\begin{equation}
    \begin{split}
    & \begin{bmatrix}
            1 & -1 & 0\\
            0 & 0 & 1\\
            0 & 0 & 0\\
        \end{bmatrix}
        \cdot
        \begin{bmatrix}
            x_1 \\
            x_2 \\
            x_3\\
        \end{bmatrix}
        =
        \begin{bmatrix}
        2 \\
        -1 \\
        0 \\
        \end{bmatrix} \\
        &Ans: x_1 = x_2 + 2; x_3 = -1\\
    \end{split}
\end{equation}






\subsection{排列}

\paragraph{偶排列}

如 2431,顺序对有${24,23}$,逆序对有${21,43,41,31}$,逆序数是4,记为$\tau(2431) = 4 $,是偶数则为偶排列。

\begin{lemma}
    对换改变奇偶性,如2431是偶排列,对换4和1后得到的2134是奇排列。

    证明:对换ab,\\
    若ab相邻:偏序函数原来查询(ab),记为$P(a,b)$, 对换后改为$P(b,a)$,反号,而b更后面的元素相关的查询不受影响,因而改变符号;\\
    若ab不相邻:记为$a x_1 \cdots x_t b$,经过t次对换变为$ x_1 \cdots x_ta b$,经过t+1次对换变为$ bx_1 \cdots x_ta $,即改变符号。
    若ab不相邻,还可以这样考虑:对换前后,与a和b有关的查询为$(a,[x_i, b]), (x_i,b)$,对换后即将其中a和b互换,影响的查询共有2t+1个,即改变符号。
    即证。
\end{lemma}




\chapter{Eigenvalue problem}

\subsection{Eigenvalue of Linear transformation}
《矩阵理论-陈大新》

\subsection{Eigenvalue of special matrix}
《矩阵理论-陈大新》

\subsection{最小多项式}
《矩阵理论-陈大新》

\subsection{圆盘定理}

《矩阵理论-陈大新》



\chapter{polynomial}
因式分解定理,多项式的根,多元多项式。


\chapter{operation}
代数运算、分块运算、乘法、秩



\chapter{Transformation}

坐标变换、像与核、特征向量、特征子空间、商空间




正交变换
规范变换

酉相似

\subsection{Elementary Transformation}
初等变换。

1) 交换两行: $ \boldsymbol A \xrightarrow {(i,j)} \boldsymbol B$ \\

2) 某行乘以不为0的数: $ \boldsymbol A \xrightarrow {\lambda(i)} \boldsymbol B$ \\

3) 某行乘以不为0的数加到另一行上: $ \boldsymbol A \xrightarrow {\lambda(i) + (j)} \boldsymbol B$

初等矩阵: 单位矩阵执行一系列初等变换得到的矩阵. 

初等变换作用于矩阵$\boldsymbol A$, 等于初等变换作用于单位阵之后得到的初等矩阵$\boldsymbol E$再作用于$\boldsymbol A$. 


\subsection{Linear Transformation}
线性变换



\subsection{Base Transformation}

[Defination: similarity] Transformation Simplification

The motivation is about the base.Changing the bases of a transformation can help simplify the computation. We want to compure $\boldsymbol y = \boldsymbol B \boldsymbol x$, in current base, $\boldsymbol B$ is difficult to compute, we have a simple transformation $\boldsymbol A$, and we want to find a good base transformation $\boldsymbol P$ where we have $\boldsymbol P \boldsymbol y = \boldsymbol A  \boldsymbol P \boldsymbol x $, therefore we have $\boldsymbol B  = \boldsymbol P ^{-1}\boldsymbol A  \boldsymbol P  $. If $ \exists \boldsymbol P$, we  note as $\boldsymbol A \sim \boldsymbol B$.


\begin{lemma}
    If $\boldsymbol A \sim \boldsymbol B$, then $\lambda_A = \lambda_B$.

[Prove] 
We use the defination of $\lambda_A $, for $\boldsymbol x$, we have $\boldsymbol P\boldsymbol B \boldsymbol P ^{-1} \boldsymbol x = \lambda_A \boldsymbol x$, which means $\boldsymbol B  (\boldsymbol P ^{-1} \boldsymbol x) = \lambda_A (\boldsymbol P ^{-1} \boldsymbol x)= \lambda_B (\boldsymbol P ^{-1} \boldsymbol x) \qed$.
\end{lemma}

For the simpest $\boldsymbol A$ is diag.







\chapter{Orthogonal}


\section{Orthogonal matrix}

正交矩阵和酉矩阵

$$
\boldsymbol{A}^T\boldsymbol{A} = \boldsymbol{A}\boldsymbol{A} ^T = \boldsymbol{I}
$$
which means, $\boldsymbol{A}^T = \boldsymbol{A} ^{-1}$

For complex matrix, we use conjugate transpose, and note as Unitary matrix.


\section{Orthogonal Bases}

\subsection{Gram-Schmidt Algorithm:GS正交化}



normalize a given set $\{ \boldsymbol \alpha_i\}$ to $\{\boldsymbol  \beta_i\}$, such that $\boldsymbol  \beta_i \boldsymbol  \beta_j = 0, \forall i,j$, and $span\{ \boldsymbol \alpha_i\} = span\{ \boldsymbol \beta_i\}$

\begin{algorithm}[H]
    \caption{Algorithm Normalize:GS}\label{algo:Normalize:GS}
    \SetAlgoLined
    \KwIn{ $\{\boldsymbol  \alpha_i\}$}
    \KwOut{ a normalized base $\{\boldsymbol  \beta_i\}$}
    $\boldsymbol \beta _1 = \boldsymbol \alpha _1$\;
    $\boldsymbol \beta _2 = \boldsymbol \alpha _2 - k_1 \boldsymbol  \beta _1$\;
    let $\boldsymbol \beta _1 \boldsymbol \beta _2 = 0 \Rightarrow k_1 = \frac{\boldsymbol \beta _1 \boldsymbol \alpha _2}{ \boldsymbol \beta _1 \boldsymbol \beta _1}$\;
    keep doing, we have
    $$\boldsymbol  \beta _k = \boldsymbol \alpha _k-\sum_{i=1}^{k}\frac{\boldsymbol \beta _i \boldsymbol \alpha _k}{\boldsymbol \beta _i \boldsymbol \beta _i} \boldsymbol \beta _i$$
    \KwRet $\{\boldsymbol  \beta_i\}$\;
\end{algorithm}








\subsubsection{description}

In 1907, Erhard Schmidt introduced an orthogonalizaiton algoritm, and he claimed the procedure was essentially the same as a paper by J. P. Gram in 1883.

form an orthogonal sequence${\boldsymbol{q}_n}$ from  a linearly independent sequence${\boldsymbol{x}_n}$ of members from inner-product space by defining ${\boldsymbol{q}_n}$ inductively as:

$$
\boldsymbol{q}_1 = \boldsymbol{x}_1,
\boldsymbol{q}_n = \boldsymbol{x}_n - \sum_{k=1}^{n-1}\frac{<\boldsymbol{q}_k,\boldsymbol{x}_n>}{||\boldsymbol{q}_k||^2}\boldsymbol {q}_k, n\geqslant2.
$$


\subsubsection{proof}

the construction is like this, first we have 
$\boldsymbol{q}_1 = \boldsymbol{x}_1$, then $\boldsymbol{q}_2 = \boldsymbol{x}_2 - k_1 \boldsymbol{q}_1$.

for now, we have 
$span(\boldsymbol{q}_1,\boldsymbol{q}_2) =span(\boldsymbol{x}_1,\boldsymbol{x}_2) $. With constraints $<\boldsymbol{x}_1,\boldsymbol{x}_2>=0$,$k_1 = \frac{<\boldsymbol{q}_1,\boldsymbol{x}_n>}{||\boldsymbol{q}_1||^2}$. Keep doing, like 

$\boldsymbol{q}_3 = \boldsymbol{x}_3 - k_1 \boldsymbol{q}_1- k_2 \boldsymbol{q}_2$. From the construction, we can see it is right.


数学归纳法(Mathematical Induction, MI)


\subsubsection{least squares problems}

\subsubsection{projection problem}

\subsubsection{example}

\subsubsection{exercises}
















\subsection{Householder Reflection}

point $\boldsymbol p$, hyperplane with normal  $\boldsymbol n$, the reflection of  $\boldsymbol p$ about the plane:
$$
\begin{aligned}
    \boldsymbol q &= \boldsymbol p - 2<\boldsymbol p ,\boldsymbol n >\boldsymbol n \\
    &= (\boldsymbol I -2\boldsymbol n \boldsymbol n^T )\boldsymbol p\\
    &= \boldsymbol H \boldsymbol p
\end{aligned}
$$
$\boldsymbol H$ is Householder matrix.





\begin{lemma}
    $\boldsymbol H$ is orthogonal.

    [prove]
    $$
    (\boldsymbol H)_{ij}=\left\{ 
        \begin{aligned}
        -2 n_in_j,\  & i \neq j\\
        1-2 n_in_j,\  & i = j\\
        \end{aligned}
         \right.
    $$

    we calculate $c =((\frac{1}{2}\boldsymbol H)(\frac{1}{2}\boldsymbol H^T))_{ij} $.For $i \neq j$, we have,
    $$
    \begin{aligned}
        c = &+(-n_in_0)(-n_jn_0)+\cdots\\
        &+(\frac{1}{2}-n_in_i)(-n_jn_i)+\cdots\\
        &+(-n_in_j)(\frac{1}{2}-n_jn_j)+\cdots\\
        &+(-n_in_{n-1})(-n_jn_{n-1})=0
        \end{aligned}
    $$
    For $i =j$, we have,
    $$
    \begin{aligned}
        c = &+(-n_in_0)(-n_in_0)+\cdots\\
        &+(\frac{1}{2}-n_in_i)(\frac{1}{2}-n_in_i)+\cdots\\
        &+(-n_in_{n-1})(-n_in_{n-1})=\frac{1}{4} \qed.
        \end{aligned}
    $$
\end{lemma}





\subsection{Application}

When we need the reflection collinear to the vector $\boldsymbol e_1 = [1,0,\cdots,0]^T$, the normal of the reflection superplane is 
$$
\boldsymbol n = \frac{\boldsymbol p - ||\boldsymbol p || \boldsymbol e_1}{||\boldsymbol p - ||\boldsymbol p || \boldsymbol e_1||}
$$

and 
$$
\boldsymbol H_1 \boldsymbol A = 
\begin{bmatrix}
\alpha _1& & \cdots &\\
0& &  &\\
\vdots& & \boldsymbol A_1' &\\
0& &  &\\
\end{bmatrix}
= 
\begin{bmatrix}
\alpha _1& & &\\
0& &  \boldsymbol A_1''&\\
\vdots& &  &\\
0& &   \cdots&\\
\end{bmatrix}
$$

$$
\boldsymbol H_k  = 
\begin{bmatrix}
\boldsymbol I_{k-1}& 0\\
0& \boldsymbol H_{k}'\\
\end{bmatrix},
\boldsymbol T_k  = 
\begin{bmatrix}
\boldsymbol I_{k-1}& 0\\
0& \boldsymbol H_{k}''\\
\end{bmatrix}
$$

where $\boldsymbol H_{k}'$ is the Householder of $\boldsymbol A_{k}'$, it is obviously orthogonal. And we have
$\boldsymbol H_n \cdots \boldsymbol H_{1} \boldsymbol A= \boldsymbol R$, 

therefore we have the QR decomposition of $\boldsymbol A$
$$
\boldsymbol A= \boldsymbol H_{1}^T \cdots \boldsymbol H_{n}^T  \boldsymbol R =  \boldsymbol Q  \boldsymbol R 
$$.

We want to rewrite each line, we can apply a householder $ \boldsymbol H$ to $ \boldsymbol A_k''^T$, and then transpolate the result, $(\boldsymbol H \boldsymbol A_k''^T)^T =\boldsymbol A_k''\boldsymbol H $, we mark the row-Householder as $ \boldsymbol T$, and we have

$\boldsymbol H_n \cdots \boldsymbol H_1 \boldsymbol A \boldsymbol T_2 \cdots \boldsymbol T_n= \boldsymbol B$,

where $\boldsymbol B$ is the bidiagonal, therefore,

$$
\boldsymbol A= \boldsymbol H_{1}^T \cdots \boldsymbol H_{n}^T  \boldsymbol B 
\boldsymbol T_{n}^T \cdots \boldsymbol T_{2}^T
$$.



\chapter{special matrix}
\section{schur定理}

\section{正规矩阵}

\section{实对称矩阵和Hermite矩阵}




\chapter{Decomposition}

\section{Transformation Decomposition}

If we have 
$\boldsymbol y = \boldsymbol B \boldsymbol x= \boldsymbol P^{-1} \boldsymbol A \boldsymbol P \boldsymbol x = \boldsymbol U \boldsymbol \Sigma  \boldsymbol U^T \boldsymbol x$, where $ \boldsymbol U $ is rotation or reflection, $ \boldsymbol \Sigma  $ is scaling, the transformation is simplified.





\section{Eigen Decomposition}

For square matrix.
 
\subsection{Defination by transformation}
  $\boldsymbol A =\boldsymbol U \boldsymbol \Sigma  \boldsymbol U^{-1} $, where $ \boldsymbol U $ is rotation or reflection, $ \boldsymbol \Sigma  $ is scaling.
 
 
\subsection{Defination by defination}
 
For the given transformation$\boldsymbol A $, if $\exists \boldsymbol v, \lambda$, such that $\boldsymbol A \boldsymbol v = \lambda \boldsymbol v$, and maybe we have some equations with different $\lambda$, we collect as $\boldsymbol  \lambda$ as a diagonal matrix, and the equations as  $\boldsymbol A \boldsymbol Q =  \boldsymbol Q\boldsymbol \lambda$, where $\boldsymbol Q = [\boldsymbol  v_0, \cdots, \boldsymbol  v_{n-1}]$, therefore we have  $\boldsymbol A =  \boldsymbol Q\boldsymbol \lambda\boldsymbol Q ^T$, we call this is the Eigen Decomposition of $\boldsymbol  A$.
 
 


\section{QR}


\paragraph{QR factorization}
the construction is like this:

$$
[\boldsymbol{a}_1,\boldsymbol{a}_2,\cdots, \boldsymbol{a}_n ]
=
[\boldsymbol{q}_1,\boldsymbol{q}_2,\cdots, \boldsymbol{q}_n ]
\cdot 
\begin{bmatrix}
    r_{11} &  r_{12} &\cdots & r_{1n}\\
    &  r_{22} & & \\
    &   &\ddots & \\
    0 & 0 & 0 &r_{nn}  \\
 \end{bmatrix}
$$


\section{SVD: Singular Value Decomposition}


\subsection{Defination}

When the matrix is not square, the form$\boldsymbol  A = \boldsymbol  U  \boldsymbol \Sigma \boldsymbol V^T$.

\begin{lemma}
    $\boldsymbol  C$ is the  Gram matrix of $\boldsymbol  A$, $  \boldsymbol\Sigma^2 = \boldsymbol \lambda_C $

    [Prove] $\boldsymbol  C = \boldsymbol  A^T \boldsymbol  A =
    \boldsymbol  V  \boldsymbol \Sigma \boldsymbol U^T
    \boldsymbol  U  \boldsymbol \Sigma \boldsymbol V^T
    =\boldsymbol  V  \boldsymbol \Sigma^2 \boldsymbol  V^T$ \qed.
\end{lemma}


\subsection{Algorithm 1: QR}

\begin{algorithm}[H]
    \caption{SVD:QR}\label{algo:SVD:QR}
    \SetAlgoLined
    \KwIn{ $\boldsymbol A_{mn}$}
    \KwOut{ $\boldsymbol U, \boldsymbol \Sigma,\boldsymbol V $}
    $\boldsymbol  C = \boldsymbol  A^T \boldsymbol  A$\;
    use symmetric QR: $\boldsymbol  C = \boldsymbol  V  \boldsymbol \Sigma^2 \boldsymbol  V^T$\;
    $\boldsymbol U = \boldsymbol A \boldsymbol  V \boldsymbol \Sigma^{-1}$\;
    \KwRet $\boldsymbol U, \boldsymbol \Sigma,\boldsymbol V $\;
\end{algorithm}


\chapter{Form}

\subsection{Jordan}

Jordan型、根子空间分解、循环子空间、多项式矩阵相抵不变量、特征方阵与相似标准型


\subsubsection{不变子空间
}

《矩阵理论-陈大新》

\subsubsection{特征值全0矩阵的Jordan标准型}

《矩阵理论-陈大新》



\subsubsection{Jordan标准型计算}


\subsection{二次}
配方法构造、对称方阵的相合、相合不变量


\chapter{参考文献说明}
《矩阵理论-陈大新》\upcite{5ch001shepherdQR2020paper}:好的观点的来源。


\bibliography{reference1}%, reference2

\end{document}
