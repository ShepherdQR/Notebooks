%%============
%%  ** Author: Shepherd Qirong
%%  ** Date: 2019-06-10 18:32:30
%%  ** Github: https://github.com/ShepherdQR
%%  ** LastEditors: Shepherd Qirong
%%  ** LastEditTime: 2024-09-15 20:14:19
%%  ** Copyright (c) 2019--20xx Shepherd Qirong. All rights reserved.
%%============


\documentclass[UTF8]{../09-Mathematics}
\begin{document}

\title{09-04-Algebra}
\date{Created on 20220605.\\   Last modified on \today.}
\maketitle
\tableofcontents


\chapter{Introduction}


\begin{lstlisting}

a: 线性代数, linear algebra
b: 群论, group theory
c: 域论, field theory
d: 李群, lie group
e: 李代数, 
f: Kac-Moody代数, 
g: 环论(包括交换环与交换代数, 结合环与结合代数, 非结合环与非结合代数等), 
h: 模论, 
i: 格论, 
j: 泛代数理论, 
k: 范畴论, 
l: 同调代数, 
m: 代数K理论, 
n: 微分代数, 
o: 代数编码理论, 
p: 代数学其他学科。

\end{lstlisting}



\chapter{介绍}

一元n次方程求根 -> 一元多项式 -> 一元多项式环  -> 环  -> 域 -> 群 

因而代数研究 代数结构和态射



\chapter{GroupTheory}

\section{GroupRepresentation}

\subsection{Introduction}
丘维声。\\
运算: 笛卡尔积$S \times S \mapsto S$是集合S的二元代数运算。现代数学的鲜明特征是研究有各种运算的集合, 称为代数系统。\\
现代数学的两大特征, 1)研究代数系统的结构;2)利用同态映射研究代数系统结构。\\
核心: 群表示论是研究G到各个线性空间的可逆线性变换群的各种同态映射, 以获得G结构的完整信息。\\
群表示论是研究群结构的最强有力的工具。应用如晶体学, 量子力学, 抽象调和分析, 组合数学, 密码学, 纠错编码\\
必备参考书, 《抽象代数基础》丘维声, 高教出版社, 《高等代数学习指导书 下》丘维声, 清华大学。

《高等代数》丘维声, 清华大学出版社, 

\subsubsection{环, 域, 群}

\begin{table}[htbp]
\newcommand{\tabincell}[2]{\begin{tabular}{@{}#1@{}}#2\end
{tabular}}
\centering
  \caption{algebra system }
  \label{tab:label}
    \begin{tabular}{cccc}
    \toprule
    名称 & 运算 & 性质 & 举例\\
    \midrule
    \multirow{2}{*}{ring}&  加法& 交换, 结合, 0元, 负元&\multirow{2}{*}{\tabincell{c}{整数集$\mathbb Z$, 偶数集$2\mathbb Z$,\\一元实系数多项式$\mathbb R(x)$,实n阶矩阵$M_n(\mathbb R)$}}\\
    \cline{2-3}
    & 乘法 & 结合, 左右分配律 & \\
    \bottomrule
    \end{tabular}
\end{table}


\paragraph{环}

交换环, 乘法可交换。单位元。\\
例子, 星期i, 记为$\bar i =\left\{ 7k+i \right\} |k\in \mathbb Z$,收集起$\bar i$可以实现整数的划分。类似的, 定义模m剩余类: 
\begin{equation}
    \mathbb Z_m =\left\{ \bar i | i=1, \cdots ,m-1 \right\}
\end{equation}
定义加法和乘法,有模m剩余类环
\begin{equation}
\begin{split}
&\bar i + \bar j =\bar{i+j}\\
& \bar i \cdot \bar j = \bar{i \cdot j}
\end{split}
\end{equation}
可逆元, 单位a: $a \in  ring R, \exists b \in R, ab=ba=e$\\
左零因子a: $a \neq 0, \exists c \neq 0,ac=0$\\
例子: $\mathbb Z_8$,零因子2,4,6;可逆元3,5,7;\\
例子: $\mathbb Z_7$, 每个非零元都可逆;\\



\paragraph{域}


域F: 有单位元e的环, 且每个非零元都可逆。
例子: 有理数集, 实数集, 复数集$\mathbb {Q,R,Z}$\\
域F中可以定义除法。



\paragraph{群}

群G:只有乘法, 结合律, 单位元, 每个元素有逆元。\\
例子: $\mathbb Z_m^*$ :$\mathbb Z_m$所有可逆元的集合, 发现只对乘法封闭,称为Zm的单位群;如域F上的所有可逆矩阵的集合$Gl_n(\mathbb F)$,只有乘法, 称为域$\mathbb F$上的一般线性群。\\
阿贝尔群: 乘法可交换。\\
例子: $GL(V)$, 域F上的线性空间V的所有可逆线性变换, 对于映射的乘法行程的V上可逆线性变换群。
子群: H<G\\

\subsubsection{等价关系与左陪集}
研究G的第一个途径: 利用子群H研究G.\\
集合的划分与等价关系。对于$a,b \in G$,定义 $a~b:\Leftrightarrow b^{-1}a \in H$, 易有关系~具有反射性(e在H中), 对称性, 传递性, 所以这是等价关系。\\
定义a的等价类: 
\begin{equation}
\begin{split}
\bar a  & :=\left\{ x \in G |x~a \right\}\\
        &:=\left\{ x \in G | a^{-1}x \in H \right\}\\
        &= \left\{ x\in G | x=ah,h\in H \right\}\\
        & =\left\{ ah |h \in H \right\}\\
        &=:aH
\end{split}
\end{equation}
aH称为以a为代表的H的一个左陪集, 是一个等价类。\\
根据等价类的性质, 有1)$aH=bH \Leftrightarrow b^{-1}a \in H$;2) aH与bH或者相等, 或者不相交(交集为空集)。所以 H的所有左陪集给出G的一个划分, 记为$G/H$,称为G关于H的左商集。G/H的基数称为G关于H的指数, 记为[G:H]。基数相同可建立双射。\\
G关于H的左陪集分解: $[G:H]=r, G=eH \bigcup a_1H \bigcup \cdots \bigcup a_{r-1}H$。\\
拉格朗日定理: 对于有限群G, 易有其元素个数$|G|=|H|[G:H]$, 即任何子群的阶是群的阶的因数。推论: 1)n阶群G的任意元素a, 有$a^n \in G$;2)素数阶群是循环群。

\subsubsection{同态}
研究G的第二个途径: 通过研究G到G'的保持运算的映射, 同态映射, 简称同态。同态要变, 是函数。\\
通常利用G到$\Omega$的同态, 等价于G在$\Omega$的作用。既可以研究G的结构, 又可以对$\Omega$的性质有了解\\
$S(\Omega)$:$\Omega$的全变换群, Full Transformation Group on Set $\Omega$, $\Omega$自身的所有双射组成的集合, 对于映射的乘法构成的一个群.\\

\subsubsection{作业}
1.环R中, 0a=a0=0 \\
2.有e的环中, 零因子不是可逆元\\
3.$\mathbb Z_m$中每个元素要么是可逆元要么是零因子;\\
4. $M_n(\mathbb F)$中, 每个矩阵是可逆矩阵或者是零因子。


\subsection{Abel群的表示}


\subsubsection{集合(Set)}
\begin{equation}
  \begin{aligned}
  Idempotent & & A \cup A = A \cap A = A \\
  Absorption & &A \cup (A \cap B) = A \cap (A \cup B) = A\\
  Commutative & &A \cup B = B \cup A, A \cap B = B \cap A\\
  Associative & &(A \cup B) \cup C = A \cup (B \cup C)\\
              & &(A \cap B) \cap C = A \cap (B \cap C)\\
  Distributive & &A \cap (B \cup C) = (A \cap B) \cup (A \cap C)\\
                & &A \cup (B \cap C) = (A \cup B) \cap (A \cup C)\\
  \end{aligned}
\end{equation}




\subsubsection{映射(Map)}
映射: $f:a \mapsto b \Leftrightarrow f(a)=b$。$f(x)读作f \ of \ x$原象, 象。映射$f$的定义域(domain)A, 陪域(codomain)B。映射得到的所有象的集合叫值域, 记作$f(A)$,或Imf。
\begin{equation}
  \begin{split}
  &f:a \mapsto b \Leftrightarrow f(a)=b\\
  &f: A \to B\\
  \end{split}
\end{equation}

映射通常关心is it one one? Is is onto? \\
Surjection, 满射, onto,到上: $f(A)=B$, $\forall b \in B, \exists a \in A, f(a) = b$. If left inverse, such as $\xi (\eta (x_A)) = x_B$, therefore surjection.\\
Injection, 单射, 1-1, 一一的, 每个a对应的b是不同的。$\forall x, y \in A, x \neq y, \therefore f(x) \neq f(y)$. If right inverse, such as $ \eta (\xi (x_C)) = x_B$, therefore injection\\
Bijection, 双射, 两个集合一一对应, isomorphic。\\
逆映射。对于$f:A \to B, g: B \to A$,有$g \circ f =1_A, f \circ g =1_B$。可逆的f是双射。\\

线性映射、线性变换是线性空间的同态映射, 有点乘和加法。\\
补空间: 域$\mathbb F$上的线性空间U<V,则$\exists W, V=U \oplus W$。对于实内积空间V, U是有限维的, $W=U^{\perp}$,正交补空间, 唯一的。\\
投影变换: 域$\mathbb F$上的线性空间$V=U \oplus W$,有$\mathbf \alpha =\mathbf \alpha _U+\mathbf \alpha _W$, 有投影变换$P_U:\mathbf \alpha \mapsto \mathbf \alpha _U$.投影变换保持加法和数乘, 是V上的线性变换。投影是同态映射。\\
几何空间: 以原点O为起点的定位向量组成的实线性空间。知道一空间点在两个坐标面的投影坐标可完全确定点。




\subsubsection{Relation}

\begin{equation}
  \begin{aligned}
  &Reflective & & \forall x \in M, x R x\\
  &Antisymmetry & & \forall x,y \in M, x R y, yRx \Rightarrow x = x\\
  &Transitive & & \forall x,y,y \in M, x R y, y R z \Rightarrow x R z\\
  \end{aligned}
\end{equation}


partial order set, poset, $(M, \preceq )$. Anti-circularity law $x_1 \preceq \cdots \preceq x_n \preceq x_1  \Rightarrow x_1 = \cdots = x_n$

quasi-order set, quoset, the relation only satisfies reflective law and transitive law, noted as $(M, \bullet \prec )$.

\begin{proposition}
  Any subset of a quoset is a quoset.
  Any subset of a poset is a poset.
  Any subset of a chain is a chain. 
\end{proposition}


\subsubsection{等价关系}

\begin{equation}
  \begin{aligned}
  &Symmetry & & \forall x,y \in M, x R y \Rightarrow  y R x\\
  &Alternative & & \forall x,y \in M, x \npreceq  y \Rightarrow  y \preceq x\\
  \end{aligned}
\end{equation}


\begin{proposition}
  Linear order, or total order: $\preceq$ and Alternative law.

  Chain: a set with a total order.

  Tower, 域扩张的塔:E>F>k, 有大小包含关系.

  nest, or sleeve: $A_1 \subset A_2 \subset \cdots  \subset A_n$,包含关系
\end{proposition}

反身性,对称性,传递性,$a \sim b$. $a \sim b \Leftrightarrow b \sim a$. $a \sim b , b \sim c \Rightarrow a \sim c$\\
$\bar x$ 是x确定的等价类,$x(M) = \left\{ y | y \in M, y \sim x \right\}$。易有
$\bar x = \bar y \Leftrightarrow x \sim y$ 。\\


定理1:集合S上等价关系$\sim$给出的等价类的集合是S的一个划分。\\
证明思路:需要证明并全,交空。交空比较难,需要研究等价类的性质。等价类的代表不唯一。\\
Step1)  It is obvious that $\cup _{a \in S} {\bar a} \subseteq S$,and for any $b \in S$, we have $b \in \bar b \in \cup _{a \in S} {\bar a}$, this means $S \subseteq \cup _{a \in S} {\bar a}  $, so $ \cup _{a \in S} {\bar a} =S $.\\
Step2) To prove $\bar x \neq \bar y \Rightarrow \bar x \bigcap \bar y = \varnothing  $, we prove the contrapositive $ \bar x \bigcap \bar y \neq \varnothing \Rightarrow  \bar x = \bar y$, and this is easy to prove.\\


\begin{proposition}
  equivalence, define $x\sim _{\bullet \prec } y:= (x \bullet \prec y) \wedge (y \bullet \prec x)$.
\end{proposition}

\begin{proposition}
  A quoset $(M, \bullet \prec)$ is a poset, if and only if, the quotient set $M / \sim _{\bullet \prec} =  M$, or say, it satisfies the anti-circularity law.
\end{proposition}


\subsubsection{群的同态Isomorphic}
同态映射: G到G'的映射$\sigma: \sigma(a,b)=\sigma(a) \sigma(b)$。单射的话是单同态。满射是满同态。双射是同构, 此时两个群同构, $G \cong G'$。\\
同态的性质: 1)单位元、逆元、子群映射过去是G’的单位元、逆元、子群。例如$G<G \Rightarrow \sigma(G)=Im\sigma <G'$, 同态的像是G'的子群\\
刻化单同态: 找到映射成单位元的原象, 定义同态的核, $Ker\sigma:=\left\{ a \in G | \sigma(a)=e' \right\}$.易有同态的核是G的子群。对于单同态, $Ker\sigma=\left\{ e \right\}$.\\
子集乘法: 类似与点乘, a的所有和b的所有的乘积方式的组合。乘法满足结合律。\\

\subsubsection{正规子群}
$ \forall k \in \ker \sigma, \forall g\in G$, we have:
\begin{equation}
\begin{split}
&\sigma(gkg^{-1})=\sigma(g)\sigma(k)\sigma(g^{-1})=e'\\
& \therefore gkg^{-1}\in \ker \sigma\\
& \therefore g \ker \sigma g^{-1} \subset \ker \sigma\\
& \& \ g^{-1}  \ker \sigma g\subset \ker \sigma\\
& \therefore g \ker \sigma g^{-1}=\ker \sigma\\
\end{split}
\end{equation}
normal subgroup正规子群:$H \lhd G :\forall g \in G, gHg^{-1}=H$, $gHg^{-1}$是g的共轭子群。\\
性质: $H \lhd G \Leftrightarrow gHg^{-1}=H,\forall g \in G \Leftrightarrow gH=Hg$,H的左右陪集相等;\\
G关于正规子群H的商群: 规定正规子群H的商群G/H乘法: $(aH)(bH)=abH$\\
\subsection{群同态基本定理}
\begin{equation}
\begin{split}
\psi : &G/ \ker \sigma \to Im \sigma\\
& a (\ker \sigma) \mapsto \sigma(a)
\end{split}
\end{equation}
看映射$\psi$的性质: 
\begin{equation}
\label{证明同构1}
\left.
\begin{aligned}
& a(\ker \sigma)=b(\ker \sigma)\\
& \Leftrightarrow b^{-1}a \in \ker \sigma\\
&\sigma(b^{-1}a)=e'\\
&\therefore \sigma(a)=\sigma(b)
\end{aligned}
\right\} \Rightarrow \psi \ is \ surjection\\
\end{equation}
从是映射、是单射、是满的, 得到是双射;\\
Let $K=\ker \sigma$, $\psi [(aK)(bK)]=\psi (abK)=\sigma(ab)=\psi(aK)\psi(bK)$,所以保持运算, 所以是同构, 所以$G/ \ker \sigma \cong Im \sigma$\\
群同态基本定理: $\ker \sigma \lhd G \ \& \   G/ \ker \sigma \cong Im \sigma  $ 



\subsection{线性表示}
$GL(V)$: 对于群G, 域$\mathbb K$上的线性空间V, G到V的所有可逆线性变换的集合, 对于映射的乘法成为一个可逆线性变换群。\\
G到GL(V)的同态$\psi $ 是G在$\mathbb K$ 上的线性表示, 简称为$\mathbb K$ 表示, 或者简称为表示。\\
V叫做表示空间, 表示次数$\deg \psi := \dim V$\\
$(\psi ,V$)\\
如用两个视图可完全确定空间曲线, 即做了两个同态。\\
$\ker \psi =\left\{ e_G \right\}$,$\psi $是忠实的;\\
$\ker \psi =G$, $\psi$是平凡的;|、
称一次的平凡表示$\psi $是G的主表示, 单位表示, 记作$1_G$;\\
$dimV=n$时, 有$\psi (g)$在基$\left\{ \alpha_1,\cdots , \alpha_n \right\}$下的矩阵$\Phi(g)$是 $\mathbb K$ 上的可逆矩阵, 由同构$GL(V) \cong GLn(\mathbb K)$, 有G到$GLn(\mathbb K)$的同态$\Phi$, 称为G在$\mathbb K$上的n次矩阵表示。\\
$\Phi $ 称为是$\psi $提供的

\subsubsection{表示的等价类}
等价关系: 对于G的2个k表示, $(\phi,V),(\psi,W)$,$\exists$v到w的线性空间的同构$\sigma$, 定义$\psi(g)\sigma =\sigma \phi (g), \forall g \in G$,这种G的所有K表示的集合$\Omega$上的二元关系, 易有具有反射性, 对称性, 传递性, 从而是等价关系。通常关注G的k表示的等价类。\\
$(\phi,V),(\psi,W)$等价, 取基后的矩阵记为$\Phi (g)| \{\alpha _i, \cdots\},\Psi (g)| \{\beta _i, \cdots\}$,同构$\sigma$把V的基映射到W上的S, 有$\sigma (\{\alpha , \cdots\})=\{\beta, \cdots\}S$, $\therefore \Psi (g)S=S\Phi(g)$.所以G在K的2个矩阵表示$\Psi, \Phi$等价: 次数一样且$\forall g \in G,\Psi =S\Phi (g)S^{-1}$. 所以群G的2个K表示$(\phi,V),(\psi,W)$等价, $\Leftrightarrow$K表示提供的矩阵表示$\Psi, \Phi$等价。
\subsubsection{例: 1次表示}
G在K上的1次矩阵表示$\Phi: G \to \mathbb K^*$;非零元集合。映射是通用的不对定义域做要求, 陪域是域的子集的映射叫函数。称为G上的K*函数, 且由于同态保持运算, 有$\Phi (gh)=\Phi (g)\Phi(h), \forall g,h \in G$, and $\Phi(e)=1$, where 1 is the unit of K*.所以一次表示是G到K*的保持运算的函数。

\subsubsection{例: 实数和加法的1次实表示}
$f_a(x)=e^{ax}$ 是 $\mathbb R, +$的1次实表示。

\subsubsection{例: 实数和加法的1次复表示}
$f_a(x)=e^{iax}$ 是 $(\mathbb R, +)$的1次实表示。
\begin{equation}
\label{fubiaohsi}
\begin{split}
f:(\mathbb R, +) &\to \mathbb C^*\\
x  &\mapsto e^{iax}\\
\end{split}
\end{equation}
 



\chapter{域论}

所有数域都包含有理数域。有理数域指含有0,1的,且对加减乘除(除数不是0)封闭的域。

最小的数域是$A = {0,1}$,包含$\sqrt{5}$的最小数域是$ x| x = a + b\sqrt{5}, a , b \in A$

{x|x=a+b√5,a,b∈A}.


最大的数域是$\mathbb{C}$


\chapter{李群}
\chapter{李代数}
\chapter{Kac-Moody代数}
\chapter{环论}
环论(包括交换环与交换代数, 结合环与结合代数, 非结合环与非结合代数等)
\chapter{模论}
\chapter{格论}
\chapter{泛代数理论}
\chapter{范畴论}
\chapter{同调代数}
\chapter{代数K理论}
\chapter{微分代数}
\chapter{代数编码理论}
\chapter{代数学其他学科}




\end{document}
