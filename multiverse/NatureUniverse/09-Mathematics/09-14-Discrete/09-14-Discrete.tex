%%============
%%  ** Author: Shepherd Qirong
%%  ** Date: 2022-05-06 20:40:40
%%  ** Github: https://github.com/ShepherdQR
%%  ** LastEditors: Qirong ZHANG
%%  ** LastEditTime: 2024-12-08 22:01:17
%%  ** Copyright (c) 2019--20xx Shepherd Qirong. All rights reserved.
%%============

\documentclass[UTF8]{../09-Mathematics}
\begin{document}

\title{09-14-Discrete}
\date{Created on 20241208.\\   Last modified on \today.}
\maketitle
\tableofcontents



% \begin{lstlisting}

% \end{lstlisting}


\chapter{Introduction}

包括:
组合数学,combinatorics

图论


\chapter{combinatorics}
组合数学,离散数学

广义的组合数学就是离散数学,狭义的组合数学是离散数学除图论、代数结构、数理逻辑等的部分。

狭义的组合数学主要研究满足一定条件的组态(也称组合模型)的存在、计数以及构造等方面的问题。 组合数学的主要内容有组合计数、组合设计、组合矩阵、组合优化(最佳组合)等。


\chapter{组合分析}
\chapter{组合设计}
\chapter{组合几何}
\chapter{编码理论 (代数码理论 )}



\chapter{图论}


\section{elementary}

\subsection{basic defination}


\begin{lstlisting}

    adjacent: vertexes u and v are adjacent, when there is an edge joining them.

    incident: vertex u is the end of the edge e, then u is incident to e.

    loop: an edge from vetex u to u. 

    parallel: two pathes with the same vertes lists.

    simple graph: no loop, no parallel edges.

    order: number of vertices.

    finite: the number of vertices is finite, and the number of edges is finite.
   

\end{lstlisting}



\paragraph{graph}

a triplet$\{V,E, \phi\}$, where V is the vertes set, E is the edge set, $\phi$ is the incidence function $\phi: V \times V \to E$

Each vertex represents an object, each edge represents a relationship.


\paragraph{isomorphic}

\paragraph{subgraph}

\paragraph{complete graph}

a complete graph with n vertices is noted as $K_n$, all vertes are adjacent each other. The number of edges is $C_n^2$





\subsection{图论的应用}


 





\end{document}
