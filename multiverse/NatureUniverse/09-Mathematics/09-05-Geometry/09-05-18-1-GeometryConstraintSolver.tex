%%============
%%  ** Author: Shepherd Qirong
%%  ** Date: 2022-04-09 22:31:44
%%  ** Github: https://github.com/ShepherdQR
%%  ** LastEditors: Qirong ZHANG
%%  ** LastEditTime: 2024-12-09 22:05:09
%%  ** Copyright (c) 2019--20xx Shepherd Qirong. All rights reserved.
%%============


\documentclass[UTF8]{../09-Mathematics}
\begin{document}

\title{09-05-18-1-GeometryConstraintSolver}
\date{Created on 20241209.\\   Last modified on \today.}
\maketitle
\tableofcontents


\chapter{Introduction}



\chapter{Books}

\begin{lstlisting}
  1

\end{lstlisting}

\chapter{Geometry Solver}




\chapter{Numerical Solver}


\section{Defination}

a geometry is a vector in d-dimention $\boldsymbol v$,

a dimension is a system of equations inequations $\boldsymbol F(\boldsymbol v)= \boldsymbol 0, \boldsymbol N(\boldsymbol v)\leqslant \boldsymbol 0 $.



\subsection{Problem}

Given the state parameters $P^{(0)}$ and the final state value $S^{(*)}$, calculate the final state parameters $P^{(n)}$.




\subsection{Solution}



\subsection{Charity}

For the outerProduct of 2 2d vectors, we define the result is negative for the result vector into the screen. For example, if $\boldsymbol u$ is on the left side of  $\boldsymbol v$, $\boldsymbol u \times\boldsymbol v < 0 $.

If before applying the constraints, point is on the left side of a line, we want the positive distance makes the point still be on the left side of a line. Therefore, we need to mark the original charity, and makes it as the input sign of the equations.


\paragraph{PL}




\subsection{Geometry}


\paragraph{Point}

$\boldsymbol v = \{p_x, p_y\}, p_x, p_y \in \mathbb R$


\paragraph{Line}

$\boldsymbol v = \{r, \theta\},r\geqslant 0,  \theta \in [0, 2\pi)$, r is the distance between the origin and the line, the x+ axis rotates  $\theta$ anticlockwise can get the direction of the line. the direction of the line is $[\cos \theta, \sin \theta]^T$.


\paragraph{Circle}

$\boldsymbol v = \{p_x, p_y, r\}, p_x, p_y \in \mathbb R, r\in [0, +\infty)$, the center of the cirsle is $[p_x, p_y]^T$.



\subsection{Constriant}

there some type of constriants, distance, angle, others.

\paragraph{D-PP}
$$
F(x_1, x_2, y_1,y_2) = \sqrt{(x_1 - x_2)^2 + (y_1 - y_2)^2}
$$
noticed that the curve $F = a \in \mathbb R$ is irregular $(x_1 - x_2 = 0) \& (y_1 - y_2 = 0) $, other than this line, we have
$$
\begin{aligned}
\frac{\partial F}{x_1} &= F^{-1}\cdot 2(x_1 - x_2)\\
\frac{\partial F}{y_1} &= F^{-1}\cdot 2(y_1 - y_2) \\
\frac{\partial F}{x_2} &= -F^{-1}\cdot 2(x_1 - x_2) \\
\frac{\partial F}{y_2} &= -F^{-1}\cdot 2(y_1 - y_2) \\
\end{aligned}
$$

\paragraph{D-PPL}

\paragraph{D-PL}

\paragraph{D-LL}





\chapter{Else}



\end{document}

