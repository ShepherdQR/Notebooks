%%============
%%  ** Author: Shepherd Qirong
%%  ** Date: 2023-05-24 21:51:38
%%  ** Github: https://github.com/ShepherdQR
%%  ** LastEditors: Qirong ZHANG
%%  ** LastEditTime: 2025-02-22 12:45:36
%%  ** Copyright (c) 2019--20xx Shepherd Qirong. All rights reserved.
%%============

\documentclass[UTF8]{../../09-Mathematics}
\begin{document}

\title{09-07-01-MathematicalAnalysis}
\date{Created on 20230524.\\   Last modified on \today.}
\maketitle
\tableofcontents


\chapter{Introduction}






\chapter{极限和连续}




\section{极限的15种求法}

极限的15种求法.pptx




\subsection{利用级数证明极限存在}


\begin{question}
    $f(x) = \frac{k+kx}{k+x}, k>1, x_{n+1}=f(x_n),x_n>0$, $\lim_{n\to+\infty}x_n =c,c=? $

    solve: $f'(x)<1-\frac{1}{k}<1$, let $a_n =x_{n+1}-x_{n}$, $\frac{a_{n+1}}{a_{n}}=  \frac{f(x_{n+1})-f(x_{n})}{x_{n+1}-x_{n}}=f'(\xi )<1$, $\therefore \sum a_n$收敛, $\therefore \lim_{n\to+\infty}x_{n+1} $存在,$c=\frac{k+kc}{k+c}$,$\therefore c=\sqrt{k} $,$\qed$
\end{question} 









\chapter{微分学}


\section{导数与求法}


\section{微分基础}


\section{微分学基本定理}



\section{高阶导数与高阶微分}








\section{多元函数-偏导数和全微分}

\section{多元函数-复合函数和隐函数微分}


\section{泰勒公式}




梯度写作列向量,Hessian matrix: 
\begin{equation}
    %\label{hessian}
\begin{aligned}
    \nabla f(\boldsymbol x)&= \left[\frac{\partial f(\boldsymbol x)}{\partial x_i},\cdots\right]\\
    \mathbf H (\boldsymbol x)&= \nabla ^2 f(\boldsymbol x)=\left[ \frac{\partial ^2 f(\boldsymbol x)}{\partial x_i \partial x_j} \right]\\
\end{aligned}
\end{equation}
 一阶导数为0,可能是极值点,同时二阶导数为0的时候就是鞍点saddle point,判断鞍点可用三阶导数。


矢量的泰勒级数展开
\begin{equation}
%\label{taylor}
f(\mathbf x_i +\mathbf \delta) \approx f(\mathbf x_i)+
\nabla ^T f(\mathbf x_i) \mathbf \delta +
\frac{1}{2}\mathbf \delta ^T \nabla ^2 f(\mathbf x_i)\mathbf \delta
\end{equation}


Given function $f(\boldsymbol x)$, we want to find a function $g(\boldsymbol y)$, such that $f^{(n)}(\boldsymbol x^*)=g^{(n)}(\boldsymbol x^*)$.

Step1: For $f^{(0)}(\boldsymbol x^*)=g^{(0)}(\boldsymbol x^*)$, we can just let $g(\boldsymbol y)=f^{(0)}(\boldsymbol x^*)$.

Step2: For $f^{(1)}(\boldsymbol x^*)=g^{(1)}(\boldsymbol x^*)$, $g(\boldsymbol y)=[f^{(1)}(\boldsymbol x^*)]^T *\boldsymbol y$. If we shift the $\boldsymbol y$, like this $g(\boldsymbol y)=[f^{(1)}(\boldsymbol x^*)]^T *(\boldsymbol y- \boldsymbol x^*)$, we find that this step has no effect on all the previous steps. The order of this step is higher than all the previous steps, so all the previous steps have no effect on this step. So all steps are isolated.

\begin{equation}
    %\label{taylor}
    f(\boldsymbol x) = f(\boldsymbol a)+
    \nabla ^T f(\boldsymbol a) (\boldsymbol x - \boldsymbol a) +
    \frac{1}{2} (\boldsymbol x - \boldsymbol a) ^T \nabla ^2 f(\boldsymbol a)(\boldsymbol x - \boldsymbol a)
\end{equation}
为什么梯度方向是上升最快的方向,因为泰勒公式中可以看到,取梯度方向时,向量共线,夹角0,模最大;同样有负梯度方向最小。通常而言,方向更重要,步长没有方向那么重要\\
把二次项也考虑进来,就叫牛顿法


\section{其他}
\subsection{中值定理}
\subsection{洛必达法则}
\subsection{单调性与极值}
\subsection{凹凸性}


\section{应用}














\chapter{积分学}

\section{换元法}

\subsection{Questions}

\begin{question}
    $y^2(x-y) = x^2, \int \frac{1}{y^2}dx$

    solve: let $y = tx$, we have $Ans = 3t-2\ln t + C $
\end{question} 

\begin{question}
    摆线上的点$a[t- \sin t, 1- \cos t]$, 求一个周期内与x轴围成的面积

    solve: let $\int _T y dx = a^2 \int _T (1- \cos t)^2 dt$,and $\int _T (1- \cos t)^2 dt = \int _T 1- 2 \cos t + (\cos t)^2 dt = 2 \pi + \int _T \frac{1 + \cos (2t)}{2}dt = 3 \pi$ we have $Ans = 3\pi a^2 $
\end{question} 

\begin{question}
    $(x^2 + y^2)^2 = 2a^2(x^2-y^2), a>0, \int \frac{dx}{y(x^2 + y^2+ a^2)}$

    $Ans = \frac{1}{2a^2} + \ln |\frac{x-y}{x+y}| + C $?
\end{question} 


\section{分部积分}
\section{反常积分}

\section{二重积分}
\section{三重积分}

\section{格林公式}
\section{对弧长的曲线积分}
\section{对坐标的曲线积分}


\section{应用}




\chapter{级数论}

\section{基础}



\subsection{收敛性}
\subsection{函数的级数展开}

\subsection{发散级数、可求和性、收敛因子}



\section{连分式论}



\section{特殊级数}


\subsection{调和级数}

\subsubsection{调和级数部分和}

$$
H_n = \sum_{k=1}^{n}\frac{1}{k}
$$
$$
\gamma = \lim_{n \rightarrow \infty}(H_n - ln(n))= \lim_{n \rightarrow \infty}(T_n)
$$

欧拉-马歇罗尼常数 (Euler-Mascheroni Constant) $\gamma = 0.5772156649$

Let us prove the limit of $T_n$ exits. 

$x\in A = [k,k+1]$, 
$$
\frac{1}{k+1} \leqslant\frac{1}{x} \leqslant\frac{1}{k} 
$$
 分别计算x在A上的积分,有:
$$
\frac{1}{k+1} \leqslant ln(k+1)-ln(k) \leqslant\frac{1}{k} 
$$
noted as $L \leqslant M \leqslant R$,all with $\sum_{1}^{n}$,

$L = H_{n+1}-1, M = ln(n+1), R = H_n$, we have $ln(n+1) \leqslant H_n \leqslant 1+ ln(n)$.

we prove $T_n \geqslant 0$, which is $H_n \geqslant ln(n)$. Since we have $H_n \geqslant ln(n+1)$, both side add $-ln(n)$, we have $T_n \geqslant  ln(1+ \frac{1}{n})>0$.

we prove $T_{n+1}- T_{n}<0$, which is $\frac{1}{n+1} \leqslant ln(n+1) - ln(n)$, which is $ \int_{n}^{n+1} \frac{1}{n}dx \leqslant \int_{n}^{n+1}\frac{1}{x}dx$, since  $x \in[n, n+1], \frac{1}{n+1} \leqslant \frac{1}{x}$.




\subsection{傅里叶级数}

把函数泰勒展开成关于1/x或其他函数的多项式:洛朗级数

把函数泰勒展开成关于1/x:1)就有曲率了?2)易收敛?

\subsection{洛朗级数}




\end{document}