%%============
%%  ** Author: Shepherd Qirong
%%  ** Date: 2023-05-24 21:51:38
%%  ** Github: https://github.com/ShepherdQR
%%  ** LastEditors: Qirong ZHANG
%%  ** LastEditTime: 2024-12-21 21:48:20
%%  ** Copyright (c) 2019--20xx Shepherd Qirong. All rights reserved.
%%============

\documentclass[UTF8]{../../09-Mathematics}
\begin{document}

\title{09-07-01-MathematicalAnalysis}
\date{Created on 20230524.\\   Last modified on \today.}
\maketitle
\tableofcontents


\chapter{Introduction}






\chapter{极限和连续}




\section{极限的15种求法}

极限的15种求法.pptx




\subsection{利用级数证明极限存在}


\begin{question}
    $f(x) = \frac{k+kx}{k+x}, k>1, x_{n+1}=f(x_n),x_n>0$, $\lim_{n\to+\infty}x_n =c,c=? $

    solve: $f'(x)<1-\frac{1}{k}<1$, let $a_n =x_{n+1}-x_{n}$, $\frac{a_{n+1}}{a_{n}}=  \frac{f(x_{n+1})-f(x_{n})}{x_{n+1}-x_{n}}=f'(\xi )<1$, $\therefore \sum a_n$收敛, $\therefore \lim_{n\to+\infty}x_{n+1} $存在,$c=\frac{k+kc}{k+c}$,$\therefore c=\sqrt{k} $,$\qed$
\end{question} 


\chapter{微分学}

\section{中值定理}
\section{洛必达法则}
\section{单调性与极值}
\section{凹凸性}

\section{多元函数-偏导数和全微分}

\section{多元函数-复合函数和隐函数微分}


\section{应用}




\chapter{积分学}

\section{换元法}

\subsection{Questions}

\begin{question}
    $y^2(x-y) = x^2, \int \frac{1}{y^2}dx$

    solve: let $y = tx$, we have $Ans = 3t-2\ln t + C $
\end{question} 

\begin{question}
    摆线上的点$a[t- \sin t, 1- \cos t]$, 求一个周期内与x轴围成的面积

    solve: let $\int _T y dx = a^2 \int _T (1- \cos t)^2 dt$,and $\int _T (1- \cos t)^2 dt = \int _T 1- 2 \cos t + (\cos t)^2 dt = 2 \pi + \int _T \frac{1 + \cos (2t)}{2}dt = 3 \pi$ we have $Ans = 3\pi a^2 $
\end{question} 

\begin{question}
    $(x^2 + y^2)^2 = 2a^2(x^2-y^2), a>0, \int \frac{dx}{y(x^2 + y^2+ a^2)}$

    $Ans = \frac{1}{2a^2} + \ln |\frac{x-y}{x+y}| + C $?
\end{question} 


\section{分部积分}
\section{反常积分}

\section{二重积分}
\section{三重积分}

\section{格林公式}
\section{对弧长的曲线积分}
\section{对坐标的曲线积分}


\section{应用}




\chapter{级数论}

\section{基础}



\subsection{收敛性}
\subsection{函数的级数展开}

\subsection{发散级数、可求和性、收敛因子}



\section{连分式论}



\section{特殊级数}


\subsection{傅里叶级数}

把函数泰勒展开成关于1/x或其他函数的多项式:洛朗级数

把函数泰勒展开成关于1/x:1)就有曲率了?2)易收敛?

\subsection{洛朗级数}




\end{document}