%%============
%%  ** Author: Shepherd Qirong
%%  ** Date: 2022-05-06 20:13:17
%%  ** Github: https://github.com/ShepherdQR
%%  ** LastEditors: Shepherd Qirong
%%  ** LastEditTime: 2022-06-12 13:04:13
%%  ** Copyright (c) 2019--20xx Shepherd Qirong. All rights reserved.
%%============


\documentclass[UTF8]{../09-Mathematics}
\begin{document}

\title{09-07-Analysis}
\date{Created on 20220605.\\   Last modified on \today.}
\maketitle
\tableofcontents


\chapter{Introduction}

%% 包括4章, 各章较为独立



\chapter{Mathematical Analysis}
a: 微分学, 
b: 积分学, 
c: 级数论,

\section{微分学}
\section{积分学}
\section{级数论}



\chapter{Non-standard analysis}

概念上又可称为实无限分析



\chapter{function theory}
函数论
a: 实变函数论, 
b: 单复变函数论, 
c: 多复变函数论, 
d: 函数逼近论, 
e: 调和分析, 
f: 复流形, 
g: 特殊函数论, 
h: 函数论其他学科

\section{实变函数论}
\section{单复变函数论}
\section{多复变函数论}
\section{函数逼近论}
\section{调和分析}
\section{复流形}
\section{特殊函数论}
\section{函数论其他学科}




\chapter{Functional Analysis}

泛函分析
a: 线性算子理论, 
b: 变分法, 
c: 拓扑线性空间, 
d: 希尔伯特空间, 
e: 函数空间, 
f: 巴拿赫空间, 
g: 算子代数 
h: 测度与积分, 
i: 广义函数论, 
j: 非线性泛函分析, 
k: 泛函分析其他学科。


\section{线性算子理论}

\section{变分法}
学习材料 张恭庆。

参考书目:
1 G.Buttazzo,M. Giaquinta, S.Hildebrandt. One-dimensional Variational Problems, An Introduction. Clarendon Press, Oxford, 1998.
2 I.M.Gelfand, S.V.Fomin. Calculus of Variations(English translated by R.A.Silverman). Prentice Hall, 1964.
3 J.Jost, X Li-Jost, Calculus of Variations. Cambridge University Press, 1998.
4 张恭庆, 变分学讲义, 高等教育出版社

\subsection{Introduction}
泛函极值、临界值问题。临界值问题的处理需要引入拓扑学的知识形成“大范围变分学”。

找到求极值的必要条件、充分条件,变分问题的求解时通过必要条件转化为常微分方程、偏微分方程求解。真正能用初等方法求解的常微分方程很有限。Dirichlet指出特殊的调和方程可以通过变分来做,漏洞是微分方程解的存在性转化为变分问题解的存在性,Dirichlet不清楚存在不存在,认为转化后解自然存在。Weierstrass专门做数学严格化的人,黎曼证明复变函数中的保形映射的存在性定理是依赖Dirichlet原理。Hilbert把Dirichlet说清楚了,说明了为什么解是存在的。从Hilbert后研究解的存在性。

变分析应用的学科:力学、物理、ODE(动力系统)、PDE(特别是椭圆型方程)、几何(如极小曲面、黎曼几何、Finsler几何、测地线、调和映射、Yang-Mils方程等,另外一类是辛几何)、拓扑学(福伦同调)、实分析、泛函分析、经济学(拉姆塞)、控制论(苏联盲人庞特里亚金)、工程最优设计、图像处理、分形(具有微分和积分的不等式、等周不等式等)、数值方法(有限元即变分问题离散化、最优化方法)、概率论(随机变分理论)。

欧拉之后2个重要的发展:\\
变分->几何(辛几何)->拓扑->大范围变分。\\
与数值方法结合,产生有限元和最优化方法,用于工程应用。


起源于1696年约翰-伯努利的最速降线问题。牛顿、莱布尼兹都有解法。其中约翰用光的折射做出来,约翰的哥哥雅各布-伯努利用变分的思想做出来。Euler, Lagrange, Weierstrass, Hamilton, Jacobi等都有很大贡献。

\subsection{经典变分学}

诺特定理。守恒律是给定泛函在群作用下的不变性。

 


\subsection{存在性}
\subsection{应用}


\section{拓扑线性空间}
\section{希尔伯特空间}
\section{函数空间}
\section{巴拿赫空间}
\section{算子代数}
\section{测度与积分}
\section{广义函数论}
\section{非线性泛函分析}
\section{泛函分析其他学科}


\end{document}


