%%============
%%  ** Author: Shepherd Qirong
%%  ** Date: 2022-05-06 20:13:17
%%  ** Github: https://github.com/ShepherdQR
%%  ** LastEditors: Qirong ZHANG
%%  ** LastEditTime: 2024-12-02 22:24:35
%%  ** Copyright (c) 2019--20xx Shepherd Qirong. All rights reserved.
%%============


\documentclass[UTF8]{../09-Mathematics}
\begin{document}

\title{09-07-Analysis}
\date{Created on 20220605.\\   Last modified on \today.}
\maketitle
\tableofcontents


\chapter{Introduction}

%% 包括4章, 各章较为独立



\chapter{Mathematical Analysis}%% empty here!!
in 09-07-01-MathematicalAnalysis.tex







\chapter{Non-standard analysis}

概念上又可称为实无限分析



\chapter{function theory}%% empty here!!








\chapter{变分法}
学习材料 张恭庆。

参考书目:
1 G.Buttazzo,M. Giaquinta, S.Hildebrandt. One-dimensional Variational Problems, An Introduction. Clarendon Press, Oxford, 1998.
2 I.M.Gelfand, S.V.Fomin. Calculus of Variations(English translated by R.A.Silverman). Prentice Hall, 1964.
3 J.Jost, X Li-Jost, Calculus of Variations. Cambridge University Press, 1998.
4 张恭庆, 变分学讲义, 高等教育出版社

\section{Introduction}
泛函极值、临界值问题。临界值问题的处理需要引入拓扑学的知识形成“大范围变分学”。

找到求极值的必要条件、充分条件,变分问题的求解时通过必要条件转化为常微分方程、偏微分方程求解。真正能用初等方法求解的常微分方程很有限。Dirichlet指出特殊的调和方程可以通过变分来做,漏洞是微分方程解的存在性转化为变分问题解的存在性,Dirichlet不清楚存在不存在,认为转化后解自然存在。Weierstrass专门做数学严格化的人,黎曼证明复变函数中的保形映射的存在性定理是依赖Dirichlet原理。Hilbert把Dirichlet说清楚了,说明了为什么解是存在的。从Hilbert后研究解的存在性。

变分析应用的学科:力学、物理、ODE(动力系统)、PDE(特别是椭圆型方程)、几何(如极小曲面、黎曼几何、Finsler几何、测地线、调和映射、Yang-Mils方程等,另外一类是辛几何)、拓扑学(福伦同调)、实分析、泛函分析、经济学(拉姆塞)、控制论(苏联盲人庞特里亚金)、工程最优设计、图像处理、分形(具有微分和积分的不等式、等周不等式等)、数值方法(有限元即变分问题离散化、最优化方法)、概率论(随机变分理论)。

欧拉之后2个重要的发展:\\
变分->几何(辛几何)->拓扑->大范围变分。\\
与数值方法结合,产生有限元和最优化方法,用于工程应用。


起源于1696年约翰-伯努利的最速降线问题。牛顿、莱布尼兹都有解法。其中约翰用光的折射做出来,约翰的哥哥雅各布-伯努利用变分的思想做出来。Euler, Lagrange, Weierstrass, Hamilton, Jacobi等都有很大贡献。

\subsection{经典变分学}

诺特定理。守恒律是给定泛函在群作用下的不变性。


\subsection{存在性}



\section{极小曲面方程}
\section{等周问题}
\section{大范围变分法}


\section{应用}






\chapter{Functional Analysis}

泛函分析


\section{函数空间}
\section{算子代数}
    \subsection{线性算子理论}

\section{测度与积分}


\section{希尔伯特空间及其线性算子理论}
\section{巴拿赫空间及其线性算子理论}
\section{线性空间理论 (向量空间 )}
    \subsection{拓扑线性空间}
    \subsection{半序线性空间}
    \subsection{其他线性空间}

\section{广义函数论}
\section{巴拿赫代数 (赋范代数 )、拓扑代数、抽象调和分析}
\section{积分变换及算子演算}
\section{谱理论}
\section{积分论 (基于泛函分析观点的 )}

\section{非线性泛函分析}
\section{泛函分析的应用}
\section{其他}




\chapter{整体分析、流形上分析}
\section{整体分析、流形上分析、突变理论}
\section{微分动力系统}


\end{document}


