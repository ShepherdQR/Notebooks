%%============
%%  ** Author: Shepherd Qirong
%%  ** Date: 2022-05-06 20:13:17
%%  ** Github: https://github.com/ShepherdQR
%%  ** LastEditors: Shepherd Qirong
%%  ** LastEditTime: 2022-06-05 22:48:29
%%  ** Copyright (c) 2019--20xx Shepherd Qirong. All rights reserved.
%%============


\documentclass[UTF8]{../09-Mathematics}
\begin{document}

\title{09-07-Analysis}
\date{Created on 20220605.\\   Last modified on \today.}
\maketitle
\tableofcontents


\chapter{Introduction}

%% 包括4章, 各章较为独立



\chapter{Mathematical Analysis}
a: 微分学, 
b: 积分学, 
c: 级数论,

\section{微分学}
\section{积分学}
\section{级数论}



\chapter{Non-standard analysis}

概念上又可称为实无限分析



\chapter{function theory}
函数论
a: 实变函数论, 
b: 单复变函数论, 
c: 多复变函数论, 
d: 函数逼近论, 
e: 调和分析, 
f: 复流形, 
g: 特殊函数论, 
h: 函数论其他学科

\section{实变函数论}
\section{单复变函数论}
\section{多复变函数论}
\section{函数逼近论}
\section{调和分析}
\section{复流形}
\section{特殊函数论}
\section{函数论其他学科}




\chapter{Functional Analysis}

泛函分析
a: 线性算子理论, 
b: 变分法, 
c: 拓扑线性空间, 
d: 希尔伯特空间, 
e: 函数空间, 
f: 巴拿赫空间, 
g: 算子代数 
h: 测度与积分, 
i: 广义函数论, 
j: 非线性泛函分析, 
k: 泛函分析其他学科。


\section{线性算子理论}
\section{变分法}
\section{拓扑线性空间}
\section{希尔伯特空间}
\section{函数空间}
\section{巴拿赫空间}
\section{算子代数}
\section{测度与积分}
\section{广义函数论}
\section{非线性泛函分析}
\section{泛函分析其他学科}


\end{document}


