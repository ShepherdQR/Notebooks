%%============
%%  ** Author: Shepherd Qirong
%%  ** Date: 2022-05-06 20:40:40
%%  ** Github: https://github.com/ShepherdQR
%%  ** LastEditors: Shepherd Qirong
%%  ** LastEditTime: 2022-06-05 23:12:43
%%  ** Copyright (c) 2019--20xx Shepherd Qirong. All rights reserved.
%%============

\documentclass[UTF8]{../09-Mathematics}
\begin{document}

\title{09-14-Others}
\date{Created on 20220605.\\   Last modified on \today.}
\maketitle
\tableofcontents


\chapter{Introduction}




\chapter{combinatorics}
组合数学,离散数学

广义的组合数学就是离散数学,狭义的组合数学是离散数学除图论、代数结构、数理逻辑等的部分。

狭义的组合数学主要研究满足一定条件的组态(也称组合模型)的存在、计数以及构造等方面的问题。 组合数学的主要内容有组合计数、组合设计、组合矩阵、组合优化(最佳组合)等。

\chapter{fuzzy mathematics}
模糊数学

由于模糊性概念已经找到了模糊集的描述方式,人们运用概念进行判断、评价、推理、决策和控制的过程也可以用模糊性数学的方法来描述。例如模糊聚类分析、模糊模式识别、模糊综合评判、模糊决策与模糊预测、模糊控制、模糊信息处理等。

\chapter{Quantum mathematics}

量子数学


\chapter{Applied mathematics}
应用数学


\end{document}
