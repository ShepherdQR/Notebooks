%%============
%%  ** Author: Shepherd Qirong
%%  ** Date: 2022-05-06 20:40:40
%%  ** Github: https://github.com/ShepherdQR
%%  ** LastEditors: Shepherd Qirong
%%  ** LastEditTime: 2023-02-24 22:08:47
%%  ** Copyright (c) 2019--20xx Shepherd Qirong. All rights reserved.
%%============

\documentclass[UTF8]{../09-Mathematics}
\begin{document}

\title{09-15-Questions}
\date{Created on 20230224.\\   Last modified on \today.}
\maketitle
\tableofcontents


\chapter{Introduction}






\chapter{ProbabilityTheory}

\section{状态空间}
\subsection{T0001 连续抛硬币}

Q1: with a random sequence of ${0, 1}$, when comes a subset ${1,0,0}$ A wins, when comes a subset ${1,1,0}$ B wins, otherwise the game keeps. The probability of A wins.


\paragraph{S1} solution 1

We draw the state-transfer graph, and calculate the probability A wins of each state as the initial state.

\begin{figure*}[h]
    \centering
      \begin{tikzpicture}[->,>=stealth',shorten >=1pt, auto, node distance=2cm,thick,
        base node/.style={circle,draw,minimum size=16pt},
        real node/.style={double,circle,draw,minimum size=35pt},
        ]

        \node[shape=circle,draw=black](0) at(0,0){000};
        \node[shape=circle,draw=black](1) at(3,0){001};
        \node[shape=circle,draw=black](2) at(6,2){010};
        \node[shape=circle,draw=black](3) at(6,-2){011};
        %%\node[shape=circle,draw=black,double](3) at(9,3){011};
        \node[shape=rectangle,draw=black](4) at(9,3){100};
        \node[shape=circle,draw=black](5) at(9,1){101};
        \node[shape=rectangle,draw=black](6) at(9,-1){110};
        \node[shape=circle,draw=black](7) at(9,-3){111};

        \path[]
        (0) edge [loop above] node {}(0)
        (0) edge node {}(1)
        (1) edge node {}(2)
        (1) edge node {}(3)
        (2) edge node {}(4)
        (2) edge node {}(5)
        (3) edge node {}(6)
        (3) edge node {}(7)

        (4) edge [ above, out=120, in=45]  node {}(0)
        (4) edge [ above, out=120, in=45]  node {}(1)
        (5) edge [out=120, in=-10] node {}(2)
        (5) edge node {}(3)
        (6) edge [out=45, in=-45] node {}(4)
        (6) edge [out=45, in=-45] node {}(5)
        (7) edge [out=45, in=-45] node {}(6)
        (7) edge [loop right]  node {}(7);

        %\draw[->, out=120, in=-50] (1.north) to (4.south);
        
      \end{tikzpicture}
    \caption{【number of solution】}\label{fig:number_of_solution}
\end{figure*}


  We calculate the probability of the machines stops at state 100.
\begin{equation}
\begin{aligned}
    &P_{100}:&1 \\
    &P_{010}:&\frac{1}{2} + \frac{1}{4} \times [ \frac{1}{2} + \frac{1}{4} \times [ \frac{1}{2} + \frac{1}{4} \times [ ... ] ] ]\\
    & & =\lim_{n \to \infty}\sum_{i=0}^{i=n}\frac{1}{2} \times (\frac{1}{4})^i=a_1\frac{1-q^n}{1-q}=\frac{2}{3} \\
    &P_{001}:&\frac{1}{2}  \times P_{010} = \frac{1}{3}  \\
    &P_{000}:&(1- \lim_{i \to \infty}(\frac{1}{2})^i) \times P_{001} = \frac{1}{3}\\
    &P_{011}:&0 \\
    &P_{110}:&0 \\
    &P_{111}:&0 \\
    &P_{101}:&\frac{1}{2}  \times P_{010} = \frac{1}{3}  \\
    &\therefore P_{A} = \frac{1}{8} \times (1 + \frac{2}{3} + 1 ) = \frac{1}{3} 
\end{aligned}
\end{equation}


\paragraph{S2} solution 2


we write the squence, which step has two posibilities. 
\addtocounter{MaxMatrixCols}{10}
\begin{equation}
    \begin{bmatrix}
\cdots & 0 & 0 & 0 & 0& 0& 0& 0& 0& 0& 0 & \cdots \\
\cdots & 1 & 1& 1& 1& 1& 1& 1& 1& 1& 1&\cdots \\
     \end{bmatrix}
     \longrightarrow
    \begin{bmatrix}
          & 0 & 0 & 0& 0& 0& 0& 0& 0& 0 & \cdots \\
         1 & 1& 1& 1& 1& 1& 1& 1& 1& 1&\cdots \\
    \end{bmatrix}
\end{equation}

It is easily to find that, when we focus to locate the sequence of ${1,0,0}$, we find that the start index of the sequence is ${0, 2,4,6,8, \cdots}$, because the index 1(which means $1,1,0,0$, and that means B wins.), we can also Analysis the index 3, 5 as the same. While the start index of the sequence is ${0,1, 2,3, \cdots}$, therefore the probability of B wins is two times of A wins.

\end{document}
