%%============
%%  ** Author: Shepherd Qirong
%%  ** Date: 2022-05-06 19:57:56
%%  ** Github: https://github.com/ShepherdQR
%%  ** LastEditors: Qirong ZHANG
%%  ** LastEditTime: 2024-12-01 23:16:57
%%  ** Copyright (c) 2019--20xx Shepherd Qirong. All rights reserved.
%%============



\documentclass[UTF8]{../09-Mathematics}
\begin{document}

\title{09-03-NumberTheory}
\date{Created on 20220605.\\   Last modified on \today.}
\maketitle
\tableofcontents


\chapter{Introduction}

\section{Contents}

\begin{lstlisting}
    a: 初等数论
    b: 解析数论
    c: 代数数论
    d: 超越数论
    e: 丢番图逼近
    f: 数的几何(几何数论)
    g: 概率数论
    h: 计算数论
    i: 组合数论
    j: 算术代数几何
    k: 数论其他学科
\end{lstlisting}




\section{Symbol}

\begin{definition}
    模m后余数相同:$a \equiv b (\mod m)$
\end{definition}




\chapter{初等数论}

\section{整数的整除性}

\subsection{因数和倍数}

\subsection{质数和合数}

\subsection{质数分布}
不大于x的质数的个数$\pi(x)$

\begin{proposition}
    质数定理
    $$  \lim_{x \to \infty} \frac{\pi(x)}{x/ \log (x)}  =1  $$
\end{proposition}


\begin{proposition}
    Goldbach猜想:大于4的偶数都是2个奇质数的和。
\end{proposition}


是否存在无数个形如$2^p-1$的数是质数


Fermat数:$F_n = 2^{2^{n}}-1, F_5$不是质数


\subsection{最大公因数和最小公倍数}

最大公因数:$(a, b)$

最小公倍数:$\{a, b\}$

a能被b除尽,即a是b的整数倍:$b|a$

辗转相除法

$a = q \cdot b + r, prove \ that:(a,b) = (r,b)$, which means that $(r+qb, b) = (r, b)$.

Mark as to prove $L =R$, Prove:
\begin{equation}
    \begin{aligned}
        &(1)\\
        &\because L|a, L|b\\
        &\because r = a-qb\\
        &\therefore L|r\\
        &\because L|b\\
        &\therefore L|(r,b) \Rightarrow L|R;\\
        &(2)\\
        &\because R|b, R|r\\
        &\because a = qb + r\\
        &\therefore R|a\\
        &\because R|b\\
        &\therefore R|(a,b) \Rightarrow R|L;\\
        &\because (1) and (2)\\
        &\therefore L = R
    \end{aligned}
\end{equation}

\begin{proposition}
     $ab = (a,b) \cdot \{a, b\}$
\end{proposition}

\begin{proposition}
    $  (a,b) = 1, a|bc \Rightarrow a|c  $
\end{proposition}

\begin{proposition}
    $ a| \prod a_i,  (a,a_1) = \cdots =(a,a_{n-1}) = 1, \Rightarrow a|a_n  $
\end{proposition}

\begin{proposition}
    算术基本定理:不计质因数的次序,正整数分解成质数连乘的形式是唯一的。
    $a =\prod p_i = L =  \prod q_i = R, \because p_1 | R, set \  p_1 = q_1, let \ L = L / p_1, R = R / q_1, keep \ doing, \therefore p_i = q_i  $

    $a =\prod p_i^{n_i}  $

\end{proposition}


\begin{proposition}
    任意4个连续整数的乘积加1是一个平方数$ a(a+1)(a+2)(a+3)+ 1 = qq \Rightarrow a(a+3) \cdot (a+1)(a+2)=(q-1)(q+1) $
\end{proposition}

\begin{proposition}
    a是整数,$6|a(a-1)(2a-1)$.Proof$a = 2m \Rightarrow a(a-1)(2a-1)=2m(9m^2 -6m-(m-1)(m+1)),a = 2m+1 \Rightarrow a(a-1)(2a-1)=2m(9m^2 +6m-(m-1)(m+1))  $
\end{proposition}

\begin{proposition}
    $ a\nmid 2 , a \nmid 3, \Rightarrow 24 | a^2+23$.Proof,分类讨论即可。
\end{proposition}

\begin{proposition}
    $ (a^n, b^n) = (a,b)^n$

    $ (na, nb) = n(a,b)$
\end{proposition}

\begin{proposition}
    $a,b \in \mathbb Z_+, \sqrt[a]{b}$如果不是整数,则不是有理分数。
\end{proposition}

\begin{proposition}
    代数方程$\prod a_ix^i = 0, a_i  \in \mathbb Z$如果有有理数根,根一定是整数。\\
    证明,设$x = \frac{p}{q}$,即需要证明$q = 1$. 带入x,有$\sum \frac{a_i p^i}{q^i} = 0$, 两边乘以$q^n$, $\sum a_i p^iq^{n-i} = 0, \therefore p^n = q \times T \Rightarrow q|p^n, \therefore q = 1$
\end{proposition}

\begin{proposition}
    $A:\{ 4t-1, t \in \mathbb Z\}$,证A中有无限质数。\\

    假设最多有k个,尝试推出矛盾。\\
    (step1) 记质数分解$m = \prod^{n}p_i \in A, \because (4T+1)(4U-1)=4\alpha -1, (4T+1)(4U+1)=4\beta +1, \therefore \exists p_i \in A$, 设A中最多有k个质数,考察$a =\prod^{k}p_i-1 $\\
    (step2) 设a是质数,$\because 1 =\prod^{k}p_i - a, \therefore p_i \nmid a, \therefore $a是第k+1的质数,矛盾;\\
    设a不是质数,a一定有质数$b \in A, b\notin \{ p_i\}$,所以存在第k+1的质数,矛盾。\\
    综上,有无限个。



\end{proposition}

\begin{proposition}
    证明$F_5 = 2^32+1 = 641 \times 6700417$不是质数?
\end{proposition}




\section{进制}

二进制的加减乘除


\section{不定方程}

\subsection{一元不定方程}
$\prod_{i=0} a_ix^i = 0, a_i  \in \mathbb Z$,对于整数解$\alpha$, we have$a_0 = -\prod_{i=1} a_i\alpha^i \Rightarrow \alpha | a_0 $


\subsection{二元一次不定方程}

$ax + by = c, a \neq 0, b \neq 0, a, b, c \in \mathbb Z$.方程总可化简,直到$(a,b) = 1$
该型方程找到特解$x_1, y_1$后,通解:$x = x_1 + bu, y = y_1 + au, u \in \mathbb Z$


\begin{proposition}
    $ (a,b) = 1 \Rightarrow \exists x,y \in \mathbb Z, ax+by = 1$

    Prove:\\
    (step1) for set $A:\{  ax+by | a,b \ is \ fixed\}$, we have$c_1, c_2 \in A \Rightarrow c_1 + c_2 \in A$.\\
    (step2) $a>b, let \ b = r_0$, we have
    \begin{equation}
        \begin{bmatrix}
           a= q_{1}r_{0} + r_{1} & (a,r_0) = (r_0, r_1) & r_1 = a -  q_{1}r_{0}\\
           r_0= q_{2}r_{1} + r_{2} & (r_0,r_1) = (r_1, r_2) & r_2 = r_0 -  q_{2}r_{1}\\
           \vdots & \vdots & \vdots\\
           r_n= q_{n+2}r_{n+1} + r_{n+2} & (r_{n},r_{n+1}) = (r_{n+1}, r_{n+2}) & r_{n+2} = r_n - q_{n+2}r_{n+1}\\
           r_{n+1}= q_{n+3}r_{n+2} + 0 & (r_{n+1},r_{n+2}) = r_{n+2} & 0= r_{n+1} - q_{n+3}r_{n+2}\\
        \end{bmatrix}
    \end{equation} 
    from column 2, we have $(a,r_0) = r_{n+2} = 1$. From cloumn 3, and $\because a, b \in A, \therefore r_i \in A, \therefore \exists x, y, ax+by = r_{n+2} = 1$

\end{proposition}




\subsection{勾股数}

$ x^2 + y^2 = z^2$, 做如下限定后$x,y,z \in \mathbb Z_+, (x,y) = 1, 2|x$, 有:$x = 2ab, y = a^2-b^2, z = a^2 + b^2, a>b, (a,b) = 1, 2\nmid (a+b)$


\begin{proposition}
    整数边长的直角三角形,斜边与一直角边长差1,3个边可表示成:$ 2b+1, 2b^2+2b, 2b^2+2b+1, b \in \mathbb Z$

    Proof $x^2 + y^2 = z^2$, 改写成等式集合A$x = 2ab, y = a^2-b^2, z = a^2 + b^2$, let $z= x+1$, so $a^2 + b^2 -2ab = 1\Rightarrow a=b+1$,带入等式集合A,即得。
\end{proposition}

\subsection{费马问题}
$ x^n + y^n = z^n$,这个不定方程没有正整数解。


\begin{proposition}
    $x^4 + y^4 = z^4$没有整数解
\end{proposition}
证明$x^4 + y^4 = z^4$没有整数解。令$u = z^2$, 即证$x^4 + y^4 = u^2$没有整数解。

step1)设存在解,即最小的正解为$u_1$,证明$(x,y) = 1$\\
设$(x,y) =d> 1, \because d^4|x^4, d^4|y^4, \Rightarrow (\frac{x}{d})^4+(\frac{y}{d})^4=(\frac{u_1}{d^2})^2 $, $ \because \frac{u_1}{d^2} < u_1$,矛盾,即证。

step2)$(x,y) = 1$,so x, y 是2个奇数,或是1奇1偶。分类讨论都是不可能的。\\

step2.1)证明不可能是2个奇数。\\
假设是2个奇数,$x = 2m+1, n=2n+1$, $L = x^4 + y^4 = (2m+1)^4 + (2n+1)^4 = 4T+2$, so $2|L= R = u^2, 4\nmid L= R = u^2$,不存在这样的u,所以不能是2个奇数。

step2.2)证明不可能是1奇1偶。\\
$x^4 + y^4 = u_1^2$改写为$(x^2)^2 + (y^2)^2 = u_1^2$,可进一步改为:
$x^2 = 2ab, y^2 = a^2-b^2, u_1 = a^2 + b^2, a>b, (a,b) = 1, 2\nmid (a+b)$

step2.2.1)设$a=2n, b=2m+1$ \\
$y^2 = a^2-b^2 \Rightarrow a^2  = b^2 + y^2 = 4U+2, \therefore 4 \nmid a^2$, 与$a=2n$矛盾。

step2.2.2)设$a=2m+1, b=2n$ \\
$\because (a,b) = 1, \therefore (a,m) = 1$, and $\because x^2 = 2ab,\therefore (\frac{x}{2})^2 = am$,因为a和m互质,所以a需要能分解为$a = c^2$, 即$am = c^2 d^2, (c,d) = 1, \therefore 2 \nmid c, b=2m = 2d^2$, \\
$y^2 = a^2-b^2 \Rightarrow b^2 + y^2 = a^2   \Rightarrow (2d^2)^2 + y^2= (c^2)^2 $,可改写为$2d^2 = 2kl, y=k^2-l^2, c^2 = k^2+l^2, (k,l)=1, d^2 = kl$\\
$d^2 = kl$,所以k和l可分解为$k = K^2, l=L^2, \therefore c^2 = K^4 + L^4$\\
$c \leqslant c^2 = a\leqslant a^2<a^2+b^2 = u_1 $,与$u_1$最小的正整数解矛盾。

step3)综上,即证不存在。


\begin{proposition}
    证明整数方程没有整数解:$x^4-4y^4 = z^2, x,y,z \in \mathbb Z$

    Proof:两边平方,有$z^4 = (x^4+4y^4)^2 - 16x^4y^4 \Rightarrow (2xy)^4 + z^4 = (x^4+4y^4)^2$,此式无解,所以原式无解。
\end{proposition}

\section{一次同余式}

\subsection{同余}


\subsubsection{同余性质}

\begin{proposition}
    reflection: $a \equiv a (\mod m) $

    symmetry:$ a \equiv b (\mod m) \Rightarrow b \equiv a (\mod m)$

    transitivity:
    $a \equiv b (\mod m), b \equiv c (\mod m) \Rightarrow a \equiv c (\mod m)$
    Proof: $a-b = q_1 m, b-c = q_2 m, \therefore a-c = tm$
\end{proposition}

\begin{proposition}\label{proposition_congruence_ka}
    $(m, n) = 1, ac \equiv bc (\mod m) \Rightarrow a \equiv b (\mod m)$
    Proof: $ac-bc = q m, \therefore a-b = tm$
\end{proposition}

\begin{proposition}
    $a \equiv b (\mod m) \Rightarrow a^n \equiv b^n (\mod m)$
    Proof: $a-b = q m, and \  a^n  =(b + qm)^n   \therefore a^n-b^n = tm$
\end{proposition}


\subsubsection{应用}

\begin{proposition}
    $10^n \mod 9 \equiv 1$, for example, $5874192 \mod 9 = (5+8+7+4+1+2) \mod 9  = 0$
\end{proposition}

\begin{proposition}
    $(a \times b) \mod 9 =  ((a \mod 9 )\times (b \mod 9) ) \mod 9 $

    $28997 \times 39459  \neq 1144192613, L = 8 \times 3 = 6 \neq 5=R$, 不相等一定没有算对,但是相等却不一定算对。
\end{proposition}


\begin{proposition}
    $(a,m) \nmid b \Rightarrow (ax+b) \mod (m) \neq 0$. Prove:$suppose \ \exists c, m | (ac+b), \therefore \exists \alpha, \alpha m = ac+b \Rightarrow b=\alpha m -ac, \because (a,m) = L, \therefore b = \alpha L, \therefore L | b$, 矛盾,即证。

    例:$2x \equiv 179(mod 562)$没有整数解
\end{proposition}

\begin{proposition}
    $(a,m) =1, m \nmid a \Rightarrow \exists x, m | (ax+b)$,证明$\exists ax+my = z, z = -b$

    例:$256x \equiv 179 (mod 337)$有整数解
\end{proposition}


\begin{proposition}
    $ad \equiv bd (\mod md) \Rightarrow a \equiv b (\mod m)$, 证明,改写一下即显然$md |(ad-bd) \Rightarrow m|(a-b)$
\end{proposition}


\begin{proposition}
    $1935|(1296x-1125) \Rightarrow 215|144x-125, x = 80,295,510,725,940,1155,1370,1585,1800$?
\end{proposition}

\subsection{孙子定理}
解同余式组
\begin{proposition}
    $x \equiv a(\mod 3), x \equiv b(\mod 5),x \equiv c(\mod 7) \Rightarrow x = 70a + 21b + 15c (\mod 105)$
\end{proposition}


\begin{proposition}
    $\{ m_k\}, \forall i,j, (m_i, m_j) = 1, \prod m_i = m_i M_i$,方程组$x \equiv b_i (\mod m_i)$的解为$x = (\sum b_i M'_i M_i)(\mod \prod m_i),  M'_i M_i \equiv 1(\mod m_i)$.\\
    Prove: $i = j, (m_i, M_j) = 1, \therefore \exists n_i, M'_j,  n_i m_i  + M'_j M_j = 1 \Rightarrow M'_j M_j \equiv 1(\mod m_i)$\\
    $i \neq j, m_i|M_j, \therefore \exists b_j, b_j M'_j M_j \equiv 0 (\mod m_i), \therefore \sum b_j M'_j M_j \equiv b_i M'_i M_i \equiv b_i (\mod m_i) $

    例:$1= x \mod 2, 2= x \mod 5, 3= x \mod 7, 4= x \mod 9$, 
    解$M = 2 \times 5 \times 7 \times 9 = 630, M_i =[315,126,90,70], M_i' =[1,1,6,4], \therefore x = 315 + 2\times 126 + 3 \times 6\times 90 + 4\times 4\times 70 = 157(\mod 630), \therefore x = 157 + 630k, k\in \mathbb Z$
\end{proposition}

\begin{proposition}
    $a \equiv x \mod m_1  \equiv x \mod m_2$,所有解是$x \equiv a \mod \{ m_1, m_2\}$, 证明的话,两边改写一下即可$m_1|(a-x),m_2|(a-x),\{ m_1, m_2\}|(a-x) $
\end{proposition}

\begin{proposition}
    $(m_1, m_2) = d, d|(b_1, b_2)$, 方程组A$x \equiv b_1(\mod m_1  ), x \equiv b_2(\mod m_2  )$,解为$x \equiv x_0(\mod (\{ m_1, m_2\}))$,其中$x_0$是方程组A的解。
\end{proposition}

\begin{proposition}
    $(n_i, n_j) = 1, n_i | m_i, \{n_1, \cdots, n_k  \}= \{m_1, \cdots, m_k  \}, \therefore $,方程组$x \equiv b_i (\mod m_i)$与方程组$x \equiv b_i (\mod n_i)$同解
\end{proposition}


\section{剩余系}



\subsection{完全剩余系}

Complete residue system

\begin{proposition}
    m的完全剩余系$\forall k \in \mathbb Z, \varphi_m (k) =  k \mod m = \alpha \in A = \{ 0, 1, \cdots, m-1 \}$, a set B, that $ \varphi_m(B) = A$
    Proof: $a-b = q m, and \  a^n  =(b + qm)^n   \therefore a^n-b^n = tm$
\end{proposition}

\begin{proposition}
    m的完全剩余系K, $\forall a, b \in K, a \neq b (\mod m)$

    给定集合T, $\forall a, b \in T, a \neq b (\mod m)$, therefore T是m的完全剩余系K
\end{proposition}

\begin{proposition}
    m的完全剩余系K, K的每个元素加a,得到的集合仍是完全剩余系。相当于平移$a\mod m$

    m的完全剩余系K,$(b,m)= 1$, K的每个元素乘以b,得到的集合仍是完全剩余系。证明参考\ref{proposition_congruence_ka}。
\end{proposition}

\subsubsection{应用}

\begin{proposition}\label{Complete_residue_system_Porduct}
    $m_1, m_2$的完全剩余系记为$R_{m_1},R_{m_2}, (m_1,m_2) = 1, \therefore R_{m_1m_2} = \{m_2x_1 + m_1x_2 | x_1 \in R_{m_1},x_2 \in R_{m_2} \}$

    Proof,即证明这$m_1m_2$个数对$m_1m_2$不同余。
    $m_2x_1 + m_1x_2 \equiv m_2y_1 + m_1y_2 (\mod m_1m_2)$, therefore
    $m_2(x_1-y_1)= m_1m_2q - m_1(x_2-y_2)$,therefore
    $m_1 | m_2(x_1-y_1) \Rightarrow m_1 | (x_1-y_1)$,therefore
    $x_1 \equiv y_1 (\mod m_1)$,即证。
\end{proposition}

\begin{proposition}
    $\{m_i\}$是k个互质的正整数,$x_i \in R_{m_i},\prod m_i = m_iM_i$, therefore $\{\sum M_i x_i \} =  R_{\prod m_i}$.证明如\ref{Complete_residue_system_Porduct}

    $\{m_i\}$是k个互质的正整数,$x_i \in R_{m_i} $, therefore 
    $$\{x_1 + m_1 x_2 + m_1m_2x_3 + \cdots + m_1m_2\cdots m_{k-1}x_k \} =  R_{\prod m_i}$$
    证明如\ref{Complete_residue_system_Porduct}
\end{proposition}



\subsection{简化剩余系}
\begin{proposition}
    m的简化剩余系:m的完全剩余系中,挑出与m互质的,包括1+mk不包括km。

    m的简化剩余系K,$(b,m)= 1$, K的每个元素乘以b,得到的集合仍是简化剩余系。
\end{proposition}



\subsection{欧拉函数、欧拉定理、费马定理}

\subsubsection{欧拉函数}

\begin{definition}
    欧拉函数$\varphi(m)$:不大于m的和m互质的正整数的个数。

    对于质数p,有$\varphi(p^l) = p^l - p^{l-1} $
    Proof:$1p, 2p, \cdots, p^{l-1}p$是p的倍数,即证。
\end{definition}

\begin{proposition}
    $a = \prod p_i ^{a_i} \Rightarrow \varphi(a) = \prod p_i^{a_i-1}(p_i-1)$

    Proof: n=1,is obvious.n=2, $p_1$的倍数$1p_1, 2p_i, \cdots, \frac{a}{p_1}p_1$,有$\frac{a}{p_1}$个,考虑$p_1p_2$的倍数,所以$\varphi(a) = a(1-\frac{a}{p_1}-\frac{a}{p_1}+\frac{a}{p_1p_2}) = a(1-\frac{a}{p_1})(1-\frac{a}{p_2})$, 继续考虑下去可以证明。

    It is easy to see that $(a,b) = 1 \Rightarrow \varphi(ab) = \varphi(a)\varphi(b)$
\end{proposition}

\begin{proposition}
    $\forall m>2, 2 |\varphi(m)$

    Proof:如果m因数分解后,若m的因数含有2,易知成立;m的因数没有2则肯定有一个奇数p,$p-1$是偶数,即证。

    因而,不大于m的和m互质的正整数,之和是$\frac{m \varphi(m)}{2}$, 证明:从小到大排列后,$\{n_k\}$, $(n_{i}, n_{\varphi(m)-i}) = 1$,求和即证。m等于1的时候也成立。
\end{proposition}

\begin{proposition}
    质数p$$\sum_0\varphi(p^i) = p^n$$
\end{proposition}





\subsubsection{欧拉定理}
\begin{definition}
    欧拉定理:$(a,m) = 1, a^{\varphi_m} \equiv 1 (\mod m)$

    Proof:对于m的简化剩余系中的元素$a_i$, we have$aa_i = a_j \Rightarrow \prod_{\varphi_m}aa_i = \prod_{\varphi_m}a_i$,即证。
\end{definition}


\begin{definition}
    费马定理:对于质数p,$p \nmid a, a^{p-1} \equiv 1 (\mod m)$

    Proof:a整除不了的质数p和a互质,$\varphi_p = p-1$,带入欧拉定理,即证。
\end{definition}

\subsubsection{应用}

\begin{proposition}
    今天周六,$t = a^{b^c}$天后是周几呢?
    Answer: $a \equiv a_1(\mod 7), 0 \leqslant a_1 leqslant 6 $, $a_1 = 0$ is Saturday too. When $1 \leqslant a_1 leqslant 6$, 费马定理$\because (a_1, 7) =1, \therefore a_1^6 \equiv 1 (\mod 7)$. \\
    $b \equiv b_1(\mod 6)$, $b_1 = 0,a_1^{b^c}\equiv 1(\mod 7)$\\
    $b_1 = 1, b^c = 6n+1, \therefore a_1^{b^c}\equiv a_1(\mod 7)$\\
    $b_2 = 2, b^c = 6n+2,or\ b^c = 6n+4 \therefore a_1^{2}, a_1^{4}$\\
    $b_2 = 3, b^c = 6n+3 \therefore a_1^{3}$\\
    $b_2 = 4, b^c = 6n+4 \therefore a_1^{4}$\\
    $b_2 = 5, b^c = 6n+5 \therefore a_1^{5}$\\
    \begin{equation}
        \begin{bmatrix}
           a_1^1 & a_1^2 & a_1^3 & a_1^4 & a_1^5\\
           1 & 1 & 1 & 1 & 1\\
           2 & 4 & 1 & 2 & 4\\
           3 & 2 & 6 & 4 & 5\\
           4 & 2 & 1 & 4 & 2\\
           5 & 4 & 6 & 2 & 3\\
           6 & 1 & 6 & 1 & 6\\
        \end{bmatrix}
      \end{equation}
    例如,$t = 773^{3169^c}, a_1 = 1, b\equiv 1(\mod 6), \therefore 3$,如今天周日则t天后是周三
\end{proposition}


\begin{proposition}
    $(a+b)(\mod m) =[a(\mod m)+b(\mod m)](\mod m) =[a(\mod m)+b(\mod m)]^n(\mod m)$

    Proof, compare$ (a+b)^2, (tm+a_1 + b)^2$, we can see it is true.
\end{proposition}

\begin{proposition}
    $(12371^{56}+34)^{28}(\mod 111) 
    \equiv = (50^{56}+34)^{28}
    \equiv ((125000^9 \times 50)^2+34)^{28}
    \equiv ((14^9 \times 50)^2+34)^{28}
    \equiv ((14^9 \times 50)^2+34)^{28}
    \equiv ((80^3 \times 50)^2+34)^{28}
    \equiv ((68 \times 50)^2+34)^{28}
    \equiv ((70)^2+34)^{28}
    \equiv ((70)^2+34)^{28}
    \equiv 70
    $

    $\varphi(111) = 72 \Rightarrow (12371^{56}+34)^{72c} \equiv 1 (\mod 111)$
\end{proposition}


\begin{proposition}
    $3^{8232010}-3^{10} \equiv 0 (\mod 24010000)$
    Proof:$t =24010000 = 2^4\times 5^4\times7^4, \therefore \varphi(t)=8232000,3^{\varphi(t) \equiv 1(\mod t)}, \therefore 3^{8232010} \Rightarrow 3^{10} $
\end{proposition}

\begin{proposition}
    $21 | (121^6-1)$

    Proof: $\varphi(21) = 12, \therefore 11^{12} \equiv 1 (\mod 21), \Box $
\end{proposition}

\begin{proposition}
    $primer \ p, p \neq 2, p \neq 5, p|9\cdots 9, (p-1)k$个9.
    
    Proof: $(p,10) = 1, (10^k, p) = 1, \therefore (10^k)^{p-1} \equiv 1 (\mod p), \Box $
\end{proposition}

\begin{proposition}
    $641 | F_5$
    
    Proof: equals to prove $641 |(2^{32}+1) 
    \Rightarrow 640 \equiv -1(\mod 641)
    \Rightarrow 5 \times 2^7 \equiv -1(\mod 641)
    \Rightarrow 5^4 \times 2^{28} \equiv 1(\mod 641)
    $. And $ \because 5^4 \equiv -2^4(\mod 641), \therefore -2^{32} \equiv 1(\mod 641), \Box$
\end{proposition}

\begin{proposition}
    primer p, $(\sum a_i)^p \equiv (\sum(a_i^p) )(\mod p)$
    
    Proof: we need to prove $L \equiv R(\mod p)$
    
    (step1) $\exists k, p|a_k, L = (\sum_{i \neq k}a_i + a_k)^p = R+Tp$

    (step2)$\forall k, p \nmid a_k, \because a_i^{p-1} \equiv 1 ( \mod p), \therefore a_i^{p} \equiv a_i ( \mod p), \therefore R \equiv (\sum a_i)(\mod p)$, so we need to prove $(\sum a_i)^p \equiv (\sum a_i)(\mod p)$. Mark $S = \sum a_i$, if$p|S$, is obvious. If$p\nmid S, S^{p-1} \equiv 1 (\mod p), \Box$
\end{proposition}


\begin{proposition}
    $1978^m \equiv 1978^n(\mod 1000), \min(m+n), m? n?$
    
    Solve: $1000 | 1978^{n-m} \Rightarrow  (2^3 \times 5^3) | 2^m \times 989^m (1978^{n-m}-1)$.

    because $989^m (1978^{n-m}-1)$ is odd, therefore $m \geqslant 3$, and therefore $5^3 | (1978^{n-m}-1), \therefore 1978^{n-m} \equiv 1 (\mod 125)$. $\because \varphi(125) = 100, \therefore 1978^{100}\equiv 1 (\mod 125)$.

    We now prove that $(n-m)|100$. Suppose $(n-m) \neq 100,  \therefore 100 = q(n-m) + r, \therefore 1978^{r}\equiv 1 (\mod 125), \because r < n-m$, 矛盾。

    For now, we have $(n-m)|100, 125 | (1978^{n-m}-1)$. $\because 125 | (1978^{n-m}-1) \therefore 1 | 1978^{n-m}, or \ 6 | 1978^{n-m}$. $\because (n-m)|100, \therefore 4 | (m-n), \therefore m-n = 4,20,100$. 

    For now we test if $1978^4 \equiv 1 (\mod 125)$.
    $L \equiv (125 \times 15 + 103)^4 
    \equiv 103^4 \equiv (3+4 \times 5^2)^{2+2} \equiv (3^2 + 2 \times 3 \times 4 \times 5^2)^2\equiv 609^2 \equiv (-16^2) \equiv 6$

    For now we test if $1978^{20 \times 5} \equiv 1 (\mod 125)$.
    $L \equiv 6^{25} \equiv [36 \times(125+91)]^5 \equiv (36 \times 91)^5 \equiv 26^5 $.We can see that for 20, we have 26, and 100 we have 1. So $m = 3, n = 103$.
\end{proposition}



\begin{proposition}
    分针目前在12,问分针走$a^{b^c}$,是几点。

\begin{equation}
    \begin{bmatrix}
       a_1^1 & a_1^2 & a_1^3 & a_1^4\\
       0 & \cdots &  &  \\
       1 & \cdots &  &  \\
       2 & 4 & 8 & 4\cdots  \\
       3 & 9 & 3\cdots & \\
       4 & 4\cdots &  &  \\
       5 & 1\cdots &  &  \\
       6 & 0\cdots &  &  \\
       7 & 1\cdots &  &  \\
       8 & 4 & 8\cdots &  \\
       9 & 9\cdots &  &   \\
       10 & 4 & 4\cdots &   \\
       11 & 1\cdots & &   \\
    \end{bmatrix}
\end{equation}


\begin{table}[htbp]
\newcommand{\tabincell}[2]{\begin{tabular}{@{}#1@{}}#2\end
{tabular}}
\centering
    \caption{mode12}
    \label{tab:mode12}
    \begin{tabular}{cc}
    \toprule
    $a_1$ & 讨论\\
    \midrule
    \multirow{2}{*}{1,5,7,11}   & b is odd: $a_1$\\
                                & b is even: 1\\
    \midrule
    \multirow{3}{*}{2}          & b = 1: 2\\
                                & b is even: 4\\
                                & b is odd and $b>1$: 8\\
    \midrule
    \multirow{2}{*}{3}          & b is odd: 3\\
                                & b is even: 9\\
    \midrule
    \multirow{1}{*}{4}          & 4\\
    \midrule
    \multirow{2}{*}{6}          & b =1: 6\\
                                & b >1: 0\\
    \midrule
    \multirow{2}{*}{8}          & b is odd: 8\\
                                & b is even: 4\\
    \midrule
    \multirow{1}{*}{9}          & 9\\
    \midrule
    \multirow{2}{*}{10}          & b =1: 10\\
                                & b >1: 4\\
    \bottomrule
    \end{tabular}
\end{table}

\end{proposition}

\section{小数、分数、实数}





\section{连分数}

\subsection{连分数基本性质}

\subsection{无限连分数}

\subsection{数论函数}




\section{复数和三角和}





\chapter{代数数论}
    \section{代数数域、域扩张}
    \section{局部数域}
    \section{分圆域}
    \section{类域论}

\chapter{数的几何(几何数论)}



\chapter{解析数论}

\chapter{二次型 (二次齐式 )}



\chapter{超越数论}
\chapter{丢番图逼近}





\chapter{数论其他学科}

\section{概率数论}
\section{计算数论}
\section{组合数论}
\section{算术代数几何}


\end{document}


