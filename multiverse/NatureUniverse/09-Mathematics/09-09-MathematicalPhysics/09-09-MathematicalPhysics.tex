%%============
%%  ** Author: Shepherd Qirong
%%  ** Date: 2022-05-06 20:29:05
%%  ** Github: https://github.com/ShepherdQR
%%  ** LastEditors: Shepherd Qirong
%%  ** LastEditTime: 2022-06-05 22:56:47
%%  ** Copyright (c) 2019--20xx Shepherd Qirong. All rights reserved.
%%============


\documentclass[UTF8]{../09-Mathematics}
\begin{document}

\title{09-09-MathematicalPhysics}
\date{Created on 20220605.\\   Last modified on \today.}
\maketitle
\tableofcontents


\chapter{Introduction}

动力系统
a: 微分动力系统, b: 拓扑动力系统, c: 复动力系统, d: 动力系统其他学科。



1、微分方程的解算: 很多物理问题, 比如在经典力学和量子力学中求解运动方程, 都可以被归结为求解一定边界条件下的微分方程。因此求解微分方程成为数学物理的最重要组成部分。相关的数学工具包括: 
常微分方程的求解
偏微分方程求解
特殊函数
积分变换
复变函数论


2、场的研究(场论): 场是现代物理的主要研究对象。电动力学研究电磁场;广义相对论研究引力场;规范场论研究规范场。对不同的场要应用不同的数学工具, 包括: 
矢量分析
张量分析
微分几何


3、对称性的研究: 对称性是物理中的重要概念。它是守恒律的基础, 在晶体学和量子场论中都有重要应用。对称性由对称群或相关的代数结构描述, 研究它的数学工具是: 
群论
表示论

4、作用量(action)理论: 作用量理论被广泛应用于物理学的各个领域, 例如分析力学和路径积分。相关的数学工具包括: 
变分法
泛函分析


\chapter{动力系统}

\section{微分动力系统}
\section{拓扑动力系统}
\section{复动力系统}
\section{动力系统其他学科}


\chapter{微分方程的解算}
\chapter{场的研究(场论)}
\chapter{对称性的研究}
\chapter{作用量(action)理论}



\end{document}
