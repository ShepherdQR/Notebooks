%%============
%%  ** Author: Shepherd Qirong
%%  ** Date: 2022-05-06 20:21:51
%%  ** Github: https://github.com/ShepherdQR
%%  ** LastEditors: Shepherd Qirong
%%  ** LastEditTime: 2022-06-12 15:05:47
%%  ** Copyright (c) 2019--20xx Shepherd Qirong. All rights reserved.
%%============


\documentclass[UTF8]{../09-Mathematics}
\begin{document}

\title{09-08-Equation}
\date{Created on 20220605.\\   Last modified on \today.}
\maketitle
\tableofcontents


\chapter{Introduction}

%% 包括3章

\chapter{Ordinary differential equation}

常微分方程
a: 定性理论, 
b: 稳定性理论, 
c: 解析理论,
d: 常微分方程其他学科。

\section{定性理论}
\section{稳定性理论}
\section{解析理论}
\section{常微分方程其他学科}



\chapter{Partial differential equation}
偏微分方程

a: 椭圆型偏微分方程, 
b: 双曲型偏微分方程, 
c: 抛物型偏微分方程, 
d: 非线性偏微分方程, 
e: 偏微分方程其他学科。

\section{椭圆型偏微分方程}
\section{双曲型偏微分方程}
\section{抛物型偏微分方程}
\section{非线性偏微分方程}
\section{偏微分方程其他学科}


\chapter{Integral equation}

积分方程


\subsection{椭圆周长近似计算公式}

\begin{equation}
    \begin{split}
        & h = \frac{(a-b)^2}{(a+b)^2}\\
        &L_{\text{Pade}} = \pi (a+b)\frac{64-3h^2}{64-16h}\\
        &L_{\text{Jocobsen}} = \frac{256-48h^2-21h^4}{256-112h^2+3h^4}\\
        & L_{\text{Pade}} = \pi (a+b)(1+\frac{3h}{10+\sqrt{4-3h}})\\
        &L_{\text{Rackaukas}} = \pi (a+b)\frac{135168-85760h-5568h^2+3867h^3}{135168-119552h+22208h^2-345h^3}
    \end{split}
\end{equation}
 


\end{document}

