%%============
%%  ** Author: Shepherd Qirong
%%  ** Date: 2022-05-06 20:32:12
%%  ** Github: https://github.com/ShepherdQR
%%  ** LastEditors: Shepherd Qirong
%%  ** LastEditTime: 2023-04-01 20:50:33
%%  ** Copyright (c) 2019--20xx Shepherd Qirong. All rights reserved.
%%============


\documentclass[UTF8]{../09-Mathematics}
\begin{document}

\title{09-10-ComputationalMathematics}
\date{Created on 20220605.\\   Last modified on \today.}
\maketitle
\tableofcontents


\chapter{Introduction}
计算数学

\begin{lstlisting}
  
a: 插值法与逼近论, 
b: 常微分方程数值解, 
c: 偏微分方程数值解, 
d: 积分方程数值解, 
e: 数值代数, 
f: 连续问题离散化方法, 
g: 随机数值实验, 
h: 误差分析, 
i: 计算数学其他学科。

\end{lstlisting}


\chapter{插值法}
给定点集求一条曲线,曲线过点集所有点。

$A = \{a, a\in \mathbb R^n\}$,插值函数P(x)过点。P(x)可以是有理函数、三角多项式、代数多项式,等等。

\section{多项式插值}
$P(x) = \sum a_n x^n$

\begin{equation}
    \begin{bmatrix}
       1 & x_0 & \cdots & x_0^n\\
       & & \ddots & \\
       1 & x_n & \cdots & x_n^n\\
    \end{bmatrix}
    \cdot
    \begin{bmatrix}
      a_1 \\
      \vdots \\
      x_n \\
    \end{bmatrix}
    =
    \begin{bmatrix}
      y_1 \\
      \vdots \\
      b_n \\
    \end{bmatrix}
\end{equation}

  Vandermonde determinant 

  \begin{equation}
    \begin{aligned}
        &
        \left|
        \begin{bmatrix}
            1 & 1 & \cdots & 1 & 1\\
            x_1 & x_2 & \cdots & x_{n-1} & x_n\\
            x_{1}^2 & x_{2}^2 & \cdots & x_{n-1}^2 & x_{n}^2\\
            &  & \vdots &  & \\
            x_{1}^{n-1} & x_{2}^{n-1} & \cdots & x_{n-1}^{n-1} & x_{n}^{n-1}\\
            x_{1}^{n} & x_{2}^{n} & \cdots & x_{n-1}^{n} & x_{n}^{n}\\
        \end{bmatrix}
        \right|
        \\
        \xrightarrow[]{L_j = L_j - L_{j-1} \cdot x_1}
        &
        \left|
        \begin{bmatrix}
        1 & 1 & \cdots & 1 & 1\\
        0 & x_2-x_1 & \cdots & x_{n-1} -x_1& x_n-x_1\\
        0 & x_{2}^2-x_{2}^1 x_1  & \cdots & x_{n-1}^2-x_{n-1}^1 x_1 & x_{n}^2-x_{n}^1 x_1\\
        &  & \vdots &  & \\
        0 & x_{2}^{n-1}-x_{2}^{n-2}x_1   & \cdots & x_{n-1}^{n-1} -x_{n-1}^{n-2}x_1 & x_{n}^{n-1}-x_{n}^{n-2}x_1 \\
        0 & x_{2}^{n}-x_{2}^{n-1} x_1 & \cdots & x_{n-1}^{n}-x_{n-1}^{n-1} x_1 & x_{n}^{n}-x_{n}^{n-1}x_1 \\
        \end{bmatrix}
        \right|
        \\
        =&
        (x_2-x_1) \cdots (x_n-x_1)
        \left|
        \begin{bmatrix}
           1 & 1 & \cdots & 1 & 1\\
           0 & 1 & \cdots & 1& 1\\
           0 & x_{2}^1  & \cdots & x_{n-1}^1 & x_{n}^1\\
           &  & \vdots &  & \\
           0 & x_{2}^{n-2}   & \cdots & x_{n-1}^{n-2} & x_{n}^{n-2} \\
           0 & x_{2}^{n-1} & \cdots & x_{n-1}^{n-1}  & x_{n}^{n-1} \\
        \end{bmatrix}
        \right|
        \\
        &
        \xlongequal[]{\left|
            \begin{bmatrix}
               1 & 1 \\
               0 & D \\
            \end{bmatrix}
        \right| = |D|}
        \prod_{1\leqslant i \leqslant j \leqslant n}^{n} (x_j-x_i)
    \end{aligned}
\end{equation}

\section{Lagrange}
\section{Newton}
\section{分段线性}
\section{Hermite}
\section{样条}

\chapter{逼近论}
给定点集,求一条曲线,曲线可以不过点集中的点。

\chapter{常微分方程数值解}
\chapter{偏微分方程数值解}
\chapter{积分方程数值解}
\chapter{数值代数}
\chapter{连续问题离散化方法}
\chapter{随机数值实验}
\chapter{误差分析}
\chapter{计算数学其他学科}



\end{document}
