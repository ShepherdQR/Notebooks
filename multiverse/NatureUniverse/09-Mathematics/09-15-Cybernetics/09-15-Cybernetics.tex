%%============
%%  ** Author: Shepherd Qirong
%%  ** Date: 2022-05-06 20:36:41
%%  ** Github: https://github.com/ShepherdQR
%%  ** LastEditors: Qirong ZHANG
%%  ** LastEditTime: 2024-12-02 22:44:03
%%  ** Copyright (c) 2019--20xx Shepherd Qirong. All rights reserved.
%%============

\documentclass[UTF8]{../09-Mathematics}
\begin{document}

\title{09-15-Cybernetics}
\date{Created on 20241202.\\   Last modified on \today.}
\maketitle
\tableofcontents


\chapter{Introduction}



控制论、信息论



\chapter{Information Throry}



香农第一定理(可变长无失真信源编码定理。霍夫曼编码也称为最佳编码, 由Huffman1952年提出, 压缩的极限是不同符号用不同的编码, 如果做到每个符号的代码长度等于它出现概率的对数, 则编码总长度就是信息熵。)
香农第二定理(有噪信道编码定理)
香农第三定理(保失真度准则下的有失真信源编码定理)

Shannon, 1948年10月, A Mathematical Theory of Communication, 提出信息熵: 

\begin{equation}
    H(X) = - \sum_i p_i logp_i
\end{equation}
提出比特单位。一段信息的信息量是固定的, 称为信息熵。无论怎么压缩, 信息熵是无失真信源编码的极限值
若编码的平均码长小于信息熵值, 必然发生差错(也就是有损)。


信道容量C计算公式: 
\begin{equation}
    C = B * log(1+ \frac{S}{N})
\end{equation}
B为信道带宽;S/N为信噪比, 通常用分贝(dB)表示。
噪声大, 信噪比接近0, C结果接近0。离路由器越远, 信噪比越小, 网速约下降。


\section{Models}
统计模型
\subsection{Fitts' Law}
Paul Fitts于1954年提出。是一个人机互动以及人体工程学中人类活动的模型。它预测了快速移动到目标区域所需的时间是目标区域的距离和目标区域大小的函数。费兹法则多用于表现指, 点动作的概念模型。无论是用手或手指进行接触, 或是在电脑屏幕上用假想的设备(例如, 鼠标)进行虚拟的触碰。

费兹法则可用多种不同公式表达, 比较普通用的是, 用一维的Shannon公式(Mackenzie,约克大学教授提出, 因其与香农定律相似而命名): 
从一个起始位置移动到一个最终目标所需的时间由两个参数来决定, 到目标的距离和目标的大小(上图中的 D与 W), 用数学公式表达为时间 T = a + b log2(D/W+1)。
T 是完成动作的时间
a代表装置开始结束的时间, b表示该装置的速度, 这些常数可从实测数据进行直线近似取得。
D是起始位置到目标中心的距离。
w是目标区域在运动维上的宽度。


\section{编码理论 (代数码理论 )}




\chapter{控制论 (控制论的数学理论 )}
    \section{线性控制系统}
    \section{非线性控制系统}
    \section{随机控制系统}
    \section{分布参数系统}
    \section{复杂系统}
    \section{其他}


\chapter{最优控制}
\chapter{逻辑网络理论}
\chapter{学习机理论}
\chapter{模式识别理论}




\end{document}
