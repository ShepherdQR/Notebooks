%%============
%%  ** Author: Shepherd Qirong
%%  ** Date: 2021-12-20 22:40:58
%%  ** Github: https://github.com/ShepherdQR
%%  ** LastEditors: Qirong ZHANG
%%  ** LastEditTime: 2024-12-01 22:19:08
%%  ** Copyright (c) 2019--20xx Shepherd Qirong. All rights reserved.
%%============


\documentclass[UTF8]{../09-Mathematics}
\begin{document}


\title{09-00-Methodology}
\date{Created on 20220605.\\   Last modified on \today.}
\maketitle
\tableofcontents




\chapter{Introduction}
Today is 20211204, and I deciede to note down all of my knowledge about the math in this notebook.




\chapter{Symbols}


\section{shortcut}

well-formed formular, wff.





\chapter{Materials}
经典作品。
\section{01-Methodology}

\section{02-History}
\section{03-Logic}
\section{04-NumberTheory}
\section{05-Algebra}
\section{06-Geometry}
\section{07-Topology}
\section{08-Analysis}
\section{09-Equation}
\section{10-MathematicalPhysics}
\section{11-ComputationalMathematics}
\section{12-ProbabilityTheory}
\section{13-Statistics}
\section{14-OperationalResearch}
\section{15-Others}



\chapter{Methodology}
方法论,数学重要思想。

重要的是怎么理解所面对的问题,领悟本质。如把画心形曲线理解成画变半径的圆。重要的在于以一种前所未有的形式把握现象。

\section{View}



prof: a method for asserting the truth.
for example: experiment and observing, sampling and counta example, judge, belief, convition.


mathematics prof: from a set of axioms, using a chain of logical deduction, to varify the given proposition.

proposition: statement with value True or False.


\begin{proposition}
    Prof
    $$  \forall n \in \mathbb{Z}, n^2 + n + 41 \in \mathbb{P} $$
    , it is a primer. Actually we can find one number to prove that it is not true, for example, when $n = 40$, or 41.
\end{proposition}

\begin{proposition}
    Prof
    $$  a^4 + b^4 + c^4  = d^4$$, this equation has no positive integer solutions. Actually there has a solution [95800, 217519, 414560,422481]
\end{proposition}

\begin{proposition}
    Prof
    $$  313(x^3 + y^3) = z^3$$, this equation has no positive integer solutions. Actually there is a solution which is a very big number.
\end{proposition}


four-color question.   

Goldbach conjecture

\section{数学的思维方式与创新-84-北大(丘维声)}
6,1039. 

\subsection{数学史上的重大创新}


\subsubsection{分析:微积分的创立和完备化}
观察问题、抓住现象主要特征,抽象出概念。探索(直觉、类比、归纳、联想、推理)。猜测。证明(依据定义、公理、已证明的定理)。\\



如求瞬时速度, $s=at^2$,$\frac{\Delta s}{\Delta t}=2at+\Delta t$,牛顿忽略$\Delta t$,叫做留数,留下来的数。\\
如何解决不等于零又等于零的矛盾?\\
delta t 趋近于0,无限,柯西引入极限的概念:函数在x0附近有定义,在x0可以没有定义,如果存在c使得x趋近于x0但不等于x0时,$|f(x)-c|$可以无限小,称c是x趋近于x0时f(x)的极限。\\
$\forall \varepsilon >0,\ \exists \delta >0$, that when $0<|x-x_0|<\delta$, we have $|f(x)-c|<\varepsilon$
\subsubsection{几何:欧几里得几何到非欧几里得几何}
从平直空间到弯曲空间。\\
从定义和公理,推导和推演。平行公设。高斯和波约,罗巴切夫斯基(1829年),平行公设只是假设。现实世界如何实现非欧几何的用处。高斯想法把球面本身看做一个空间。后来黎曼发展了。弯曲空间的几何是黎曼几何,如球面上的直线定义为大圆的一部分,这样发现过已知直线外一点不存在其平行线。在双曲几何模型下可以实现罗巴切夫斯基几何。\\
\subsubsection{代数学中}
伽瓦罗,代数学从研究方程的根,到研究代数系统的结构和保持运算的映射。

\subsection{集合的划分}
交空并全的划分方法:模n同余是$\mathbb Z$的一个二元关系。两个集合的笛卡尔积\\
a与b模n同余:$(a,b) \in \bigcup _{i=0}^{n-1} H_i \times H_i \subseteq \mathbb Z \times \mathbb Z$.抽象:非空集合s,$S\times S$的子集W是S是上的二元关系,有关系的记为aWb\\





\section{Reference}






\chapter{数学参考工具书}

\section{数学词典}

\section{数学表}
    \subsubsection{乘法表、因数表、质数表}
    \subsubsection{倒数表}
    \subsubsection{乘方与开方表}
    \subsubsection{对数表}
    \subsubsection{三角函数表}
    \subsubsection{积分表}
    \subsubsection{概率论、数理统计用表}
    \subsubsection{特殊函数表}
    \subsubsection{计算数学用表}
\section{计算工具}

\end{document}
