%%============
%%  ** Author: Shepherd Qirong
%%  ** Date: 2021-12-20 22:40:58
%%  ** Github: https://github.com/ShepherdQR
%%  ** LastEditors: Shepherd Qirong
%%  ** LastEditTime: 2023-03-25 18:55:09
%%  ** Copyright (c) 2019--20xx Shepherd Qirong. All rights reserved.
%%============


\documentclass[UTF8]{../09-Mathematics}
\begin{document}


\title{09-00-Methodology}
\date{Created on 20220605.\\   Last modified on \today.}
\maketitle
\tableofcontents




\chapter{Introduction}
Today is 20211204, and I deciede to note down all of my knowledge about the math in this notebook.




\chapter{Simbols}


\section{shortcut}

well-formed formular, wff.



\chapter{Methodology}
方法论,数学重要思想。





\section{数学的思维方式与创新-84-北大(丘维声)}
6,1039. 

\subsection{数学史上的重大创新}


\subsubsection{分析:微积分的创立和完备化}
观察现象主要特征,抽象出概念。探索。猜测。证明。\\
如求瞬时速度, $s=at^2$,$\frac{\Delta s}{\Delta t}=2at+\Delta t$,牛顿忽略$\Delta t$,叫做留数,留下来的数。\\
如何解决不等于零又等于零的矛盾?\\
delta t 趋近于0,无限,柯西引入极限的概念:函数在x0附近有定义,在x0可以没有定义,如果存在c使得x趋近于x0但不等于x0时,$|f(x)-c|$可以无限小,称c是x趋近于x0时f(x)的极限。\\
$\forall \varepsilon >0,\ \exists \delta >0$, that when $0<|x-x_0|<\delta$, we have $|f(x)-c|<\varepsilon$
\subsubsection{几何:欧几里得几何到非欧几里得几何}
从平直空间到弯曲空间。\\
从定义和公理,推导和推演。平行公设。高斯和波约,罗巴切夫斯基(1829年),平行公设只是假设。现实世界如何实现非欧几何的用处。高斯想法把球面本身看做一个空间。后来黎曼发展了。弯曲空间的几何是黎曼几何,如球面上的直线定义为大圆的一部分,这样发现过已知直线外一点不存在其平行线。在双曲几何模型下可以实现罗巴切夫斯基几何。\\
\subsubsection{代数学中}
伽瓦罗,代数学从研究方程的根,到研究代数系统的结构和保持运算的映射。

\subsection{集合的划分}
交空并全的划分方法:模n同余是$\mathbb Z$的一个二元关系。两个集合的笛卡尔积\\
a与b模n同余:$(a,b) \in \bigcup _{i=0}^{n-1} H_i \times H_i \subseteq \mathbb Z \times \mathbb Z$.抽象:非空集合s,$S\times S$的子集W是S是上的二元关系,有关系的记为aWb\\

\subsubsection{等价关系}
反身性,对称性,传递性,$a \sim b$. $a \sim b \Leftrightarrow b \sim a$. $a \sim b , b \sim c \Rightarrow a \sim c$\\
$\bar a$ 是a确定的等价类,$\left\{ x \in S |x \sim a \right\}$。易有
$\bar x = \bar y \Leftrightarrow x \sim y$ 。\\


定理1:集合S上等价关系$\sim$给出的等价类的集合是S的一个划分。\\
证明思路:需要证明并全,交空。交空比较难,需要研究等价类的性质。等价类的代表不唯一。\\
Step1)  It is obvious that $\cup _{a \in S} {\bar a} \subseteq S$,and for any $b \in S$, we have $b \in \bar b \in \cup _{a \in S} {\bar a}$, this means $S \subseteq \cup _{a \in S} {\bar a}  $, so $ \cup _{a \in S} {\bar a} =S $.\\
Step2) To prove $\bar x \neq \bar y \Rightarrow \bar x \bigcap \bar y = \varnothing  $, we prove the contrapositive $ \bar x \bigcap \bar y \neq \varnothing \Rightarrow  \bar x = \bar y$, and this is easy to prove.\\







\section{Space}

\subsection{Operation Defination}

\subsubsection{Element}

we define the basic element as following, where $ \boldsymbol e_i $ means $x_i = 1, x_j = 0$ for all $j \neq i$. When we say a vector, we normally mean a column vector.

\begin{equation}
\vec{x}  = \boldsymbol x = [x_1, x_2,\dots]^T
= \begin{bmatrix}
    x_1 \\
    \vdots \\
    x_n
\end{bmatrix}
= \Sigma x_i \boldsymbol{e_i} 
\end{equation}

We define Kronecker sign to simply the description of $\boldsymbol e_i \cdot \boldsymbol e_j $.

\begin{equation}
    \begin{split}
    &\delta _{ij}:=
    \begin{cases}
    &1,\qquad i = j\\
    &0,\qquad i \neq j\\
    \end{cases}\\
    \end{split}
\end{equation}

The set of bases $\{ \boldsymbol e_i  \}  \xrightarrow{apply} \boldsymbol{x} \longrightarrow    \{ x_i \}   $.
%%\stackrel{apply}
%%\xrightarrow[under]{up} 


\subsubsection{Dot Product}

We define in algebra, $ \boldsymbol{x} \cdot \boldsymbol{y} := \sum{x_iy_i \delta _{ij}} = \boldsymbol{x}^T \cdot \boldsymbol{y}$.

Then the defination is restricted to the choose of the coordinate system. We take a look a the product with reflect $T : \boldsymbol x \rightarrow  \boldsymbol{T} \cdot \boldsymbol{x}$,

\begin{equation}
    (\boldsymbol A \cdot \boldsymbol  B)^T = (a_{ik}b_{kj})^T = c_{ij}^T = c_{ji} = b_{jk}a_{ki} = \boldsymbol B^T \cdot \boldsymbol A^T
\end{equation}
we have 
\begin{equation}
    (\boldsymbol T \cdot \boldsymbol  x)^T(\boldsymbol T \cdot \boldsymbol  y) =  \boldsymbol x^T (\boldsymbol T^T \boldsymbol T) \boldsymbol y = [(\boldsymbol T^T \boldsymbol T) \boldsymbol x]^T \boldsymbol y
\end{equation}

We name T a Contractive mapping when $\boldsymbol T^T \boldsymbol T \leqslant   \theta, 0 \leqslant \theta \leqslant 1$.

\subsubsection{geometry Properties}

\begin{equation}
    \begin{split}
    &\parallel \boldsymbol{x} \parallel := \sqrt{\boldsymbol x \cdot \boldsymbol x}\\
    &\cos {\theta_{x,y}} : = \frac
    {\boldsymbol x \cdot \boldsymbol x}
    {\parallel \boldsymbol{x} \parallel \cdot \parallel \boldsymbol{y} \parallel}\\
\end{split}
\end{equation}

\subsubsection{Add}

\begin{equation}
    \begin{split}
    & \boldsymbol x + \boldsymbol y := \sum (x_i + y_i)\boldsymbol e_i\\
    & k \cdot \boldsymbol x := \sum kx_i\boldsymbol e_i\\
\end{split}
\end{equation}

Law $\boldsymbol{x} + \boldsymbol{y} = \boldsymbol{y} + \boldsymbol{x}$,
law $ (\boldsymbol{x} + \boldsymbol{y} )+ \boldsymbol{z} = \boldsymbol{x} +( \boldsymbol{y} + \boldsymbol{z})$ is not obvious in the view of Set Theory.




\section{Euclid空间}
有序的n元组的全体称为n维Euclid空间,记为$\mathbb R^n$,称$\boldsymbol p=(p_i)_{i=1}^n \in \mathbb R^n$是$\mathbb R^n$的一个点。\\
为便于研究,本论文以$ \mathbb R^3$为背景空间,所涉及的函数默认为可微实值函数。如果实函数$f$的任意阶偏导数存在且连续,则称函数是可微的(或无限可微的,或光滑的,或$C^\infty$的)。\\
由于微分运算是函数的局部运算,限制所讨论函数的定义域在$ \mathbb R^3$中的任意开集,所讨论的结论仍然成立。\\
自然坐标函数:定义在$\mathbb R^n$上的实值函数$x_i: \mathbb R^n \to  \mathbb R$,使得$\boldsymbol p=(p_i)_{i=1}^n = \left( x_i(\boldsymbol p) \right)_{i=1}^n   $\\
切向量:由$\mathbb R^n$ 中的二元组构成,$\boldsymbol v_{\boldsymbol p}=(\boldsymbol p,\boldsymbol v)$,其中$\boldsymbol p$是作用点,$\boldsymbol v$是向量部分\\
切空间$T_p  \mathbb R^n$: 作用点$\boldsymbol p \in \mathbb R^n$的所有切向量的集合。利用向量加法与数量乘法使某点的切空间称为向量空间,与背景空间存在非平凡同构。\\
向量场$\boldsymbol V$:作用于空间点的向量函数,$\boldsymbol V(\boldsymbol p)\in T_p  \mathbb R^n $\\
逐点化原理:$(\boldsymbol V+\boldsymbol W)(\boldsymbol p)=\boldsymbol V(\boldsymbol p)+\boldsymbol W(\boldsymbol p),\ (f \boldsymbol V)(\boldsymbol p)= f(\boldsymbol p)\boldsymbol V (\boldsymbol p)$\\
自然标架场:定义$\boldsymbol U_i=(\delta _j^i)_{j=1}^n$,按Einstein求和约定,有$\boldsymbol V(\boldsymbol p)=v^i(\boldsymbol p)\boldsymbol U_i(\boldsymbol p)$,称$v^i$为场的Euclid坐标函数,其中Kronecker $\delta$函数定义为:
\begin{equation}
\label{Kronecker_delta}
\delta _i^j=\left\{ 
    \begin{aligned}
    1,\  & i =j\\
    0,\  & i \neq j\\
    \end{aligned}
     \right.
\end{equation}

\section{Reference}









\end{document}
