%%============
%%  ** Author: Shepherd Qirong
%%  ** Date: 2019-08-10 19:42:00
%%  ** Github: https://github.com/ShepherdQR
%%  ** LastEditors: Qirong ZHANG
%%  ** LastEditTime: 2024-12-02 22:23:16
%%  ** Copyright (c) 2019--20xx Shepherd Qirong. All rights reserved.
%%============

\documentclass[UTF8]{../09-Mathematics}
\begin{document}

\title{09-06-Topology}
\date{Created on 20220605.\\   Last modified on \today.}
\maketitle
\tableofcontents


\chapter{Introduction}

a: 点集拓扑学, 
b: 代数拓扑学, 
c: 同伦论, 
d: 低维拓扑学,
e: 同调论, 
f: 维数论, 
g: 格上拓扑学, 
h: 纤维丛论, 
i: 几何拓扑学, 
j: 奇点理论, 
k: 微分拓扑学, 
l: 拓扑学其他学科。



\chapter{一般拓扑}
    \section{拓扑空间 (空间拓扑 )}
    \section{维论}
    \section{模糊拓扑学 (不分明拓扑学 )}


\chapter{点集拓扑学}






\chapter{AlgebraicTopology}

\section{代数拓扑}

\subsection{摘要}
拓扑空间概念、性质、构造方法(如映射锥)
基本群的计算方法
奇异同调群3个定理: 同伦不变性, 正合序列, 切除定理
奇异上同调(有环结构), 泛系数定理, K\H unneth定理
代数拓扑通过寻找拓扑不变量给拓扑空间做分类。通过函子, 把输入的拓扑空间变成群, 把映射对应为同态, 把同胚对应为同构。梦想是通过证明同构能够断言空间同胚。梦想还未实现, 目前三维流形的分类为完成。
\subsection{拓扑空间}
通常研究连续映射、度量空间\\
性质: 紧致性(任意开覆盖有子覆盖), 连通性(不能表示成不相交的开子集之并), 道路连通, 分离性\\
同胚: 对于对于拓扑空间X和Y, 称$X\cong Y$, 如果对于$ X \autorightleftharpoons{f}{g}Y $, 有$ g\circ f =1_X,f\circ g=1_Y $\\
拓扑性质: 同胚意义下不变的性质

 
\begin{comment}
    dsa 
    X \autorightleftharpoons{d969696}{3}Y
    \overset{f}{ \underset{g}{\rightleftharpoons} } 
    \xlongequal[d]{dfafdsf}
    \autorightleftharpoons{f}{g} Z

    \begin{equation}
    \label{homeomorphism}
    \begin{split}
        &\text{if: }X \autorightleftharpoons{f}{g}Y\\
        &\text{where: }g\circ f =1_X,f\circ g=1_Y\\
        &\text{then: }X\cong Y\\
    \end{split}
    \end{equation}

\end{comment}

\section{组合拓扑}
\section{同调和上同调群}
\section{同伦论}
\section{纽结理论}
\section{拓扑K-理论}



\chapter{解析拓扑学}
    \section{流形的几何}
    \section{微分拓扑}
    \section{微分流形}
    \section{纤维丛 (纤维空间 )}



\chapter{低维拓扑学}
\chapter{同调论}
\chapter{维数论}
\chapter{格上拓扑学}

\chapter{几何拓扑学}
\chapter{奇点理论}
\chapter{微分拓扑学}
\chapter{Else}





\end{document}
