%%============
%%  ** Author: Qirong ZHANG
%%  ** Date: 2022-05-06 20:38:41
%%  ** Github: https://github.com/ShepherdQR
%%  ** LastEditors: Qirong ZHANG
%%  ** LastEditTime: 2025-01-07 20:21:09
%%  ** Copyright (c) 2019 Qirong ZHANG. All rights reserved.
%%  ** SPDX-License-Identifier: LGPL-3.0-or-later.
%%============



\documentclass[UTF8]{../../09-Mathematics}
\begin{document}

\title{09-13-04-OptimizationTheory}
\date{Created on 20250105.\\   Last modified on \today.}
\maketitle
\tableofcontents






\chapter{Introduction}

最优化的数学理论


\section{Books}

\begin{lstlisting}
    最优化理论与方法, 袁亚湘
\end{lstlisting}





\chapter{问题定义}
\chapter{一维搜索}
\chapter{Newton Method}

\section{梯度下降法}

$f(\boldsymbol x^*)=f(\boldsymbol x)+\nabla f(x)(\boldsymbol x^*-\boldsymbol x)$,
for nonlinear function, we have $f(\boldsymbol x^*) \leqslant f(\boldsymbol x)+\nabla f(x)(\boldsymbol x^*-\boldsymbol x)$, which is $R=f(\boldsymbol x^*) - f(\boldsymbol x)  \leqslant \nabla f(x)(\boldsymbol x^*-\boldsymbol x)=\nabla f(x) \cdot \alpha \boldsymbol d $, $min R \Rightarrow \boldsymbol d = -(\nabla f(x))^T$

$R=\boldsymbol J \Delta \boldsymbol x \Rightarrow  \Delta \boldsymbol x = \boldsymbol J^+ R$, via $ \Delta \boldsymbol x$, we move from the current position to the next position, when $R=\delta$, we reach the target position.

\chapter{共轭梯度法}
\chapter{拟牛顿法}

\chapter{非二次模型最优化方法}
\chapter{非线性最小二乘问题}
\chapter{约束优化最优性条件}
\chapter{二次规划}
\chapter{逐步二次规划}

\chapter{罚函数法}
\chapter{可行方向法}
\chapter{信赖域法}
\chapter{非光滑优化}



\chapter{END}



\end{document}

