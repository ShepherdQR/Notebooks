%%============
%%  ** Author: Qirong ZHANG
%%  ** Date: 2023-02-24 22:22:15
%%  ** Github: https://github.com/ShepherdQR
%%  ** LastEditors: Qirong ZHANG
%%  ** LastEditTime: 2024-12-29 18:59:56
%%  ** Copyright (c) 2019 Qirong ZHANG. All rights reserved.
%%  ** SPDX-License-Identifier: LGPL-3.0-or-later.
%%============


\documentclass[UTF8]{../03-Chemistry}
\begin{document}

\title{03-05-AnalyticalChemistry}
\date{Created on 20241228.\\   Last modified on \today.}
\maketitle
\tableofcontents


\chapter{Introduction}

分析化学

\chapter{分析化学基础理论}



\chapter{分析作业方法与技术}
\subsubsection{分析实验}
\subsubsection{化学仪器}
\subsubsection{试剂、反应}
\subsubsection{试样、分解}
\subsubsection{富集方法、分离方法}
\subsubsection{沉淀法}
\subsubsection{溶剂萃取法}
\subsubsection{柱液相色谱法}
\subsubsection{其他方法}
\subsubsection{分析自动化}







\chapter{无机分析}






\chapter{定性分析 (定性分析学 )}
\section{半微量及微量定性分析、显微结晶分析}
\section{湿法分析}
\section{干法分析}
\section{其他方法}






\chapter{定量分析 (定量分析学 )}
\section{重量分析}
    \subsubsection{半微量、微量及超微量分析}
\section{容量分析 (滴定分析法 )}
    \subsubsection{微量容量分析}
    \subsubsection{中和法}
    \subsubsection{氧化还原滴定法}
    \subsubsection{沉淀法}
    \subsubsection{络合物形成法}
    \subsubsection{非水溶液滴定法}
\section{结构分析}
\section{价态分析}
\section{状态分析}








\chapter{有机分析}
\section{有机定性分析}
    \subsubsection{半微量、微量、超微量及痕量有机定性分析}
    \subsubsection{化合物分析}
    \subsubsection{元素有机定性分析}
    \subsubsection{功能团分析}
    \subsubsection{有机点滴分析}
\section{有机定量分析}
    \subsubsection{半微量、微量、超微量及痕量有机定量分析}
    \subsubsection{元素有机定量分析}
    \subsubsection{有机重量分析}
    \subsubsection{有机容量分析}
    \subsubsection{功能团的测定}
\section{结构分析}
\section{价态分析}
\section{状态分析}








\chapter{仪器分析法 (物理及物理化学分析法 )}
\section{电化学分析法}
    \subsubsection{电导分析法}
    \subsubsection{库仑分析 (电量分析法 )}
    \subsubsection{电解分析法}
    \subsubsection{极谱分析}
    \subsubsection{电势分析法和离子选择性电极分析法}
\section{磁化学分析法}
\section{光化学分析法 (光谱分析法 )}
    \subsection{原子发射光谱分析法}
        \subsubsection{激光光源的光谱分析法}
    \subsection{可见和紫外分光光度法}
    \subsection{红外光谱分析法}
    \subsection{X射线荧光分析法}
    \subsection{γ射线分析法}
    \subsection{微波光谱分析法}
    \subsection{拉曼光谱分析法}
    \subsection{激光光谱分析法}
\section{放射化学分析法、活化分析}
\section{超声波分析法}
    \subsubsection{波谱分析}
    \subsubsection{能谱分析}
    \subsubsection{质谱分析}
\section{色谱分析}
    \subsubsection{气相色谱分析法}
    \subsubsection{液相色谱分析法}
    \subsubsection{吸附色谱分析法}
    \subsubsection{分配色谱分析法}
    \subsubsection{离子交换色谱分析法}
    \subsubsection{络合色谱分析法}
    \subsubsection{纸上电泳分析法}
    \subsubsection{热色谱分析法}
\section{毛细管分析、电毛细管分析}
    \subsubsection{物理化学常数测定分析法}
    \subsubsection{氢离子浓度指数 (PH )的测定}
    \subsubsection{其他物理及物理化学分析法}







\chapter{元素及化合物的分离方法}
\section{色层吸附分析 (层析法 )}
    \subsubsection{吸附层析}
    \subsubsection{气相层析}
    \subsubsection{离子交换层析法}
\section{萃取法}
\section{蒸馏法}
\section{汞电极分离法}
\section{渗碳法}
\section{物相分析}
    \subsubsection{气体色层分析}
    \subsubsection{分子筛分析法}
    \subsubsection{热扩散法}
    \subsubsection{薄膜色层法}
    \subsubsection{环炉技术}
    \subsubsection{浮选法}
    \subsubsection{超离心机法}
    \subsubsection{离子交换膜法}









\chapter{气体分析}
\section{气体定性分析}
    \subsubsection{微量气体定性分析}
    \subsubsection{混合气体分析}
\section{气体定量分析}
    \subsubsection{微量气体定量分析}
\section{气体物理及物理化学分析法}
    \subsubsection{气体电化学分析法}
    \subsubsection{气体光学分析法}
    \subsubsection{气体量热分析、热导分析}
    \subsubsection{气体光声分析}
    \subsubsection{气体色层分析}
    \subsubsection{空气分析、含毒气体分析}
    \subsubsection{工业气体分析}







\chapter{液体分析、水分析}
\section{水分析}







\chapter{END}

\end{document}
