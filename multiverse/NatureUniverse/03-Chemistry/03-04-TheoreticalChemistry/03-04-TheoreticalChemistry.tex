%%============
%%  ** Author: Qirong ZHANG
%%  ** Date: 2023-02-24 22:22:15
%%  ** Github: https://github.com/ShepherdQR
%%  ** LastEditors: Qirong ZHANG
%%  ** LastEditTime: 2025-01-01 23:27:00
%%  ** Copyright (c) 2019 Qirong ZHANG. All rights reserved.
%%  ** SPDX-License-Identifier: LGPL-3.0-or-later.
%%============


\documentclass[UTF8]{../03-Chemistry}
\begin{document}

\title{03-04-TheoreticalChemistry}
\date{Created on 20241228.\\   Last modified on \today.}
\maketitle
\tableofcontents


\chapter{Introduction}

物理化学 (理论化学 )、化学物理


\chapter{结构化学}
\section{结构化学研究方法}
    \subsubsection{化学显微术}
\section{化学键理论}
    \subsection{化学键的量子力学理论}
        \subsubsection{量子化学、量子力学计算}
        \subsubsection{分子轨道理论}
    \subsection{化学键物理学}
\section{化学键的种类}
\section{分子间的相互作用、超分子化学}
\section{络合物化学 (配位化学 )}
\section{立体化学}






\chapter{化学热力学、热化学、热力学平衡}
\section{化学热力学 (反应热力学 )}
    \subsubsection{第一定律和第二定律及各种热力学函数在化学过程中的应用}
    \subsubsection{第三定律在化学过程中的应用}
    \subsubsection{不可逆过程热力学}
\section{化学亲合力}
\section{热化学}
    \subsubsection{热效应}
\section{热力学平衡}
    \subsubsection{平衡原理}
    \subsubsection{相平衡}
    \subsubsection{化学平衡}
\section{体系的物理化学分析}
    \subsection{原理}
    \subsection{实验方法}
    \subsection{单组分体系}
    \subsection{多组分体系}
        \subsubsection{金属体系}
        \subsubsection{盐类体系}
        \subsubsection{有机物体系}
        \subsubsection{非金属体系}
        \subsubsection{混合体系}
    \subsection{物理化学分析的应用}








\chapter{化学动力学}
\subsection{化学反应速度的理论}
\subsection{化学反应的机理和动力学}
\subsection{单相反应与多相反应}
    \subsubsection{气相反应}
    \subsubsection{液相反应、溶液反应}
    \subsubsection{固相反应、局部化学反应}
    \subsubsection{气固反应与液固反应}
\subsection{同位素交换反应}
\subsection{高压和超高压反应}
\subsection{放电反应}
\subsection{周期性反应}




\chapter{化学反应过程}

\section{化学反应过程的研究方法}
    \subsubsection{新技术的应用}

\section{一般化学反应过程}
    \subsubsection{合成}
    \subsubsection{分解、裂化}
    \subsubsection{氢化、氢解和脱氢}
    \subsubsection{水化、水解和脱水}
    \subsubsection{还原、还原剂}
    \subsubsection{氧化、氧化剂}
    \subsubsection{卤化、卤化剂}
    \subsubsection{卤化、卤化剂}

\section{催化反应过程}
    \subsection{催化原理}
    \subsection{催化反应}
        \subsubsection{单相催化反应 (均相催化 )}
        \subsubsection{多相催化反应 (非均相催化 )}
            \paragraph{接触催化}
    \subsection{接触催化过程的产品}
    \subsection{催化剂}
        \subsubsection{催化剂的活性}
        \subsubsection{催化剂的中毒和再生}
        \subsubsection{催化剂的衰老}
        \subsubsection{催化剂的种类}
    \subsection{负催化作用、负催化剂}
    \subsection{加氢、脱氢的催化}

\section{低温化学、深度冷冻化学}

\section{生物化学过程}
\section{光化学反应过程}
\section{电化学反应过程}
\section{高压和减压反应过程}
    \subsubsection{高压反应过程理论}
    \subsubsection{高压反应类型及应用}
    \subsubsection{高压原理及设备}
    \subsubsection{超高压反应 (1000大气压以上 )}
    \subsubsection{减压反应过程}

\section{高温反应过程}
    \subsubsection{高温反应过程}
    \subsubsection{电热反应过程}
    \subsubsection{超高温反应过程}


\section{燃烧过程}
    \subsubsection{燃料与燃烧}
    \subsubsection{固体燃烧过程}
    \subsubsection{液体燃烧过程}
    \subsubsection{气体燃烧过程}
    \subsubsection{燃烧生成物}

\section{爆炸和爆破}
    \subsubsection{爆炸}
    \subsubsection{爆破}

\section{其他化学反应过程}
    \subsubsection{放射化学反应过程}
    \subsubsection{等离子射流化学反应过程}





\chapter{光化学、辐射化学、超声波作用的化学过程}
\section{光化学}
    \subsubsection{光化学反应原理}
    \subsubsection{光化学反应动力学}
    \subsubsection{气相光化学反应}
    \subsubsection{液相光化学反应、溶液光化学反应}
    \subsubsection{固相光化学反应}
    \subsubsection{感光化学}
    \subsubsection{由萤光发生的光化学反应、化学冷光}
    \subsubsection{激光化学}
    \subsubsection{有机物和无机物的各种光化学反应}
\section{辐射化学}
    \subsubsection{在气体系统中的反应}
    \subsubsection{在液体系统中的反应}
    \subsubsection{在固体系统中的反}
    \subsubsection{其他辐射化学反应}
\section{超声化学}






\chapter{溶液}
\section{液态溶液}
    \subsection{溶液理论}
    \subsection{溶解度}
    \subsection{溶液中离子溶剂化作用、离子溶剂化热与自由能}
    \subsection{扩散与渗透作用、溶液的渗透压力}
    \subsection{冰点、沸点}
    \subsection{溶液性质}
        \subsubsection{物理力学性质}
        \subsubsection{磁性}
        \subsubsection{光学性质}
        \subsubsection{热力学性质、溶液热力学}
        \subsubsection{导热性}
    \subsection{电解质溶液}
    \subsection{非电解质溶液}
\section{固体溶液}
\section{气态溶液 (气体的混合体 )}
\section{熔盐}
\section{熔盐和溶液中的结晶作用}







\chapter{电化学、电解}
\section{电化学工业基础}
    \subsection{基础理论}
    \subsection{原料和辅助物料}
    \subsection{机械与设备}
    \subsection{生产过程}
    \subsection{产品类型、性质}
    \subsection{电化学工厂}
    \subsection{三废处理与综合利用}






\section{水的电解}
    \subsubsection{各种电解槽}
    \subsubsection{水的电解生产过程}
    \subsubsection{产品应用}
    \subsubsection{水电解工厂 (车间 )}
\section{电解质溶液理论}
    \subsection{强电解质溶液}
        \subsubsection{氯化钠 (食盐 )水溶液电解工业}
            \paragraph{氯和氢氧化钠的生产}
            \paragraph{次氯酸钠 (漂白液 )的生产}
            \paragraph{氯酸钠的生产}
            \paragraph{高氯酸钠的生产}
        \subsubsection{氯化钾水溶液的电解工业}

    \subsection{弱电解质溶液与中强电解质溶液}
    \subsection{中和与水解}
    \subsection{酸碱平衡、酸碱理论}
    \subsection{气体在液体中的溶液}
    \subsection{水溶液}
    \subsection{非水溶液}
\section{非水溶液}
    \subsubsection{可逆电池 (化学电池 )}
    \subsubsection{电极电势}
    \subsubsection{电动势与热力学函数的关系}
\section{电解与电极作用}
    \subsection{电解 (电解学 )}
    \subsection{电极过程}
        \subsubsection{阴极过程}
        \subsubsection{阳极过程}
\section{金属的溶解和腐蚀的电化学理论}
\section{气体电化学 (放电反应 )}

\section{电解氧化还原过程的工业生产}
\subsection{电解氧化过程的工业生产}
    \subsubsection{电解氧化过程的无机化工生产}
    \subsubsection{电解氧化过程的有机化工生产}
\subsection{电解还原过程的工业生产}
    \subsubsection{电解还原过程的无机化工生产}
    \subsubsection{电解还原过程有机化工生产}

\section{界面电解}
\section{电泳和电渗在化工中的应用}
\section{电解冶金}
    \subsubsection{水溶液电解冶金}
    \subsubsection{熔融物电解冶金}

\section{化学电源}

\section{电镀工业}
    \subsection{单一金属的电镀}
        \subsubsection{镀铬}
        \subsubsection{镀镍}
        \subsubsection{镀锡}
        \subsubsection{镀铜}
        \subsubsection{镀锌}
        \subsubsection{镀银}
        \subsubsection{镀镉}
        \subsubsection{镀金}
    \subsection{合金的电镀}
    \subsection{非金属材料的电镀}
    \subsection{电铸}
        \subsubsection{铁的电铸}
        \subsubsection{镍的电铸}
        \subsubsection{铜的电铸}
        \subsubsection{银的电铸}
        \subsubsection{银的电铸}
    \subsection{电抛光}
    \subsection{阳极氧化}
    \subsection{气体电化学工业生产}




\chapter{磁化学}
\section{磁化学}
\section{等离子体化学}



\chapter{表面现象的物理化学}
\section{表面现象的理论}
    \subsubsection{表面化学 (界面化学 )}
\section{表面活性物质的化学}
\section{吸附}
    \subsection{吸附理论}
        \subsubsection{吸附作用理论}
        \subsubsection{气体和蒸汽的吸附作用}
        \subsubsection{溶液中的吸附作用}
        \subsubsection{色层吸附作用}
        \subsubsection{离子交换吸附作用}
            \paragraph{离子交换理论}
            \paragraph{离子交换剂}
    \subsection{化学吸附、物理吸附、等温吸附}
    \subsection{吸附剂}
\section{粘附}
\section{湿润现象}
\section{毛细现象}
\section{其他表面现象}








\chapter{胶体化学 (分散体系的物理化学 )}
\section{胶体}
    \subsection{胶体结构}
    \subsection{胶体性质}
        \subsubsection{分子动力性质}
        \subsubsection{热力学性质}
        \subsubsection{胶体的电磁学性质及电化学性质}
        \subsubsection{光学性质}
        \subsubsection{结构力学性质}
    \subsection{胶体研究方法}
    \subsection{胶体稳定性}
    \subsection{特殊胶体系统}
    \subsection{胶体溶液、溶胶}
        \subsubsection{疏液溶胶、疏液胶体}
        \subsubsection{亲液溶胶、亲液胶体}
    \subsection{凝胶及软胶}
\section{粗分散体系}
    \subsubsection{研究方法}
    \subsubsection{膜、悬浮体}
    \subsubsection{乳状液}
    \subsubsection{泡沫}
    \subsubsection{粉末、糊膏}
\section{胶体系统陈化、传动凝结作用}





\chapter{半导体化学}
    \subsubsection{半导体晶体结构}
    \subsubsection{半导体表面化学}
    \subsubsection{半导体分析化学}
    \subsubsection{半导体物理化学、化学物理}
    \subsubsection{有机半导体化学}


\chapter{END}

\end{document}
