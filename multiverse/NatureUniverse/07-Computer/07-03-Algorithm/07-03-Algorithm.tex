%%============
%%  ** Author: Shepherd Qirong
%%  ** Date: 2022-06-05 00:00:41
%%  ** Github: https://github.com/ShepherdQR
%%  ** LastEditors: Shepherd Qirong
%%  ** LastEditTime: 2023-02-15 23:45:46
%%  ** Copyright (c) 2019--20xx Shepherd Qirong. All rights reserved.
%%============

\documentclass[UTF8]{../computerUniverse}
\begin{document}
\title{07-03-Algorithm}
\date{Created on 20220605.\\   Last modified on \today.}
\maketitle
\tableofcontents








\chapter{计算模型合集}
各种高效实用的计算模型

\chapter{数据结构}










\subsubsection{数组}
 a[i] = a+i*len ;//i from 0;
 a[i][j]  按行存:a+ (i*len +j )*len ;
 a[i][j]  按列存:a+ (j*len +i )*len ;

 5*5的二维数组a,各元素2字节,a[2][3]行有限存,地址?
 2*5+3 = 13, 13*2 = 26, a+26;
\subsubsection{线性表}

\subsubsection{树与二叉树}

二叉树遍历:前序、后续、中序。
反向构造二叉树:利用 前+中,或后+中遍历结果,推出树的结构。只利用前+后不行。

树转二叉树:第一个孩子在左,兄弟都是右。

查找二叉树:左<根<右


最优二叉树、哈夫曼树:带权路径长度最小。 
1,2,8,4构造哈夫曼树:
step1: 1,2-->3; 3,8,4;
step2: 3,4-->7;7,8
so:        15
      7       8
   3    4
1    2
权值: 1*3+2*3 +4*2+8*1 = 25

线索二叉树:前序、后续、中序,列举各元素后,叶子LR指针指向前后元素。

平衡二叉树:
任意结点左子树与右子树深度差不大于1。
平衡度=左子树深度-右子树深度。


\subsubsection{图}
有向图,无向图。
完全图。

存储:
邻接矩阵。n个点,n*n。i到j有邻接边,Rij=1,否则为0。
邻接表,$V1-->[2,6,--]-->[4,1,--]-->[6,50,^]//$V1到2号结点距离6,到4号结点距离1,到6号结点距离50

【遍历】深度优先,广度优先。

【拓扑排序,AOV网络】有向边表示活动之间开始的先后关系。

【图的最小生成树,普里姆算法】留下的权值最小。
树没有环路,n个节点的树边最多n-1个。
染色红,逐个收集最短的一个元素进来。注意过程中不能形成环。

【图的最小生成树,克鲁斯卡尔算法】从最短的边开始选边。




\chapter{随机算法}


\section{随机数生成}

\subsection{Mersenne Twister}
梅森旋转(Mersenne Twister, MT)算法,常用是伪随机数生成算法。算法描述如算法\ref{algo:2ch001MersenneTwister}所示。

      \begin{algorithm}[ht]
        \caption{Mersenne Twister}\label{algo:2ch001MersenneTwister}
        \SetAlgoLined
        \KwIn{the index is noted as $x_{in}$, the seed number is noted as ${\rm seed}$}
        \KwOut{random number $x_{out}$}
        Initialization:$[w, n, m, r], a, f$,$(u,d),(s,b),(t,c),l$, $MT_0 \leftarrow {\rm seed}$\;
        \For{$i \leftarrow 1$ \KwTo $n-1$}{
          $MT_{i} \leftarrow f \cdot \left\{ MT_{i-1} \oplus [MT_{i-1} >> (w-2)] +i \right\}$
        }
        \For{$i \leftarrow 0$ \KwTo $n-2$}{
          $M_c \leftarrow$ the commposition of the higest $w-r$ bits of $MT_i$ and the lowest $r$ bits of $MT_{i+1}$\;
          $M_c \leftarrow M_c >> 1$\;
          \If{the lowest bit of $M_c$ is $1$}{
            $M_c \leftarrow M_c \oplus a$
          }
          $MT_i \leftarrow MT_{i+m} \oplus M _c$
        }
        $x \leftarrow MT_{x_{in}}$\;
        $x_{out} \leftarrow x \oplus [ (x >> u) \ \& \  d]$\;
        $x_{out} \leftarrow x \oplus [ (x << s) \ \& \ b]$\;
        $x_{out} \leftarrow x \oplus [ (x << t) \ \& \  c]$\;
        $x_{out} \leftarrow x \oplus  (x >> l) $\;
        \KwRet $x_{out}$\;
      \end{algorithm}






\chapter{算法合集}
一般难解问题的高效实用算法





\subsection{计算几何}

\subsection{分布式算法}
\subsection{并行算法}





\subsection{算法}
又穷,确定,有效。

【复杂度】时间,空间

时间复杂度:$1,log_2n,n,nlog_2n,n^2,n^3,...,2^n$

\subsubsection{查找}

【顺序查找】
平均查找长度:$\frac{n+1}{2}$
time,O(N)

【二分查找】
有序排列。
比较次数最多$\left\lfloor log_2n\right\rfloor +1$
time,O($log_2n$)

【散列表】
例如,存储空间10,p=5,散列函数$h=key\%p$,
存储3,8,12,17,9:线性探测: 3,4,2,5,6
冲突解决:线性探测,伪随机数。

\subsubsection{排序}

稳定、不稳定。【 一样的数,保持原顺序,叫稳定】

【插入式:直接插入】
新的一个与已经排好的比,插入到位置

【插入式:希尔】
数据少时插入排序效率可以。

例如10个元素,先
d=n/2=5,隔5个一组,插入排序;
d = d/2  =2,取奇数是3;隔3个一组,插入;
d = d/2 = 1,全体插入排序。


【选择式:直接选择】
每次选剩余最小的。

【选择式:堆排序】完全二叉树。。
小顶堆:$k_i<=k_{2i}, k_i<=k_{2i+1}$
大顶堆:$k_i>=k_{2i}, k_i>=k_{2i+1}$ 所有孩子都更小

从小到大排列:建小顶堆--》取顶--》建小顶堆--》。。。。

例如构造大顶堆:
step1,数组顺序构造完全二叉树。
step2,最后一个非叶子节点,与其2个孩子调整为大顶堆;
       倒数第2个非叶子节点,依次调整。如果有子树,要调整后继续调整子树。
       




【交换式:冒泡】

【交换式:快速】



【归并排序】

【基数排序】




\chapter{算法综合案例}
面向应用的大尺度难解问题的工程实用算法


\chapter{工程算法集成和相应软件体系结构}

\chapter{工程算法分析和评价体系}

\end{document}
