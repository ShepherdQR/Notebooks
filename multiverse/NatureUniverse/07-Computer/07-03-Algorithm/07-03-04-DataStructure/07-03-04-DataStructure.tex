%%============
%%  ** Author: Qirong ZHANG
%%  ** Date: 2022-06-05 00:00:41
%%  ** Github: https://github.com/ShepherdQR
%%  ** LastEditors: Qirong ZHANG
%%  ** LastEditTime: 2025-01-26 23:44:19
%%  ** Copyright (c) 2019 Qirong ZHANG. All rights reserved.
%%  ** SPDX-License-Identifier: LGPL-3.0-or-later.
%%============


\documentclass[UTF8]{../../computerUniverse}

%\SetAlgoSkip{} % 用于移除算法与文字之间的额外间距

\begin{document}
\title{07-03-04-DataStructure}
\date{Created on 20250126.\\   Last modified on \today.}
\maketitle
\tableofcontents



\chapter{Basic}

数据结构

\begin{lstlisting}

\end{lstlisting}

线性表是一种逻辑结构。其存储结构分为顺序存储(如数组),链式存储(如链表)。




\chapter{数组}

\section{基础数组}


a[i] = a+i*len ;//i from 0;
a[i][j]  按行存: a+ (i*len +j )*len ;
a[i][j]  按列存: a+ (j*len +i )*len ;

5*5的二维数组a,各元素2字节,a[2][3]行有限存,地址?
2*5+3 = 13, 13*2 = 26, a+26;


\section{后缀数组}



 
\chapter{链表}
 


\chapter{栈}


\chapter{队列}

\section{基础队列}

\section{优先队列}



\chapter{哈希表}

\section{哈希函数}
哈希函数: y=H(x), 输出长度不变;相同输入每次得到相同输出;输入差距小也会导致输出差距大,输入差距大也可能导致输出相同。x求y容易,y求x困难。

MD5, message digest algorithm 5
SHA-1, SHA-2, source hash algorithm.
MD5, SHA-1 存在安全隐患。




\chapter{堆}
上浮和下沉,用于实现priority queues


\chapter{树}


\section{二叉树}

二叉树遍历: 前序、后续、中序。
反向构造二叉树: 利用 前+中,或后+中遍历结果,推出树的结构。只利用前+后不行。

树转二叉树: 第一个孩子在左,兄弟都是右。


查找二叉树: 左<根<右
1) 左子树的值<根的值<右子树的值;
2) 一直向左达到最小值,一直向右达到最大值;
3) 增加节点: 从根开始,向末端方向,插入值更小就左转,否则右转,到达末端增加一个叶子节点;
4) 删除节点A: A的左子树的最大节点替代删除的A的位置;
5) 扩展: 平衡二叉查找树;B树(m个节点的形状平衡的) 。



最优二叉树、哈夫曼树: 带权路径长度最小。 
1,2,8,4构造哈夫曼树: 
step1: 1,2-->3; 3,8,4;
step2: 3,4-->7;7,8
so:        15
      7       8
   3    4
1    2
权值:  1*3+2*3 +4*2+8*1 = 25

线索二叉树: 前序、后续、中序,列举各元素后,叶子LR指针指向前后元素。

平衡二叉树: 
任意结点左子树与右子树深度差不大于1。
平衡度=左子树深度-右子树深度。

\section{可持久化线段树}
\section{树链剖分}


\chapter{图}
有向图,无向图。
完全图。

存储: 
邻接矩阵。n个点,n*n。i到j有邻接边,Rij=1,否则为0。
邻接表,$V1-->[2,6,--]-->[4,1,--]-->[6,50,^]//$V1到2号结点距离6,到4号结点距离1,到6号结点距离50

【遍历】深度优先,广度优先。

【拓扑排序,AOV网络】有向边表示活动之间开始的先后关系。

【图的最小生成树,普里姆算法】留下的权值最小。
树没有环路,n个节点的树边最多n-1个。
染色红,逐个收集最短的一个元素进来。注意过程中不能形成环。

【图的最小生成树,克鲁斯卡尔算法】从最短的边开始选边。







\chapter{END}

\end{document}
