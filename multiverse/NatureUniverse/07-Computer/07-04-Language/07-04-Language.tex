%%============
%%  ** Author: Shepherd Qirong
%%  ** Date: 2022-06-05 00:07:26
%%  ** Github: https://github.com/ShepherdQR
%%  ** LastEditors: Shepherd Qirong
%%  ** LastEditTime: 2022-10-13 22:04:16
%%  ** Copyright (c) 2019--20xx Shepherd Qirong. All rights reserved.
%%============




\documentclass[UTF8]{../computerUniverse}

\begin{document}


\title{07-04-Language}
\date{Created on 20190819.\\   Last modified on \today.}
\maketitle
\tableofcontents


\chapter{Introduction}

主要包括


\chapter{形式化方法(形式验证)}
形式语义、代数规范、范畴论、类型论和重写技术等。
\section{形式语义}


\chapter{语义理论(自然语言、程序语言)}
\section{计算语义}


\chapter{类型理论}

\chapter{多值逻辑}
\chapter{模糊逻辑}
\chapter{推理技术}





\chapter{编译原理}


【词法分析】根据词法规则,对源程序逐个字符扫描,识别出一个个“单词”符号,词汇检查。
【语法分析】根据语法规则将单词符号序列分解成各类语法单位,如“表达式”“语句”和“程序”等。语法规则
就是各类语法单位的构成规则,主要针对结构的检查。
【语义分析】分析各语法结构的含义,检查源程序是否包含语义错误,主要针对句子含义的检查。



\chapter{编程范式}

\section{并发编程}
\section{函数式编程}
\section{命令式编程}
\section{逻辑编程}

\section{面向对象编程}
    
\subsection{UML}
united modeling language. 

通信图中,:冒号前面是对象名,后面是类名,之间连线上是消息。

模块结构图的主要组成有:模块、调用、数据、控制信息和转接符号

\subsection{Principles}
Open for extension, closed for modification. 对扩展开放,对修改封闭。类的改动通过增加而不是修改代码实现。
Single Responsibility. 类对外提供一种功能。模块只能有一种被修改的理由。
Dependence inversion. 依赖接口。业务层与实现层都向协议层靠拢。
interface segegation. 一个接口提供一种功能。
Liskov substitution. 抽象类出现的地方都可以用其实现类替换。
composite or aggregate reuse. 优先使用组合,因为继承时父类的修改会影响子类,对象组合会降低依赖关系。
law of demeter一个对象应对其他对象尽量少的了解。


\subsection{设计模式}
观察者模式:具体的观察者将自己注册到事件,具体的事件知道自己的观察者、体现类对扩展开放,对修改关闭。


\subsection{Others}




\begin{lstlisting}
   http://api.open-notify.org/astros.json

   https://openweathermap.org/current

\end{lstlisting}


建议把xy push到局部vector中,把局部vector push到vec中

Notepad++里Run可以设置用.bat来编译运行,bat里面要先cd到其目录

penetration test, using attack tools.

进程因为创建而就绪,因为调度而运行,因为IO事件而阻塞,IO事件完毕或时间片用完回到就绪状态,运行完成后消亡。

c++看gcc-cp目录

它是个编译器,也可以画图写别的pythonjava之类的

我都没有研究过,我需要找点资料好好学习一下,个人觉着c用起来更像是地址的别名,fortran的更像是地址的别名的别名,两个的比较应该找一些开源的转换库看看,

\begin{lstlisting}
    using namespace std;在局部使用挺好的
   using FP = void (*) (int, const std::string&); //Replace using typedef
\end{lstlisting}


伸缩型数组,在c中比存指针成员的好处是1)不占1个地址的存储空间,2)最适合制作动态buffer,因为可以直接就把buffer的结构体和缓冲区一起分配。
\begin{lstlisting}
   struct MyStruct
   {
       int			i;
       float		j;
       double		arr[0];		//伸缩型数组成员
       //double	arr[];		//这样声明也可以
   };
   pf = malloc(sizeof(MyStruct) + N*sizeof(double));/*只分配一块内存*/
\end{lstlisting}
c\#生态还是局限


假如一种语言不允许运行期动态分配内存的话,确实只要有stack就够了,内存的数据区只划分成两大块,一块用于独立于过程的数据(编译、连接时已知大小),剩下的全部做为一个stack。C、C++这样的允许动态分配内存的语言,当然还是要有一个heap的,这样内存就划分成三个区域了。在引入线程概念之前,确实如你所想的,除了运行时动态确定大小的数据必须放在heap里面,其他都可以放在stack里,我们可以把stack设置的足够大。然而,后来线程出现了。每个线程有自己独立的stack,一个程序可以创建成百上千个线程,那么stack的大小也就只能是一个很小的值,内存才不会被用光。在windows系统里,每个线程的stack默认只有1MB,LINUX默认8MB,这样,stack就成了紧缺资源,必须非常有节制地使用,体量比较大的数据都要由程序员手工分配到heap里去并手工管理其生命期。

java 半编译半解释

反码:符号位不变,其余位翻转。
正整数的补码:是原码;
负整数的补码:是反码+1;
注意运算可能溢出。

\begin{lstlisting}
   a = 3*5,a*4;//60
   typedef int MyINT;
   using MyINT = int;

   decltype(i) j = 2;//the type of j is the same as i.
initializer_list


 \end{lstlisting}

 每一个练习封装成一个函数收集在一个.hpp里面,在一个main里面调用相应的练习函数

 我认为核心还是数据,设计类的方法要围绕着数据来,公开出去的方法要维护好隐藏起来的成员变量

 量子加密RSA

 原来stl的模板采用的那叫静态多态,编译期间翻译,所以模板的声明实现分离在头文件里实例一份就行了





\chapter{程序设计语言的设计与实现}




\chapter{程序设计语言学习}




\section{CPP}

\subsection{Introduction}

\subsubsection{History}

\paragraph{C++11}

1)并行编程:(Native Concurrency),通过内存模型、线程、原子操作;
2)泛型编程:统一初始化表达式、auto、decltype、移动语义;
3)系统编程:constexpr、POD;
4)支持库构建:内联命名空间、集成构造、右值引用。


\subsubsection{Books}


Bjarne Stroustrup推荐要掌握的的5种语言:
C++、Java、Python、Ruby和JavaScript


c++ primer v5, the author is the creator of the first c++ compiler.

the c++ programming language v4, the author is the creator of the c++.

the c++ standard library: a tutorial and reference.



\subsection{代码规范}

修改注释,例如
---[[修改的代码]]-----



\subsection{Basic Grammer}

\paragraph{sizeof}

程序存储分布有三个区域:栈、静态和动态。

sizeof 操作符,计算的是对象在栈上的投影体积。

不管指针指向的内容在什么地方,sizeof 得到的都是指针的栈大小

C++ 中对引用的处理比较特殊;sizeof 一个引用得到的结果是 sizeof 一个被引用的对象的大小。



\begin{lstlisting}
    关于sizeof的两个精巧的宏实现。
非数组的sizeof:
#defne _sizeof(T) ( (size_t)((T*)0 + 1) )

数组的sizeof:
#define array_sizeof(T)  ( (size_t)(&T+1)  - (size_t)(&T) )
\end{lstlisting}


\paragraph{Reference}
The reference is ofen used as parameters of a function. 
rreference is a const pointer

\begin{lstlisting}
    // const T& or const T cannot be passed to T&
    %* \dots *)
    int var(1);
    int &ref = var; //ref is always the second name of var.
    %*
        \dots
    *)
    int& func(const double& iVal,...){...}
    func() = 3; // We can use the function just as a variable
\end{lstlisting}

 
\paragraph{Const}

\begin{lstlisting}
    // const T* can be transformed to T* using (T*)
    %* \dots *)
    int var(1);
    const int* pCanNotModifyVar = &var; //const T* p CANNOT modify the variable it points to.

    const int& num = 10;//ok
    int& num = 10;//wrong
\end{lstlisting}
const object function can only be called by const object.\\

\begin{lstlisting}
    void testFuncAdd20210718_2()
{
    const int a = 10;
	int* pConstModifier = (int*)&a;// compiler finds and allocates memory
	const int  *q = &a;
	{
		*pConstModifier = 20;
	}
	std::cout <<a <<std::endl;//10
	std::cout <<*pConstModifier <<std::endl;//20
	std::cout << (&a == pConstModifier ) <<std::endl;//1
	std::cout << (q == pConstModifier ) <<std::endl;//1
}
\end{lstlisting}



\paragraph{IO}

dec,hex, oct

\paragraph{RTTI与异常处理}
RTTI,运行时类型信息,在类层次结构中漫游,可向上、向下、平行转换,可实现反射、高级调试等功能。

异常处理:安全的longjmp机制。
\begin{lstlisting}
    void function(void) throw(something){//...}
\end{lstlisting}
构造函数中抛出异常会产生不完整的类,永远不会调析构。解决方法:做一个有T*数据成员的模板类A,我们创建的类B中的int* pVal要在B的构造中new,现在改为存A<int>pVal,在B的构造中new,这样new委托给了类A,在我们的类B的构造中throw也不会导致内存泄露。



\subsection{Function}
函数定义中可以为最右边连续若干参数有省却值,可以用于扩充函数参数时减少对原有代码的修改。
\paragraph{Inline} \quad\\
We use Inline to tell Compiler to do copy-paste. The function is short and called many times.调用时间少了,可执行文件体积增大

完成形参的构造(如调用类的拷贝构造等)之后,再进入函数体内。
\begin{lstlisting}


    
\end{lstlisting}
\paragraph{Oveload} \quad\\
重载。Functions with the same name but the variables are different.We donot care the returned type.




\subsection{Object}

1) Do not set data member public.
2) Try the best to use reference as the io of functions.
3) Use const as most as possible.
4) use initial list in constructor.
结构化程序设计没有封装和隐藏的概念,导致数据结构与函数之间相互的关系、函数之间的调用关系不明确。
把数据结构和对其进行的操作方法放在一起。

stack object, also called auto object. 
static object will be created once, it will not be delete until the program ends.
global object, outside every {}, its scale is the whole program.

类的实现也可以直接在.h中。
\subsubsection{Basic}
给出了构造函数,编译器不默认提供无参构造函数。
复制构造函数,只有1个
\begin{lstlisting}
    C(const C& iC);
    C();//临时对象,下一个语句就析构了。
\end{lstlisting}
返回值是类A的实例时,函数返回时调用A的复制构造函数

声明时可给出变量的初始值,后面可在构造函数中重新赋值。


解决菱形继承,在第一级继承时使用virtual:public,这样后面每次实例化时要调用虚基类的构造函数。
\begin{lstlisting}
    //类型转换构造函数
    C(int iIn)
    C c1 = 12, c2;
    c2 = 9;//error
    c2 = C(9); // ok
\end{lstlisting}



\begin{lstlisting}
    //When we new a class: 
    C* pC = new C(1,2);
    // The compiler transformed into 3 steps:

    void* mem = operator new(sizeof C);//Inside it calls malloc(n);
    pC = static_cast<C*>(mem);
    pC->C::C(1,2);

    // When delete a class:
    delete pC;
    // It turns into:
    C::`C(pC);//Clear the data.
    operator delete(pC);//Insid it calls free(pC); //free the memory.
\end{lstlisting}



参数对象消亡时调用析构函数;函数返回时是生成临时对象返回,在临时对象调用的那条语句之后,临时对象消亡,调用析构。

When we do something in constructor, we need to make sure the function works well in copy constructor.

在构造函数的初始化列表中对类的成员类进行初始化,,顺序是类中成员声明顺序。

\subsubsection{Decoration}
没有使用成员变量的方法,可以通过空的类指针调用,相当于翻译后传了一个空的this指针。static方法也可以空类指针调用,static方法参数中没有this指针 \\
static和global在main结束后按入栈的顺序析构。\\
sizeof doesn't contain static variables.\\
friend function, a class must declare WhichClass::WhichFunction is its friend, so that the specific function can access its private members. Also the same as friend class.\\
friend relationship cannot be passed, or be inherited.\\
每个对象的空间中都有this指针。(x)\\

\begin{lstlisting}
  提示编译器不生成默认复制构造函数:A(CONST A&)=delete;
  提示编译器提供默认构造函数 A()=default;
  默认情况一旦写了构造函数,编译器就不生成无参的默认构造函数了。
  委托构造函数 A():A(0,0,0){}
\end{lstlisting}

const function cannot modidfy class members. 
final, means this function or class cannot be inherited.
override, compiler will check the function, and if it is not override a function in its base, it tells you the error.
Outsied, a const class can only use const functions.



\subsubsection{关系}
Composition,复合, A has a B。类A有1个B的成员变量。
deque 读作 dek, queue,读作 q。
构造时由内而外;析构时由外而内。

Delegation, 委托, 类A有1个B的指针的成员变量。Composition by reference。pimpl,pointer to implementation。 Handle, body。隔离了接口与具体的实现。编译防火墙。reference counting。共享。copy on write,当多人共享,其中某类想要改变共享的类时,拷贝一份,是他脱离共享。

Inheritance, is a, 继承。子类的对象中有父类的成分。父类的析构函数需要是virtual的。
非虚函数,不希望derived class override的函数。
虚函数,希望derived class override的函数。通过子类对象调用父类的虚函数,这延缓了执行逻辑,设计模式叫做template method。Application framework的常用手法,例如打开文件的流程。
pure virtual function, derived class 一定要override。
\begin{lstlisting}
(*(this->vptr)[n])(this);

\end{lstlisting}


1份数据,多个窗口,或多种表达形式。






\subsection{Template}
sortable, comparable, assignable.

Data type int is a comparable modle.

Iterators connects the containers with algorithms.






\subsection{Memory}
%%\paragraph{Memory} \quad\\

\begin{lstlisting}
new是先分配memory(调用operator new,内部调用malloc返回void*,再static_cast转到类指针),再调用构造函数

delete先调用析构函数,再调用operator delete(内部调用free)释放内存
\end{lstlisting}

Dynamic
\begin{lstlisting}
    T* p = new T;
    if(p){
        delete p;
        p = nullptr;
    }
    int arraySize(6);
    T* p = new T[arraySize];
    if(p){
        delete[] p;
        p = nullptr;
    }
\end{lstlisting}



\subsection{STL}
\subsubsection{Contianer}
\begin{lstlisting}

Sequence and reversible:
list 

Sequence and random access:
vector, deque

Associative and reversible:
    set, multiset, map, multimap  

    array, , forward_list,  

unordered_set, unordered_multiset, unordered_map, unordered_multimap  


\end{lstlisting}




\section{HTML}

\subsection{Basic}
Html,超文本标志语言。前端做的多。
asp
aspx
jsp
php

用adobe的 Dw开发。


\section{JAVA}

\subsection{Basic Grammer}
\begin{lstlisting}
    DOS command
    cd\      //back to the root
    md xxx   // make directory
    rd xxx   // remove directory
    del test.txt     // delete the file
    rd /s /q testFolder
 \end{lstlisting}


\chapter{Else}












\end{document}
