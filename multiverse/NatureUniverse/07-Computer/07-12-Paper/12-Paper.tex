%%============
%%  ** Author: Shepherd Qirong
%%  ** Date: 2022-06-05 00:25:44
%%  ** Github: https://github.com/ShepherdQR
%%  ** LastEditors: Shepherd Qirong
%%  ** LastEditTime: 2024-05-21 22:36:50
%%  ** Copyright (c) 2019--20xx Shepherd Qirong. All rights reserved.
%%============


\documentclass[UTF8]{../computerUniverse}

\begin{document}

\title{11-Program}
\date{Created on 20240413.\\   Last modified on \today.}
\maketitle
\tableofcontents


高项

\chapter{整体项目明细}

\subsection{项目作用}


整合分散的养老服务资源;
人养老需求和养老机构提供的服务进行准确的对接;

\subsection{六大功能}

养老数据汇聚中心:收集、清洗、整合的多方数据,包括机构每日动态数据(如某福利院当天完成的服务信息)、民政局当前养老项目进展与项目补贴信息、老人当日每5分钟的穿戴设备信息、养老服务人员服务信息、养老投诉、平台安全等信息;
养老服务监督管理系统:记录养老服务内容,评估质量,以提高养老服务水平
养老信息统计管理系统:多维度统计养老信息,整合分散的养老服务资源,用于全区动态分配养老资源;
养老综合业务支撑系统:实现从数据到展示的中间支持系统
养老数据采集系统:多种信息的采集、传输、记录;
互联网+养老对外服务系统:用户界面、网页的形式访问资源

\subsection{功能点}

【信息源】
养老服务站签约服务商:机构情况、服务类别、服务质量;
民政局:项目补贴计算、养老项目管理;
老人:健康档案;
养老服务人员:服务信息;

养老投诉监管、平台安全、数据权限。


\subsection{功能要求}

养老大数据可视化:养老服务中心、服务站覆盖的老人情况、养老服务机构情况、养老服务人员情况、养老服务人次情况、养老服务金额、养老补贴等数据

养老服务质量监管:通过养老服务类别信息维护及变更和养老服务内容信息维护及变更,实现全区服务机构的养老服务备案管理。

服务补贴及经费管理:服务项目补贴标准、补贴比例、补贴策略
补贴额度限定、各部门根据人群分类差异化设置补贴标准、补贴次数等;
支持多维度、多角度报表查询统计,包括服务情况统计、补贴金额明细、经费核算明细

老人健康管理:以会员健康档案为核心,以医养结合健康管理为主线,以移动互联网云计算为创新点。

养老服务机构管理:养老院、福利院、养老驿站、服务商家(家政公司、社区超市、医疗机构等),管理养老服务机构的信息、收费标准、服务范围、养老机构间的合作关系和上下级关系等。


养老服务人员管理:所属机构,人员基本信息,服务星级,奖项成果,服务状态

养老项目管理:政府购买服务项目、政府补贴项目、政府扶持项目。建立养老项目档案,通过项目类别对地区养老项目进行有序,规范管理

养老信息统计管理:机构运营状态,机构服务状态,设施运营状态,设施开放时间,设施服务类型,服务员状态,服务员技能,社团状态,社团评级,健康信息,自理信息,为全区老龄工作领导决策提供辅助支撑

养老投诉监管:对投诉内容统一规范管理,建立投诉档案库,实现投诉内容多维度查阅

数据采集:养老服务管理中心和智慧养老服务运营中心的数据打通

基础管理:平台的信息安全、权限管理、基础信息维护和管理。






\subsection{假设日志}

项目名称:
准备日期:

编号    分类        归属    条目描述    责任人  到期时间    活动名称    状态    备注
1       假设条件    技术    5家试点单位在老人个性化养老字典中的字段无冲突。
XXX     具体日期XX    会议    已结束

2       制约因素    成本    批准的项目预算内,完成项目。
XXX     具体日期XX    专家判断    已结束


\subsection{问题日志}

项目名称:
准备日期:

编号    分类    条目描述    创建时间    计划解决时间     责任人   决议    状态  问题解决时间   备注
1       管理    开发和测试对需求理解不一致   2023年3月12日,2023年3月31日,    xxxx  沟通会议    已解决    2023年3月23日
1       采购    视频探视终端增加采购5台   2023年8月12日,2023年8月31日,    xxxx  采购    已解决    2023年8月23日


\subsection{变更日志}

项目名称:
准备日期:

编号    分类    变更描述    创建时间    提交人    状态    处理    备注
1       范围    “养老数据汇聚中心”系统加增可自由组合的"动态条件"查询功能
xxx    20230302    已批准    已实施,相关组件已纳入版本管理
1       采购    视频探视终端增加采购5台   2023年8月12日,    
xxxx  采购   已批准   已实施,相关组件已纳入版本管理


\subsection{经验教训登记册}

项目名称:
准备日期:

编号    类别    条目描述    影响    经验教训    记录人    记录日期   备注
1       管理    开发和测试对需求理解不一致    影响开发进度
客户、需求人员、开发、测试开会,各自阐明观点,客户和需求人员解释,达成一致。加强沟通。     xxxx    2023年3月23日

1       采购    视频探视终端增加采购5台   做好管理,针对新需求,考虑成本储备,加强与卖方沟通,做好采购工作,    xxxx  采购    已解决    2023年8月23日



\subsection{需求文件}


序号    需求事项     需求描述       需求类别    优先级
1       并发访问    养老数据汇聚中心支持1000用户同时访问    性能需求    高
2       响应时间    每次查询响应时间不超过2秒           性能需求    高
2       吞吐量      支持每日查询次数大于60万次          性能需求    高


\subsection{需求跟踪矩阵}
项目名称:“智慧养老综合信息管理平台”开发。
项目负责人及联系方式:xxx
%成本中心:
%项目描述:在12个月时间里开发完成具备养老数据汇聚中心、养老服务监督管理系统、养老信息统计管理系统、养老综合业务支撑系统、数据采集系统、互联网+养老对外服务系统等六大功能的综合信息管理平台。预算为12个月,投入为430万人民币。


%标识        需求描述    业务需要、机会、目的和目标  项目目标    WBS可交付成果   产品设计    产品开发    测试案例
标识        需求描述    需求来源  WBS可交付成果   产品设计    产品开发    测试用例  测试用例状态
$FR_001$     收集、整合的多方数据,包括机构每日动态数据(如某福利院当天完成的服务信息)、民政局当前养老项目进展与项目补贴信息、老人当日每5分钟的穿戴设备信息、养老服务人员服务信息、养老投诉、平台安全等信息;
项目原始;养老数据汇聚中心;评审通过;开发完成;    $TC_001$  测试通过

$FR_001_01$   多方数据收集子模块
项目原始;养老数据汇聚中心系统的子模块;评审通过;开发完成;    $TC_001_01$  测试通过

\subsection{CCB成员主要包括}

我方的领导l人,
建设单位的领导1人,
监理工程师1人,
项目组管理人员1人,
项目组技术人员1人。


\subsection{WBS}

1.0 智慧养老综合信息管理平台
1.1 项目管理
1.2 第三方服务
1.3 系统开发
1.4 系统测试
1.5 系统试运行
1.6 系统正式运行
1.7 培训

1.3.1 养老数据汇聚中心子系统开发
%1.3.1.1 养老数据汇聚中心子系统
1.3.1.1.1 多方数据收集子模块
1.3.1.1.2 多方数据处理子模块
1.3.1.1.3 多方数据展示子模块

\subsection{WBS字典}

编号    任务名称    任务描述    资源    工作结果    评价标准/质量    责任人
1.1 项目管理        根据计划实施指导项目管理;项目组成员、全体干系人;项目按计划完成;各里程碑节点审核通过、试运行通过、正式上线;xxx
1.2 第三方服务      根据采购计划实施采购    采购成员,卖方;  完成实物资源、服务采购;    根据采购合同采购;系统验收通过;xxx    


\subsection{横道图}
横道图:
设计:2022.12
需求调研2023。[1,2]
原型设计2023.[2,3]
软件开发2023.[4,7]
软件测试2023.[5,8]
软件部署2023.[9,9]
软件试运行2023.[10,11]
验收2023.[11]



\subsection{责任分配矩阵}
技术经理1人,我;

需求分析1人,李工;

全栈工程师4人:张工、陈工、赵工、王工;

嵌入式和硬件工程师1人:腾工;

质量与测试2人:李工、郑工;

其他人员包括配置、采购、财务等3人【周,孙、王】

\subsection{干系人分类}

我们利用权力利益方格(级别、对项目的关心程度),将干系人分为四类:
1.民政局养老服务指导中心领导张某、养老综合信息管理平台负责人刘某、我司领导唐某、试点单位负责人,权力高利益高,需重点管理;
2.民政局其他相关负责人(如信息管理部门负责人)、监理工程师,权力高利益低,需令其满意;
3.项目组成员及专家,权力低利益高,需随时告知;
4.供应商、老人代表及其他干系人,权力低利益低,需随时监督。


\subsection{项目章程}

项目名称:“智慧养老综合信息管理平台”开发。
总体里程碑进度表:2022年12月6日启动,于2023年11月10日结束。
项目经理:XXX;联系电话:XXXXXXXXXXX。
项目立项依据:为创新推广互联网+养老服务模式,高效整合辖区养老服务资源,整体提升区域综合养老服务供给能力,规范智慧养老服务体系,科学精准实施养老行业监管,市民政局决定搭建智慧养老综合信息管理系统,建立全区养老服务智慧监管平台,实现
全区养老服务数字化监管。
项目目标:在12个月时间里开发完成具备养老数据汇聚中心、养老服务监督管理系统、养老信息统计管理系统、养老综合业务支撑系统、数据采集系统、互联网+养老对外服务系统等六大功能的综合信息管理平台。预算为12个月,投入为430万人民币。
项目成功的标准:5家指定试点单位上线运行测试通过。
项目干系人:
1.刘某某:项目发起人,养老综合信息管理平台负责人;
2.张某某:民政局养老服务指导中心领导刘某,负责监督项目;
0.XXX:项目经理,负责计划、监控项目,对项目质量负责;
iii.刘某某:养老行业专家,负责对接各养老机构,提供业务需求;
iiii.李某某:技术经理,负责为项目提供适当的资源和培训。
签名:(以上所有干系人签名)


\subsection{项目团队人员}

公司采用强矩阵组织形式实施本项目,

全职:12
项目经理:我;
架构师1人、需求分析2人、全栈工程师5人、嵌入式和硬件工程师1人、质量与测试2人;

兼职 3人
其他人员包括配置、采购、财务等3人,由组织安排职能部门内的人员,采用虚拟团队的方式远程兼职参与。


\subsection{采购合同}

采购合同

甲方(订货人): xx
乙方(供应商):A公司
根据《中华人民共和国民法典》规定,经甲乙双方充分协商,特订立合同,以便共同遵守。第一条、产品名称、品种、规格和质量
1、产品名称、数量及其金额

序号    名称                            数量    单价(元)  金额(元)  备注
1       智能化自助服务体检系统终端      5       53000.00    265000.00 
2       智能化老人自助服务终端          5       3700.00     18500.00  
3       一键呼叫主机                5       2600.00     13000.00    
4       售餐刷卡机                  5       2200.00     11000.00        
5       数据汇聚中心服务器          4       33000.00    132000.00
6       。。。
(注:其他如交换机、路由器、应用层中间件、底层中间件、存储、智能卡相关设备等不一一列举。)

2、产品的技术指标(包括质量要求),按国家标准执行


第二条、合同价款及付款方式
本合同总价款为574000.00元整(伍拾柒万肆千圆整)。本合同签订后,甲方向乙方支付定金60000.00元整(陆万圆整),在乙方将上述产品送至甲方指定地点并经甲方验收合格后,甲方一次性将剩余款项付给乙方。

第三条、产品质量
1.乙方保证所提供的产品货真价实,来源合法,无任何法律纠纷和质量问题,如果乙方所提供产品与第三方出现了纠纷,由此引起的一切法律后果均由乙方承担。
2.如果甲方在使用上述产品过程中,出现产品质量问题,乙方负责调换若不能调换,予以退还。

第四条、违约责任
1.甲乙双方均应全面履行本合同约定,一方违约给另一方造成损失的,应当承担赔偿责任。
2.乙方未按合同约定供货的,按延迟供货的部分款、每延迟一日承担货款的万分之一违约金,延迟15日以上的,除支付违约金外,甲方有权解除合同。
3.甲方未按照合同约定的期限结算的,应按照中国人民银行有关延期付款的规定,延迟一日,需支付结算货款的万分之一的违约金;延迟30日以上的除支付违约金外,乙方有权解除合同。
4.甲方不得无故拒绝接货,否则应当承担由此造成的损失和运输费用。5.合同解除后,双方应当按照本合同的约定进行对帐和结算,不得刁难。

第五条、供货时间
乙方需在项目启动之初(2023年2月10日前)提交服务器、存储设备、交换机、防火墙、应用中间件,其余设备需在项目进入到系统测试前(2023年9月10日)提交。

第六条 手写修改本合同无效

第七条其他约定事项
本合同一式两份,自双方签字之日起生效。如果出现纠纷,双方均可向有管辖权的人民法院提起诉讼。


甲方(签字盖章):XXX 日期:2023年1月17日
乙方(签字盖章):A公司 日期: 2023年1月17日


\subsection{采购策略}

1、采购交付方法:乙方需在项目启动之初提交服务器、存储设备、交换机、防火墙、应用中间件,其余设备需在项目进入到系统测试前提交,具体日期以采购合同为准。

2、合同支付类型:需求明确、采用总价合同。

3、采购阶段:采购合同签订后分成3个阶段实施,服务准备阶段、服务阶段、服务完成阶段。
服务准备阶段,提供基础设备如服务器等;
服务阶段,结合项目进度计划,按时段提供设备;
服务完成阶段,完成所有设备交付并验收通过。


\subsection{采购管理计划【todo】}
采购负责人:采购部小周。
拟使用是预审合格卖方:A公司和B公司


\subsection{整合管理-管理计划内容}

子管理计划(10个,含需求管理计划)
基准:范围、进度、成本
其他组件:变更管理计划、配置管理计划、绩效测量基准、生命周期、开发方法、管理审查。


\chapter{句子}

在项目中保持适应性和韧性。

\chapter{P25}

\chapter{概述、过渡、结尾}

\section{概述}

为深入贯彻《中共中央 国务院关于加强新时代老龄工作的意见》《国务院关于印发“十四五”国家老龄事业发展和养老服务体系规划的通知》,落实《智慧健康养老产业发展行动计划(2021—2025年)》,某市民政局于2022年10月启动了“智慧养老综合信息管理平台”的项目建设。该项目通过构建老年人综合信息管理体系,实现辖区范围养老机构、日间照料中心、社区食堂等养老服务设施信息统一监管,全面提升健康养老服务综合能力,形成多元化的居家和社区养老服务新格局,以满足老年人多层次、多样化养老服务需求,切实增加老年人的获得感、幸福感、安全感,形成可推广、可复制、可持续的居家和社区养老服务经验。


我司于2022年11月中标该项目,中标金额530万元,其中软件部分390万元。

我担任此项目的项目经理,全面负责该项目规划、实施、监控和收尾过程的管理工作。

项目自2022年12月启动,于2023年11月顺利通过验收,历时12个月,圆满完成项目任务。
项目涉及辖内养老数据采集、养老大数据可视化、养老服务机构管理、养老服务质量监管、养老投诉监督等多方面;
服务端采用基于SOA的webService设计,nacos进行服务的动态发现与管理,前置nginx负载均衡,使用无状态令牌机制实现单点登录;
数据库选用完全兼容MySQL的MariaDB集群,采用前置Redis、分库分表、读写分离等手段进一步提高系统吞吐量;
物联网部分,智能化自助服务体检系统终端使用国产Mcu GD32F407+国产实时操作系统RTThread进行嵌入式开发,远程视频探望终端数据传输采用ONVIF协议。
项目主要建设内容包括养老数据汇聚中心、养老服务监督管理系统、养老信息统计管理系统、养老综合业务支持系统、养老数据采集系统、互联网+养老对外服务系统等六大功能。

%%公司采用强矩阵组织形式实施本项目,包括技术经理1人、需求分析2人、全栈工程师5人、嵌入式和硬件工程师1人、质量与测试2人;其他人员包括配置、采购、财务等3人,由组织安排职能部门内的人员,采用虚拟团队的方式远程兼职参与。


%%项目涉及辖内养老大数据可视化、养老服务质量监管、服务补贴及经费管理、老人健康管理、养老服务机构管理、养老服务人员管理、养老项目管理、养老信息统计管理、养老投诉监督、养老数据采集等多方面;

%%物联网部分,智能化自助服务体检系统终端使用国产Mcu GD32F407+国产实时操作系统RTThread进行嵌入式开发,智能售餐平台、智能化老人自助服务终端采用deepin国产操作系统,远程视频探望终端数据传输采用ONVIF协议,智能穿戴设备(孝心手环、智能枕垫等)以MQTT作为IOT通信协议,经无线网络实时发布至服务器。



\section{过渡【管理】}

%%由于本项目涉及题材敏感、驻场地点特殊、应用环境少见等特点;

由于本项目涉及数据源多样、驻场单位多,
且需要充分考虑建设单位、用户单位、公众群体等多方面对可交付成果的期望;
此外本项目还广泛涉及物联网、工业互联网等嵌入式开发,个人信息安全保护工作,这些因素都会增加交付产品和服务的复杂度,以及项目管理过程中的难度,为项目带来更多风险和挑战。

因此,我和我的团队从建设初期就认识到项目XX管理的重要性,运用科学的项目管理方法,做好项目XX管理工作。本文结合该项目实践经验,从规划XX管理,......等几个方面,就该项目的XX管理展开论述。




\section{过渡【绩效域】}

%%由于本项目涉及题材敏感、驻场地点特殊、应用环境少见等特点;

由于本项目涉及数据源多样、驻场单位多,
且需要充分考虑建设单位、用户单位、公众群体等多方面对可交付成果的期望;
此外本项目还广泛涉及物联网、工业互联网等嵌入式开发,个人信息安全保护工作,这些因素都会增加交付产品和服务的复杂度,以及项目管理过程中的难度,为项目带来更多风险和挑战。


在项目整个生命周期过程中,有效执行本绩效域能够实现预期目标,主要包含:xxxxxx。

为保障项目如期保质保量交付,我按照项目XXX绩效域的通用要求,结合本项目实际,重点围绕以下

项目过程、制约因素、关注过程和能力、沟通管理、实物资源管理、采购管理、监控变更、学习与持续改进

八个绩效要点,有效推动项目的价值交付。



\section{结尾}

历时12个月的开发和实施,智慧养老综合信息管理平台于2023年11月成功上线,同步应用于辖内多处试点养老服务站。上线至今,系统保持稳定可靠无故障运行,深受市民政局与试点单位的好评。在项目开发过程中,本人及项目团队始终保持高度的责任心和使命感,对XX管理尤为重视,密切关注XX,XX的过程,使项目的XX管理始终处于可控状态。

本项目的规划和实施,充分体现出政府对养老产业的支持与示范引领作用,本项目成果有力推动了市区养老产业信息化发展进程,为满足人民群众日益增长的健康养老需求提供了坚实支撑。

同时,全体成员也在这次项目实施中,对国家的智慧养老产业布局有了更加深入的了解,有效地提升了项目经验和技术水平。

%也希望我能在今后的工作中,利用自己项目管理方面的的专业知识,为人民点亮数字生活,为数字强国贡献自己的力量。

我将吸取本项目的经验教训,并保持空杯心态,不断学习提升项目管理、战略和商务以及领导力技能,利用自己项目管理方面的的专业知识,为人民点亮数字生活,为数字强国贡献自己的力量。


%本项目的成功完成需要感谢新区领导及同仁的信任和支持,需要感谢公司各部门对项目的指导和帮助,需要感谢全体项目成员的不懈努力。当然,本项目也存在不足,由于项目干系人数量较多且关系复杂,收集需求过程产生了进度偏差,暴露出识别干系人及收集需求工作不够扎实。不过,通过同干系人的积极沟通和引导,以及事先规划的进度储备,我们有效控制了项目进度。(结合主题〉我将吸取本项目的经验教训,并保持空杯心态,不断学习提升项目管理、战略和商务以及领导力技能,





\section{过程材料}
\begin{lstlisting}

    本项目的规划和实施,是社会主义现代化建设中对人民群众生老病死的情感关怀,使用到本项目的人民群众,或许心中都充满着离别和哀思,谨希望由我们团队创造出的优质产品和服务,能够慰诫和温暖他们。

    器端采用适合SOA和微服务架构的Go语言进行开发,分布式架构,

    
    促进典型智慧健康养老产品和服务推广应用,推动智慧健康养老产业发展,

现组织开展2022年智慧健康养老产品及服务推广目录申报工作。


    
    通过搭建和使用智慧养老综合信息管理系统,建立全区养老服务智慧监管平台,实现养老服务数字化监管。
,
通过构建老年人综合信息管理三级体系,实现辖区、街道、社区分级监督管理所辖区域的智慧养老服务运营情况。实现辖区范围养老机构、日间照料中心、社区食堂等养老服务设施信息统一监管,建立健康养老大数据,为健康养老事业、产业精准施策和科学、规范管理提供可靠支撑。

围绕政府引导、企业运营、市场化运作、社会力量参与,积极开展“互联网+”居家和社
区养老服务,全面提升居家和社区养老服务综合能力,形成多元化的居家和社区养老服务新格局,以满足老年人多层次、多样化养老服务需求,切实增加老年人的获得感、幸福感、安全感,形成可推广、可复制、可持续的居家和社区养老服务经验。

\end{lstlisting}




%%%%%%%%%%%%%%%%%%%%%%
%%%%%%%%%%%%%%%%%%%%%%
\chapter{整合管理【1】【P18】【参考3】}



\section{过渡}

%%由于本项目涉及题材敏感、驻场地点特殊、应用环境少见等特点;

由于本项目涉及数据源多样、驻场单位多,
且需要充分考虑建设单位、用户单位、公众群体等多方面对可交付成果的期望;
此外本项目还广泛涉及物联网、工业互联网等嵌入式开发,个人信息安全保护工作,这些因素都会增加交付产品和服务的复杂度,以及项目管理过程中的难度,为项目带来更多风险和挑战。

因此,我和我的团队从建设初期就认识到项目XX管理的重要性,运用科学的项目管理方法,做好项目XX管理工作。本文结合该项目实践经验,从规划XX管理,......等几个方面,就该项目的XX管理展开论述。





\section{制定项目章程}

编写一份正式批准项目并授权项目经理在项目活动中使用组织资源的文件的过程。

明确项目和组织战略目标的直接联系,确定项目的正式地位,展示组织对项目的承诺。


发起人根据合同、项目评估报告(项目建议书、可行性研究报告)等立项文件,主持编制了项目章程,组织甲乙双方的干系人、养老领域专家、试点单位负责人等,召开了项目启动会,市民政局领导批准签发了项目章程。项目章程规定

项目目标:在12个月时间里开发完成综合信息管理平台。
总体里程碑进度表:2022年12月6日启动,于2023年11月10日结束。
项目成功的标准:5家指定试点单位上线运行测试通过。
章程任命我为项目经理,批准我在项目实施中使用组织的资源。

%%项目章程还包括:整体项目风险、预先审批的财务资源。。。

我在参与(协助)编制项目章程过程中,我已经开始干系人的识别,利用权力利益方格(级别、对项目的关心程度)详细分类记录已识别的干系人,例如民政局养老服务指导中心领导张某,权力高利益高,需重点管理。同时我也拟定了11名同事为团队成员名单。


同时我编制假设日志用于记录项目管理中的制约因素假设条件,典型条目如在技术分类中,假设5家试点单位在老人个性化养老字典中的字段无冲突。

章程批准后,项目正式启动,我正式负责此项目的管理工作。%%主持,,申请调配资源



\section{制定项目管理计划}

"制订项目管理计划是定义、准备和协调项目计划的所有组成部分,并把它们整合为一份综合项目管理计划的过程。
本过程的主要作用:生成一份综合文件,用于确定所有项目工作的基础及其执行方式。
"



我采用逐步细化的原则指导项目管理计划的制定。
项目启动后,我组织团队成员召开项目开工会议,明确了项目目标、主要可交付成果、总体里程碑,
讨论项目总体描述,让项目团队成员分组构建 $初步$ 的子管理计划。
我对民政局领导、养老领域专家、福利院等试点单位负责人等逐一访谈以进一步明确需求,完善、 $细化$ 各个子管理计划。
【故事】2家养老服务机构试点单位就系统进场时间问题产生争执,我进行冲突管理,积极引导沟通,康复中心设备多、环境更复杂,最终达成一致康复中心先于养老院入驻。

进而我根据项目章程和各个子管理计划,组织团队成员并邀请组织中2名有经验的项目经理,采用头脑风暴的数据收集技术,集思广益$充实$项目管理计划。

由于项目需要满足老人信息管理、药库管理、老人银行、健康监控、护理管理等诸多场景,
涉及养老院、老年康复中心、老年公寓、医养结合等多种信息源,
对范围、进度、成本、风险、采购等方面进行了详细规划,
反复斟酌 
项目范围说明书、进度模型、项目预算
%% 范围基准、进度基准、成本基准

同时,为了保证项目收尾前,项目管理计划$更新$有序进行,我制定了基准变更的控制和批准流程。

最终我组织各方负责人、专家$评审$项目管理计划,根据评审意见进一步修改完善,并获得审批,
为项目顺利开展与绩效测量提供了指导计划与基线标准。%%马上进入实施阶段





\section{指导与管理项目工作}
"指导与管理项目工作是为实现项目目标而领导和执行项目管理计划中所确定的工作,并实施已批准变更的过程。
本过程的主要作用是对项目工作和可交付成果开展综合管理,以提高项目成功的可能性。"


采用Coding一站式研发管理平台收集项目进展状况和KPI指标,

每日站会,每周PDCA会议,团队成员每月PBC总结。定期举行问题解决会和进展跟进会议。会议中发现的问题,能当场解决的当场解决,需要协调的,明确几天内给予答复,形成会议纪要,安排专人跟踪落实,并将上次会议遗留问题的予以汇总说明,做到有始有终。

每月召开一次干系人会议,将项目进度情况、成本支出、需要协调问题等情况进行汇报,并听取他们的反馈意见,及时对项目进行合理并正式的变更,形成了问题日志和变更请求。

2023年3月中旬的周PDCA会议上,开发人员刘工反映某变更开发难度大,该变更是为响应某日民政局某领导临时提出的需求,径专家分析和CCB审批通过后,我列为高优先级的变更,需求描述是在“养老数据汇聚中心”系统中加增可自由组合的"动态条件"查询功能,取代原先定义好的固定条件查询,刘工投入了1周的时间用于实现该新增需求,且未能完全实现需求,导致原有的进度落后。

我将该问题作为高优先级问题,记录到问题日志中,
对此,我将可能新增的需求综合考虑需求紧迫性、技术实现难易度、变更影响范围、需求的处理时间、对预算和进度应急储备的影响等方面,
提出处理方案与变更请求,
采取纠正措施,对于养老信息动态查询功能,与刘工进一步讨论需求,安排技术更熟悉的赵工协助解决技术难题。
同时邀请民政局相关负责人给团队成员讲解论养老数据查询典型场景,并一起讨论关注的组合查询信息,限定查询范围。
同时安排技术专家开展数据库查询培训。
经过两周迭代,顺利完成变更,追平进度,将这一实践记入经验教训登记册。

质量方面,配合QA定期进行质量审计工作。如发现的UI界面交互帧率达到60FPS、UI界面多处细节优化点问题,
一方面保持现有技术作为备选方案以减轻新功能效果不理想的不利影响,另一方面在整个项目期间定期安排熟悉前端控件的成员改进Vue控件、升级前端组件等措施,经过申请与批准后动用预算储备2万元用于升级智能售餐平台双目相机、多屏显示器等硬件规格。

6月份,发现小李和小刘因为现场压力大,产生焦虑,我工作之余组织聚餐,缓解压力,营造积极向上的团队氛围。

通过项目工作的开展、工作绩效数据的收集,为改善工作绩效、生成高质量可交付物,奠定了坚实的基础。


\section{管理项目知识}
"管理项目知识是使用现有知识并生成新知识,以实现项目目标并且帮助组织学习的过程。管理项目过程的主要作用:
1、利用已有的组织知识来创造或改进项目成果;
2、使当前项目创造的知识可用于支持组织运营和未来的项目或阶段。"


重视开发文档管理,每月PBC总结会请团队成员分享经验教训

注重团队成员彼此赋能,不定期组织研讨会,研究养老政策,分享养老数据实践经验,总结项目工作中的教训,更新知识到经验教训登记册。
如养老数据汇聚中心开发中遇到的智能腰带心率变异度异常数据剔除处理经验,7个摄像头数据时延与卡滞导致的进度拖延风险处理经验等。

每周五下午3点是愉快的座谈会时间,团队成员畅所欲言,充分交流近期实践中遇到的新经验,分享显性知识,如5月份某次座谈会中,测试工程师孙某,交流中得知他对最新的IT技术知识、测试工具了解较多,我鼓励他积极探索新的测试框架如Playwright应用到养老数据采集系统多数据源测试中,丰富了团队的测试工具库。

我还鼓励大家多进行人际交往活动,不断收获隐性知识,在组织的春游团建活动中,邀请发起人参与,通过共绘蓝图等协作游戏,加强了成员间信任感,提高了成员对团队的认同感,提高了团队协作能力。

%%再比如,我根据多年的项目经验,项目实施过程中,人员的变动会对项目造成影响,尤其是关键岗位人员,容易造成进度延误,成本增加等风险。要提早做好AB角色配置。在本项目中,我给各个小组配备了副组长,辅助组长的工作,一旦组长离职,副组长可以很快接收项目,开展工作。


\section{监控项目工作}

"监控项目工作是跟踪、审查和报告整体项目进展,以实现项目管理计划中确定的绩效目标的过程。

本过程的主要作用:
1、让干系人了解项目的当前状态并认可为处理绩效问题而采取的行动;
2、通过成本和进度预测,让干系人了解项目的未来状态。
"

监控项目工作其实就是按照一定的频率监测项目是否按照计划的进度、成本等实施。其实在每周五的内部例会上,就能对项目监控起到一个辅助的作用。项目3月开始实施,前期进展都符合计划,但是到了5月中旬的时候,因为受到清明、五一假期的影响,很多同事请假,虽然制定进度计划的时候把这个因素考虑进去了,但实际上还是比计划进度要稍慢一些,但当时的成本是有一定结余的,所以我决定通过赶工处理,我基于目前的现状生成了工作绩效报告。

2023年6月某周的PDCA会议结束后,我对项目绩效进行了例行测量,挣值分析中发现 SPI值异常偏低,说明项目存在进度落后。经过根本原因分析,发现是某开发人员在与建设单位的沟通中,私自响应了民政局某领导临时提出的需求,要求在“养老数据汇聚中心”系统中加增可自由组合的"动态条件"查询功能,取代原先定义好的固定条件查询,同时提出UI界面交互帧率60FPS、UI界面多处细节优化点,触发了范围蔓延风险。

清楚原因后,我评估了工作量,认为会带来较高的进度和成本风险。随后我召开了项目团队会议,向全体成员再次明确项目需求和范围的变更控制流程。
对于已发生的范围变更和进度落后,我们采取了赶工加班,开发未实现部分查询功能的同时并行推进已完成部分查询的测试过程,
同时安排使用高质量的资源如升级前端Babylon.js版本、安排熟悉前端控件的成员改进Vue控件等纠偏操作,
最终经过两周的迭代,追平进度,顺利实施完此次变更,

民政局相关领导对系统改进很满意,表示对整个项目更加有信心,
同时福利院负责人也表示养老数据查询更加方便高效,项目团队也士气大增。


%%成功对此次事件导致的范围、进度、成本等多方面的风险进行了有效控制。

\section{实施整体变更控制}
"实施整体变更控制是审查所有变更请求、批准变更,管理对可交付成果、项目文件和项目管理计划的变更,并对变更处理结果进行沟通的过程。
%%本过程审查对项目文件、可交付成果或项目管理计划的所有变更请求,并决定对变更请求的处置方案。
本过程的主要作用是确保对项目中已记录在案的变更做出综合评审。如果不考虑变更对整体项目目标或计划的影响就开展变更,往往会加剧整体项目风险。
"

为防止频繁变更对项目的不利影响,项目初期,我们成立了由甲方、乙方、监理方组成的项目控制委员会CCB。所有的变更,都需书面提出变更申请,经CCB批准后才能实施。

项目过程中,试点单位某养老院核心功能区装修工期延误,有较高风险导致设备装调就绪工期延误,我与养老院负责人沟通后,用户发起了延迟部署服务监督管理系统和部分数据采集系统的变更请求,我初步审查通过后,组织项目团队成员一起讨论变更方案,

首先,采用资源平滑技术,将该单位的进场工作安排在其他4家单位之后,前4家装调时充分总结经验教训以提高该单位装调效率,根据责任分配矩阵进场事项由张工负责;
同时安排装调人员与该单位负责人保持密切沟通,实时观察进展,确认工作最早开始时间,防止出现紧前活动一再延期的情况。
并与该单位就餐饮服务管理系统等其他模块装调进度、智慧养老院系统软件部分的现阶段交付物等方面积极开展满意度评估,尽量减少其他活动的返工时间,充分保证足够的成本储备与进度储备,确保项目进度按计划进行。

总结整理后正式提出变更请求,并提出更新进度计划,经CCB召开变更控制会审查批准后,我根据更新的进度计划,并组织团队成员按变更方案实施,对变更持续进行监控和评估,确保项目有条理推进。两周后,追平进度,项目稳定开展。

\section{结束项目或阶段}

"结束项目或阶段是终结项目、阶段或合同的所有活动的过程。所有活动的过程。本过程的主要作用:
1、存档项目或阶段信息,完成计划的工作;
2、释放组织团队资源以展开新的工作。
"

%%本时期主要是完结项目管理过程的所有活动,主要是总结经验教训,并释放组织的资源。
2023年10月在3家试点养老服务站试运行并做相应调整,如老人用餐期间数据采集密集需要系统加大数据处理能力、孝心手环遗失处理等,试运行后正式上线并验收通过。

在收尾阶段,我们切实做好了项目收尾的相关工作,

我们依据项目章程、项目管理计划、协议等,通过组织会议,邀请甲方,公司领导,监理等相关干系人参与验收。
项目产品、项目文档,可交付成果,对功能进行测试,测试符合合同要求,甲方一致认可,验收通过。

在验收会上,民政局某领导尤其对养老投诉监管系统的闭环管理功能表示满意。

在验收会后,把项目产生的代码、文档等按照合同的要求进行和移交。
我组织了内部的项目总结会,肯定了大家的付出,
同时总结了项目建设中的机遇与挑战,将这些经验教训更新到了组织过程资产库中,为今后的项目提供参考。



\section{过程材料}
\begin{lstlisting}

    由于养老院、老年康复中心、老年公寓、医养结合等多种信息源数据采集与管理风险;多种不同单位的沟通风险;
    的多种运行环境不同物联网设备(智能卡、无线通信芯片、网络视频服务系统等)采购成本超支风险;
\end{lstlisting}








%%%%%%%%%%%%%%%%%%%%%%
%%%%%%%%%%%%%%%%%%%%%%
\chapter{范围管理【6】【P69】【参考3】}


\section{过渡}

由于本项目涉及数据源多样、驻场单位多、干系人要求多,

在本项目的管理工作中,涉及采购
医疗自助智能一体机、智能卡系统、服务器、交换机等硬件设备,

广泛涉及物联网、智能穿戴设备等嵌入式开发,对硬件设备的性能和质量方面有较高的要求,

提高了项目XX管理过程中的难度,为项目带来更多风险和挑战。

因此,我和我的团队从建设初期就认识到项目XX管理的对项目成功的重要性,运用科学的项目管理方法,做好项目XX管理工作,
确保项目的最终成功。

本文结合该项目实践经验,从
XXXX,XXXX,XXX
等几个方面,就该项目的XX管理展开论述。




\section{规划范围管理}
"规划范围管理是为了记录如何定义、确认和控制项目范围及产品范围,而创建范围管理计划的过程。
本过程的主要作用是在整个项目期间对如何管理范围提供指南和方向。


项目初期,我邀请了民政局领导、养老综合信息管理平台负责人、我司领导、试点单位负责人、团队成员,
依据项目章程、项目管理计划等开会讨论项目范围和产品范围,
通过专家判断、备选方案分析等方法,形成了本项目的范围管理计划和需求管理计划。

范围管理计划用于指导制订项目范围说明书,并进一步分解创建WBS,
我安排需求分析师李工负责需求分析与梳理,强调尤其要做好民政局养老服务指导中心领导张某的需求收集和管理,项目范围说明书也要经过该领导审批和签字确认。确认范围时邀请甲方信息部门负责人和第三方进行评审。

需求管理计划中着重强调了充分利用集团公司的PMIS系统用于需求活动的规划、跟踪、报告,变更全流程要符合规定的变更控制流程。


%范围管理计划内容有:制定项目范围说明书;根据详细项目范围说明书创建WBS;确定如何审批和维护范围基准;正式验收已完成的项目可交付成果。
%需求管理计划内容有:如何规划、跟踪和报告各种需求活动;配置管理活动;需求优先级排序过程;测量指标及使用这些指标的理由;反应哪些需求属性将被列入跟踪矩阵。


\section{收集需求}
"
"收集需求是为实现目标而确定,记录并管理干系人的需要和需求的过程。
本过程的主要作用是为定义产品范围和项目范围奠定基础。

本项目实施前,我们首先采用标杆对照技术,根据公司知识库中的类似项目的相关设计文档,和项目合同文档进行了横向比较作为需求收集的参考,

然后我和需求分析师,与民政局对接人、试点单位负责人进行访谈深入了解需求,
进而召集团队成员采用头脑风暴、焦点小组等技术,引导讨论项目、进一步挖掘需求。

最后,我们综合考虑范围管理计划、需求管理计划、干系人管理计划、干系人登记册等内容,最终形成了一致认可的需求文件(下表1)和记录相关属性的需求跟踪矩阵(下表2)。

生成了需求文件,如根据合同中规定的技术指标,养老数据汇聚中心支持1000用户同时访问,识别为高优先级的性能需求。




\section{定义范围}
"
"定义范围是制定项目和产品详细描述的过程。
本过程的主要作用是描述产品、服务或成果的边界和验收标准。


本过程中,我带领相关干系人,通过项目的范围管理计划、需求文件、风险登记册等内容,

采用备选方案分析、多标准决策分析、引导等方法进行分析研判,

形成了项目范围说明书和更新的需求跟踪矩阵。

项目范围说明书包括:
1、产品范围描述(包括各子系统模块等)
2、项目的可交付物(应用系统等)
3、系统验收标准(功能满足需求、系统运行稳定、相关文档齐全等)
4、项目的除外责任(该项目涉及的机房的主电不包含在该项目范围中)。

本说明书经过审查、批准后,作为范围基准的重要组成部分,以支持未来项目决策,提供了项目沟通的基础,为后续的范围确认、范围控制等提供了依据。



\section{创建WBS}
"
"创建工作分解结构(WBS)是把项目可交付成果和项目工作分解成较小、更易于管理的组件的过程。
本过程的主要作用是为所要交付的内容提供架构。
"

我和我的团队,
基于项目的范围说明书等文件,秉承100\%分解的原则将工作自上而下逐层分解,
把“智慧养老综合信息管理平台”项目作为第一层,
把项目生命周期的各阶段作为第二层(项目管理、第三方服务等),
把产品和项目可交付成果放在第三层(如系统开发包括养老数据汇聚中心子系统开发等6个子系统开发),
第四层是进一步细化分解,如养老数据汇聚中心子系统开发包括多方数据收集子模块、处理、展示子模块等。



并为WBS工作单元分配了标识与编码,最终分解的工作包控制在8/80小时,并明确了唯一责任人。如根据责任分配矩阵试点单位进场事项由张工负责。

对于养老数据采集系统,数据源众多,采集场景不明确,采用滚动式规划分解,我们先作为规划包,等需求明确后再分解。

最后组织客户代表参加会议来评审工作分解是否完整、充分必要且详细,便于实施中的甲乙双方的进度与成本的监控。至此被批准的项目范围说明书、WBS(如下表1)以及 WBS字典(如下表2)组成了项目的范围基准。




\section{确认范围}
"确认范围是正式验收已完成的项目可交付成果的过程。本过程的主要作用:
1、使验收过程具有客观性;
2、通过确认每个可交付成果来提高最终产品、服务或成果获得验收的可能性。



由于项目涉及技术层面较广且复杂,在项目各阶段都需要对阶段的成果质量进行控制,在确认范围前,我带领团队成员进行了质量控制工作。
%%贯穿于项目的始终。

质量控制小组对根据质量标准和要求对工作交付成果进行核查,针对不合格的交付物进行责任倒查,项目组实施人员对于不合格的交付成果进行整改,直至符合标准。

2023年3月中旬,根据质量管理计划,我们对“养老数据汇聚中心”系统开发进行质量审计,质量测量结果显示,UI界面交互帧率为50FPS,没有达到技术指标要求的60FPS。根本原因分析发现是当前代码使用的算法过于复杂,导致在数据基数增加后,程序计算量过大。充分评估且变更申请批准后,启动纠偏程序,经过两周的迭代,追平进度,顺利实施完此次变更。
民政局相关领导对系统改进很满意,表示对整个项目更加有信心,
同时福利院负责人也表示养老数据查询更加方便高效,项目团队也士气大增。
%%针对这一问题,用核对单对功能设计开发要点逐一检查,核对条目包括设计接口、交互逻辑、数据库、视频传输、多屏显示等方面,为了找到影响最大的原因,我们使用帕累托图,将几种原因产生的影响进行排序,最终发现 80\%的原因为当前代码使用的算法过于复杂,导致在数据基数增加后,程序计算量过大。



项目组按照既定的里程碑时间表将已经质量检查过的交付物交由监理、业主方,以项目管理计划、需求文件作为依据,使用检查和群体决策的方法对本项目的阶段性的成果进行确认。
确认的内容主要包括现场查找养老信息查看数据是否准确,视频探视终端是否画面正常,交互时页面帧率60FPS等。

此过程中,我们发现不同干系人关注点不同,
比如我们公司管理层关注投入产出是否合理;
客户关注是否完成了产品或服务﹔
我作为项目经理比较关注项目制约因素,是否必须完成,时间资源是否足够,风险和预备解决办法等;
项目团队成员关注自己参与负责的元素。

对于有些不好确定的验收项目,比如是否达到等保3级,我们邀请第三方评估并给出合格证书。最终得到业主签字确定,形成了验收的可交付成果和工作绩效信息。


%在确认范围前,我带领团队成员进行了质量控制工作。质量控制也可以和确认范围同时进行,质量控制是内部进行,强调的是可交付成果的正确性及是否满足质量要求;而范围确认通常在阶段末进行,强调的可交付成果是否被客户接受和认可。
%质量控制属于内部检查,由执行组织的质量部门实施,确认范围是由外部干系人对项目可交付成果进行检查验收。





\section{控制范围}

"
"控制范围是监督项目和产品的范围状态,管理范围基准变更的过程。
本过程的主要作用是在整个项目期间保持对范围基准的维护。"


在本项目中,我组织项目组成员定期收集分析工作绩效数据,并根据范围管理计划、需求文件、需求跟踪矩阵,召开项目状态评审会,通过审查项目范围找出偏差并分析纠偏,防止范围蔓延。


2023年3月中旬,我们进行了例行的风险审查会议,经过分析我们发现项目中需求泛滥会达到35\%:随着用户界面原型的完成和与建设单位的沟通,目前增加了约占10\%的需求。如民政局某领导临时提出的需求变更,要求在“养老数据汇聚中心”系统中加增可自由组合的"动态条件"查询功能,取代原先定义好的固定条件查询,并提出UI界面交互帧率达到60FPS、UI界面多处细节优化点,存在范围蔓延风险。
对此,我综合考虑需求紧迫性、技术实现难易度,
依据范围管理计划、变更管理计划、需求跟踪矩阵等项目文件,
使用偏差分析和趋势分析评估出此变更对项日其它方而的的影响,
%%变更影响范围、需求的处理时间、对预算和进度应急储备的影响等方面,

我提交变更申请到到更控制委员会(CCB),审批通过后我立即根据相应应对策略,指导变更的实施,
如对于养老信息动态查询功能,安排技术更熟悉的赵工优先解决技术难题,

经过7个工作日完成变更,追平进度。将变更登记到变更日志和经验教训登记册中,以供后续继续跟踪。


作为项目经理的我深知,变更流程的重要性,只有严格遵守执行变更流程才能防止项目随意变更,防止需求蔓延和镀金行为的发生。


\section{过程材料}
\begin{lstlisting}

\end{lstlisting}





%%%%%%%%%%%%%%%%%%%%%%
%%%%%%%%%%%%%%%%%%%%%%
\chapter{进度管理【7】【todo107-143】【参考后3篇】}



\section{过渡}

由于本项目涉及数据源多样、驻场单位多、干系人要求多,

在本项目的管理工作中,涉及采购
医疗自助智能一体机、智能卡系统、服务器、交换机等硬件设备,

广泛涉及物联网、智能穿戴设备等嵌入式开发,对硬件设备的性能和质量方面有较高的要求,

提高了项目XX管理过程中的难度,为项目带来更多风险和挑战。

因此,我和我的团队从建设初期就认识到项目XX管理的对项目成功的重要性,运用科学的项目管理方法,做好项目XX管理工作,
确保项目的最终成功。

本文结合该项目实践经验,从
XXXX,XXXX,XXX
等几个方面,就该项目的XX管理展开论述。



\section{规划进度管理}

为规划、编制、管理、执行和控制项目进度而制定政策、程序和文档的过程
作用:
为如何在整个项目期间管理项目进度提供指南和方向


项目启动后,我立即召集项目组成员,并邀请养老行业专家,组织举行了进度管理计划制订会议

按照项目管理计划、项目章程等文件,参考公司以往项目的组织过程资产,考虑各类事业环境因素,通过分析技术、专家判断最终讨论制定了本项目的进度管理计划,其中包括:
进度模型:预测型;
计量单位:人日;
控制临界值为5\%,(超过10\%为严重偏离基准),超过临界值时必须立即审查,并做出书面报告。

然后请组织内专家进行评审,根据意见修改和完善。接着我组织业主方领导和其他干系人,对进度管理计划进行评审,评审会上我详细介绍了进度计划的内容,得到了业主方的认可,并通过了评审。

在全体人员参与下,最终完成了—份详细、科学的管理计划,为后续项目的顺利开展奠定了基础。
%%项目的进度管理计划的制定,为后续项目的顺利开展奠定了基础。



\section{定义活动}

识别和记录为完成项目可交付成果而须采取的具体行动的过程
作用:
将工作包分解为进度活动,作为对项目工作进行进度估算、规划执行、监督和控制的基础


在项目范围管理中,已经通过分解6个子系统创建了wBS,在进度管理中,我们采用分解方法,将 wBS中得到的工作包进一步细化为具体可执行的活动。

整理进度活动生成活动清单,如养老数据汇聚中心子系统开发包括多方数据收集子模块、处理、展示子模块等。

通过这份清单,项目组成员清楚知道需要做的具体工作,同时我还梳理了项目的里程碑清单,包括收集需求,软件开发,功能演示、试运行、验收等,

确定了各阶段的里程碑交付物,如需求说明书、初步设计方案、详细设计方案、系统初验报告、试运行报告、终验报告等。


总体里程碑进度表:2022年12月6日启动,于2023年11月10日结束。
里程碑清单:
2023年3月31日,养老数据汇聚中心上线测试
2023年10月31日,3家试点养老服务站试运行



\section{排列活动顺序}
识别和记录项目活动之间关系的过程
作用:
定义工作之间的逻辑顺序,以便在既定的所有项目制约因素下获得最高的效率


我通过公司的项目管理软件,根据活动清单、里程碑清单、进度管理计划等文件,对各个活动根据轻重缓急等因素,进行排序。

考虑各项活动之间的关系,项目的提前量与滞后量,制定出了项目进度网络图,

使我在调动项目资源,分配项目工作时有了很好的参考依据。




\section{估算活动持续时间}
根据资源估算的结果,估算完成单项活动所需工作时段数的过程
作用:
确定完成每个活动所需花费的时间量


我参照项目进度管理计划、活动清单、活动属性等,

在项目进行中,查找公司历史项目的数据对本项目的开发进度进行类比估算,算出该活动需要的时间。

同时通过乐观、悲观、最可能(最快、最慢、正常)所需时间对本项目的部署调试进行三点估算,计算出该活动的持续时间。用同样的方法对项目的各个活动进行时间估算,预留项目的进度储备,如养老数据汇聚中心预留15\%的时间,以便如期交付。




\section{制订进度计划}

分析活动顺序、持续时间、资源需求和进度制约因素,创建进度模型,从而落实项目执行和监控的过程
作用:
为完成项目活动而制定具有计划日期的进度模型


制定进度计划就是对整个项目的交付工作进行排期,通过关键链路法、资源优化技术等方式,让整个项目保质保量如期交付。

我参考项目进度管理计划、活动清单、资源日历、风险登记册等文件,考虑项目的提前量与滞后量,通过进度计划编制工具,用甘特图的方式制定出了项目进度计划。

同时,我将项目进度计划发项目组成员讨论、修改,完善后提交评审,评审通过后,形成进度基准,确保项目的进度可控。

%%通过进度计划的制定,形成进度基准、进度数据等文件,确保项目的进度可控。进度计划(甘特图)如下表所示:




\section{控制进度}
监督项目状态,以更新项目进度和管理进度基准变更的过程
作用:
在整个项目期间保持对进度基准的维护

在项目实施过程中,我参考项目进度计划、资源日历、进度数据等材料,通过项目管理软件、资源优化技术,来确保整个项目保持正常进度。

%%在需求调研的时候,由于需求调研团队新员工较多,导致需求调研的进度延迟。我立即向公司申请资源,调配经验丰富的老员工过来追赶进度。

原型开发过程中,
由于客户的小需求一直在变动,经常变更。我和客户说明情况,组织产品经理等项目干系人和客户加强沟通,每三天进行一次原型确认,减少变更,控制进度
采用资源平滑的资源优化技术,将设计和测试交错进行,加强质量控制避免返工,提高工作效率。

2023年3月某周的PDCA会议结束后,我对项目绩效进行了例行测量,发现 SPI值异常偏低,经过根本原因分析,发现是某开发人员在与建设单位的沟通中,私自响应了民政局某领导临时提出的需求变更,要求在“养老数据汇聚中心”系统中加增可自由组合的"动态条件"查询功能,取代原先定义好的固定条件查询,并提出UI界面交互帧率60FPS、UI界面多处细节优化点,触发了范围蔓延风险。

在了解原因后,我评估了工作量,认为会带来较高的进度和成本风险。随后我召开了项目团队会议,向全体成员再次明确项目需求和范围的变更控制流程。

对于已发生的范围变更和进度落后,我们采取了赶工加班、使用高质量的资源如升级前端Babylon.js版本、安排熟悉前端控件的成员改进Vue控件等纠偏操作,

最终经过两周的迭代,顺利实施完此次变更,追平进度,成功对此次事件导致的范围、进度、成本等多方面的风险进行了有效控制。



%界面样式、养老数据搜索算法、


%%其中也有个别活动出现了进度延迟,一是在需求调研的时候,由于需求调研团队新员工较多,导致需求调研的进度延迟。我立即向公司申请资源,调配经验丰富的老员工过来追赶进度。二是在原型设计的过程中,由于客户的小需求一直在变动,经常变更。我和客户说明情况,组织产品经理等项目干系人和客户加强沟通,每三天进行一次原型确认,减少变更,控制进度。三是在软件开发、测试的过程中,公司测试人员测出的 bug较多,每次都是测试完再发开发人员修改。我组织开发工程师和测试工程师进行沟通,最终我们采用边测试边修改的并行工作模式来控制进度。除了这几种方式外,在项目管理过程中,我还通过非关路径上的资源调配到关键路径上、加强质量控制避免返工、关注里程碑节点等方式来控制进度,保证项目如期交付




\section{过程材料}
\begin{lstlisting}


\end{lstlisting}










\chapter{成本管理【4】}

\section{过渡}

由于本项目涉及数据源多样、驻场单位多、干系人要求多,

在本项目的管理工作中,涉及采购
医疗自助智能一体机、智能卡系统、服务器、交换机等硬件设备,

广泛涉及物联网、智能穿戴设备等嵌入式开发,对硬件设备的性能和质量方面有较高的要求,

提高了项目XX管理过程中的难度,为项目带来更多风险和挑战。

因此,我和我的团队从建设初期就认识到项目XX管理的对项目成功的重要性,运用科学的项目管理方法,做好项目XX管理工作,
确保项目的最终成功。

本文结合该项目实践经验,从
XXXX,XXXX,XXX
等几个方面,就该项目的XX管理展开论述。


\section{规划成本管理}

规划成本管理是确定如何估算、预算、管理、监督和控制项目成本的过程。
本过程的主要作用是在整个项目期间为如何管理项目成本提供指南和方向。




为了更好地规划成本管理,我邀请了税务专家、公司业务骨干、高层领导等成员共同对项目进行了分析,依据项目管理计划、项目章程、组织过程资产等信息,形成了成本管理计划。


在成本管理计划中规定了计量单位为元,。
控制临界值为5\%,(超过10\%为严重偏离成本基准),超过临界值时必须立即审查,并做出书面报告。
绩效测量规则中的EVM技术,活动开始记为【todo】



成本分为直接成本和间接成本两大类。直接成本就是直接可以归属于项目工作的成本,如团队差旅费、项目使用的物料等,间接成本就是指管理费用及公司内部分摊的一些费用,比如税金、福利等。

成本管理计划制定后经过评审最终才形成了本项目的成本管理计划,为成本管理提供指导和方法。




\section{估算成本}

估算成本是对完成项目工作所需资源成本进行近似估算的过程。
本过程的主要作用是确定项目所需的资金。


如赵工资源成本的800元每日,需要投入10天,则赵工的资源总成本为800*10=8000元


成本估算是对整个项目活动所需资金进行近似估算的过程。主要作用是确定项目所需的资金。只有估算做得科学且全面,成本预算才能更准确,

我们依照成本管理计划、范围基准、进度计划、资源管理计划、风险登记册,

用专家判断、类比估算、自下而上的方式进行估算,

首先我们分析和确定了成本的构成科目:人力资源成本、办公设备费用、硬件采购费用等,然后针对不同科目的特点,使用类比、专家判断的方式,根据WBS分解的工作包,在根据进度计划的具体活动,估算出每个活动的各个科目成本,

例如: his开发工程师和互联网医院的开发工程师成本是不一样的,我们要估算出每个活动具体的工期和人员工资、差旅费等,并形成项目成本估算表:

1.电子病历高级开发工程师2000元/天,需要100人日,中级开发工程师1200元/天,需要200人日。

2.his高级实施工程师1600元/天,需要120人日,中级实施工程师800元/天,需要200人日等。

为应对“已知-未知”的风险,为每个整个项目准备了20万元的应急储备,最后我们将成本估算及估算依据进行了评审工作。


\section{制定预算}

制定预算是汇总所有单个活动或工作包的估算成本,建立一个经批准的成本基准的过程。

本过程的主要作用是,确定可以依据其来进行监督和控制项目绩效的成本基准。项目预算包括经批准用于执行项目的全部资金,而成本基准是经过批准且按时间段分配的项目预算,包括应急储备,但不包括管理储备。


\section{控制成本}

控制成本是监督项目状态,以更新项目成本和管理成本基准变更的过程。
本过程的主要作用是在整个项目期间保持对成本基准的维护。










\section{过程材料}
\begin{lstlisting}
\end{lstlisting}




\chapter{质量管理【5】}

\subsection{核对单}

编号    模块名称            核对功能点  步骤    预期结果    是否通过
1       养老数据汇聚中心    数据分类检索    
点击分类下拉菜单,选择条目如“主题”;输入搜索关键词;点击查询按钮。
逐条展示搜索结果,当无搜索结果时展示搜索结果为0条。

2       养老服务监督管理系统    提交建议
点击“我要提建议”按钮,弹出建议类型和建议文本填写区,点击不同类型的建议按钮设置类型,填写文本,点击提交。
能够正确设置类型,填写文本,点击提交后提示“建议提交成功”。


\subsection{核查表}

缺陷/日期   20240303    20240310    20240317    20240324    合记

智能穿戴设备数据处理模块    1           2       2       2
福利院当天完成的服务信息处理模块    1           1       4       2
养老服务人员服务信息管理模块    3           1       0       0
养老信息处理平台安全        0          1       0       0




\section{过渡}

由于本项目团队成员对养老行业了解不深,缺少相关开发经验;
且需要开发的系统较多,工作量大,同时包含后端、WEB前端、PDA端、智能卡技术、大屏看板等平台的软件开发;
此外本项目还广泛涉及物联网、工业互联网等嵌入式开发,以及等保二级质量要求,这些工作的质量风险较高,易引发返工,
还涉及外购品的质量问题,
这些因素都会增加项目风险管理的难度,为项目带来更多的挑战。

因此,我和我的团队从建设初期就认识到项目风险管理的重要性,运用科学的项目管理方法,做好项目风险管理工作。本文结合该项目实践经验,从规划风险管理,识别风险,实施定性风险分析,实施定量风险分析,规划风险应对,实施风险应对,监督风险等几个方面,就该项目的风险管理展开论述。

\section{规划质量管理}


"规划质量管理是识别项目及其可交付成果的质量要求、标准,并书面描述项目将如何证明符合质量要求、标准的过程。
本过程的主要作用是为在整个项目期间如何管理和核实质量提供指南和方向。

首先,我根据项目章程、范围基准、需求跟踪矩阵、干系人登记册等项目文件,制订了初步的项目质量管理计划。
编制过程中,我
采用标杆对照技术,参考组织以往类似项目中的质量管理计划资料,识别出本项目应达到的成本、进度和资源使用等质量标准和要求。
采用成本效益分析技术,如养老服务人员服务信息管理模块,采用开发完测试预计成本8人日,若开发过程中开展2次评测每次各1人日,则开发完只需要再加3人日评测,综合分析采用分段评估测试的方按,预计质量评估与测量成本为5人日。同样分析得到质量审计和测试成本为10人日。


然后,我和公司的QA组长、民政局项目对接人、以及项目质量保证人员和测试人员等进行会议,对如何达到项目质量标准和要求进行了研讨,采用流程图、思维导图等快速收集质量要求,细化质量管理计划,并制订质量测量标准、过程改进计划等项目文件的初稿,会后我配合QA落实会议意见并修改完善,及时组织召开包含多方关系人参与的专家论证会,评审通过后,向全体项目组成员发布,报送项目甲方单位,并对项目管理文件进行了必要的更新。


\section{管理质量}

"
"管理质量是把组织的质量政策用于项目,并将质量管理计划转化为可执行的质量活动的过程。
本过程的主要作用:
1、提高实现质量目标的可能性;
2、识别无效过程和导致质量低劣的原因;3、使用控制质量过程的数据和结果向干系人展示项目的总体质量状态。
"

我注重过程改进,加强质量审计,保证工作质量。
2023年3月中旬,根据质量管理计划,我们对“养老数据汇聚中心”系统开发进行质量审计,质量测量结果显示,
目前查询功能只支持固定条件查询,未能实现可自由组合的"动态条件"查询功能;同时发现UI界面多处细节优化。

针对差距与不足,我积极、主动地提供协助,以改进过程的执行、提高团队生产效率,我首先进一步分析问题原因,确定根本原因,逐条描述质量优化点,针对每项具体质量问题提出纠正措施建议,如安排熟悉前端控件的成员改进Vue控件等纠偏操作,汇总生成质量报告。进而充分评估进度、时间成本的影响,提出变更请求。

2023年6月中旬,测试工程师孙某指出养老数据采集系统多数据源测试中采用了新的Playwright测试框架,我识别出这种新工具能有效提高测试效率,将其更新到团队的测试工具库中,并在后续试运行前的测试阶段多次应用。

在此项目实施过程中项目组一共召开如
需求规格说明书评审;总体方案评审、测试方案及用例评审、上线方案评审
等共计60余次评审会。

项目实施过程中,我配合QA全流程跟踪和记录了项目各项具体工作的开展效果,对于取得良好实施效果的工作进行了表彰,对于产生不好结果的工作进行了纠正,同时积极采纳我司内外部同类项目的先进经验。
在项目的双周例会上,我会邀请相关工作具体负责人介绍切身的经验和教训。与此同时,在项目的每个关键节点前,我配合QA会组织专家论证会,邀请相关质量审计专家进行研讨,总结项目开展中的经验和教训,并将形成的文档及时更新进项目文件中。

通过有效的管理质量活动,对项目实施中不能满足质量要求的活动及时进行整改,出现偏差时候及时纠正,持续改进使产品质量不断提升。



\section{控制质量}

"控制质量是为了评估绩效,确保项目输出完整、正确且满足客户期望,而监督和记录质量管理活动执行结果的过程。
本过程的主要作用:
1、核实项目可交付成果和工作已经达到主要干系人的质量要求,可供最终验收;
2、确定项目输出是否达到预期目的,这些输出需要满足所有适用标准、要求、法规和规范。"



在项目的各阶段,我都会组织QA小组成员参照验收标准、质量核对单等文件,通过统计抽样、使用质量控制图等方式进行质量控制,确保各阶段的可交付成果满足要求,且已批准的变更请求均实施到位。

2023年3月中旬,根据质量管理计划,我们对“养老数据汇聚中心”系统开发进行质量审计,质量测量结果显示,UI界面交互帧率为50FPS,没有达到技术指标要求的60FPS。针对这一问题,用核对单对功能设计开发要点逐一检查,核对条目包括设计接口、交互逻辑、数据库、视频传输、多屏显示等方面,
为了找到影响最大的原因,我们使用帕累托图,将几种原因产生的影响进行排序,最终发现 80\%的原因为当前代码使用的算法过于复杂,导致在数据基数增加后,程序计算量过大。


%采用核对单进行数据收集,采用文件分析、过程分析、根本原因分析等方法进行数据分析,定位问题为前端控件需要优化、多屏显示器等硬件规格需要更新,用鱼骨图(如下图1)、直方图等数据表现技术直观展示缺陷原因,形成质量报告。


在了解原因后,我评估了进度、成本,决定提出变更申请针对代码进行优化,并最终通过了CCB的批准,

对于已发生的范围变更和进度落后,我们采取了赶工加班、使用高质量的资源如升级前端Babylon.js版本、安排熟悉前端控件的成员改进Vue控件等纠偏操作,最终经过两周的迭代,我们提升了代码执行效率,顺利解决了系统响应时间长的问题。顺利实施完此次变更,追平进度,成功对此次事件导致的范围、进度、成本等多方面的风险进行了有效控制。






\section{过程材料}
\begin{lstlisting}
 

质量审计是评估项目活动是否遵循了组织和项目的政策、过程与程序的一种过程审计。质量审计的目的包括5个:
一是识别全部正在实施的良好及最佳实践,
二是识别全部违规做法、差距及不足、
三是分享所在组织或行业中类似项目的良好实践,
四是积极、主动地提供协助,以改进过程的执行,从而帮助团队提高生产效率,
五是强调每次审计都应对组织经验教训的积累做出贡献。

\end{lstlisting}




%%%%%%%%%%%%%%%%%%%%%%%%%%%%%%
%%%%%%%%%%%%%%%%%%%%%%%%%%%%%%
\chapter{风险管理【8】}

\section{过渡}

由于本项目团队成员对养老行业了解不深,缺少相关开发经验;
且需要开发的系统较多,工作量大,同时包含后端、WEB前端、PDA端、智能卡技术、大屏看板等平台的软件开发;
此外本项目还广泛涉及物联网、工业互联网等嵌入式开发,以及等保二级质量要求,这些工作的质量风险较高,易引发返工,
还涉及外购品的质量问题,
这些因素都会增加项目风险管理的难度,为项目带来更多的挑战。

因此,我和我的团队从建设初期就认识到项目风险管理的重要性,运用科学的项目管理方法,做好项目风险管理工作。本文结合该项目实践经验,从规划风险管理,识别风险,实施定性风险分析,实施定量风险分析,规划风险应对,实施风险应对,监督风险等几个方面,就该项目的风险管理展开论述。

\section{规划风险管理}

规划风险管理是定义如何实施项目风险活动的过程,其作用在于确保风险管理水平、方法和可见度与项目实际风险程度相匹配,与组织和其他干系人的重要程度相匹配。

项目建设初期,我和项目团队成员根据项目章程、项目管理计划、干系人登记册等文件,制定了初步的风险管理计划,包括概率和影响矩阵的数值定义,人员角色的分工,应急与管理储备的预算及其使用方案等内容。

在概率和影响矩阵的影响值定义中,我们使用公式(工期影响/4周)+(成本影响/60万)计算,即假设一个20\%概率的风险会造成2周工期延误,15万成本超支,那么其影响力=2周/4周+15万/60 万=0.75,其概率影响值=0.75*0.2=0.15。

采用风险分解结构(RBS)构建风险类别,RBS0级是项目风险所有来源,RBS1级分为技术风险、管理风险、商业风险、外部风险等4个方面,RBS2级进一步细化分解,如RBS1.1为养老数据类型随项目开发不断增长导致的项目范围定义风险,RBS2.1为与民政局负责人、福利院对接人、老人、老人家属等多方面沟通风险。

我们邀请了民政局对养老行业的专家、组织内有丰富信息系统项经验的专家、养老机构负责人等共同分析和讨论,形成了正式的风险管理计划。

科学合理的风险管理计划可以有效提高项目韧性,提高项目的成功率,为项目顺利实施奠定可靠的基础。


\section{识别风险}
识别风险是识别整体项目风险与单个项目风险的来源,并记录风险特征的过程。
本过程的主要作用是,描述和记录风险特征,使得当风险触发时项目团队能够恰当地应对已识别的风险。

项目规划阶段,我组织团队成员对项目合同、需求文件、范围说明书、采购文件等项目文件进行了文件分析,采用头脑风暴的方式对项目实施过程中可能产生的风险进行了识别,同时对风险产生的根本原因、可能造成的后果和可采取的应对措施进行宁分析。

最后,将上述成果归纳总结,形成了风险登记册,其中包括:

由于该项目需要向市民政局和5个试点单位养老机构同时进行需求的收集,因此可能导致归口不一,需求不一致的范围风险;

由于开发团队对养老行业领域专业知识缺乏产生的技术风险;

由于养老院、老年康复中心、老年公寓、医养结合等多种信息源数据采集与管理风险;多种不同单位的沟通风险;

由于满足老人信息管理、药库管理、老人银行、健康监控、护理管理等多场景的多种运行环境不同物联网设备(智能卡、无线通信芯片、网络视频服务系统等)采购成本超支风险;

等主要风险共计17项。

随着项目工作的迭代推进,采用每日站会信息收集、与敬老院等试点单位负责人访谈等多种技术,识别出风险共计6项,如某试点单位装修工作可能导致的进度延期风险。




\section{实施定性风险分析}

实施定性风险分析是通过评估单个项目风险发生的概率和影响及其他特征,对风险进行优先级排序,从而为后续分析或行动提供基础的过程。
本过程的主要作用是重点关注高优先级的风险。

项目过程中,每2个月定期开展定性风险分析会议,首次会议我邀请了组织内资深的风险评估专家,与团队成员一起针对风险登记册中记录的单个风险条目,进行了概率和影响程度的评估,以此绘制出了概率影响矩阵,并进行了优先级排序。经评估,本项目实施中优先级最高的前三项风险依次为:
(1)需求不明导致的范围风险,如网络视频服务系统布置的采集设备种类与数量、
(2)采购问题导致的成本超支和进度延期风险,如AI抓拍型网络摄像机与平台报警模块结合现场安装调试可能产生成本与进度问题、
(3)专业知识缺乏导致的技术风险,如养老院、敬老院、福利院等不同单位的业务流程缺乏细致了解。
如排名第1的风险,概率和影响矩阵的影响值定义,是70\%概率发生的风险,会造成4周工期延误,30万成本超支,那么其影响力=4周/4周+50万/60 万=1.17,其概率影响值=1.17*0.7=0.82。在风险报告中重点分析了其紧迫性、可管理性、可控性、可检测性等风险特征。

随项目进行,风险优先级不断变化并不断更新风险登记册,如在第4次定性风险分析会议中,需求不明导致的范围风险已经成为低概率和影响的风险,列入风险登记册中的观察清单。

\section{实施定量风险分析}

实施定量风险分析是就己识别的单个项目风险,和潜在的未知风险进行定量分析的过程。作用在于:量化整体风险的最大可能性,提供额外的信息以支持风险应对的规划。

项目过程中,
我和团队成员对优先级位列前三的风险进行了定量风险分析,使用蒙特卡洛模拟,基于风险概率的正态分布经验数据,
计算结果为:范围风险可对项目带来约3周返工时间,并因此带来41万左右的成本超支;
采购风险和技术风险分别可对项目带来2周时间进度落后,并因此带来22万左右的成本超支。

最后,我们将分析出来的量化结果,更新到了风险登记册和风险报告中,为后续规划风险应对措施提供了数据基础。



\section{规划风险应对}

规划风险的应对是为了应对项目风险,而制定可选方案、选择应对策略并商定应对行动的过程。
本过程的主要作用:
1、制定应对整体项目风险和单个项目风险的适当方法;
2、分配资源,并根据需要将相关活动添加进项目文件和项目管理计划中。


项目期间,我与团队成员对风险登记册中已识别的风险展开了讨论,按照优先级顺序和量化的风险分析结果,对每一项风险制定了应对预案,

包括:

针对需求不明带来的范围风险,我们进行两周一轮的冲刺和与客户的演示确保项目朝着正确方向发展,同时预备了12万元预算与2周时间管理储备;

针对采购带来的交期和来料品质风险,我们安排技术部门配合采购专员尽早确定硬件方案,尽早实施采购,安排人员对每一种采购品执行来料检验流程,此外预备了10万元预算的管理储备;

针对团队成员行业相关知识的不足,我们为进度计划追加1周时间,邀请了民政局相关领域的专家进行养老制度的讲解、各试点单位负责人对各自养老机构制度及运作流程培训,提高了团队成员的知识和专业技能。


\section{实施风险应对}

%%实施风险应对是对己规划的风险应对措施进行条件触发判断,并启用措施的过程。
实施风险应对是执行商定的风险应对计划的过程。
本过程的主要作用:
1、确保按计划执行商定的风险应对措施;
2、管理整体项目风险入口、最小化单个项目威胁,以及最大化单个项目机会。



项目过程中,如我们所预计那样,由于试点单位某养老院核心功能区装修工期延误,导致设备装调就绪工期延误,我们随即启用了应急预案,采用资源平滑技术,将该单位的进场工作安排在其他4家单位之后,前4家装调时充分总结经验教训以提高该单位装调效率。同时安排装调人员与该单位负责人保持密切沟通,实时观察进展,确认工作最早开始时间,防止出现紧前活动一再延期的情况。并与该单位就餐饮服务管理系统等其他模块装调进度、智慧养老院系统软件部分的现阶段交付物等方面积极开展满意度评估,尽量减少其他活动的返工时间,充分保证足够的成本储备与进度储备,确保项目进度按计划进行。%智能穿戴设备



\section{监督风险}
监督风险是在整个项目期间,监督风险应对计划的实施,并跟踪已识别风险、识别和分析新风险,以及评估风险管理有效性的过程。
本过程的主要作用是,保证项目决策是在整体项目风险和单个项目风险当前信息的基础上进行。



2023年3月中旬,我们进行了例行的风险审查会议,经过分析我们发现项目中需求泛滥会达到35\%:随着用户界面原型的完成和与建设单位的沟通,目前增加了约占10\%的需求。如民政局某领导临时提出的需求变更,要求在“养老数据汇聚中心”系统中加增可自由组合的"动态条件"查询功能,取代原先定义好的固定条件查询,并提出UI界面交互帧率达到60FPS、UI界面多处细节优化点,存在范围蔓延风险。
对此,我将可能新增的需求综合考虑需求紧迫性、技术实现难易度、变更影响范围、需求的处理时间、对预算和进度应急储备的影响等方面,

采取开拓的策略应对高优先级机会,如对于养老信息动态查询功能,确保功能需求明确,安排技术更熟悉的赵工优先解决技术难题;

采取减轻与提高并用的策略应对影响整体项目的风险,如帧率优化、细节优化等,一方面保持现有技术作为备选方案以减轻新功能效果不理想的不利影响,另一方面在整个项目期间定期安排熟悉前端控件的成员改进Vue控件、升级前端组件等措施,并动用预算储备2万元用于升级智能售餐平台双目相机、多屏显示器等硬件规格。

对于不断调试的智能穿戴设备,按需求的响应时间为7个工作日为期限,对于响应超时的需求、多次响应仍不满意的需求,提高风险等级。

最终各种风险审计信息添加至风险登记册,形成了一份正式的风险项以供后续继续跟踪。




\section{过程材料}
\begin{lstlisting}
    并与主要干系人沟通后,
    
    对于重要变更,如对于动态查询功能,确保变更控制明确,变更前充分考虑进度、成本、质量等方面影响。

    对于在持续优化的需求,如帧率优化、细节优化等,在整个项目期间定期安排熟悉前端控件的成员改进Vue控件、升级前端组件等措施,按需求的响应时间为30天为期限,对于响应超时的需求、多次响应仍不满意的需求,提高风险等级。

    随着项目开发过程的陆续开展,开发人员后期经常需要在试点殡仪馆内与工作人员沟通,有时还需要直接与殡葬设备进行接触测试,如火化机、冷柜的PLC通信测试,可能会导致人员心理不适或精神压力过大,从而引发人力资源风险。我们在会议上对此风险进行评估与分析,为其制订了潜在的应对方案即安排心理医生对团队成员进行心理疏导,最终将其添加至风险登记册,形成了一份正式的风险项以供后续继续跟踪。
    
    
    2023年3月某周的PDCA会议结束后,我对项目绩效进行了例行测量,发现 SPI值异常偏低,经过根本原因分析,发现是某开发人员在与建设单位的沟通中,私自响应了民政局某领导临时提出的需求变更,要求在“养老数据汇聚中心”系统中加增可自由组合的"动态条件"查询功能,取代原先定义好的固定条件查询,并提出UI界面交互帧率60FPS、UI界面多处细节优化点,触发了范围蔓延风险。在了解原因后,我评估了工作量,认为会带来较高的进度和成本风险。随后我召开了项目团队会议,向全体成员再次明确项目需求和范围的变更控制流程。对于已发生的范围变更和进度落后,我们采取了赶工加班、使用高质量的资源如升级前端Babylon.js版本、安排熟悉前端控件的成员改进Vue控件等纠偏操作,最终经过两周的迭代,顺利实施完此次变更,追平进度,成功对此次事件导致的范围、进度、成本等多方面的风险进行了有效控制。

\end{lstlisting}

\chapter{资源管理【9】【todo】}


\section{过渡}

由于本项目团队成员对养老行业了解不深,缺少相关开发经验;

项目涉及干系人众多,包括民政局领导、养老综合信息管理平台负责人、我司领导、试点单位负责人、项目团队成员、供应商等。

且需要开发的系统较多,工作量大,同时包含后端、WEB前端、PDA端、智能卡技术、大屏看板等平台的软件开发;
%%此外本项目还广泛涉及物联网、工业互联网等嵌入式开发,这些工作的质量风险较高,易引发返工,还涉及外购品的质量问题,这些因素都会增加项目风险管理的难度,为项目带来更多的挑战。

因此,我和我的团队从建设初期就认识到项目XX管理的重要性,运用科学的项目管理方法,做好项目XX管理工作。本文结合该项目实践经验,从等几个方面,就该项目的XX管理展开论述。

\section{规划资源管理}


规划资源管理是定义如何估算、获取、管理和利用团队以及实物资源的过程。
本过程的作用是根据项目实际情况确定适用于项目资源管理的方法和程度。


我和我的团队依据项目章程、范围基准、项目进度计划等,参考组织提供的模版,结合本项目实际情况,首先制定了资源管理计划初稿。然后,邀请公司资源管理领域的专家及院方领导等主要干系人对其进行讨论优化,并评审通过,形成了本项目的资源管理计划和团队章程,为接下来的资源估算、获取、控制等提供了依据。

其中,资源管理计划分为团队管理计划和实物资源管理计划,分别用于指导本项目的人力资源管理和实物资源管理;团队章程包括团队价值观、团队共识、沟通指南、会议指南、冲突处理过程等。


\section{估算活动资源}

估算活动资源是估算执行项目所需的人力资源,以及实物资源的类型、数量的过程。本过程的作用是明确项目所需的资源种类、数量等。我和我的团队依据资源管理计划、活动清单等,采取自下而上的方式对本项目的人力资源和实物资源进行估算。首先,根据活动清单估算每一个活动的所需资源,再汇总至相应工作包;然后汇总各个工作包的资源估算,汇总至相应控制账户;最后汇总各控制账户的资源估算,得到本项目的资源需求。例如:实物资源需求包括:一体式电脑130台、扫码墩90只、大办公室一间,培训会议室一间等;人力资源需求包括:JAVE开发工程师13人,测试工程师5人等。此外,考虑到项目必然存在的不确定性,对于实物资源,我特意申请了一定的应急储备;对于人力资源,我通过与公司人力资源部和各职能科室沟通,得到了优先满足本项目的承诺。


\section{获取资源}

获取资源是获取项目所需的人力资源、实物资源和其他资源的过程,本过程的作用是:1.指导资源的选择;2.将获取的资源分配给相应的活动。我深刻的认识到,能否获得最适合项目的各种资源,是项目成功的关键因素之一。所以,我和我的团队依据资源管理计划、资源需求,采用多标准决策分析,对于实物资源,主要从可用性、成本、性能等方面进行考量;对于人力资源,再结合经验、知识、技能、态度、国际因素等方面进行综合评价,然后采用预分派、谈判等方式获取最适合本项目的各种实物资源和人力资源。例如:有多年医院信息系统开发经验的程序员王工,我们采用预分派的方式提前向公司申请确定;JAVE开发工程师、测试工程师等我通过谈判的方式从公司人力资源部门和各职能科室获取;一体式电脑、扫码墩等我们从公司采购部门获取;大办公室、培训会议室等场地我们向医院方申请解决。最后,我们得到了本项目的项目团队派工单、物资资源分配单和资源日历。


\section{建设团队}

建设项目团队是提高项目团队工作能力,促进团队成员互动,改善团队氛围,以提高项目
绩效的过程,不涉及实物资源。本过程的作用是改进团队协作,激励团队成员以及提升项目整体绩效。多年的项目管理经验使我有清醒的认识:一个成功的团队不是一蹴而就的,形成、震荡、规范、发挥、解散是项目团队必须经历的过程。我根据资源管理计划、团队章程、项目团队派工单等,采取多种措施,努力使项目团队快速度过形成、震荡等阶段,尽快发挥出团队的战斗力。例如:团队形成初始,我就将全部成员安排在医院提供的大办公室集中办公,以便大家加快了解,及时沟通;再例如:参考马斯洛需求层次理论,我向院方争取了工作餐,让团队成员可以在职工食堂免费就餐;在医院附近,安排了标准宿舍;视项目进度不定期组织培训和团队建设活动等。


\section{管理团队}

管理项目团队是跟踪团队成员表现,解决问题并管理团队变更以优化项目绩效的过程,不涉及实物资源。本过程的作用是影响团队行为、管理冲突以及解决问题。现代的项目管理理论表明:一团和气的团队不一定是一个高效的团队,冲突不可避免,如何精准地识别冲突的来源,及时地处理冲突,才是团队管理的关键。例如:2023年1月,程序员王工和测试工程师李工发生了争执,相持不下。我马上向双方了解情况,原来是李工在测试工作中发现,LIS标本扫码到采集界面信息提示间隔3秒,而该项质量测量指标明确要求小于0.5秒,他认为是王工的程序有问题;而王工以其多年的医院信息系统开发经验为保证,认为程序没有问题,而是李工的测试用例不完善。我首先认可了李工对工作的认真态度,同时也表达了王工技术能力的信任,等他们冷静下来后鼓励双方一起发散思维,进行根本原因分析,可以采用因果图、直方图等工具识别问题根源。最后,发现是扫码器自带驱动与电脑操作系统不匹配引起的传输响应延迟,更换扫码墩后,扫码响应时间变为0.3秒,最终以合作的方式解决了这个问题。


\section{控制资源}

控制资源是确保按照计划为项目分配实物资源,以及根据计划监督实物资源实际使用情况,并采取必要纠正措施的过程,不涉及人力资源。本过程的作用是:1.确保所分配的资源适时、适地得用于项目;2.资源在不再需要时被释放。我和团队根据资源管理计划、物质资源分配单等定期进行绩效审查和趋势分析,确定实物资源的实际使用情况与计划使用情况保持一致,发现偏差,及时纠正。例如:2023年3月,医院方因业务需要紧急加开了3个人工收费窗口,需要增加相应的一体式电脑、扫码墩等设备,我们立即启用了应急储备,满足了需求;再例如:在集成平台业务集中培训结束后,我们将业务培训的方式转变为现场流动指导,所以将培训会议室交还给医院方,另做他用。





\chapter{沟通管理【3】}


\section{过渡}

由于本项目团队成员对养老行业了解不深,缺少相关开发经验;

项目涉及干系人众多,包括民政局领导、养老综合信息管理平台负责人、我司领导、试点单位负责人、项目团队成员、供应商等。

且需要开发的系统较多,工作量大,同时包含后端、WEB前端、PDA端、智能卡技术、大屏看板等平台的软件开发;
%%此外本项目还广泛涉及物联网、工业互联网等嵌入式开发,这些工作的质量风险较高,易引发返工,还涉及外购品的质量问题,这些因素都会增加项目风险管理的难度,为项目带来更多的挑战。

因此,我和我的团队从建设初期就认识到项目XX管理的重要性,运用科学的项目管理方法,做好项目XX管理工作。本文结合该项目实践经验,从等几个方面,就该项目的XX管理展开论述。


\section{规划沟通管理}

"规划沟通管理是基于每个干系人或干系人群体的信息需求、可用的组织资产,以及具体项目的需求,为项目沟通活动制定恰当的方法和计划的过程。本过程的主要作用:
1、及时向干系人提供相关信息;
2、引导干系人有效参与项目;
3、编制书面沟通计划。"



项目正式启动后,我带领项目团队以项目章程、项目管理计划(资源管理计划和干系人参与计划)和干系人登记册(和需求文件)为依据,着手进行沟通管理计划编制。

首先我和项目团队召开会议对干系人登记册中的所有干系人进行沟通需求分析,将所有干系人分成领导层、技术人员层、以及其他层,针对各层人员制定了不同的沟通方法。

对干系人登记册中的所有干系人进行沟通需求分析,利用权力利益方格(级别、对项目的关心程度),将干系人分为四类:
1.民政局养老服务指导中心领导张某、养老综合信息管理平台负责人刘某、我司领导唐某、试点单位负责人,权力高利益高,需重点管理;
2.民政局其他相关负责人,权力高利益低,需令其满意;
3.项目组成员及专家,权力低利益高,需随时告知;
4.供应商、老人代表及其他干系人,权力低利益低,需随时监督。

进而采用相关沟通技术和沟通模型,考虑了政策意识、文化意识等形成了项目的沟通管理计划。

民政局相关领导张某、我司领导、监理方等,关注里程碑节点信息,在项目阶段总结会议中向他总结汇报,并通过邮件报告每月总结会进展;
平台负责人刘某,需要及时了解项目的详细信息并给项目团队反馈,通过每周PDCA会议、月度总结会议、阶段总结会议等促进他的参与,每日站会也积极邀请他以面对面或视频会议的形式参加。
项目成员间实时沟通通过集团公司办公软件、与项目外干系人联系通过企业微信群、微信、电话联系等方式。

最终结合公司PMO发布的沟通管理计划模板,形成了一份有效的沟通管理计划。



\section{管理沟通}
"管理沟通是确保项目信息及时且恰当地收集、生成、发布、存储、检索、管理、监督和最终处置的过程。
本过程的主要作用是,促成项目团队与干系人之间的有效信息流动。"


在本项目中我带领项目团队根据资源管理计划、沟通管理计划、干系人参与计划等文件,利用沟通技术、沟通技能、项目报告等工具,及时搜集、整理、发布项目绩效信息,采用不同的方式进行了沟通管理工作,并形成了项目沟通记录。为了更好的促进项目团队和干系人之间的有效沟通,我采用多种沟通方式,
例如:


民政局领导和我司领导,我每月通过邮件发送绩效报告和综合进度、质量、风险等信息形成项目报告,对于我司领导我着重汇报项目资金使用情况。我经常与民政局领导进行电话沟通,让其了解养老综合信息管理平台项目的实际进展情况。在项目里程碑报告时我通过图表、演示视频的方式面对面汇报。如2023年3月的汇报中,民政局养老服务指导中心领导上手体验“养老数据汇聚中心”功能,直观感受到项目的效能,提高了领导的参与度,增强了领导对项目成功的信心,同时领导提出UI界面交互帧率达到60FPS的场景要求、UI界面多处细节优化等问题,为提高汇聚中心的质量收集了新需求。

我们通过会议、演示、微信群等方式,积极和试点单位负责人、老人代表等干系人进行互动,搜集和整理他们提出的问题,并管理他们的参与度;

针对项目组成员和监理方,我们采用推式和拉式的沟通方式,把项目日报通过邮件方式推送给相关人员,把项目和公司的一些资料放在群文件里,方便大家使用。

团队成员撰写项目报告中的技术开发报告、每月在公司知识库发布知识,期间强调书面沟通的5C原则,正确的语法和拼写,简洁的表述,清晰的目的,连贯的思维,善用控制语句和承接。

  
    

\section{监督沟通}
"监督沟通是确保满足项目及其干系人的信息需求的过程。
本过程的主要作用是,按沟通管理计划和干系人参与计划的要求优化信息传递流程。"


在2023年1月份的月度例会中,民政局养老服务指导中心领导指出,平台的养老信息统计管理系统对于养老服务部门的工作也有帮助,养老服务处李某对这块功能比较感兴趣,我将其识别为权力高利益高分类,更新到干系人登记册中。

2023年3月月度例会中,养老服务处李某临时提出了新需求,要求在“养老数据汇聚中心”系统中加增可自由组合的"动态条件"查询功能,取代原先定义好的固定条件查询,我将她加入到干系人参与度评估矩阵中,期望为支持,评估为支持,将其需求信息更新到干系人登记册。

我将该问题作为高优先级问题,记录到问题日志中,
对此,我将可能新增的需求综合考虑需求紧迫性、技术实现难易度、变更影响范围、需求的处理时间、可能出现的风险、对预算和进度应急储备的影响等方面,提交变更请求及影响分析到变更控制委员会,开会决定是否批准该变更申请。

会中我说明由于甲方事先没有考虑到此模块,此模块可以极大提升工作效率,应该具有此功能,且实施这一变更对成本和进度的影响在可控范围内。经过CCB的讨论最终通过该变更,输出了变更请求。采取纠正措施,对于养老信息动态查询功能,与该模块负责人刘工进一步讨论需求,安排技术更熟悉的赵工协助解决技术难题。将赵工登记入干系人登记册。合理的沟通和对干系人需求的有效管理,也促进了本项目的顺利实施。


\section{过程材料}


%%有甲方领导、甲方项目接口人、我公司高层领导、我公司团队成员、监理方等等。甲方领导只需要关注一些里程碑节点的信息,我们打算每月一次进行当面汇报。甲方项目接口人,需要及时了解项目的详细信息,需要及时和我们进行反馈,我们打算每周进行一次例会,我公司高层领导主要关注成本、进度、风险等相关信息,我打算每月一次当面汇报,我公司团队成员,需要关注各自每天的工作信息,我打算每天在QQ群里进行沟通,监理方,我们也是每周进行一次例会的形式进行沟通。
%对于重点管理的干系人中,张某
%最终汇总上述信息并经过评审,编制出项目沟通管理计划(如下概要图表),确定了在项目的各阶段,分别用何种工具、以何种方式对特定的干系人进行何种信息的沟通工作。
%%,提高了项目成功的可能性


  %%汇报前我们先明确沟通的目的,使用对方能接受的方式进行沟通,我非常重视沟通技巧,当前养老系统开发模块样件 最后衡量沟通的效果,并记录在项目沟通记录中。
    %%使用恰当的沟通方法,按沟通计划进行信息收集,并将各种项目信息按沟通管理计划分发到相关干系人。
    %%给业主单位的领导则更侧重于进度进展情况;而对总监理工程师则更侧重于质量测量结果。该过程产生了项目沟通记录(如下概要图表),并且通过合理的沟通,与干系人建立信任关系,推动了项目的顺利开展。


\begin{lstlisting}


\end{lstlisting}




%============================
%============================
%============================
%============================
%============================
\chapter{干系人管理【10】}

\section{过渡}

%%由于本项目涉及题材敏感、驻场地点特殊、应用环境少见等特点;

由于本项目涉及数据源多样、驻场单位多、干系人要求多,
且需要充分考虑建设单位、用户单位、老人群体等多方面对可交付成果的期望;
此外本项目还广泛涉及物联网、工业互联网等嵌入式开发,个人信息安全保护工作,这些因素都会增加交付产品和服务的复杂度,以及项目管理过程中的难度,为项目带来更多风险和挑战。

因此,我和我的团队从建设初期就认识到项目XX管理的重要性,运用科学的项目管理方法,做好项目XX管理工作。本文结合该项目实践经验,从规划XX管理,......等几个方面,就该项目的XX管理展开论述。

\section{识别干系人}

定期识别项目干系人,分析和记录他们的利益、相互依赖性、影响力和潜在影响。
作用是建立起团队对干系人和干系人群体的适度关注。

我组织团队成员根据项目章程、干系人参与计划、沟通管理计划、协议等文件,使用头脑风暴、文件分析、干系人分析等技术,将识别出的干系人及其影响力、利益等情况记录在干系人登记册中。

我们利用权力利益方格(级别、对项目的关心程度),将干系人分为四类:
1.民政局养老服务指导中心领导、养老综合信息管理平台负责人、我司领导、试点单位负责人,权力高利益高,需重点管理;
2.民政局其他相关负责人,权力高利益低,需令其满意;
3.项目组成员及专家,权力低利益高,需随时告知;
4.供应商、老人代表及其他干系人,权力低利益低,需花最少精力监督。
识别干系人是个反复的过程,随着项目进展,我们对干系人登记册定期检查更新。

采用标杆对照技术,已初步分类的权力利益方格与其他养老项目进行对比,将民政局办公室负责软件装调的孙某作为专家成员,加入到权力低利益高的分组。

采用优先级排序的决策技术,综合考虑干系人权力、利益、影响力、参与程度等多种因素,最重要的干系人前三位分别是民政局养老服务指导中心领导、养老综合信息管理平台负责人、我司领导。

在2023年1月份的月度例会中,民政局养老服务指导中心领导指出,平台的养老信息统计管理系统对于养老服务部门的工作也有帮助,养老服务处李某对这块功能比较感兴趣,我将其识别为权力高利益高分类,更新到干系人登记册中。


%%离退休干部处


\section{规划干系人参与}

根据干系人的需求、期望、利益和对项目的潜在影响,规划干系人参与方法。
作用是提供与干系人有效互动的可行计划。

我们根据沟通管理计划、资源管理计划、干系人登记册等文件,采用标杆对照、优先级排序及干系人参与度评估矩阵等技术来编制干系人参与计划。


其次我们建立了干系人参与度评估矩阵,将识别出的干系人的目前参与程度和期望参与程度信息填在矩阵中,分为不知晓、抵制、中立、支持和领导。如
民政局养老服务指导中心领导,期望为领导,评估为支持
养老综合信息管理平台负责人,期望为领导,评估为领导
我司领导,期望为支持,评估为支持。
试点单位某养老院对接人冯某,期望为支持,评估为中立。

结合干系人参与度评估矩阵,针对不同类别的干系人制定了不同的管理和沟通策略,一并汇总到干系人管理计划,作为我们团队后期干系人管理的指导。%针对每个干系人的互动计划并提交评审通过。

如项目里程碑节点前向民政局领导展示项目进展、演示现阶段实现功能,根据领导指导意见优化阶段交付物。

阶段交付物开发中,邀请老人试用智能化自助服务体检系统。


\section{管理干系人参与}


与干系人进行沟通和协作,以满足他们的需求和期望,并处理问题,以促进干系人合理参与的过程。本过程的执行要点是:促进干系人参与到项目中。本过程中,我按照干系人管理计划和沟通管理计划制定的策略,充分应用各种沟通方法、管理技能以及人际关系技能,促进各干系人参与到项目中来。

对照干系人参与度评估矩阵,对于当前参与程度和期望参与度一致的干系人,我们会继续使用当前的策略促进参与;对于参与度不一致的干系人,我们会针对性地实施策略。

主要采取的措施如下:

1.针对民政局领导、我司领导,我会在月末和关键里程碑点,及时汇报项目进度、成本、质量和风险等情况;

2.针对平台负责人刘某,每次周例会,都会主动邀请他参加,每次月度例会他都出席。在项目团队组织春游、秋游时,也会优先考虑他的空闲时间,确定团建的具体日期;

3.至于项目内成员,我们采用集中办公为主,辅以虚拟团队的形式,线下会议与线上会议相结合,有突破性进展时,会及时举行团建活动;

4.针对项目涉及到公司内部相关行政人员,在项目需要时,随时告知项目进展情况,以获取项目资源;

5.对于供应商,我们主要采用随时监督进行管理。




\section{监督干系人参与}

"监督干系人参与是监督项目干系人的关系,并通过修订参与策略和计划来引导干系人合理参与项目的过程。
本过程的主要作用是,随着项目进展和环境变化,维持或提升干系人参与活动的效率和效果。"

2023年3月月度例会中,养老服务处李某临时提出了新需求,要求在“养老数据汇聚中心”系统中加增可自由组合的"动态条件"查询功能,取代原先定义好的固定条件查询,我将她加入到干系人参与度评估矩阵中,期望为支持,评估为支持,将其需求信息更新到干系人登记册。

我将该问题作为高优先级问题,记录到问题日志中,
对此,我将可能新增的需求综合考虑需求紧迫性、技术实现难易度、变更影响范围、需求的处理时间、对预算和进度应急储备的影响等方面,
采取纠正措施,对于养老信息动态查询功能,与该模块负责人刘工进一步讨论需求,安排技术更熟悉的赵工协助解决技术难题。将赵工登记入干系人登记册。

当开发需求出现歧义时,积极与平台负责人、试点单位负责人冯某沟通明确,如通过视频会议讨论“养老数据汇聚中心”的UI界面交互帧率达到60FPS的场景要求、UI界面多处细节优化等问题,引导干系人积极参与项目,经过直观感受到项目的效能,2周后再次评估干系人参与度时冯某变为支持。



\section{过程材料}

\begin{lstlisting}
   

\end{lstlisting}





\chapter{采购管理【2】}

\section{过渡}

由于本项目涉及数据源多样、驻场单位多、干系人要求多,

在本项目的管理工作中,涉及采购
医疗自助智能一体机、智能卡系统、服务器、交换机等硬件设备,

广泛涉及物联网、智能穿戴设备等嵌入式开发,对硬件设备的性能和质量方面有较高的要求,

提高了项目XX管理过程中的难度,为项目带来更多风险和挑战。

因此,我和我的团队从建设初期就认识到项目XX管理的对项目成功的重要性,运用科学的项目管理方法,做好项目XX管理工作,
确保项目的最终成功。

本文结合该项目实践经验,从
合同的签订管理、合同的履行管理、合同的变更管理、合同的档案管理、合同的违约索赔管理
等几个方面,就该项目的XX管理展开论述。


\section{规划采购管理}

规划采购管理是记录项目采购决策、明确采购方法,及识别潜在卖方的过程。
本过程的主要作用是确定是否从项目外部获取货物和服务,如果是,则还要确定将在什么时间、以什么方式获取什么货物和服务。
货物和服务可从执行组织的其他部门采购,或者从外部渠道采购。


在项目开始后,
我多次去民政局、试点单位走访调研,充分了解建设方需求,
组织甲方代表、公司采购部代表、公司技术负责人、以及团队成员等相关干系人,召开采购专题会议,
以项目章程、项目管理计划、组织过程资产为依据,开展采购规划。

首先开展
自制与外购分析:我们通过计算得出,自制成本:硬件150万元,软件100万元;外购成本:硬件120万元,软件120万元。
同时考虑到软件部分需要紧密结合民政局养老部门需求,专业性强;硬件部分供应商能够提供更好的设备质保,
最终我们决定软件自制,硬件外购。招标采购的硬件设备,需求明确,并结合招标文件中规定,使用总价合同方式签订合同。
本项目中设备的采购主要是通过公开招标的方式进行采购。

其次制定采购策略
采购策略中明确采购交付方法:乙方需在项目启动之初提交服务器、存储设备、交换机、防火墙、应用中间件,其余设备需在项目进入到系统测试前提交,具体日期以采购合同为准。

并制定
供方选择标准,基于质量和技术方案得分,其中技术标权重0.6,经济标权重0.4,按综合评分选择供货合作商。

最终得到采购管理计划、采购工作说明书(规格、数量、技术参数、到货时间、服务和责任等)、招标文件为采购管理提供指南和方向。

%%通过自制和外购分析、供方选择标准、招标文件、采购管理计划、采购策略、采购工作说明书。

\section{实施采购}

实施采购是获取卖方应答、选择卖方并授予合同的过程。
本过程的主要作用是,选定合格卖方并签署关于货物或服务交付的法律协议。本过程的最后成果是签订的协议,包括正式合同。


首先我们通过本市政府采购中心网站登记了招标信息,发布了招标公告。

然后我们成立了评标委员会,评标委员会设置有5人组成,其中包括技术领域的专家3名,经济领域的专家1名,公司管理人员1名。

截止投标之日共收到5家供应商投标,经过资格审查共有4家符合要求。

开标时,由投标人检查投标文件的密封情况,经确认无误后,由工作人员当众拆封宣读投标文件内容;评标委员会按照评标标准和方法,对4家符合条件的供应商进行评审、比较、打分,并根据打分结果进行了排名,评标委员会出具了书面评标报告,并推荐了排名第一的A公司作为供应商。

确定供应商后,我们向中标人发出中标通知书,并同时将中标结果通知所有未中标的投标人。

公示结束后,30日内,我们与A公司签订了采购合同(详见附件)。根据招标文件要求,为了确保采购合同顺利执行,A公司也缴纳了履约保证金15 万元。

合同的付款方式:按照项目关键里程碑设置多个进度点,每个里程碑能如期完成的支付特定约定款项,一般为合同的20\%—30\%。未能按期完成的,每延期1工作日扣千分之一。平台项目能如期完工,并顺利通过试运行及验收,按合同说明共支付90\%的合同额。项目建设完成后,留10\%作为尾款在供方免费维护期过后一年付清。最终签订合同,金额为110万,工期6个月。


\section{控制采购}

控制采购是管理采购关系、监督合同绩效、实施必要的变更和纠偏,以及关闭合同的过程。
本过程的主要作用是,确保买卖双方履行法律协议,满足项目需求。


在2022年2月第二周,我用挣值分析方法对项目绩效进行评价时,发现SPI=0.98,CPI=1,进度滞后,经根本原因分析,
采购合同签订后到首次交付之间横跨春节假期,由于供应商方面的原因导致购买的服务器比合同约定的时间晚了4个工作日,导致我方工作人员不能按时工作。我们及时和甲方、供货方进行了沟通,取得了甲方的谅解,供货方也同意加派人手进行设备的安装和调试,及时进行了纠偏。


在现场安装环节,供货方反馈由于甲方机房现状的限制,机房的网络拓扑需要变更。我们严格采用项目变更控制流程,由供货方提出变更申请,我方进行变更初审和论证,CCB进行审查和批准,供货方变更实施,我方和监理方进行监控,完成后进行变更效果验证,确保项目回归正常的流程。最后更新了项目文件和采购管理计划。


在本项目中,远程视频探望终端原计划需求数量是5台,2023年8月,民政局提出要增加5台,根据政府采购法的规定,为保证原有采购项目一致性或者服务配套的要求,可以继续从原供应商处添购,且添购资金总额不超过原合同采购金额10\%,于是我们向CCB提出了增购5台远程视频探望终端的合同变更申请,CCB审批后,我们本着“公平合理”的原则与A公司协商,先确定了变更的数量以及供货细节,再确定变更设备的价格按原中标价进行核算。最终达成一致,变更得以顺利实施,确保了项目在规定时间内完工。

%%如2023年3月的汇报中,民政局养老服务指导中心领导针对“养老数据汇聚中心”功能,提出UI界面交互帧率达到60FPS的场景要求,我们针对问题经根本原因分析发现是监视设备与数据线路问题。。。。

\section{过程材料}

\begin{lstlisting}
   

\end{lstlisting}







\chapter{合同管理【ADD】}


\section{过渡}

由于本项目涉及数据源多样、驻场单位多、干系人要求多,

在本项目的管理工作中,涉及采购
医疗自助智能一体机,

此外本项目还广泛涉及物联网、工业互联网等嵌入式开发,个人信息安全保护工作,这些因素都会增加交付产品和服务的复杂度,以及项目管理过程中的难度,为项目带来更多风险和挑战。

因此,我和我的团队从建设初期就认识到项目XX管理的对项目成功的重要性,运用科学的项目管理方法,做好项目XX管理工作。
减少合同纠纷,确保项目的最终成功。

本文结合该项目实践经验,从
合同的签订管理、合同的履行管理、合同的变更管理、合同的档案管理、合同的违约索赔管理
等几个方面,就该项目的XX管理展开论述。


\section{合同签订管理}

合同签订管理就是在保证公平公正的情况下签订一份对买卖双方具有一定约束力和法律效力的合同进行管理的过程。

项目确立初期,我组织项目团队成员开会,给大家发放了项目管理计划、干系人登记册、风险登记册、需求文件、项目进度计划等材料。
最终确定对于项目实施过程中涉及到的服务器、存储设备、交换机、防火墙、应用中间件、负载中间件等硬件设施采用公开招标的方式获得。

招标过程严格按照招投标流程进行,通过公开招标最终确定A公司为我们的供应商。

公示结束后,30日内,我们与A公司签订了采购合同(详见附件)。根据招标文件要求,为了确保采购合同顺利执行,A公司也缴纳了履约保证金15 万元。




\section{合同履行管理}

合同履行管理包括对合同的履行情况进行跟踪管理,主要指对合同当事人按合同规定履行应尽的义务和应尽的职责进行检查,及时、合理地处理和解决合同履行过程中出现的问题,包括合同争议、合同违约和合同索赔等事宜。

在合同履行过程中,我们首先协商解决,按照《合同法》有关合同争议处理如下规定进行处理。

履行费用不明确的,由履行义务一方承担等等。如我们在合同履行过程中,对智能化自助服务体检系统终端运输至五家试点单位的费用承担产生了争议,于是我们根据《民法典》中“履行费用不明确的,由履行义务一方承担”这一原则解决了争议。




\section{合同变更管理}
项目的建设过程中难免出现一些不可预见的事项,包括要求修改或变更合同条款的情况,如改变系统功能、开发进度、成本支付及双方各自承担的职责等。

在本项目中,远程视频探望终端原计划需求数量是5台,后来民政局提出要增加5台,根据政府采购法的规定,为保证原有采购项目一致性或者服务配套的要求,可以继续从原供应商处添购,且添购资金总额不超过原合同采购金额10\%,于是我们向CCB提出了增购5台远程视频探望终端的合同变更申请,CCB审批后,我们本着“公平合理”的原则与A公司协商,先确定了变更的数量以及供货细节,再确定变更设备的价格按原中标价进行核算。最终达成一致,变更得以顺利实施,确保了项目在规定时间内完工。



\section{合同档案管理}

合同档案管理(文本管理)是整个合同管理的基础。

合同文本是合同内容的载体,我们重点关注合同文本与合同文本格式。

1)合同的正本、副本的管理,合同签订时我们就采取的是一式4份合同,其中正副本各两份,双方各执2份。
文本原件放在档案室本项目专属保密柜中,合同扫描后作为重要的项目文件,登记录入PMIS系统。

2)对合同文本格式的管理,我们所有合同一律采用计算机打印,明确规定了手写旁注和修改无效。



\section{合同违约索赔}

合同违约是指信息系统项目合同当事人一方或双方不履行或不适当履行合同义务,应承担因此给双方造成的经济损失的赔偿责任。

在本项目中,
采购合同签订后到首次交付之间横跨春节假期,由于供应商方面的原因导致购买的服务器比合同约定的时间晚了4个工作日,导致我方工作人员不能按时工作,
于是,我通过索赔的流程对供货方提出了索赔。首先我发出了索赔意向通知书,然后在28天内又提交了详细的索赔资料,最后,供货方也认可了本次索赔,对我方进行了经济补偿并承诺在后面的调试安装过程中通过赶工的方式来弥补工期。


\section{过程材料}

\begin{lstlisting}
   
\end{lstlisting}





\chapter{团队绩效域【6】【0】}



一、团队文化

团队文化反映了项目团队中个体的工作和互动方式,每个团队都会展现出自己的团队文化。作为项目经理,我深刻体会到建立一个安全、尊重、无偏见的团队文化是非常重要的。我和团队成员共同努力,通过透明、诚信、积极的讨论、支持、尊重、勇气、庆祝成功建立我们的团队文化。

关于透明,首先团队集中办公,能够促进大家交流,随时分享传递关于项目开展过程中的所需信息。同时,我设置BVC,展示项目进度网络图、高优先级风险状态、高优先级问题状态,促进大家获取信息,关注进度制约因素以及相应的风险和问题。

当项目通过关键里程碑验收,我会组织庆祝小仪式,并且请示公司领导莅临项目慰问团队,使大家获得认可,并朝着项目整体目标稳定前行。

项目开展过程中,我们重视积极地讨论。例如,形成阶段,我经常组织办公室下午茶,让大家分享参与过的智慧交通项目,使大家相互熟悉并促进交流。震荡阶段,我及时引导冲突发生的相关人员,通过合作探讨促进相互理解和配合。规范阶段,大家开始在定期组织的经验教训交流会上积极发言,并不断增加知识和技能。发挥阶段,我引导团队积极讨论并自主制定决策,以提升参与度和责任。



二、高绩效团队

如何聚拢一盘散沙的个体形成高绩效团队我认为是需要重点关注的。我和团队共同通过开诚布公的沟通、共识、共享责任、信任、协作、适应性、韧性、赋能、认可打造高绩效团队。首先,我们以每个小组为单位,在开发库范围内实行集体代码所有,以共享责任,促进小组内部各成员之间的工作配合和协作。对于小组之间的协作,我灵活采用不同权利进行促进。震荡阶段,研发租和实施组关于数据系统和MEC边缘计算单元的对接频繁冲突,我适当使用惩罚权利,颁布绩效考核规则,惩罚协作不利的双方,以快速建立团队协作。规范阶段,我逐渐增加奖励权利,通过评选“每月之星”等方法,并匹配绩效加分,促进团队成员之间更默契的协作。由于我在整个项目期间始终保持对目标的执着以及面对困难时候的坚毅,可能也适当提升了团队成员的韧性。而对于车辆管理系统,由于创新程度较高,并且相关需求不够明确。例如是否对救护车、消防车提供应急情况下的智能全路段放行功能等,同时这些功能实施标准也难以衡量。为提高适应性,受斯泰西复杂性模型启发的不确定性和复杂性模型,开发1组基于迭代的敏捷进行开发,在短时间内探讨可行性,并根据评估和反馈快速调整。产品经理在每周的冲刺后同甲方以及团队充分沟通,重写排列待办事项列表,以促进适应性,并实现价值交付。




三、领导力技能

领导力技能对每位团队成员都是非常重要的,包含建立和维护愿景、人际关系技能、批判性思维以及激励。本文重点探讨其中的人际关系技能。人际关系技能包含决策、情商以及冲突管理。情商包含自我意识、自我管理、社交意识和社交技能。是评估自己的情绪、动机、优势、劣势,控制破环性感受和冲动,以及理解他人展现同理心并建立融洽关系的技能。项目中冲突是不可避免的,冲突来源于资源稀缺、进度优先级排序、个人工作风格差异等。基于具体冲突,

我首先分析冲突是正面的还是负面的,并观察团队成员是否有效解决了冲突。并且在冲突升级之前,灵活运用冲突解决的5个方法进行解决,包含合作解决、撤退/回避、缓和/包容、妥协/调解、强迫/命令。对于项目中的决策,可使用单方面决策、群体决策、发散汇聚模式进行处理。对于影响不大,并且不涉及小组外协作的事项,通常使用单方面决策,在小组内自行开展,以提高效率。对于跨小组协作的协作,我们尽量使用发散汇聚方式开展,以充分汲取团队知识并提升效率。例如,开发Ⅰ组计划变更数据库主从配置以解决数据转换问题。由于该决策涉及实施团队MEC配置工作,我建议小组发出会议通知,明确会议决策主题、时间、地点、参会人员等。使参会人员提前思考最佳决绝方案,并在会议上充分沟通交流。促进了团队协作。





\chapter{干系人绩效域【7】【0】}



\section{识别}

定期识别项目干系人,分析和记录他们的利益、相互依赖性、影响力和潜在影响。
作用是建立起团队对干系人和干系人群体的适度关注。

我组织团队成员根据项目章程、干系人参与计划、沟通管理计划、协议等文件,使用头脑风暴、文件分析、干系人分析等技术,将识别出的干系人及其影响力、利益等情况记录在干系人登记册中。



在2023年1月份的月度例会中,民政局养老服务指导中心领导指出,平台的养老信息统计管理系统对于养老服务部门的工作也有帮助,养老服务处李某对这块功能比较感兴趣,我将其识别为权力高利益高分类,更新到干系人登记册中。

\section{理解与分析}


根据干系人的需求、期望、利益和对项目的潜在影响,规划干系人参与方法。
作用是提供与干系人有效互动的可行计划。

我们根据沟通管理计划、资源管理计划、干系人登记册等文件,采用标杆对照、优先级排序及干系人参与度评估矩阵等技术来编制干系人参与计划。


其次我们建立了干系人参与度评估矩阵,将识别出的干系人的目前参与程度和期望参与程度信息填在矩阵中,分为不知晓、抵制、中立、支持和领导。如
民政局养老服务指导中心领导,期望为领导,评估为支持
养老综合信息管理平台负责人,期望为领导,评估为领导
我司领导,期望为支持,评估为支持。
试点单位某养老院对接人冯某,期望为支持,评估为中立。

结合干系人参与度评估矩阵,针对不同类别的干系人制定了不同的管理和沟通策略,一并汇总到干系人管理计划,作为我们团队后期干系人管理的指导。%针对每个干系人的互动计划并提交评审通过。

如项目里程碑节点前向民政局领导展示项目进展、演示现阶段实现功能,根据领导指导意见优化阶段交付物。

阶段交付物开发中,邀请老人试用智能化自助服务体检系统。


\section{优先级排序}


我们利用权力利益方格(级别、对项目的关心程度),将干系人分为四类:
1.民政局养老服务指导中心领导、养老综合信息管理平台负责人、我司领导、试点单位负责人,权力高利益高,需重点管理;
2.民政局其他相关负责人,权力高利益低,需令其满意;
3.项目组成员及专家,权力低利益高,需随时告知;
4.供应商、老人代表及其他干系人,权力低利益低,需花最少精力监督。
识别干系人是个反复的过程,随着项目进展,我们对干系人登记册定期检查更新。

采用标杆对照技术,已初步分类的权力利益方格与其他养老项目进行对比,将民政局办公室负责软件装调的孙某作为专家成员,加入到权力低利益高的分组。

采用优先级排序的决策技术,综合考虑干系人权力、利益、影响力、参与程度等多种因素,最重要的干系人前三位分别是民政局养老服务指导中心领导、养老综合信息管理平台负责人、我司领导。


\section{参与}




与干系人进行沟通和协作,以满足他们的需求和期望,并处理问题,以促进干系人合理参与的过程。本过程的执行要点是:促进干系人参与到项目中。本过程中,我按照干系人管理计划和沟通管理计划制定的策略,充分应用各种沟通方法、管理技能以及人际关系技能,促进各干系人参与到项目中来。

对照干系人参与度评估矩阵,对于当前参与程度和期望参与度一致的干系人,我们会继续使用当前的策略促进参与;对于参与度不一致的干系人,我们会针对性地实施策略。

主要采取的措施如下:

1.针对民政局领导、我司领导,我会在月末和关键里程碑点,及时汇报项目进度、成本、质量和风险等情况;

2.针对平台负责人刘某,每次周例会,都会主动邀请他参加,每次月度例会他都出席。在项目团队组织春游、秋游时,也会优先考虑他的空闲时间,确定团建的具体日期;

3.至于项目内成员,我们采用集中办公为主,辅以虚拟团队的形式,线下会议与线上会议相结合,有突破性进展时,会及时举行团建活动;

4.针对项目涉及到公司内部相关行政人员,在项目需要时,随时告知项目进展情况,以获取项目资源;

5.对于供应商,我们主要采用随时监督进行管理。




\section{监督}

"监督干系人参与是监督项目干系人的关系,并通过修订参与策略和计划来引导干系人合理参与项目的过程。
本过程的主要作用是,随着项目进展和环境变化,维持或提升干系人参与活动的效率和效果。"

2023年3月月度例会中,养老服务处李某临时提出了新需求,要求在“养老数据汇聚中心”系统中加增可自由组合的"动态条件"查询功能,取代原先定义好的固定条件查询,我将她加入到干系人参与度评估矩阵中,期望为支持,评估为支持,将其需求信息更新到干系人登记册。

我将该问题作为高优先级问题,记录到问题日志中,
对此,我将可能新增的需求综合考虑需求紧迫性、技术实现难易度、变更影响范围、需求的处理时间、对预算和进度应急储备的影响等方面,
采取纠正措施,对于养老信息动态查询功能,与该模块负责人刘工进一步讨论需求,安排技术更熟悉的赵工协助解决技术难题。将赵工登记入干系人登记册。

当开发需求出现歧义时,积极与平台负责人、试点单位负责人冯某沟通明确,如通过视频会议讨论“养老数据汇聚中心”的UI界面交互帧率达到60FPS的场景要求、UI界面多处细节优化等问题,引导干系人积极参与项目,经过直观感受到项目的效能,2周后再次评估干系人参与度时冯某变为支持。










\chapter{不确定性绩效域【5】【0】}




1、风险

风险是不确定性的一个方面,消极风险称为威胁,积极风险称为机会。在本项目中,我带领项目团队认真做好了风险管理工作。比如我们最开始规划了风险管理,写了风险管理计划。其次,我们识别了本项目中的风险,比如有资源风险、技术风险、进度风险等,然后,我们再分别对这些风险按发生概率和发生后的影响对风险进行了排序,比如排在最前面的是进度风险、其次是技术风险和资源风险。对于进度风险我们还进行了定量风险分析,然后采用了主动接受的策略,比如增加应急储备。最后,我们还做好了风险应对的实施和风险监控工作。通过我们全过程的良好的风险管理,本项目中的风险并没对我们产生不可控制的影响。



2、模糊性

模糊性有两类,概念模糊性和情景模糊性。概念模糊性,即缺乏有效的理解,通过正式地确立共同的规则并定义术语,可以减少概念模糊性。当可能出现多个结果时,就会出现情景模糊性。有多种解决方案可以解决情景模糊性的问题,比如渐进明细、实验和原型法。
在本项目中,一个项目团队成员对我说,我们上周报告的质量没什么问题。于是,我问他,到底是上周的质量没问题,还是说这个情况是上周报告的?你这样给我汇报,这就出现了概念模糊性。为了避免这样的问题,我及时正式确定了共同的规则,并要求大家汇报的时候尽力用词清晰,用专业的术语。比如:我们在质量报告中,比如明确说明该报告的编制日期、检查区间。另外,对于情景模糊性,我是这样处理的。曾经,我们因为系统的界面设计有一些模糊性,我们不知道到底哪种设计更适合各干系人,于是,我采用了渐进明细的方法,我根据用户的基本需求,先设计1个最简单的,然后,让用户给我们提一些意见,不断的更新完善,综合满足用户的需求。当然,我们也可以采用实验,比如让各干系人来使用、体验,让他们提出各设计的缺陷和需要改进的地方,我再来选择一个比较适合的。除此之外。原型法也是不错的解决办法,比如我把各种设计的原型给用户,让他们去测试、使用,最后选1个大部分人接受的。



3、复杂性

复杂性是由于人类行为、系统行为和模糊性而造成的难以管理的项目、项目集或其环境的特征,当有许多相互关联的影响以不同的方式表现出来并相互作用时,就会存在复杂性。在复杂的环境中,单个要素的累积会导致无法预见或意外的结果。
在本项目中,基于系统的复杂性,比如我们在设计的时候,就要求各模块之前低耦合,要让各功能模块都是独立的一部分﹐比如本项目中XX模块出现问题不会影响XX模块的正常使用。针对重新构建的复杂性,我们采用专家判断的方式,聘请具有经验的专家进行头脑风暴分析,从不同的角度来看待系统。针对过程的复杂性,我们要求各干系人,特别是用户频繁参与项目,实时沟通,随时了解收集信息,及时进行完善。


4、不确定性的应对方法

项目中必然存在不确定性,任何活动的影响都无法准确预测,而且可能会产生一系列的不确定性。在本项目中,针对使用需求方面的不确定性,我采用了如下的步骤。
(1)收集信息:我制定了明确的需求管理计划,带领需求分析人员尽力全面的分析目前确定的需求。
(2)为多种结果做好准备:我制定了应急计划,备选方案,以便在用户需求发生变化时,及时进行调整。
(3)集合设计:通过考虑项目的成本与质量、风险、进度等多种因素,选择了一个比较好的方案
(4)增加韧性:培养团队成员敏捷开发、适应性的意识和经验,在用户需求变更的时候能够及时进行迭代和适应、应对变化。




\chapter{度量绩效域【4】【0】}

一、制定有效的度量指标

制定有效的度量指标有助于确保对项目进行度量,并向干系人报告。制定有效度量指标需要遵循SMART特征,即:具体的、有意义的、可实现的、相关的、及时的。因此,我们制定度量指标时,确保符合基准和需求、能够实现、能够带来价值。我们制定的关键绩效指标分为提前指标和滞后指标。提前指标用于预测项目的变化趋势,发现不利的变化趋势,分析根本原因并采取相应措施。滞后指标用于测量项目可交付物或重大事件,反应过去绩效或状况。由于本项目激光雷达、毫米波雷达等硬件设备根据项目实际进度按照T+1方式分批配送,我们设置了提前指标,供应商出库完成率,以提前获取资源供给信息。对于基准,使用滞后指标,通过分析进度偏差、成本偏差,识别问题并采取相应措施。




二、度量内容及相应指标

度量内容、参数和方法取决于项目目标、预期成果和项目环境。常见的度量指标包括可交付物的度量指标、交付的度量指标、基准的度量指标、价值的度量指标、干系人的度量指标、资源的度量指标、预测的度量指标。基准绩效的度量指标主要针对进度基准和成本基准,例如,综合监管系统系统,计划 12月1日开始,12月28日完成,成本估算30万元,要求CPI高于0.98,SPI高于1.00。对于干系人,我们定期开展评估,并使用NPS-100\%-+100\%的净推荐值了解干系人满意度。预测性度量指标用于预测未来情况,以便决定是否采取适当应对措施。例如,我们计算完工尚需绩效指数TCPI,了解剩余工作所需成本与剩余预算的比率。当TCPI小于1时反应剩余预算可支持完成项目,当TCPI大于1是,反应按照当前绩效,剩余预算不足以完成项目。



三、展示度量信息和结果

通常情况下,可使用仪表盘、大型可见图表、看板、燃烧图展示度量信息和结果。对于开发1组基于流程的敏捷开发车辆管理系统时,使用看板展示用户故事状态。包含准备就绪、开发和单元测试、开发完成、系统测试、完成等。通过看板监控提前期和周期时间了解团队效率和开发速度,并及时发现瓶颈和延迟问题。使用看板也可以发现并预防团队开发不在看板中的工作,促进价值交付。由于团队集中办公,我还设置了BVC,展示项目进度、高优先级风险信息、高优先级问题状态,以增加透明。同时,在我们使用控制图测试车辆位置感知的偏离,以便识别需要采取纠正措施的检查点。



四、度量陷阱

项目开展过程中,我们要避免落入度量陷阱。常见的度量陷阱包含霍桑效应、虚荣指标、士气低落、误用度量指标、确认偏见、相关性与因果关系混淆。相关性与因果关系混淆是指两个指标可能仅仅是具有相关性,但是理解为一个因素导致了另一个因素的因果性。例如,有一次,路侧系统时延显示139ms,超过100ms标准。同时,MEC边缘计算单元也给出时延告警。团队认为是MEC引起了整体系统的时延。我纠正了错误的逻辑,并带领团队详细分析,避免了度量陷阱。为了避免士气低落,对于开发1组基于流程的敏捷开发车辆管理系统时,我们有意识的根据团队能力限制在制品,一方面提高团队士气,另一方面保持团队专注,通过避免多任务切换提高团队效率。



五、基于度量进行诊断

我们对度量指标制定临界值,并对超出临界值的度量进行分析。团队不应等到突破临界值才采取行动,应根据趋势和预测主动解决偏差。例如,对于车辆水平位置感知偏离,我们使用控制图进行监控。其中,规格界限是根据要求制定的,反映了可允许的最大值和最小值。控制界限是通过标准的统计原则,通过标准的统计计算获得。反映了一个稳定过程的自然波动范围。我们设置规格上限0.55m,规格下限0m,控制上限0.50m,控制下限0.05m。有一次,控制图反馈0.46、0.33、0.32、0.30、0.22、0.21、0.15。连续7点呈现相同变化方向,我们识别了需要开展纠正措施的检查点。然后,通过因果图展现所有相关原因,包含RTK差分基站设置、数据库主从配置、车辆传感器前装等,最终发现根本原因是RTK设置问题。



六、持续改进

度量的最终目的是为了持续改进,有助于避免问题或缺陷、防止绩效下降、促使项目团队学习、改进项目或产品绩效、推动决策、更好的创造价值等。例如,当发现整体系统时延是由于RSU自身时延原因造成的,我们发起了变更,将原本到货验收通过统计抽样的方式,更改为对每套设备进行严格的性能测试,避免设备运用至现场后导致系统问题才分析并寻找原因。对于RTK差分基站设置问题,我们一方面记入经验教训登记册,另一方面加入核对单并向相关成员开展培训,防止同样错误再次发生。







\chapter{规划绩效域【1】【0】}

\section{一、规划的影响因素}


由于本项目需求明确、范围稳定,结合公司以往项目经验适合采用预测型开发方法(但其中养老数据采集系统通过适应型方法开发)。
%%远程视频探望终端

同时,实施项目过程中涉及采购
医疗自助智能一体机、智能卡系统、服务器、交换机
等硬件设备,需要预先做好采购规划,

广泛涉及物联网、智能穿戴设备等嵌入式开发,对硬件设备的性能和质量方面有较高的要求,

综合如上因素,项目初期我们开展了充分详细的规划。

而对于养老数据采集系统,数据源众多,采集场景、相关需求不明确。例如远程视频探望终端安装位置等,同时这些功能实施标准也难以衡量。

受斯泰西复杂性模型启发的不确定性和复杂性模型,小李和小刘基于迭代的敏捷进行开发,在短时间内探讨可行性,并根据评估和反馈快速调整。因此,在项目初期建立高层级愿景,并在待办列表细化会议上明确下个冲刺的重要规划。


\section{二、项目估算}


影响估算的因素包含区间、准确度、精确度、信心。
项目规划初期,我们进行粗略量级估算估算,准确性在-25\%-75\%。随着规划工作深入开展,相关信息逐渐明确,达到-5\%-10\%确定性估算。更提升了估算的信心。

对于设备资源、材料资源,我们使用绝对估算,例如
智能化老人自助服务终端5套,一键呼叫主机5套,数据汇聚中心服务器4套

对于开发1组基于迭代的敏捷开发养老数据采集系统,使用相对估算评估用户故事规模及预计开发时间,例如孝心手环数据采集功能需要1个故事点,远程视频探望终端数据集成需要3个故事点。

同时,为应对估算不准,对各故事点预留相应应急储备。


\section{三、团队组成和结构规划}

人才是成功之本,我们非常重视团队组成和结构规划。首先,在投标环节我们根据项目开始时间、持续时间、经验,预先承诺了系统架构师、产品经理以及测试工程师。对于其他成员,例如开发工程师,我们确定需要计算机通讯相关专业,具有2年及以上研发经验,熟悉机器学习、深度学习等AI能力,并精通主流V2X功能。由于车路协同项目创新程度较高,同时需要研发专家。综合考虑成本、时间、能力、经验等因素,我们通过虚拟团队方式获取。同时,为了对冲虚拟团队造成沟通困难,我们申请了视频会议系统并计划使用向日葵远程控制程序进行应对。


\section{四、沟通规划}


项目开展过程中,有效的沟通能够提升管理层的支持、促进干系人合理参与、并形成清晰的业务目标。我们依据干系人登记册中己识别干系人的权利、需求、期望、影响程度,结合需求文件中干系人的沟通需求分析对于不同干系人所需信息及沟通频率。并使用权力利益方格,结合信息需求的紧迫性、技术的可用性、信息的敏感性和保密性、易用性和项目环境,开展对于不同干系人所需信息的适当的沟通技术和沟通方法。同时,通过识别千系人参与度差距,开展沟通风格评估,以识别偏好的沟通信息、方式、方法等。项目开展过程中由于干系人众多,我们通过干系人优先级排序,重点聚焦权力大、利益大的干系人。


\section{五、实物资源规划}


由于本项目需要激光雷达、毫米波雷达、全景摄像机、RTK差分基站等设备,数量多、造价高,集中配送到货将带来极大的资金压力,同资金限制相冲突。同时,还会增加仓储成本,并提高丢失、损坏的风险。因此,我们计划根据项目实际进度需求,按照T+1方式分批配送实物资源,并根据实际到货验收情况分批次支付货款,以此拉长配送周期并且符合资金限制,同时降低仓储成本以及丢失损坏的风险。为了达到该目标,计划项目开展过程中由实施团队根据每个路口环境改造的开始时间提交配送申请,采购管理员在供应链系统监督供应商是否出库并反馈物流信息,以满足实物资源需求。


\section{六、采购规划}

由于本项目硬件设备,数量多、造价高,并涉及公共交通安全。同时市面上相关产品繁多,性能、参数、标准各不相同。为了确保整体平台稳定和兼容,我们详细描述了拟采购产品的接口类型、数据格式、通讯协议、性能参数。例如,对于激光雷达需要1550nm半固态定向雷达,视场角900*300,探测精度+/-5cm,全反射可靠探测距离300m,数据协议UDP,网络接口10000Base-t,工作平台Linux,时间同步IEEE1588v2等。同时,综合多项指标设置供方选择标准,商务评分20分,报价评分10分,技术评分70分。例如,商务评分中同类业绩3分,具有IS09001、IS045001等6项认证各得1分,投标文件编制2分,等细则和分值。



\section{七、变更规划}

由于本项目涉及公共交通安全,因此项目规划阶段便设置了严格的变更控制流程(除车辆管理系统外)。变更的原则是遵循基准、变更控制流程化、明确组织分工、评估可能影响、妥善保存变更相关信息。对于具体变更,变更人提交书面变更申请。我进行变更初审,确保变更是必要的,信息充分,并同干系人就变更信息达成共识。接着,开发团队进行方案论证,CCB审查变更。然后我监督基准的调整以及变更的实施。变更完成后同团队开展效果评估以及收尾。对于开发1组基于迭代的敏捷进行开发,遵循了不同的价值观:响应变化重于遵循计划、可用的软件重于详尽的文档、个体和互动重于过程和工具、客户合作重于合同谈判。因此,2周时间盒冲刺完成后召开评审会,获取反馈和评估,之后产品经理通过调整待办事项列表将变更纳入项目,以优化价值交付。


\section{八、度量指标和一致性}


确定度量指标、基准、临界值,以及确定测试和评估方法是规划绩效域的重要工作。由于本项目涉及公共交通安全,因此在规划阶段我们充分开展度量指标的规划。包括基准绩效的度量指标、价值的度量指标、干系人的度量指标、资源的度量指标、可交付物的度量指标、交付的度量指标、预测型的度量指标等。可交付物的度量指标用来描述产品、服务或成果。例如,我们使用控制图监控车辆位置感知偏离。规格界限根据要求制定,反映了可允许的最大值和最小值,控制界限根据标准的统计规则,通过标准的统计计算获得,反映了一个稳定过程自然波动范围。规格上限设置为0.55米,规格下限设置为0.00米,控制上限为0.50米,控制下限为0.05米。预测型度量指标用来预测未来情况,以识别是否需要采取纠正。我们计划通过计算完工尚需绩效指数TCPI,判断剩余工作所需成本与剩余预算的比值,如果比值小于1则剩余预算充足,如果比值大于1,则按照当前情况剩余预算肯定不足。




\chapter{开发方法和生命周期绩效域【2】【0】}



一、交付节奏

交付节奏是指项目可交付物的时间安排和频率。项目可以一次性交付、多次交付、定期交付、持续交付。一次性交付的项目只在项目结束时交付。多次交付的项目在整个项目期间的不同时间交付。定期交付与多次交付非常相似,但需要遵循固定的交付计划。持续交付是将项目特性增量交付给客户,通常通过使用小批量工作和自动化技术完成。本项目除车辆管理系统外,其他部分需求明确、范围稳定,结合以往V2X交付经验以及客户期望,项目验收后一次性交付价值。而对于车辆管理系统,由于创新程度较高,并且相关需求不明确,需要在短时间内探讨可行性,根据评估和反馈快速调整。团队频繁的、小批量的交付增量时,能够更快速、更准确地理解客户的真正需求。因此仅对于该子系统将持续交付。



二、开发方法

开发方法是在项目生命周期内创建产品、服务或成果的方法。当前,行业普遍认同三种开发方法,分别是预测型开发方法、混合型开发方法和适应型开发方法。预测型开发方法又称瀑布型开发方法。这种开发方法相对稳定,范围、进度、成本、资源和风险可以在项目生命周期早期预先明确。混合型开发方法是适应型方法和预测型方法的结合体,该方法中预测型方法的要素和适应型方法的要素均会涉及。混合型开发方法的适应性比预测型方法强,但比纯粹的适应型方法的适应性弱。适应型方法在项目开始时建立高层级的愿景,之后在项目进行过程中在最初已知需求的基础上,按照用户反馈、环境或意外事件来不断说明、完善、更新或替换。



三、开发方法的选择

由于本项目需求明确、范围稳定,结合公司以往V2X项目经验适合采用预测型开发方法。同时,项目涉及公共交通安全,根据《智能网联汽车道路测试管理规范》,XX市车路协同联席小组需要评审项目计划,重点包含测试路段选址、项目质量标准、路侧系统整体时延、云控平台接入并发数等内容。因此,需要通过事先做好规划确保所有安全需求都得到识别、规划、创建、整合和测试。而对于车辆管理系统,由于创新程度较高,并且相关需求不够明确。例如是否对救护车、消防车提供应急情况下的智能全路段放行功能等,同时这些功能交付标准事先难以量化,同时会随着项目研发过程不断发现新的重要的需求。受斯泰西复杂性模型启发的不确定性和复杂性模型,开发1组基于迭代的敏捷进行开发,在短时间内探讨可行性,并根据评估和反馈快速调整。基于迭代的敏捷同预测型开发方法的价值观是不同的,敏捷认为个体和互动重要于过程和工具,可用的软件重要于详尽的文档,客户合作重要于合同谈判,拥抱变化重要于遵循计划。团队将以2周的时间盒进行冲刺,通过评审会频繁小批量的交付价值,并通过更新待办事项列表将价值融入项目,以优化价值的交付。



四、协调交付节奏和开发方法及生命周期

由于该项目涉及预测型及基于迭代的敏捷开发。本环节将重点讨论其中基于迭代的敏捷的内容。

1.创建产品代办列表:我们通过用户故事来创建需求,例如我们创建的一个用户故事“作
为公共交通的监管者,我希望系统能向自动驾驶车辆推送第一视角和第三视角观测到的路侧信息,将原本因道路交叉口由车辆视野阻挡产生的交通事故率降低12\%”,同时,我们确保用户故事制定时清晰、简洁、一致、完整、可跟踪、可核实。这个阶段,我们创建的用户故事是高层级的,并且有些用户故事是史诗级用户故事,需要进一步分解细化。

2.待办事项列表细化:产品负责人会使用1小时时间盒同团队细化待办事项列表,根据用户价值进行优先级排序。一方面让团队了解下个迭代的用户故事,以及故事之间的关系。另一方面提高适应性,根据干系人价值实时将变更融入。

3.执行冲刺:首先,开展冲刺计划会,产品经理和团队对冲刺目标达成共识,根据团队稳定开发速度将相应用户故事纳入冲刺,并进行分工。其次,我召开每日Scrum,由团队成员自行主持,利用15分钟反馈各自计划、进度及问题。会后我根据停车场中汇总的问题提供相应支持。

4.冲刺评审:迭代完成,我组织评审会,产品负责人和客户接受可交付成果或提供反馈。这样,团队成员就可以得到充分反馈,防止他们朝着错误的方向前进。

5.冲刺回顾:我们在迭代结束或者认为需要开展回顾的时候召开回顾会议。回顾针对定性的和定量的数据,然后利用这些数据找到根源,设计对策,并制订计划。同时,我和团队对改进事项做好优先级排序,并确保控制改进的数量。


\chapter{项目工作绩效域【8】【0】}


1.项目过程
项目绩效域重点关注项目是否能够创造价值,进而关注过程是否有效率、有效果。在本项目中,我主要采用定期过程审计和召开论证总结会的方法来优化过程。

比如,在过程审计中,我发现身份认证组费时费力地开发人脸的生物识别方式。于是我组织身份认证组和需求分析组召开了简短的论证总结会,分析讨论出甲方部分厅级领导因眼睛近视加老花导致近距离操作平板需摘掉眼镜,因此开发人脸识别无实际意义,并不会为项目带来额外价值。故我们一致决定删减人脸识别的开发过程,增加与甲方确认具体生物识别方式需求的过程。

2.制约因素
政务类信息系统项目的制约因素重点突出法律法规,特别是行业领域内规范性文件的要求。在本项目中,专网接入组考虑OAAPP登录的便捷性,原计划采用SSL VPN接入方式。但我在监控项目工作时发现:甲方的OA部署在财政业务专网上,且财政业务专网接入有明确的规范性文件,文件要求专网必须与公网物理隔离。SSLVPN接入本质上属于逻辑隔离,不符合财政业务专网接入规范,用户存在跨网操作的安全风险,最终我们将接入方式修改为VPDN接入。



3.关注过程和能力
项目工作关注过程和团队的工作能力是为了始终掌握团队成员的工作进展情况,进而聚焦价值交付。在本项目中,我致力于构建成员敢说真话、表达异议、听得进不同意见的团队。比如,在项目技术方案论证会上,成员就身份认证实现方式,展开了热烈的讨论。讨论焦点方案一是采用oAuth方式,实现统一认证、单点登录,即用户通过独立的身份认证APP只认证一次,便可登录OA APP和预算管理一体化等财政业务系统。焦点方案二是采用报文方式,OA APP及业务系统均单独集成身份认证。考虑到甲方对移动办公系统的实际需要重点体现在OA APP上,且甲方部分年纪较大同志对于操作便捷性的要求,我们最终选用了报文方式。但每一位成员的勇敢表达都是助力于项目成功的经验积累。


4.沟通管理
管理好沟通是取得甲方干系人支持的重要条件,特别是面对有严格汇报结构的甲方,我必须采取行之有效的沟通流程,才能提高可交付成果的被接受度。在本项目中,我就OA APP办文流程的需求确定,设计出了一套沟通流程:针对办文秘书、各处室综合岗,我制作了问卷调查,包括会签意见汇总、另行增加会签处室等工作中的常见细节。

针对办公室正、副主任,我在分析整理问卷调查结果的基础上,利用思维导图核对并细化技术上可实现的需求,形成初步的SRS并胶装成册。逐级呈递SRS,并演示测试原型。当完成OA APP的全部迭代后,再与办公室主任、分管副厅长逐级沟通确认定稿。最终实现了有效的沟通管理。

5.实物资源管理
实物资源管理要尽可能把项目需要的实物资源都想到、想全,并随着项目执行与变更及时调整。由于本项目是多模块并行开发,我尤为重视实物资源的变化情况。比如,OA APP在完成最后的一次迭代进行单元测试时,我发现需要配备8台模拟终端,才能全过程演示收文办理流程。但最初规划时只配备了5部模拟终端,测试人员需不停地切换各类角色的OA测试账号才能完整演示。考虑到OA APP阶段性交付会上需给甲方良好的使用特性和体验感,我又补充了3部模拟终端,最终在交付会上为甲方关键干系人顺畅地完成了全过程收文办理流程演示。


6.采购管理
采购在项目执行过程中可随时进行,但预先规划采购会缩短到货的等待时间,特别是项目高层级目标已明确的情况下,可以就所需资源开展自制/外购分析。比如,本项目中手写签批平板上的终端管控软件(非主体非关键),它主要实现平板只能接入特定网络、只能下载、安装特定应用等管控功能,我们团队也有能力自行开发。自行开发的好处是可以开拓业务、巩固经验;但需要工期、存在适配风险。考虑到项目工期紧张、任务重,最终决定采购平板的终端管控软件,降低风险,提高项目成功的机会。


7.监控变更
项目执行过程中不可避免地产生变更,故监控好变更是项目实现价值交付的根本保证。在本项目中,我组织团队成员严格定义了变更控制流程。比如,项目执行到中后期,甲方新任办公室副主任提出变更,要求调整OA APP中文件批办单的排版样式,调整必然引发历史文件批办单的显示错误,修改在途数据必然会引起进度、成本的偏差。于是,我组织团队成员基于当前已完成的项目工作,将变更技术方案提交CCB,最终CCB批准了变更的同时,也确认了在途数据修改的解决方案,以及进度延期、追加部分工时数费用。此项变更的全过程记录在变更日志中,查有可据。

8.学习与持续改进
学习的精髓在于温故知新,持续改进,更好地创造价值。特别是总结经验教训,好的是经验、差的是教训。比如,我根据本项目总结出经验有:

一是涉及业务需求高、技术要求低的功能模块,比如OA APP的六类办文流程及运转节点设置,其开发方法最好采用适应型,通过创建原型进行多次迭代,以适应甲方多位干系人的多变需求;
二是变更申请、需求确认等要形成书面文字材料,呈递甲方领导签字确认;

三是和甲方领导沟通技术实现方式要通俗易懂,尽量不使用信息化专业术语,要善用类比论证。




\chapter{交付绩效域【3】【0】}


一、价值交付

由于本项目除车辆管理系统外,其他需求明确、范围稳定,结合公司以往V2X经验适合使用预测型开发方法,价值将在最终验收完成后一次性交付。而对于车辆管理系统,由于创新程度较高,并且相关需求不够明确。例如是否对救护车、消防车提供应急情况下的智能全路段放行功能等,同时这些功能实施标准也难以衡量。受斯泰西复杂性模型启发的不确定性和复杂性模型,开发1组基于迭代的敏捷进行开发,在短时间内探讨可行性,并根据评估和反馈快速调整。基于迭代的敏捷同预测型开发方法的价值观是不同的,敏捷认为个体和互动重要于过程和工具,可用的软件重要于详尽的文档,客户合作重要于合同谈判,拥抱变化重要于遵循计划。团队将以2周的时间盒进行冲刺,通过评审会频繁小批量的交付价值,并通过更新待办事项列表将变更融入项目,以优化价值的交付。




二、可交付物

在交付项目可交付物的过程中,重点需要关注需求启发、不断演变和发现的需求、管理需求、定义范围和管理范围。由于项目涉及管委会工信局、公安局、规划局、道路与交通运输局等局办,干系人需求描述模糊,并存在冲突,我们有意识使用用户故事和引导进行需求启发。例如,我们通过编写用户故事“作为公共交通的监管者,我希望系统能够向自动驾驶车辆推送第一视角、第三视角观察到的路侧情况,将因道路交叉口车辆阻挡视线而造成的交通事故率降低 12\%”。通过使用用户故事,方便需求获取、沟通和评估。同时,在制定需求文件以及用户故事时候,我们遵循清晰、简洁、一致、完整、可跟踪、可核实。对于其中使用预测型开发方法开发的项目内容,我们使用需求跟踪矩阵管理需求。需求跟踪矩阵是一种将需求从其来源连接到能满足需求的可交付成果的表格。使用需求跟踪矩阵将需求同项目目标或业务目标联系起来,使需求具有价值。同时提供了跟踪需求的办法,使项目结束时需求可交付。其中开发1组基于流程的敏捷则使用产品代办列表和燃尽图对需求进行管理和监控。为了应对范围蔓延,我们还设置了严格的变更控制流程,申请者提交变更申请,我开展初审,实施团队进行方案论证,CCB审批变更,同时我监督基准的调整和变更的实施,变更结束后我们开展效果评估及收尾。对于车辆管理系统的研发,团队使用适应型开发方法,通过代办事项列表细化会调整工作优先级,以管理不断演变和新发现的需求。




三、质量

范围和需求聚焦于需要交付的内容,而质量聚焦于需要达到的绩效水平。项目管理过程中,需要在质量和满足质量所付出的成本之间寻找平衡,我们使用质量成本规划质量管理,包含一致性成本和非一致性成本。一致性成本包含预防成本和评价成本,非一致性成本包含内部失败和外部失败。预防成本是预防产品、服务或成果缺陷所产生的成本,例如我们开展对员工的培训。评价成本是评价产品、服务或成果而产生的成本。例如,我们对路侧系统开展时延测试产生的成本。内部失败是项目内部发现失败的成本。例如,控制质量过程中,我们发现数据系统和MEC数据库主从配置不一进行返工造成的成本。外部失败是客户发现失败的成本。例如验收环节客户发现云控可视化系统UI界面甲方元素需要调整,进行返工造成的成本。我们依据行业数据库以及公司同类项目历史数据,寻找到预防和评估成本的最佳平衡,用来规避失败成本。



\chapter{其他}
\begin{lstlisting}
 
\end{lstlisting}

\end{document}





