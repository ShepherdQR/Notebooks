%%============
%%  ** Author: Shepherd Qirong
%%  ** Date: 2022-06-05 00:25:44
%%  ** Github: https://github.com/ShepherdQR
%%  ** LastEditors: Shepherd Qirong
%%  ** LastEditTime: 2022-06-05 23:31:45
%%  ** Copyright (c) 2019--20xx Shepherd Qirong. All rights reserved.
%%============


\documentclass[UTF8]{../computerUniverse}

\begin{document}

\title{07-10-Application}
\date{Created on 20220605.\\   Last modified on \today.}
\maketitle
\tableofcontents

\chapter{Introduction}

\section{与自然科学相关}
% %%在计算数学,自然科学,工程和医药:
% \subsection{代数(符号)计算}
% \subsection{计算生物学(生物信息学)}
% \subsection{计算化学}
% \subsection{计算数学}
% \subsection{计算神经科学}
% \subsection{计算数论}
% \subsection{计算物理}
% \subsection{计算机辅助工程}
% \subsection{计算流体动力学}
% \subsection{有限元分析}
% \subsection{数值分析}
% \subsection{科学计算(计算科学)}

\section{与社会科学相关}
% %%社会科学、艺术、人文和专业领域的计算:
% \subsection{社区信息学}
% \subsection{计算经济学}
% \subsection{计算金融}
% \subsection{计算社会学}
% \subsection{数字人文(人文计算)}
% \subsection{计算机硬件的历史}
% \subsection{计算机科学史}
% \subsection{人文信息学}

\section{与数据相关}
% \subsection{数据库}
%     \subsubsection{分布式数据库}
%     \subsubsection{对象数据库}
%     \subsubsection{关系数据库}
% \subsection{数据管理}
% \subsection{数据挖掘}
% \subsection{信息架构}
% \subsection{信息管理}
% \subsection{信息检索}
% \subsection{知识管理}
% \subsection{多媒体,超媒体}
%     \subsubsection{声音和音乐计算}
% \subsection{人机交互}
% \subsection{图像处理与科学可视化}



\subsection{科学计算(计算科学)}
\section{工程与科学计算方面,对数值模拟进行的多学科的应用研究}

\chapter{代数(符号)计算}
\chapter{计算生物学(生物信息学)}
\section{计算神经科学}

\chapter{计算化学}

\chapter{计算数学}
\section{计算数论}
\section{数值分析}

\chapter{计算物理}
\section{计算流体动力学}
\section{有限元分析}

\chapter{计算机辅助工程}
\section{CAD/CAM技术的理论研究、CAD/CAM系统的软件开发平台研制}


\chapter{计算社会学}
\section{社区信息学}

\chapter{计算经济学}
\section{电子商务}
\section{计算金融}


\chapter{数字人文(人文计算)}

\chapter{计算机历史}
\section{计算机硬件的历史}
\section{计算机科学史}



\chapter{数据库}

\section{基本概念}
数据库,database,按照数据结构来组织、存储和管理数据的仓库。
MySQL,需要熟练。关系型数据库。
程序+数据=软件。
应用程序是操作和查询数据库服务器,数据库服务器响应和提供数据。
数据库:增删改查。
数据库管理系统,dbms。
MySQL免费,小项目,一百万以下的数据。\\
甲骨文ORACLE,可以提供支持,大项目。\\
老的:sybase,银行、电信系统。db2。\\
分布式:HBASE,mongoDB。\\
嵌入式SQLite
SQL Server
Postgre SQL
数据库系统(Database System)包括数据库(Database),数据库管理系统(Database Management System),应用开发工具,管理员及用户。




\section{数据库}
B-S体系中,系统安装、修改、维护都只在服务器端。

\subsection{三级模式}
物理数据库(对应一个文件)《-》内模式(数据的存储形式)《-》概念模式(数据划分为表)《-》外模式(各种用户视图)《-》用户

\subsection{数据库设计}
需求分析:数据处理要求、应用要求,输出数据流图、数据字典、数据说明书;

概念结构设计,输出ER模型【重点】,用户数据模型;

逻辑结构设计:转换规则、规范化理论,输出关系模式,视图、完整性约束、应用处理说明书;

物理设计:结合DBMS特性、硬件特性、OS特性等。

\subsubsection{ER模型}
实体,方框表示,如学生、课程;
属性,椭圆表示,如学号、姓名;
联系,菱形,如学生-----M-选课-N-----课程。

集成:画局部,合成整体的。集成会存在属性冲突、命名冲突、结构冲突等。

关系O对应的三个实体之间的联系数量分别为A,B,C,则最少转换为4个关系模式(每个实体1个,关系至少一个)。

\subsubsection{关系代数}

关系代数表达式的等价转换;
业务场景对应的关系代数表达式。

并
交
差
笛卡尔积
投影
选择
连接

\subsection{规范化理论}
解决数据冗余、更新异常、插入异常、删除异常等。

\subsubsection{函数依赖}
部分函数依赖:如当主键为(学号,课程)的复合键时,可以确定姓名;但是主键的一部分--学号,可以确定姓名。
传递函数依赖:A->B->C,有A->C


\subsubsection{键}
超键:唯一标识元组,可以存在冗余属性。
候选键:唯一标识元组,不存在冗余属性。
主键:任选一个候选键。
外键:其他关系的主键。关联各表。

\subsubsection{范式}
1NF,属性值不可分,原子值;
2NF,由1NF消除非主属性对候选键的部分依赖得到;
3NF,由2NF消除非主属性对候选键的传递依赖得到;
BCNF,由3NF消除主属性对候选键的传递依赖得到。
越高级,数据表拆分越细。粒度小,性能下降。

\subsubsection{模式分解}
表的拆分。
保持函数依赖:冗余的依赖不需要保留。

无损分解:可以还原。

\subsection{并发}
事务机制:一系列操作作为一个整体,原子性、一致性、隔离性、持续性。

PV操作可实现资源互斥使用。

\subsubsection{并发问题}

丢失更新,T1与T2都读取A,分别将A-5,写回。有可能某个进程后写回导致A只是-5

不可重读读取:T1读取数据计算后,在T1校验计算结果时,T2进程修改数据,导致T1校验失败。

错误数据读出:T1修改了数据,T2读到了修改的数据,T1进行回滚恢复了数据,导致T2的数据无意义。

\subsubsection{封锁协议}
S封锁,X封锁。
一级:事务T修改数据R之间必须加X锁。防止丢失修改。
二级:一级,且T读取R之间S锁,读完释放S锁,防止丢失修改、读脏数据。
三级:一级,且T读取R之间S锁,事务完成再释放S锁,防止丢失修改、读脏数据、数据重复读。
两段锁协议:可串行,可能死锁。

更新 = 读+写。

\subsection{其他}
\subsubsection{完整性约束}
提高数据可靠性。

实体完整性约束:主键不能为空、重复。
参照完整性约束:外键,如员工的部门,应该是对应的部门表中的主键或还未分配部门时为空。
用户自定义完整性约束:如性别、年龄的限制

\subsubsection{安全}
用户标识和鉴定,账户、校验
存取控制,用户的操作权限
密码存储和传输;
视图和保护,视图授权;
审计,记录用户对数据库的操作。

\subsubsection{数据备份}
静态备份,关闭数据库,复制。快速,易归档;
动态备份,运行时复制。有选择性备份、恢复某个表,灵活,出错导致问题更大。

备份策略,如连续7天:完增增增差增增
完全备份
差量备份,与上次完全备份的差异
增量备份,与上次备份的差异。


海量,
转储,

增删改查,先写日志文件,然后再实际处理。

\subsubsection{故障与恢复}
可预期故障:程序中预先设置rollback语句;
不可预期:如算术溢出、违反存储保护等,可以由DBMS的恢复子系统通过日志,撤销对数据库的修改;
系统故障:检查点法;
介质故障:使用日志。

\subsubsection{数据仓库}
数据仓库特点:
面向主题,而一般数据库面向业务组织数据。
集成的,
相对稳定的,一般不修改与删除。
反映历史变化

数据源进行抽取、清理、装载、刷新,得到数据仓库。
OLAP服务器,提供查询、报表、分析、数据挖掘等工具。
部门级的数据仓库整合,得到企业级的数据仓库。

\subsubsection{数据挖掘}
方法:决策树、神经网络、遗传算法、关联规则挖掘算法

分类:
关联分析,数据之间的隐藏关系分析;
序列模式分析,分析数据之间的因果关系;
分类分析,记录做标记,按标记分类;
聚类分析


\subsubsection{反规范化技术}
规范化程度低,存在冗余;高了导致数据表多,查询效率低。

牺牲空间和规范化程度来提高查询效率。

技术:
增加派生性冗余列;
增加派生性冗余列;
重新组织表;
分割表


\subsubsection{大数据}
多种不同类型进行联合分析。

数据量大,PB级或以上
需要快速处理
数据多样性
数据有价值

cookies售卖;百度广告推送。

深度分析,关联分析,回归分析

集群平台。

大数据处理系统:高可扩展性;高性能;高度容错;支持异构环境;分析延迟小;开放接口易用;成本低;向下兼容。


    \section{分布式数据库}
    \section{对象数据库}
    \section{关系数据库}

    \section{MySQL基本操作}
    \subsection{数据库基础操作}
    \begin{comment}
        创建的root密码是123456
        登录:cmd:mysql -h主机名空格-u用户名空格-p密码
        cmd mysql -uroot -p回车
        mysql -hlocalhost -uroot -p -P3306
        登录同时打开数据库:
        mysql -uroot -p -D 数据库名字
    
        修改密码:mysqladmin -uroot -p旧密码 password 新密码\\
        
        
        退出: exit
        quit
        \q
        ctrl+c
        
        修改命令提示符的名称:如修改为shepherd,每次关闭mysql后失效
        mysql -uroot -p --prompt='shepherd>'
        mysql -uroot -p --prompt='\h~\a~\d~\D'
        h是host,a是用户名,d是选择的数据库,D是日期
    
        查询版本号:mysql -V
        mysql --version
    
        ;和\g 是执行改行命令
        \c 不执行改行
        help,\h或者 ? 加上关键字可以查找帮助
    
        支持折行写,单引号不要折行;
        MySQL关键字大写。
        库、表、字段名称要避开关键字,否则需要反引号以区分。
    
    
        批命令:
        select user();
        select version();
        select now();
        select current_date();
        select current_time();
        select current_timestamp();
        以上几行保存,在cmd中输入以下:\\
        \. E:\Notes\Notes_Latex\MySQL_study\Codes\t001.sql  
    
    
    
        数据库的基本操作
    
        DATABASE 和 SCHEMA 一样
        show databases; -- 查看数据库
        create database t001; -- 注意双减号注释,前后都需要空格
        show databases;
        select database(); -- a查看当前选定的库
        use t001; -- 打开数据库
        select database();
        DROP DATABASE IF EXISTS t001; -- c删除库
    
        CREATE DATABASE IF NOT EXISTS t007 DEFAULT CHARACTER SET 'GBK'; -- 指定编码
        SHOW WARNINGS;
        SHOW CREATE DATABASE t007;
        ALTER DATABASE t007 DEFAULT CHARACTER SET 'UTF8'; -- 修改编码
        SHOW CREATE DATABASE t007;
    
    
    \end{comment}
    
    
    \subsection{表}
    表由行row和列column组成
    关系模型(二维表):不同的属性叫字段;每行数据是一个记录;每个记录都不相同的字段叫主键。\\
    做表,先确定字段,确定每个字段的类型。
    
    \begin{comment}
        CREATE TABLE IF NOT EXISTS table_name(
            字段名称 字段类型 [可选:完整性约束条件]
    
        ) EIGEN=存储引擎, charset=编码方式
    
        数值类型:
            整形
            TINYINT (-128, 127) or (0, 255),即8次幂,1字节
            SMALLINT 16次幂,2字节
            MEDIUMINT 24次幂,3字节
            INT (-2^31, 2^31-1) or (0, 2^24-1),32次幂,4字节
            BIGINT 64次幂,8字节
            BOOL, BOOLEAN, 等价于TINYINT(1),1字节
    
            浮点型
            FLOAT[(M,D)],M是总位数,D是小数位数,单精度浮点数精度大约7位小数。不设置时根据硬件条件设置
            取值范围(-3.4e38, -1.17e-38) & 0 & (1.175e-38, 3.4e38), 4字节
            DOUBLE[(M,D)],M是总位数,D是小数位数,取值范围(-1.79e308, -2.22e-308) & 0 & (2.22e-308, 1.79e308), 8字节
            DECIMAL[(M,D)],M是总位数,D是小数位数,类似DOUBLE,内部以字符串形式存储,字节M+2
    
            字符串类型
            CHAR(M), 定长串, M字节, M=[0,255]
            VAARCHAR(M), 变长串, 所占字节为L+1   小于M,M=[0,2^16-1],65536-1
            TINYTEXT, L < 2^8,L+1字节
            TEXT, L < 2^16,L+2字节
            MEDIUMTEXT, L < 2^24,L+3字节
            LONGTEXT, L < 2^32,L+4字节
            ENUM('VALUE1', 'VALUE2',...), 枚举,1或2字节,取决于枚举个数,最多2^16-1个值
            SET('VALUE1', 'VALUE2',...), 集合,1或2或3或4或8字节,取决于set成员个数,最多64个值
    
            日期类型:
            DATE, 3字节
            TIME, 3字节
            YEAR, 1字节
            DATETIME, 8字节
            TIMESTAMP 4字节
    
            二进制类型:
            不常用。
        
    
    \end{comment}
    
    
    
    
    \subsection{表间联系}
    表a是学生表,有学生的学号姓名班级等信息,表b是课程表,是所有开课课程的序号、课名称、老师等信息,表a和b没有公共元素,表c是选课表,有学号和所选课程号、成绩等信息。表c联系起表a和b。
    需要建5个键,3个表要有3个主键,同时表c要有2个外键投射到表a和表b\\
    一对一的表,常用在登录表和详细表。方便查询。
    
    \subsection{SQL Language}
    Structured Query Language,结构化查询语言。包括数据定义语言(DDL),数据操作语言(DML)。
    
    \subsubsection{DDL数据定义语言}
        
        \begin{comment}
        CREATE TABLE 创建表
        DROP TABLE 删除表
        ALTER TABLE 修改表+( ADD 字段名 类型, MODIFY 字段名 新字段名 新字段类型, DROP 待删除字段, RENAME TO 表的新名字)
        TRUNCATE TABLE 清空表
        DESCRIBE 查看表结构
        建立主键、外键等约束(PRIMARY KEY, NOT NULL, UNIQUE, DEFAULT '男', AUTO_INCREMENT)
        UNIQUE 意思是不能有重复的。
        \end{comment}
    
    \subsubsection{DML数据操作语言}
    增删改
    
    INSERT INTO 表名(字段列表) VALUSES ('XX','XX'),...,('XX','XX'); 日期和字符串要加引号,int格式不需要引号。注意顺序,数值,数量对应好。
    
    UPDATE 表名 SET 字段名=字段值 WHERE 条件
    DELETE FROM 表名 WHERE 条件
    
    数据查询最重要:
    (1)单表查询
    SELECT DISTINCT 字段列表 FROM 表名 WHERE 条件
    ORDER BY 排序依据
    GROUP BY 分组依据
    HAVING 分组筛选条件
    
    limit语句
    
    单表查询
    
    特殊查询
    
    \subsubsection{DQL数据查询语言}
    数据查询
    
    \subsubsection{DCL数据控制语言}
    控制数据的使用权限
    
    
    
    \subsubsection{视图}
    SQL-> 外模式,包括各个视图 -> 模式,包括各个基本表, -> 内模式,包括各个存储文件
    
    create view 名称 as select 选择的项目
    
    
    
    
    \subsubsection{索引}
    普通索引
    
    唯一索引:可以空
     
    主键索引:不能空
    
    组合索引
    
    create [unique] index 名称 on 表名(字段名)
    几十万行记录能感觉到几毫秒的速度提高。
    加入索引会提高查询速度,增删的速度会降低。
    大表,表中数据经常查询,建索引。
    like 查询不支持索引
    
    
    \subsection{数据库设计}
    需求分析:分析业务和数据处理要求
    
    概要设计:设计数据库和E-R模型图
    详细设计:E-R图转换为多张表,逻辑设计,
    代码编写:
    
    
    
    subsubsection{概念模型}
    实体,客观世界中的对象
    实体的属性
    实体间的联系:n对m的
    
    
    subsubsection{E-R图}
    entity-relationship
    E-R图:矩形是实体,属性是圆形,联系是菱形
    比如,教师-学生-课程;
    比如,论坛用户-发帖-跟帖-版块
    比如,商店-顾客
    
    subsubsection{e-R图到关系范式的转换}
    1对1的关系:部门-经理,部门这个表中加入经理编号这个自造的字段,把经理表中的主键加入。
    1对多:科室-医生,把医生表加入科室编号。
    m对n:学生-课程,构造3个表,中间的联系做一个表,转化成1对多
    顾客-商品,构建新表(订购)
    
    subsubsection{关系范式}
    评估表的质量,评估规范标准。
    1NF,2NF, 3NF, BCNF,  4NF, 5NF。范式越高,质量越高,代码难度越难写。
    项目快就范式低,项目人力资源充足,就提高范式
    通常3范式就可以。
    1NF,目标是确保每列的原子性,每一列不可分割。不满足1NF的数据库不是关系数据库。
    2NF,实体属性完全依赖主键,不能只依赖主键的一部分。如选课表(学号,姓名,课程号,成绩,系名,系主任),复合主键(学号+课程号),这不符合2NF,因为这里如姓名,姓名之和主键的一部分(学号)有关系,不是和整个主键有关系。更好的做法是分成两个表选课表(学号,课程号,成绩),学生表(学号,姓名,系名,系主任)
    3NF,在2NF基础上,任意非主属性都不传递依赖于主关键字。
    如系主任之和系名有关,系名和学号有关。相当于把一些部件放在子装配体里面。应该把学生表分成两个表,学生表(学号,姓名,系代号),系别表(系代号,系名,系主任);
    
    
    冗余设计:
    在业务频繁的时候增加冗余。
    
    水平分表与垂直分表:
    数据大时。
    水平分表,把数据放在不同表中,如按照学号奇偶划分,大表可以提高查询速度,表中数据有独立性时如记录数据中的不同时期等的数据。缺点是会增加复杂度。
    垂直分表,把不常用的信息放在另外表中。管理有冗余,查询所有信息时需要join命令。
    
    读写分离
    数据库备份三份,1个master库用于更新insert,update,delete,2个slave库用于查询select,read,
    
    




    \section{多数据库系统集成技术研究}

\chapter{数据处理}
\section{数据管理}
\section{数据挖掘}
\section{信息架构}
\section{信息管理}
\section{信息检索}
\section{知识管理}
\section{知识工程}

\section{多媒体,超媒体}
\subsection{声音和音乐计算}

\chapter{人机交互}
\chapter{图像处理与科学可视化}



\chapter{Else}
\section{高性能存储系统,处理机同步通信机制}
\section{超常指令字(VLIW)系统结构}
\section{格点计算模型及体系结构的研究}
\section{波分复用WDM全光网中的路由及波长分配算法的研究}
\section{计算机在信息产业中的应用}
\section{计算机在制造产业中的应用}
\section{各个领域中计算机应用软件的开发技术}
\section{过程工程}
\section{计算机应用工程化}
\section{以机器人足球为标准问题的MAS体系结构与合作规划}
\section{MAS中的策略协作学习}
\chapter{CIMS及其它先进制造技术}
\section{信息系统}
\section{大规模科学与工程计算}
\section{计算机图形学及可视化技术,计算机辅助设计}



\end{document}





