%%============
%%  ** Author: Shepherd Qirong
%%  ** Date: 2022-06-04 23:55:01
%%  ** Github: https://github.com/ShepherdQR
%%  ** LastEditors: Shepherd Qirong
%%  ** LastEditTime: 2022-06-05 20:28:34
%%  ** Copyright (c) 2019--20xx Shepherd Qirong. All rights reserved.
%%============

\documentclass[UTF8]{../computerUniverse}
\begin{document}

\title{07-02-ComputerScience}
\date{Created on 20220605.\\   Last modified on \today.}
\maketitle
\tableofcontents



\chapter{计算理论}

\section{自动机理论(形式语言)}


\section{可计算性理论}
\section{计算复杂性理论}
\section{并发理论}



\chapter{计算机架构}







\subsection{数据表示}

10进制转换成二进制,短除法,最后余1或0停止,结果倒着写。
原码,最高位的符号,数的绝对值;
反码,正数同原码,负数是符号位为1, 数的绝对值的相反数
补码,正数同原码,负数是符号位为1,数的绝对值的相反数加上1
移码,在补码基础上,将符号位取反,使得正数表示起来比负数大。


\begin{lstlisting}
   浮点数表示 N=M*R^K;
1.0e3 + 1.2e2
= 1.0e3 + 0.12e3 //阶数对齐,
= 1.02e3 // 尾数计算;
= 1.02e3 // 格式化

两个浮点数相加诚,首先需要比较阶码大小,使小阶向大阶看齐(称为“对阶”)。即小阶的尾数向右移位(相当于小数点左移),每右移一位,其阶码加1,直到阶码相等,右移的位数等于阶差。

\end{lstlisting}







\subsection{校验码}


码距:编码系统中任意两个码字的最小距离。
码组内检测k个错误码,最小码距d>=k+1;
码组内纠正k个错误码,最小码距d>=2k+1。


奇偶校验Parity Code,


\subsubsection{CRC}
循环冗余校验码,Cyclic Redundancy Check,CRC。
可检测错误。思路是传输数据末尾增加一些位,使得计算指定的运算后余数为0。
如模2除法,写还是按照除法竖式计算,算的时候按位异或不计进位。
CRC举例:
1)例如对于多项式$x^4+x^3+x+1$,各次的系数为:11011,作为模2除法的除数;
2)多项式有5位,原始信息末尾补5个0;
3)计算模2除法,得到余数;
4)则最终得到:原始信息+余数

\subsubsection{Hamming Code}

\begin{lstlisting}
海明码 Hamming Code,n位数据,k位校验,满足$2^K-1>=N+K$. 
如求1011的海明码;
1)求得校验位需要3位,分别放在1,2,4位(2的i次方上),分别记为a,b,c。
2)从右往左写: 1    0    1    c    1    b    a
               7    6    5    4    3    2    1
   7=111; 6 = 110; 5 = 101;  3 = 011
SO c 由7,6,5位异或,c=1 xor 0 xor 1 = 0
   b, 7,6,3, b =1 xor 0 xor 1 = 0
   a, 7,5,3 a = 1 xor 1 xor 1 = 1
\end{lstlisting}








\section{计算机系统体系结构}





计算机体系结构分类-Flynn,根据 指令流与数据流
SISD,单处理器系统
SIMD,并行处理机,阵列处理机
MISD,流水线计算机,理论的模型
MIMD,多计算机

CISC复杂,指令:多,使用频率差别大,可变长;支持多种寻址方式;微程序控制技术实现;研制周期长。
RISC精简,指令:少,使用频率差别小,定长,大部分为单周期指令,只有Load和Store操作内存;支持寻址方式少;通用寄存器(速度快),硬布线逻辑控制为主,适合用流水线;优化编译,高级语言支持性好。


流水线:
取址2ns,分析2ns,执行1ns;
流水线周期:执行时间最长的一段, 2ns。
100条指令用时:理论公式 2+2+1 + 99 * 2 = 203ns,第1条指令完整+剩余的*流水线周期
100条指令用时:实践公式 2+2+2 + 99 * 2 = 204ns, 第1条指令每一步按照流水线周期算+剩余的*流水线周期
吞吐率:Though Put rate, TP,指令条数/流水线执行时间, 100/203;
吞吐率极限:1/流水线周期
流水线加速比:不用流水线的时间/用流水线的时间


程序计数器PC用于存放计算机将要执行的指令所在的存储单元地址
指令寄存器IR保存从存储器中取出的指令(正在执行的指令)
地址寄存器存放当前CPU所访问的内存单元地址
指令译码器对指令寄存器IR中的指令进行译码分析

\section{指令集}
\subsection{指令及并行处理(ILP)体系结果的理论与技术}
\subsection{对称多处理器(SMP)并行体系结构的理论与技术}
\subsection{多线程机制}


\section{系统设计理念}

\subsection{可重构设计}
\subsection{可扩展设计}

\subsection{并行与分布系统容错性、可用性、可靠性技术}





\chapter{分布式计算}
\subsection{分布式资源管理、故障恢复、进程动态迁移、分布式存取控制技术}
\section{网格计算}


\chapter{并行计算}
\subsection{机群并行处理体系结构、互连技术、程序设计环境以及计算密集型应用在机群系统中的实现}
\subsection{指令级并行关键技术研究}
\section{高性能计算}



\chapter{计算机架构示例与相关软件技术}

\section{并行计算机系统}
\subsection{当代主流并行机的体系结构模型、存储技术的研究}

\section{分布式计算机系统}

\section{多机系统}
\subsection{工作站机群、网络和网格等环境下的并行分布式计算模型}

\section{网络计算}

\section{量子计算}


\section{计算机系统体系结构软件仿真环境构建方法研究}












\end{document}

