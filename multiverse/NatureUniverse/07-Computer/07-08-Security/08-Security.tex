%%============
%%  ** Author: Shepherd Qirong
%%  ** Date: 2022-06-05 00:25:44
%%  ** Github: https://github.com/ShepherdQR
%%  ** LastEditors: Shepherd Qirong
%%  ** LastEditTime: 2022-06-05 20:32:17
%%  ** Copyright (c) 2019--20xx Shepherd Qirong. All rights reserved.
%%============


\documentclass[UTF8]{../computerUniverse}

\begin{document}

\title{07-08-Security}
\date{Created on 20220605.\\   Last modified on \today.}
\maketitle
\tableofcontents


计算机安全性和可靠性


\chapter{密码学}


\chapter{信息安全技术}


\subsection{系统安全分析}
保密性,【最小授权,防暴露,信息加密,物理保密】
完整性,【安全协议,校验码,密码校验,数字签名,公证】
可用性,【IP过滤,路由选择控制】
不可抵赖,【数字签名】


安全的五个基本要素
●机密性(确保信息不暴露给未授权的实体或进程)
●完整性(只有得到允许的人才能修改数据,并能够判别数据是否己被篡改)
●可用性(得到授权的实体在需要时可访问数据)
●可控性(可以控制授权范围内的信息流向和行为方式)
●可审查性(对出现的安全问题提供调查的依据和手段)


\subsubsection{加密}
对称,【加密和解密秘钥一样】【加密强度低,秘钥分发困难】
DES,替换+移位,速度快;
3DES,56位的K1和K2,K1加--K2解--K1加 
AES,
RC-5,
IDEA,


非对称,【加密和解密秘钥不一样】【加密速度慢】
RSA,
Elgamal,基础是Diffie-Hellman秘钥交换算法;
ECC,
背包算法,Rabin,D-H等


\subsubsection{摘要}

单向散列函数,单向Hash函数,定长的散列值。
MD5, SHA,SHA更长更安全。


\subsubsection{数字签名}
A: 我的名字--》信息摘要--》我的私钥加密得到签名;
对方:1)收到明文名字--》信息摘要;【数字签名,识别身份的作用】
      2)收到的签名,用A的公钥解密,得到信息摘要;【验证】
      3)比较上述两个摘要是否相等。



\subsubsection{数字信封与PGP}
A:原文,对称加密;秘钥用B的公钥加密后发送给B。
B:收到电子信封,用私钥解密信封,去除秘钥解密出原文。

秘钥用加密时间长的复杂的非对称加密。

PGP证书,是电子邮件、文件存储加密。
可以将文件用PGP加密后存到云盘,更安全。

数字证书:秘钥与数字签名结合在一起。CA机构颁发。验证数字证书上颁发机构的签名。


【试设计】邮件要加密传输,最大附件500M,发送者不可抵赖,第三方截获的话无法篡改。

发送端A:
邮件正文 --》随机秘钥K,对称加密--》邮件密文;
邮件正文--》信息摘要--》数字签名(私钥)--》摘要密文;
秘钥K --》数字信封技术,非对称加密(公钥)--》信封;

接收端B:
信封--》非对称加密(私钥)--》秘钥K
邮件密文--》随机秘钥K,对称加密--》邮件正文;邮件正文--》邮件摘要
摘要密文--》解密签名(公钥)--》邮件摘要,与上一步的摘要验证;

\subsubsection{病毒}
引导区:。主引导记录病毒感染硬盘的主引导区,如大麻病毒、2708病毒、火炬病毒等;分区引导记录病毒感染硬盘的活动分区引导记录,如小球病毒、Girl病毒等。
宏:TaiwanNo.1,Nuclear宏病毒
木马:冰河,ICMP类型的木马,灰鸽子和蜜蜂大盗,PassCopy和暗黑蜘蛛侠
蠕虫:震网(Stuxnet)

\section{数据安全}


\chapter{容错计算技术}


\end{document}





