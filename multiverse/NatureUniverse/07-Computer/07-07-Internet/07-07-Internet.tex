%%============
%%  ** Author: Shepherd Qirong
%%  ** Date: 2022-06-05 00:19:55
%%  ** Github: https://github.com/ShepherdQR
%%  ** LastEditors: Shepherd Qirong
%%  ** LastEditTime: 2022-06-05 20:31:24
%%  ** Copyright (c) 2019--20xx Shepherd Qirong. All rights reserved.
%%============

\documentclass[UTF8]{../computerUniverse}

\begin{document}

\title{07-07-Internet}
\date{Created on 20220605.\\   Last modified on \today.}
\maketitle
\tableofcontents





\chapter{云计算}
\section{计算网络及其应用}
\section{网络计算环境下的知识处理、网络体系结构、网络管理}
\section{高速计算机网络和网络服务质量}



\chapter{信息论}
\section{多媒体信息在网络中的传输及处理}



\chapter{高速互连网络}
\section{互联网络体系结构}


局域网在最低两层。
\subsection{OSI、RM七层模型}
二层以上通过协议保障安全。

物理层,二进制传输,中继器、集线器;【安全上,隔离,屏蔽】
数据链路层,以帧为单位,网桥、交换机、网卡;【安全上,链路加密,PPTP,L2TP】
【这两层:以太网,ATM,帧中继等】

网络层,分组传输与路由选择,三层交换机、路由器、IP【安全上,防火墙,IPSec】
【IP,ICMP,IGMP,等】

传输层,端到端。【安全上,TLS,SET,SSL】
【TCP,UDP】【UDP没有验证】

会话层,建立、管理、终止会话;
表示层,数据格式表示、加密、压缩;
应用层,具体实现功能。
【这3层,DHCP等】【安全上,PGP,Https,SLL】

\subsection{网络技术标准与协议}
TCP-IP协议族,Internet,可扩展,可靠;
IPX-SPX协议,局域网即时战略游戏等,路由;
NETBEUI协议,不支持路由,IBM,快速



\subsubsection{常见协议功能}

RIP(Routing Information Protocol,路由信息协议)是使用最久的协议之一。RIP是一种分布式的基于距离向量的路由选择协议,RIP协议是施乐公司20世纪80年代推出的,主要适用于小规模的网络环境。RIP协议主要用于一个AS(自治系统)内的路由信息的传递

OSPF路由协议是用于网际协议(IP)网络的链路状态路由协议。该协议使用链路状态路由算法的内部网关协议(IGP),在单一自治系统(AS)内部工作。适用于IPv4的OSPFv2协议定义于RFC 2328,RFC 5340定义了适用于IPv6的OSPFv3。

POP3:邮件收取

SMTP:邮件发送

FTP:20数据端口/21控制端口,文件传输协议

HTTP:超文本传输协议,网页传输

DHCP: IP地址自动分配

SNMP:简单网络管理协议

DNS:域名解析协议,记录域名与IP的映射关系

TCP:可靠的传输层协议

UDP:不可靠的传输层协议

ICMP:因特网控制协议,PING命令来自该协议

IGMP:组播协议

ARP:地址解析协议,IP地址转换为MAC地址

RARP:反向地址解析协议,MAC地址转IP地址

BCP (边界网关协议)是运行于TCP上的一种自治系统的编由协议。






\subsubsection{TCP三次握手}
A给B,B给A,A给B。保证传输可靠

\subsubsection{DHCP协议}
动态分配IP地址。169.254.0.0和0.0.0.0是假的。
租约8天。
\subsubsection{DNS协议}
域名与IP转换。
迭代查询:可以丢给别人【根域名服务器--》顶级域名服务器--》权限域名服务器---》。。。】,直接给出反馈,不盘根究底。
递归查询:最终答案【本地域名服务器】


\subsection{子网划分}
网络前缀+主机号。
B类168.195.0.0分27个子网,子网掩码?
10101000,~~
因为$27<2^5$,
所以255,255,11111000,0
所以255,255,248,0

B类168.195.0.0划分子网,每个子网主机700,求掩码?
$2^k-2>=700$, 所以10位地址,所以255,255,252,0


210.115.192.0/20,可划分多少c类子网?
因为后面20-16=4,中间还有$2^4=16$个

\subsection{无线网}
局域,WLAN,WIFI
城域,WMAN,WIMAX
广域,WWAN,3G,4G
个人,WPAN,bluetooth

\subsection{网络接入}
有线:公用交换电话PSTN
数字数据网DDN 
ISDN,可以打电话时上网;
ADSL,非对称数字用户,电话线通信,下行8M,上行512K;
同轴光纤,HFC,上行下行对称。

无线:WIFI,BLUETOOTH,红外IrDA,WAPI;
3G,4G







\section{协议测试}
MIME它是一个互联网标准,扩展了电子邮件标准,使其能够支持,与安全无关。

\section{高性能通信机制与策略的研究}
\section{计算机支持的协同工作(CSCW)}
\section{Web技术软件工程和环境}
\section{互联网络下的协同工作环境}
\section{实时与多媒体技术}
\section{IPv6协议的中间件和软件应用}


\chapter{普适计算}
\chapter{无线计算(移动计算)}


\chapter{网络信息安全}
\subsection{加密和认证}
HTTPS用ssl协议对报文进行封装

对称加密,如DES,3DES或TDEA, RC5, IDEA, AES,适合大量明文传输;
非对称加密,如RSA;
DES是共享秘钥加密。

信息摘要算法:SHA-1, MD5
MD5,信息以512位分组,结果128位。


CA的公钥是验证CA签名的依据,所以不同CA互换公钥是用户互信的必要条件


采样频率大于等于工作频率的2倍,能够恢复实际波形。




\chapter{应用}
\section{远程教学}
\section{以Intention形式化为核心的BDI建模}
\section{基于多主体技术的Internet信息检索和用户建模}
\section{群件与网络技术研究}


\end{document}


