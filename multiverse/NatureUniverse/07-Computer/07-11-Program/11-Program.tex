%%============
%%  ** Author: Shepherd Qirong
%%  ** Date: 2022-06-05 00:25:44
%%  ** Github: https://github.com/ShepherdQR
%%  ** LastEditors: Shepherd Qirong
%%  ** LastEditTime: 2024-05-05 21:39:39
%%  ** Copyright (c) 2019--20xx Shepherd Qirong. All rights reserved.
%%============


\documentclass[UTF8]{../computerUniverse}

\begin{document}

\title{11-Program}
\date{Created on 20240413.\\   Last modified on \today.}
\maketitle
\tableofcontents


高项


\chapter{重点}


\section{阶段二}
五大过程组,十大知识域,输入,输出,工具,49个子过程的定义、作用、ITTO工具

单代号、双代号、时标网络图的绘制,计算关键路径;
PV,EV,AC公式


\section{概念}
\subsubsection{区分1}

项目日历: 规定可以开展进度活动的可用工作日和工作班次,区分开展进度活动的可用时间段和不可用时间段。

资源日历:识别每种具体资源的可用时间、班次、上下班时间、周末、节假日。

物质资源分配单: 材料、设备、用品、地点、其他实物资源。

项目团队派工单:团队成员的角色和职责、姓名、姓名插入到项目管理计划的其他部分,如项目组织图、进度计划。




\chapter{基础}

\chapter{案例分析}

\chapter{English}

\begin{lstlisting}
    三点估算/计划评审技术(PERT)
    关键路径法(CPM) 

    precede event  紧前事件
    successor event  紧后事件,
    dummy activity  虚活动,

    analogous estimating 类比估算
    parametric estimating 参数估算
    opreation costs 运营成本
    implement costs 实施成本
    contingency reserve 应急储备
    management reserve 管理储备
    budget at completion, bac, 完工预算
    earned value management, evm, 挣值管理

    EV, earned value,  挣值
    
    PV,  planned value, 计划价值
    SV,schedule variance, 进度偏差 SV = EV-PV
    SPI, schedule performance index, 进度绩效指数 SPI = EV/PV
    
    AC, actual cost, 实际成本
    CV, cost variance, 成本偏差 CV = EV-AC
    CPI,cost performance index, 成本绩效指数 CPI = EV/AC

    BAC, budget at completion, 完工预算
    ETC, estimate to completion, 完工尚需估算 EAC = BAC - EV ; EAC = (BAC-EV)/CPI
    EAC, estimate at completion,完工估算 EAC = AC + ETC
    VAC, variance at completion, 完工偏差,VAC = BAC - EAC
    TCPI, to-complete performance index ,完工尚需绩效指数, TCPI = (BAC-EV)/(BAC-AC)



\end{lstlisting}



\chapter{计算}


\section{立项管理}

\section{整合管理}

\section{范围管理}
包括范围、需求

\section{进度管理【重点】}

\subsection{单代号网络图}
单代号网络图,active on node,AON,又称
前导图,Precedence diagramming method, PDM。方框是活动节点,箭头是逻辑关系。

依赖关系,根据紧前活动和紧后活动的关系,按照FS排列有4种。

FS:A finishes,才有B starts,如比赛与颁奖;系统设计与设计评审;系统分析与需求评审;确定项目范围与制定WBS;装硬件与装软件;验证身份合法与数据传输;
FF:如文件编写与文件编辑;
SS:如地基浇灌与找平;设备运行与设备实时监控;
SF:如第2个人值班与第1个人值班。新系统上线与旧系统下线。

FS配合earlist和latest,英国标准BS6046有节点标示,分别是ES,工期,EF;活动名称;LS,总浮动时间,LF

提前量(lead),正数,如FS+6,意思是紧前活动结束,+6个时间单位,紧后活动开始。

\subsection{双代号网络图}
双代号网络图,箭线图,arrow diagramming method,ADM

箭线图,active on the arrow,AOA

箭头是活动,节点是事件。箭尾是活动的紧前事件,箭头是活动的紧后事件。

\subsection{关键路径法【*****】}
关键路径法,critical path method,最长活动路径。

总浮动时间,total float,不影响进度的最早开始时间可推迟。
自由浮动时间,free float,不影响所有紧后活动的最早开始时间可推迟。是紧后工作的最早开始时间减去本任务的最早完成时间。

浮动float,等于 时差slack。

时标逻辑图,又称时标网格图。

计算方法:
关键路径时长:找时间最长的。
某活动最早开始时间:流到某活动前最长时间。

关键路径的总浮动时间通常为0.最早开始时间和最晚开始时间不同的任务是非关键任务。

某活动延期x天,计算总工期延后多少天:从左开始计算每个事件完成的时间,计算。注意可伸缩的,注意虚活动。

\subsection{PERT估算}

PERT估算,program evaluation and technique.

最可能时间,ml;
最乐观时间,o;
最悲观时间,p;
期望E = (o+4*ml+p)/6;
标准差,standard deviation, $\sigma = (p-o)/6$

\section{成本管理【重点】}

\section{质量管理}

\section{资源管理}

\section{沟通管理}



\section{风险管理}

\section{采购管理}

\section{干系人管理}

\section{其他}
\subsection{版本管理}
\subsection{组合管理}
\subsection{知识产权}


\section{信息技术}

\section{运筹学}


\chapter{论文}


\end{document}





