%%============
%%  ** Author: Shepherd Qirong
%%  ** Date: 2022-05-08 20:35:45
%%  ** Github: https://github.com/ShepherdQR
%%  ** LastEditors: Qirong ZHANG
%%  ** LastEditTime: 2024-12-01 22:11:35
%%  ** Copyright (c) 2019--20xx Shepherd Qirong. All rights reserved.
%%============


\chapter{Introduction}

自然界。
包括NaturalScience自然科学:

生物、化学、物理、地球科学、空间科学,


FormalScience形式科学:

计算机科学,系统理论,数学




\chapter{生物学}    %01


\section{空气生物学}
\section{解剖学}
% \subsection{比较解剖学}
% \subsection{人体解剖学}
\section{生物化学}
\section{生物信息学}
\section{生物物理学}
\section{生物技术}
\section{植物学}
% \subsection{民族植物学}
% \subsection{藻类学}
\section{细胞生物学}
\section{时间生物学}
\section{计算生物学}
\section{冷冻生物学}
\section{发育生物学}
% \subsection{胚胎学}
% \subsection{畸形学}
\section{生态}
% \subsection{农业生态学}
% \subsection{民族生态学}
% \subsection{人类生态学}
% \subsection{景观生态}
\section{内分泌学}
\section{民族生物学}
\section{人类动物学}
\section{进化生物学}
\section{遗传学}
% \subsection{表观遗传学}
% \subsection{行为遗传学}
% \subsection{分子遗传学}
% \subsection{群体遗传学}
\section{组织学}
\section{人类生物学}
\section{免疫学}
\section{湖沼学}
\section{林奈分类学}
\section{海洋生物学}
\section{数理生物学}
\section{微生物学}
% \subsection{细菌学}
% \subsection{原生生物学}
\section{分子生物学}
\section{真菌学}
\section{神经科学(大纲)}
\section{行为神经科学}
\section{营养(大纲)}
\section{古生物学}
\section{寄生虫学}
\section{病理}
% \subsection{解剖病理学}
% \subsection{临床病理学}
% \subsection{皮肤病理学}
% \subsection{法医病理学}
% \subsection{血液病理学}
% \subsection{组织病理学}
% \subsection{分子病理学}
% \subsection{手术病理}
\section{生理}
% \subsection{人体生理学}
% \subsection{运动生理学}
\section{结构生物学}
\section{系统学(分类学)}
\section{系统生物学}
\section{病毒学}
% \subsection{分子病毒学}
\section{异形生物学}


\section{动物学}    %02
% \subsection{动物通讯}
% \subsection{道歉学}
% \subsection{蛛形学}
% \subsection{节肢动物学}
% \subsection{烟道学}
% \subsection{苔藓动物学}
% \subsection{癌症学}
% \subsection{动物学}
% \subsection{刺胞动物学}
% \subsection{昆虫学}
%     \subsubsection{法医昆虫学}
% \subsection{民族动物学}
% \subsection{行为学}
% \subsection{蠕虫学}
% \subsection{爬虫学}
% \subsection{鱼类学}
% \subsection{无脊椎动物学}
% \subsection{哺乳动物学}
%     \subsubsection{犬儒学}
%     \subsubsection{猫科}
% \subsection{软糖学}
%     \subsubsection{贝壳学}
%     \subsubsection{利马科}
%     \subsubsection{条义学}
% \subsection{多足学}
% \subsection{生态学}
% \subsection{线虫学}
% \subsection{神经行为学}
% \subsection{物性}
% \subsection{鸟类学}
% \subsection{浮游学}
% \subsection{灵长类动物学}
% \subsection{动物切开术}
% \subsection{动物符号学}


\chapter{化学}    %03
\section{农化}
\section{分析化学}
\section{天体化学}
\section{大气化学}
\section{生物化学(大纲)}
\section{化学生物学}
\section{化学工程(大纲)}
\section{化学信息学}
\section{计算化学}
\section{宇宙化学}
\section{电化学}
\section{环境化学}
\section{飞秒化学}
\section{味道}
\section{流动化学}
\section{地球化学}
\section{绿色化学}
\section{组织化学}
\section{氢化}
\section{免疫化学}
\section{无机化学}
\section{海洋化学}
\section{数理化学}
\section{机械化学}
\section{药物化学}
\section{分子生物学}
\section{分子力学}
\section{纳米技术}
\section{天然产物化学}
\section{神经化学}
\section{酿酒学}
\section{有机化学(大纲)}
\section{有机金属化学}
\section{石化}
\section{药理}
\section{光化学}
\section{物理化学}
\section{物理有机化学}
\section{植物化学}
\section{高分子化学}
\section{量子化学}
\section{放射化学}
\section{固态化学}
\section{声化学}
\section{超分子化学}
\section{表面化学}
\section{合成化学}
\section{理论化学}
\section{热化学}



\chapter{地球科学}    %04
\section{土壤学}
\section{环境化学}
\section{环境科学}
\section{宝石学}
\section{地球化学}
\section{大地测量学}
\section{自然地理}
% \subsection{大气科学/气象学(大纲)}
% \subsection{生物地理学/植物地理学}
% \subsection{气候/古气候学/古地理学}
% \subsection{海岸地理学/海洋学}
% \subsection{土壤学/土壤学或土壤科学}
% \subsection{地球生物学}
% \subsection{地质学(纲要)(地貌,矿物学,岩石学,沉积学,洞穴,构造地质学,火山)}
% \subsection{地质统计学}
% \subsection{冰川学}
% \subsection{水文学(大纲)/湖沼学/水文地质学}
% \subsection{景观生态}
% \subsection{第四纪科学}

\section{地球物理学}
\section{古生物学}
% \subsection{古生物学}
% \subsection{古生态学}



\chapter{空间科学}    %05
\section{天体生物学}
\section{观测天文学}
% \subsection{伽马射线天文学}
% \subsection{红外天文学}
% \subsection{微波天文学}
% \subsection{光学天文学}
% \subsection{射电天文学}
% \subsection{紫外线天文学}
% \subsection{X射线天文学}
\section{天体物理学}
% \subsection{引力天文学}
% \subsection{黑洞}
\section{星际介质}
\section{数值模拟}
% \subsection{天体物理等离子体}
% \subsection{星系形成与演化}
% \subsection{高能天体物理学}
% \subsection{流体力学}
% \subsection{磁流体动力学}
% \subsection{恒星形成}
\section{物理宇宙学}
\section{恒星天体物理学}
% \subsection{日震学}
% \subsection{恒星演化}
% \subsection{恒星核合成}
\section{行星科学}



\chapter{物理学}    %06

\section{理论物理}%TheoreticalPhysics
% \subsection{数学物理}
% \subsection{电磁场理论}
% \subsection{经典场论}
% \subsection{相对论和引力场}
% \subsection{量子力学}
% \subsection{统计物理学}
%    \subsubsection{统计力学}
% \subsection{其他}

\section{力学}
% \subsection{牛顿力学}
% \subsection{固体力学}
% \subsection{空气动力学}
% \subsection{流体动力学}


\section{声学}
% \subsection{物理声学}
% \subsection{非线性声学}
% \subsection{量子声学}
% \subsection{超声学}
% \subsection{水声学}
% \subsection{应用声学}

\section{光学}
% \subsection{几何光学}
% \subsection{物理光学}
% \subsection{非线性光学}
% \subsection{光谱学}
% \subsection{量子光学}
% \subsection{信息光学}
% \subsection{导波光学}
% \subsection{发光学}
% \subsection{红外物理}
% \subsection{激光物理}
% \subsection{应用光学}
% \subsection{原子、分子和光学物理}


\section{热学}
% \subsection{热力学}
% \subsection{热物性学}
% \subsection{传热学}


\section{电磁学}
% \subsection{电学}
% \subsection{静电学}
% \subsection{静磁学}
% \subsection{电动力学}
% \subsection{电磁波}
%   \subsubsection{微波}
% \subsection{无线电}
%   \subsubsection{量子无线电}
%   \subsubsection{超高频无线电}
%   \subsubsection{统计无线电}


\section{粒子物理}
% \subsection{电子物理}
%   \subsubsection{量子电子}
%   \subsubsection{电子电离与真空物理}
%   \subsubsection{带电粒子光学}
% \subsection{原子分子物理}
%   \subsubsection{原子与分子理论}
%   \subsubsection{原子光谱}
%   \subsubsection{分子光谱}
%   \subsubsection{波谱学}
%   \subsubsection{原子与分子碰撞过程}


\section{凝聚态物理学}
% \subsection{凝聚态理论}
% \subsection{金属物理学}
% \subsection{半导体}
% \subsection{电介质}
% \subsection{晶体}
% \subsection{非晶态}
% \subsection{液晶}
% \subsection{薄膜}
% \subsection{低维物理}
% \subsection{表面与界面}
% \subsection{固体发光}
% \subsection{磁学}
% \subsection{超导}
% \subsection{低温}
% \subsection{高压}


\section{等离子体物理}
% \subsection{热核聚变等离子体}
% \subsection{低温等离子体}
% \subsection{等离子体光谱学}
% \subsection{凝聚态等离子体}
% \subsection{非中性等离子体}


\section{核物理}
% \subsection{核结构}
% \subsection{核能谱}
% \subsection{低能核反应}
% \subsection{中子物理学}
% \subsection{裂变}
% \subsection{聚变}
% \subsection{轻粒子核物理}
% \subsection{重离子核物理}
% \subsection{中高能核物理}

\section{高能物理}
% \subsection{基本粒子物理学}
% \subsection{宇宙线}
% \subsection{粒子加速器}
% \subsection{高能物理实验}

\section{其他}
% \subsection{应用物理学}
% \subsection{天体物理学}
% \subsection{生物物理学}
% \subsection{实验物理}
% \subsection{地球物理学}
% \subsection{医学物理学}
% \subsection{固态物理学}






\chapter{计算机科学}


\section{理论}
% \subsection{计算理论}
%   \subsubsection{自动机理论(形式语言)}
%   \subsubsection{可计算性理论}
%   \subsubsection{计算复杂性理论}
%   \subsubsection{并发理论}
% \subsection{计算机架构}
%   \subsubsection{量子计算}
% \subsection{分布式计算}
% \subsubsection{网格计算}
% \subsection{并行计算}
% \subsubsection{高性能计算}

\section{算法}
% \subsection{数据结构}
% \subsection{计算几何}
% \subsection{分布式算法}
% \subsection{并行算法}
% \subsection{随机算法}

\section{编程语言}
%\subsection{编程语言的逻辑基础}
%   \subsubsection{形式化方法(形式验证)}
%   \subsubsection{语义理论(自然语言、程序语言)}
%   \subsubsection{类型理论}
%   \subsubsection{多值逻辑}
%   \subsubsection{模糊逻辑}
%   \subsubsection{推理技术}

% \subsection{编译原理}
% \subsection{编程范式}
%     \subsubsection{并发编程}
%     \subsubsection{函数式编程}
%     \subsubsection{命令式编程}
%     \subsubsection{逻辑编程}
%     \subsubsection{面向对象编程}
% \subsection{程序设计语言的设计与实现}


\section{软件}
% \subsection{系统软件}
%   \subsubsection{操作系统}
% \subsection{应用软件}


\section{硬件}
% \subsection{芯片设计}
% \subsection{超大规模集成电路设计}
% \subsection{嵌入式}


\section{计算机通信(网络)}
% \subsection{云计算}
% \subsection{信息论}
% \subsection{高速互连网络}
% \subsection{普适计算}
% \subsection{无线计算(移动计算)}
% \subsection{网络信息安全}

\section{计算机安全性和可靠性}
% \subsection{密码学}
% \subsection{信息安全技术}
% \subsection{容错计算}



\section{人工智能}
% \subsection{认知科学}
%     \subsubsection{自动推理}
%     \subsubsection{计算机视觉}
%     \subsubsection{机器学习}
%     \subsubsection{人工神经网络}
%     \subsubsection{自然语言处理(计算语言学)}
% \subsection{专家系统}
% \subsection{机器人}


\section{与其他领域结合}
% %%在计算数学,自然科学,工程和医药:
% \subsection{代数(符号)计算}
% \subsection{计算生物学(生物信息学)}
% \subsection{计算化学}
% \subsection{计算数学}
% \subsection{计算神经科学}
% \subsection{计算数论}
% \subsection{计算物理}
% \subsection{计算机辅助工程}
% \subsection{计算流体动力学}
% \subsection{有限元分析}
% \subsection{数值分析}
% \subsection{科学计算(计算科学)}

% %%社会科学、艺术、人文和专业领域的计算:
% \subsection{社区信息学}
% \subsection{计算经济学}
% \subsection{计算金融}
% \subsection{计算社会学}
% \subsection{数字人文(人文计算)}
% \subsection{计算机硬件的历史}
% \subsection{计算机科学史}
% \subsection{人文信息学}

% \subsection{数据库}
%     \subsubsection{分布式数据库}
%     \subsubsection{对象数据库}
%     \subsubsection{关系数据库}
% \subsection{数据管理}
% \subsection{数据挖掘}
% \subsection{信息架构}
% \subsection{信息管理}
% \subsection{信息检索}
% \subsection{知识管理}
% \subsection{多媒体,超媒体}
%     \subsubsection{声音和音乐计算}
% \subsection{人机交互}
% \subsection{图像处理与科学可视化}





\chapter{系统理论}
\section{1.计算机系统与应用}
\section{2.应用数学}
\section{3.管理工程}
\section{4.控制理论}


\chapter{数学}
\section{主条目:数学和数学大纲}
\section{主条目:数学学科分类}
\section{纯数学}
\section{数理逻辑与数学基础}
\section{直觉逻辑}
\section{模态逻辑}
\section{模型论}
\section{证明论}
\section{递归论}
\section{集合论}
\section{代数(大纲)}
\section{结合代数}
\section{范畴论}
\section{拓扑理论}
\section{微分代数}
\section{场论}
\section{群论}
\section{集团代表}
\section{同调代数}
\section{K理论}
\section{晶格理论(秩序理论)}
\section{李代数}
\section{线性代数(向量空间)}
\section{多重线性代数}
\section{非结合代数}
\section{表征理论}
\section{环论}
\section{交换代数}
\section{非交换代数}
\section{通用代数}
\section{分析}
\section{复杂分析}
\section{功能分析}
\section{算子理论}
\section{谐波分析}
\section{傅立叶分析}
\section{非标分析}
\section{常微分方程}
\section{p进分析}
\section{偏微分方程}
\section{真实分析}
\section{微积分(大纲)}
\section{概率论}
\section{遍历理论}
\section{测度论}
\section{积分几何}
\section{随机过程}
\section{几何(轮廓)和拓扑}
\section{仿射几何}
\section{代数几何}
\section{代数拓扑}
\section{凸几何}
\section{差分拓扑}
\section{离散几何}
\section{有限几何}
\section{伽罗华几何}
\section{一般拓扑}
\section{几何拓扑}
\section{积分几何}
\section{非交换几何}
\section{非欧几何}
\section{射影几何}
\section{数论}
\section{代数数论}
\section{解析数论}
\section{算术组合}
\section{几何数论}
\section{应用数学}
\section{近似理论}
\section{组合数学(大纲)}
\section{编码理论}
\section{密码学}
\section{动力系统}
\section{混沌理论}
\section{分形几何}
\section{博弈论}
\section{图论}
\section{信息论}
\section{数学物理}
\section{量子场论}
\section{量子引力}
\section{弦理论}
\section{量子力学}
\section{统计力学}
\section{数值分析}
\section{行动调查}
\section{分配问题}
\section{决策分析}
\section{动态规划}
\section{库存理论}
\section{线性规划}
\section{数学优化}
\section{最佳维护}
\section{实物期权分析}
\section{调度}
\section{随机过程}
\section{系统分析}
\section{统计(大纲)}
\section{精算学}
\section{人口统计学}
\section{计量经济学}
\section{数理统计}
\section{数据可视化}
\section{计算理论}
\section{计算复杂性理论 [1] }

