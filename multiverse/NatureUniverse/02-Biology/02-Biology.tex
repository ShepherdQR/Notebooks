%%============
%%  ** Author: Shepherd Qirong
%%  ** Date: 2021-12-11 17:07:45
%%  ** Github: https://github.com/ShepherdQR
%%  ** LastEditors: Qirong ZHANG
%%  ** LastEditTime: 2024-12-29 19:34:09
%%  ** Copyright (c) 2019--20xx Shepherd Qirong. All rights reserved.
%%============



\documentclass[UTF8]{../NatureUniverse}

\begin{document}

\title{02-Biology}
\date{Created on 20241229.\\   Last modified on \today.}
\maketitle
\tableofcontents

\chapter{Introduction}

Biology:普通生物学、细胞生物学、遗传学、生理学、生物化学、生物物理学、分子生物学




\chapter{生物科学总论}
\subsubsection{生物学的普及读物}
\subsubsection{物理学词典}


\chapter{普通生物学}
\subsubsection{生命的起源}
\subsubsection{生物演化和发展}
\subsubsection{生物形态学}
\subsubsection{生态学}
\subsubsection{生物分布与生物地理学}
\subsubsection{保护生物学}
\subsubsection{水生生物学}
\subsubsection{寄生生物学}
\subsubsection{生物分类学}






\chapter{细胞生物学}
\subsubsection{细胞的形成与演化}
\subsubsection{细胞遗传学}
\subsubsection{细胞形态学}
\subsubsection{细胞生理学}
\subsubsection{细胞生物化学}
\subsubsection{细胞生物物理学}
\subsubsection{细胞分子生物学}






\chapter{遗传学}
\subsubsection{遗传与变异}
\subsubsection{杂交与杂种}
\subsubsection{人工选择与自然选择}
\subsubsection{遗传学分支学科}
\subsubsection{微生物遗传学}
\subsubsection{植物遗传学}
\subsubsection{动物遗传学}
\subsubsection{人类遗传学}







\chapter{生理学}
\section{普通生理学}
\section{神经生理学}
\section{分析器生理学}
\section{运动器官生理学}
\section{内分泌生理学}
\section{循环生理学}
\section{呼吸生理学}
\section{消化生理学}
    \subsubsection{排泄生理学}
    \subsubsection{生殖生理学}
    \subsubsection{新陈代谢与营养}
    \subsubsection{特殊环境生理学、生态生理学}
    \subsubsection{比较生理学与进化生理学}







\chapter{生物化学}
\section{一般性问题}
\section{蛋白质}
\section{核酸}
\section{糖}
\section{脂类}
\section{酶}
\section{维生素}
\section{激素}
\section{生物体其他化学成分}
\section{其他}
    \subsubsection{物质代谢及能量代谢}
    \subsubsection{体液代谢}
    \subsubsection{器官生物化学}
    \subsubsection{比较生物化学}
    \subsubsection{应用生物化学}








\chapter{生物物理学}
\section{理论生物物理学}
\section{生物声学}
\section{生物光学}
\section{生物电磁学}
\section{生物热学}
\section{生物力学}
\section{物体化学生物学}
\section{物理因素对生物的作用}
\section{其他}
    \subsubsection{辐射生物学 (放射生物学 )}
    \subsubsection{仿生学}
    \subsubsection{空间生物学}









\chapter{分子生物学}
\section{生物大分子的结构和功能}
\section{生物膜的结构和功能}
\section{生物小分子的结构和功能}
\section{分子遗传学}
\section{生物能的转换}
\section{基因工程 (遗传工程 )}
\section{生物工程学}
    \subsubsection{仿生学}
    \subsubsection{细胞工程}
    \subsubsection{酶工程}
    \subsubsection{生物工程应用}
\section{环境生物学}
\section{古生物学}
\section{微生物学}
    \subsubsection{细菌学}
\section{植物学}
    \subsubsection{普及读物}
    \subsubsection{植物学词典}
\section{动物学}
    \subsubsection{普及读物}
    \subsubsection{动物学词典}
\section{昆虫学}
\section{人类学}
    \subsubsection{人类起源论}
    \subsubsection{人类遗传学}




\chapter{其他(未分类)}





\section{空气生物学}
\section{解剖学}
% \subsection{比较解剖学}
% \subsection{人体解剖学}
\section{生物化学}
\section{生物信息学}
\section{生物物理学}
\section{生物技术}
\section{植物学}
% \subsection{民族植物学}
% \subsection{藻类学}

\section{时间生物学}
\section{计算生物学}
\section{冷冻生物学}
\section{发育生物学}
% \subsection{胚胎学}
% \subsection{畸形学}
\section{生态}
% \subsection{农业生态学}
% \subsection{民族生态学}
% \subsection{人类生态学}
% \subsection{景观生态}
\section{内分泌学}
\section{民族生物学}
\section{人类动物学}
\section{进化生物学}
\section{遗传学}
% \subsection{表观遗传学}
% \subsection{行为遗传学}
% \subsection{分子遗传学}
% \subsection{群体遗传学}
\section{组织学}
\section{人类生物学}
\section{免疫学}
\section{湖沼学}
\section{林奈分类学}
\section{海洋生物学}
\section{数理生物学}


% \subsection{原生生物学}

\section{真菌学}
\section{神经科学(大纲)}
\section{行为神经科学}
\section{营养(大纲)}
\section{古生物学}
\section{寄生虫学}
\section{病理}
% \subsection{解剖病理学}
% \subsection{临床病理学}
% \subsection{皮肤病理学}
% \subsection{法医病理学}
% \subsection{血液病理学}
% \subsection{组织病理学}
% \subsection{分子病理学}
% \subsection{手术病理}
\section{生理}
% \subsection{人体生理学}
% \subsection{运动生理学}
\section{结构生物学}
\section{系统学(分类学)}
\section{系统生物学}
\section{病毒学}
% \subsection{分子病毒学}
\section{异形生物学}


\section{动物学}    %02
% \subsection{动物通讯}
% \subsection{道歉学}
% \subsection{蛛形学}
% \subsection{节肢动物学}
% \subsection{烟道学}
% \subsection{苔藓动物学}
% \subsection{癌症学}
% \subsection{动物学}
% \subsection{刺胞动物学}
% \subsection{昆虫学}
%     \subsubsection{法医昆虫学}
% \subsection{民族动物学}
% \subsection{行为学}
% \subsection{蠕虫学}
% \subsection{爬虫学}
% \subsection{鱼类学}
% \subsection{无脊椎动物学}
% \subsection{哺乳动物学}
%     \subsubsection{犬儒学}
%     \subsubsection{猫科}
% \subsection{软糖学}
%     \subsubsection{贝壳学}
%     \subsubsection{利马科}
%     \subsubsection{条义学}
% \subsection{多足学}
% \subsection{生态学}
% \subsection{线虫学}
% \subsection{神经行为学}
% \subsection{物性}
% \subsection{鸟类学}
% \subsection{浮游学}
% \subsection{灵长类动物学}
% \subsection{动物切开术}
% \subsection{动物符号学}


\chapter{END}


\end{document}
