%%============
%%  ** Author: Shepherd Qirong
%%  ** Date: 2021-12-11 17:07:45
%%  ** Github: https://github.com/ShepherdQR
%%  ** LastEditors: Qirong ZHANG
%%  ** LastEditTime: 2024-12-29 19:01:42
%%  ** Copyright (c) 2019--20xx Shepherd Qirong. All rights reserved.
%%============



\documentclass[UTF8]{../NatureUniverse}

\begin{document}

\title{04-Crystallography}
\date{Created on 20241229.\\   Last modified on \today.}
\maketitle
\tableofcontents

\chapter{Introduction}


晶体学




\chapter{几何晶体学}
\section{晶体对称性}
    \subsubsection{对称性理论}
    \subsubsection{点群和有限图形的对称性}
    \subsubsection{空间群和点阵图形的对称性}
    \subsubsection{晶系、晶类}
\section{点阵和倒易点阵}
\section{晶体外形和晶体投影}
    \subsubsection{测角技术与仪器}
    \subsubsection{晶体投影}
    \subsubsection{晶体外形规律}
    \subsubsection{晶体外形数据}
    \subsubsection{晶体习性}






\chapter{X射线晶体学}
\section{晶体对X射线、电子和中子的衍射理论}
\section{衍射实验及数据处理}
    \subsubsection{劳厄法}
    \subsubsection{周转法、回摆法及魏森伯法}
    \subsubsection{倒易点阵直接照相法}
    \subsubsection{粉末法}
    \subsubsection{低角散射 (小角散射 )}
    \subsubsection{漫散射}
    \subsubsection{电子衍射与中子衍射}
    \subsubsection{扩展X射线吸收精细结构 (EXAFS )}
\section{结构分析}
    \subsubsection{粉末法中单胞的确定}
    \subsubsection{空间群的测定}
    \subsubsection{傅立叶综合法 (帕特森投影及电子云分布法 )及重原子法}
    \subsubsection{周相问题}
    \subsubsection{结构分析所用的模拟及计算工具}
    \subsubsection{结构参数的准确测定}
    \subsubsection{点阵常数的准确测定}







\chapter{晶体物理}
\section{晶体的物理性质}
\section{晶体的各向异性}
    \subsubsection{晶体的矢量和张量性质}
\section{晶体的力学性质}
    \subsubsection{点阵力学}
    \subsubsection{弹性与滞弹性}
    \subsubsection{范性形变}
    \subsubsection{其他}
\section{晶体的光学性质}
    \subsubsection{电光、弹光、非线性光学效应}
    \subsubsection{折射、反射}
    \subsubsection{发光现象}
\section{晶体的声学性质}
\section{晶体的热学性质}
\section{晶体的磁学性质}
\section{晶体的电学性质}
\section{晶体物理实验}








\chapter{晶体化学}
\section{晶体结构数据 (结构报告 )}
    \subsubsection{金属和合金体系}
    \subsubsection{矿物}
    \subsubsection{无机物}
    \subsubsection{硅酸盐}
    \subsubsection{氧化物体系}
    \subsubsection{有机物}
\section{晶体化学的规律性}
    \subsubsection{晶体中的化学键}
    \subsubsection{原子半径、离子半径及极化率}
    \subsubsection{密堆积和配位}
    \subsubsection{同晶型和多晶型}
    \subsubsection{化学组成和结构间的关系}
    \subsubsection{水合物和结晶水}
    \subsubsection{晶体中的氢键}
    \subsubsection{有序、无序转变}
    \subsubsection{结构与性能间的关系}
\section{系统晶体化学}
    \subsection{元素的晶体化学}
    \subsection{金属和合金晶体化学}
    \subsection{无机物晶体化学}
    \subsection{硅酸盐晶体化学}
    \subsection{有机物晶体化学}
        \subsubsection{高聚物晶体化学}
        \subsubsection{蛋白质、生化物质晶体化学}
        \subsubsection{络合物、螯合物和元素有机物晶体化学}








\chapter{非晶态和类晶态}
\section{非晶态}
\section{丝缕结构}
\section{类晶态}
    \subsubsection{微晶}
    \subsubsection{液晶}
    \subsubsection{准晶体}
\section{无定形态和琉璃态}
\section{非晶态和类晶态材料的应用}







\chapter{晶体结构}
\section{复相在晶体中的分布}
\section{孪生晶体}
\section{晶粒间界}
\section{粒度分布}
\section{晶体中的应力}
\section{观察、分析晶体结构的实验方法}
    \subsubsection{显微镜技术}
    \subsubsection{光测弹性学}
    \subsubsection{X射线方法}
    \subsubsection{衍射方法}









\chapter{晶体缺陷}
\subsubsection{点缺陷、面缺陷、体缺陷}
\subsubsection{位错}
\subsubsection{色心}
\subsubsection{高能辐射在晶体中的效应}
\subsubsection{杂质}
\subsubsection{其他缺陷}








\chapter{晶体生长}
\section{晶体生长理论}
\section{晶体生长工艺}
    \subsubsection{溶液法}
    \subsubsection{高温超高压法}
    \subsubsection{焰熔法 (维尔纳叶法 )}
    \subsubsection{熔盐法 (助熔剂法 )}
    \subsubsection{提拉法}
    \subsubsection{浮区法}
    \subsubsection{气相-固相反应}
    \subsubsection{固相-固相反应、应变退火法}
    \subsubsection{其他生长方法}
\section{再结晶}
\section{晶须}
\section{单晶体的检验}
    \subsubsection{单晶体的定向}
    \subsubsection{锥光偏振仪技术}
    \subsubsection{X射线拓扑技术}
    \subsubsection{电子自旋共振技术}
    \subsubsection{电子探针分析技术}
    \subsubsection{分光光度计技术}
    \subsubsection{位错密度的测定}
\section{晶体加工}
\section{区域提纯 (区熔提纯 )}






\chapter{晶体物理化学过程}
\subsubsection{扩散}
\subsubsection{相变}
\subsubsection{表面现象和表面性能}
\subsubsection{玻璃的晶化}
\subsubsection{晶化过程的热力学与动力学}
\subsubsection{应用晶体学}














\chapter{END}


\end{document}
