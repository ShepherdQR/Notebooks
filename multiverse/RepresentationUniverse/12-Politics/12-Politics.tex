%%============
%%  ** Author: Shepherd Qirong
%%  ** Date: 2022-06-05 14:55:53
%%  ** Github: https://github.com/ShepherdQR
%%  ** LastEditors: Qirong ZHANG
%%  ** LastEditTime: 2024-12-17 23:25:26
%%  ** Copyright (c) 2019--20xx Shepherd Qirong. All rights reserved.
%%============


\documentclass[UTF8]{../RepresentationUniverse}
\begin{document}

\title{12-Politics}
\date{Created on 20220605.\\   Last modified on \today.}
\maketitle
\tableofcontents


\chapter{Introduction}



\chapter{政治理论}


\section{阶级、阶级斗争理论}
    \subsubsection{阶级的产生与消亡}
    \subsubsection{阶级矛盾、阶级斗争与社会发展}
    \subsubsection{社会阶层、利益集团理论}

\section{革命理论}
    \subsubsection{革命的起源和本质}
    \subsubsection{各历史阶段的革命}
    \subsubsection{奴隶革命}
    \subsubsection{农民革命}
    \subsubsection{资产阶级革命}
    \subsubsection{民族民主革命}
    \subsubsection{无产阶级革命}
    \subsubsection{不断革命论与革命发展阶段论}


\section{国家理论}
\subsection{国家的起源、发展和消亡}
\subsection{国家与民族、国家与人民}
\subsection{国家政治制度}
    \subsubsection{奴隶制国家}
    \subsubsection{封建制国家}
    \subsubsection{资本主义国家}
    \subsubsection{社会主义国家}
\subsection{国家体制}
    \subsubsection{最高政权组织}
    \subsubsection{选举}
    \subsubsection{权利与义务}

\subsection{国家行政管理}
    \subsubsection{国家行政机构}
    \subsubsection{人事管理}
    \subsubsection{公安学}
        \paragraph{公安管理学}
        \paragraph{刑事侦察学 (犯罪对策学、犯罪侦察学 )}
        \paragraph{保卫学}
        \paragraph{治安管理学}
        \paragraph{预审学}
        \paragraph{消防管理}
        \paragraph{交通管理}
        \paragraph{技术装备}
    \subsubsection{监察、监督}
    \subsubsection{地方行政管理、地方自治}


\section{无产阶级革命与无产阶级专政理论}
    \subsection{民主革命与社会主义革命的关系}
    \subsection{无产阶级领导权与革命同盟军}
    \subsection{武装夺取政权的道路}
    \subsection{革命的战略和策略}
    \subsection{无产阶级专政}
        \subsubsection{民主和专政}
        \subsubsection{民主与集中、自由与纪律}
    \subsection{社会主义向共产主义过渡}


\section{政党理论}
    \subsubsection{政党的起源和本质}
    \subsubsection{资产阶级政党}
    \subsubsection{无产阶级政党}
    \subsubsection{领袖、政党、阶级、群众}


\section{民族、殖民地问题理论}
    \subsubsection{民族解放运动理论}
    \subsubsection{民族自决问题}
    \subsubsection{民族平等与民族团结}
    \subsubsection{殖民地问题}
    \subsubsection{战争与和平问题理论}
    \subsubsection{国际主义与爱国主义}


\section{政治流派和思潮}


\section{其他政治理论问题}
    \subsubsection{自由、平等、博爱}
    \subsubsection{民主、人权、民权}
    \subsubsection{其他}


\section{政治学史、政治思想史}
    \subsection{世界政治思想史}
        \subsubsection{古代}
        \subsubsection{中世纪 (476-1640年 )}
        \subsubsection{近代 (1640-1917年 )}
        \subsubsection{现代 (1917年- )}
    \subsection{社会主义思想史}
    \subsection{中国政治思想史}
    \subsection{各国政治思想史}






\chapter{国际共产主义运动}
\section{共产主义运动理论}
\section{共产主义运动初期 (1846-1864年 )}
\section{第一国际 (国际工人协会,1864-1876年 )}
    \subsubsection{马克思主义者与小资产阶级思潮的斗争}
    \subsubsection{第一国际时期小资产阶级思潮}
    \subsubsection{第一国际会议}

\section{巴黎公社 (1871年 )}
    \subsubsection{第一国际解散后的共产主义运动 (1877-1889年 )}

\section{第二国际 (1889-1900年 )}
    \subsubsection{恩格斯领导下的第二国际}
    \subsubsection{列宁主义者与第二国际修正主义的斗争 (1900-1914年 )}
    \subsubsection{第二国际时期修正主义理论}
    \subsubsection{第二国际会议}

\section{十月社会主义革命 (1917年 )}
\section{共产国际 (第三国际,1919年3月-1943年6月 )}
    \subsubsection{第三国际的准备和成立}
    \subsubsection{马克思、列宁主义者与机会主义、修正主义的斗争}
    \subsubsection{第三国际时期修正主义理论}
    \subsubsection{第三国际会议}
    \subsubsection{共产国际的解散 (1943年6月 )}
    \subsubsection{托派}

\section{共产党、工人党情报局 (1947年9月-1956年4月 )}
    \subsubsection{情报局时期的分歧}
    \subsubsection{情报局会议}
    \subsubsection{情报局的解散 (1956年4月 )}

\section{当代国际共产主义运动 (1956年4月- )}
    \subsubsection{当代国际共产主义运动中的分歧}
    \subsubsection{当代国际代表会议}
    \subsubsection{各国共产党的相互关系}
    





\chapter{中国共产党}
\section{党的领导人著作}
\section{建党理论}
\section{党章}
    \subsubsection{党章学习参考资料}

\section{党的组织、会议及文献}
    \subsubsection{中共中央文件}
    \subsubsection{地方组织、会议及其文献}
    \subsubsection{会议文献参考资料}

\section{党史}
    \subsubsection{新民主主义革命时期 (1919-1949年 )}
    \subsubsection{社会主义革命和建设时期 (1949年- )}
    \subsubsection{党的地方组织史料}
    \subsubsection{党史参考资料}

\section{党的总路线和总政策}
\section{党的领导}
    \subsubsection{领导原则与方法}
    \subsubsection{群众路线}

\section{党的建设}
    \subsection{思想建设}
        \subsubsection{思想教育、路线教育}
        \subsubsection{纠正党内错误思想}
        \subsubsection{党的作风}
        \subsubsection{党内教育}
            \paragraph{党校}
            \paragraph{党课}
        \subsubsection{党的宣传工作}

    \subsection{组织建设}
        \subsubsection{组织原则}
            \paragraph{民主集中制}
            \paragraph{党的团结与统一}
            \paragraph{党的纪律}
        \subsubsection{党的组织工作}
        \subsubsection{党的干部工作}
        \subsubsection{建党工作}
        \subsubsection{党的纪律检查工作}

    \subsection{党员}
        \subsubsection{党员标准}
            \paragraph{预备党员}
        \subsubsection{党员的权利和义务}
        \subsubsection{党性锻炼}
        \subsubsection{组织生活}

    \subsection{整风整党运动}

    \subsection{党的基层工作}
        \subsubsection{厂矿企业}
        \subsubsection{农村、乡镇企业}
        \subsubsection{部队}
        \subsubsection{财贸}
        \subsubsection{国家机关}
        \subsubsection{文教、卫生}
        \subsubsection{街道}
        \subsubsection{其他}

    \subsection{中国共产党与各国共产党的关系}
    \subsection{中国共产主义青年团}
        \subsubsection{建团理论}
        \subsubsection{团章}
            \paragraph{团章学习参考资料}
        \subsubsection{团的组织、会议及其文献}
            \paragraph{地方组织、会议及其文献}
            \paragraph{会议文献的学习参考资料}

        \subsubsection{团史}
        \subsubsection{团的建设}
            \paragraph{思想建设}
                \subparagraph{团校、团课、教材}
            \paragraph{组织建设}
            \paragraph{团员}
        \subsubsection{团的工作}
    
    \subsection{世界各国共产党}
        \subsubsection{亚洲各国共产党}
        \subsubsection{非洲各国共产党}
        \subsubsection{欧洲各国共产党}
        \subsubsection{大洋洲各国共产党}
        \subsubsection{美洲各国共产党}









\chapter{工人、农民、青年、妇女运动与组织}

\section{工人运动与组织}
\subsection{工人运动理论}
\subsection{世界工人运动与组织}
    \subsubsection{五一国际劳动节}
    \subsubsection{国际工人组织与会议}
    \subsubsection{世界工会联合会 (世界工联 )}
    \subsubsection{产业工会国际组织与会议}
    \subsubsection{其他国际性工会组织}
    \subsubsection{世界工人生活状况}
    \subsubsection{国际工人运动史}

\subsection{中国工人运动与组织}
    \subsubsection{党对工人运动的领导}
    \subsubsection{工会章程、条例}
    \subsubsection{工人组织与会议}
    \subsubsection{中华全国总工会}
    \subsubsection{产业工会组织与会议}
    \subsubsection{地方工会组织与会议}
    \subsubsection{工会工作}
        \paragraph{组织工作}
        \paragraph{思想政治教育工作}
        \paragraph{职工培训工作}
        \paragraph{生产工作}
        \paragraph{文化宣传工作}
        \paragraph{生活福利工作}
        \paragraph{财务工作}
    \subsubsection{工人生活状况}
    \subsubsection{地方工人运动与组织}
    \subsubsection{工人运动史}

\subsection{各国工人运动与组织}



\section{农民运动与组织}
\subsection{农民运动理论}
\subsection{世界农民运动与组织}
    \subsubsection{世界农民生活状况}
    \subsubsection{世界农民运动史}

\subsection{中国农民运动与组织}
    \subsubsection{党对农民运动的领导}
    \subsubsection{章程、条例}
    \subsubsection{农民组织与会议}
    \subsubsection{农民工作}
        \paragraph{组织工作}
        \paragraph{思想政治教育工作}
        \paragraph{学习}
        \paragraph{生活}
        
    \subsubsection{农民生活状况}
    \subsubsection{地方农民运动与组织}
    \subsubsection{农民运动史}

\subsection{各国农民运动与组织}





\section{青年、学生运动与组织}
\subsection{青年、学生运动理论}
\subsection{世界青年学生运动与组织}
    \subsubsection{世界青年节、世界青年联欢节}
    \subsubsection{国际青年组织与会议}
    \subsubsection{国际学生组织与会议}
    \subsubsection{国际少年儿童会议与活动}
    \subsubsection{世界青年、学生、儿童生活状况}
    \subsubsection{世界青年、学生运动史,青年社会生活史}

\subsection{中国青年学生运动与组织}
    \subsubsection{党对青年、学生运动的领导}
    \subsubsection{五四青年节}
    \subsubsection{章程、条例}
    \subsubsection{青年组织与会议}
    \subsubsection{学生组织与会议}
    \subsubsection{少年儿童组织与活动}
        \paragraph{少先队}

    \subsubsection{青年工作}
        \paragraph{组织工作}
        \paragraph{思想政治教育工作}
        \paragraph{学习}
        \paragraph{生活}

    \subsubsection{青年、学生、儿童生活状况}
    \subsubsection{地方青年、学生运动与组织}
    \subsubsection{青年、学生运动史,青年社会生活史}
   
\subsection{各国青年、学生运动与组织}



\section{妇女运动与组织}
\subsection{妇女运动理论}
\subsection{世界妇女运动与组织}
    \subsubsection{三八国际妇女节}
    \subsubsection{国际妇女组织与会议}
    \subsubsection{国际民主妇女联合会 (国际民主妇联 )}
    \subsubsection{世界母亲大会}
    \subsubsection{国际保卫儿童大会}
    \subsubsection{世界妇女生活状况}
    \subsubsection{世界妇女运动史、妇女社会生活史}

\subsection{中国妇女运动与组织}
    \subsubsection{党对妇女运动的领导}
    \subsubsection{章程、条例}
    \subsubsection{妇女组织与会议}
    \subsubsection{中华全国妇女联合会 (全国妇联 )}
    \subsubsection{地方妇女联合会}
    \subsubsection{妇女工作}
        \paragraph{组织工作}
        \paragraph{思想政治教育工作}
        \paragraph{学习}
        \paragraph{生活}
        \paragraph{其他}
        
    \subsubsection{妇女生活状况}
    \subsubsection{地方妇女运动与组织}
    \subsubsection{妇女运动史、妇女社会生活史}

\subsection{各国妇女运动与组织}





\chapter{世界政治}
\section{世界政治概况}
    \subsubsection{发展中国家 (总论 )}
    \subsubsection{发达国家 (总论 )}
    \subsubsection{美俄政治 (总论 )}
    \subsubsection{社会主义国家政治 (总论 )}
    \subsubsection{资本主义国家政治 (总论 )}

\section{世界人民革命斗争}
    \subsubsection{反对侵略扩张的斗争}
    \subsubsection{反对种族歧视的斗争}
    \subsubsection{反对法西斯主义、复活军国主义、复仇主义的斗争}
    \subsubsection{反对“和平演变”的斗争}

\section{世界政治制度与国家机构}
    \subsection{政治制度}
    \subsection{行政管理}
        \subsubsection{国家机构}
        \subsubsection{人事}
        \subsubsection{公安}
            \paragraph{政策}
            \paragraph{制度}
            \paragraph{教育、训练}
            \paragraph{国际组织及其活动}

        \subsubsection{监察、监督}
        \subsubsection{地方行政、地方自治}
        \subsubsection{移民、侨民}

    \subsection{情报机构及其活动}


\section{世界政治事件}
\section{世界社会结构}
    \subsubsection{民族问题}
    \subsubsection{政党和政治团体及其活动}
    \subsubsection{社会调查分析}

\section{社会福利与社会救济}
\section{社会生活与社会问题}
\section{世界政治制度史}










\chapter{中国政治}
\section{政策、政论}
    \subsection{方针、政策}
    \subsection{政论}
    \subsection{报刊社论}
    \subsection{评论}
        \subsubsection{港、澳、台地区评论}

\section{中国革命和建设问题}
    \subsubsection{无产阶级的革命领导权}
    \subsubsection{武装夺取政权的道路}
    \subsubsection{革命统一战线}
    \subsubsection{社会主义革命和社会主义建设总路线}
    \subsubsection{革命与生产}
    \subsubsection{中国特色社会主义建设问题}
    \subsubsection{中国革命的特殊问题}
    \subsubsection{社会主义革命和社会主义建设成就}

\section{政治制度与国家机构}
    \subsection{政治制度}
        \subsubsection{选举}
        \subsubsection{公民权利与义务}
        \subsubsection{国家表征}
        
    \subsection{全国人民代表大会}
    \subsection{国务院}
    \subsection{地方各级人民代表大会}
    \subsection{地方各级人民政府}
    \subsection{中国人民政治协商会议}



\section{国家行政管理}
\subsection{国家机关工作与人事管理}
    \subsubsection{国家机关工作}
    \subsubsection{人事管理}
    \subsubsection{监察、监督}

\subsection{公安工作}
    \subsubsection{公安行政工作}
        \paragraph{制度}
        \paragraph{组织机构与人事管理}
        \paragraph{教育、训练}
        \paragraph{政治工作}

    \subsubsection{犯罪侦察、刑事侦察工作}
    \subsubsection{保卫工作、保密工作}
    \subsubsection{治安工作}
        \paragraph{治安基层组织}
        \paragraph{户籍管理}
        \paragraph{公共秩序管理}
        \paragraph{特种行业、危险物品管理}
        \paragraph{出入境管理}

    \subsubsection{交通管理}
    \subsubsection{消防工作}
    \subsubsection{看守所、收审所管理}

\subsection{民政工作}
    \subsubsection{社会保障与社会福利}
    \subsubsection{干部离退休工作}
    \subsubsection{优抚安置}
    \subsubsection{移民}
    \subsubsection{救灾}
    \subsubsection{信访工作}
    \subsubsection{其他}

\subsection{民族工作}
    \subsubsection{民族政策}
    \subsubsection{民族事务与民族问题}
    \subsubsection{民族区域自治}
    \subsubsection{各少数民族状况}

\subsection{侨务工作}
    \subsubsection{华侨政策}
    \subsubsection{华侨事务与华侨问题}
    \subsubsection{归国华侨}
    \subsubsection{国外华侨}

\subsection{宗教工作}
    \subsubsection{宗教政策}
    \subsubsection{宗教事务与宗教问题}
    \subsubsection{宗教团体}

\subsection{群众自治工作}


\section{思想政治教育和精神文明建设}
\subsection{学习丛书、文集}
    \subsubsection{学习文选}

\subsection{学习和应用马克思列宁主义、毛泽东思想、邓小平理论}
\subsection{革命传统教育}
\subsection{形势教育、国情教育}
\subsection{国际主义教育、爱国主义教育}
\subsection{道德教育}
    \subsubsection{职业道德教育}
    \subsubsection{集体主义教育、纪律教育}
    \subsubsection{社会公德教育}
    \subsubsection{革命英雄主义、革命乐观主义教育}
\subsection{家庭、婚姻道德教育}



\section{政治运动、政治事件}
\subsection{1949年10月-1966年5月}
    \subsubsection{土地改革}
    \subsubsection{镇压反革命}
    \subsubsection{抗美援朝}
    \subsubsection{三反、五反运动}
    \subsubsection{肃反}
    \subsubsection{整风、反右派斗争}
    \subsubsection{社会主义教育运动 (四清运动 )}
    \subsubsection{其他}
\subsection{1966年5月-1976年10月}
\subsection{1976年10月-1978年12月}
\subsection{1979年1月-}




\section{阶级结构与社会结构}
\subsection{我国社会主义历史时期的阶级和阶级斗争}
\subsection{阶级、阶层}
    \subsubsection{工人阶级}
    \subsubsection{农民阶级}
    \subsubsection{小资产阶级}
    \subsubsection{民族资产阶级}
    \subsubsection{知识分子阶层}
    \subsubsection{其他}

\subsection{中国人民政治协商会议}
    \subsubsection{全国委员会}
    \subsubsection{各地方委员会}

\subsection{民主党派及其活动}
    \subsubsection{中国国民党革命委员会}
    \subsubsection{中国民主同盟}
    \subsubsection{中国民主促进会}
    \subsubsection{中国民主建国会}
    \subsubsection{中国农工民主党}
    \subsubsection{中国致公党}
    \subsubsection{九三学社}
    \subsubsection{台湾民主自治同盟}
    \subsubsection{其他}

\subsection{社会调查和社会分析}
\subsection{社会生活与社会问题}
    \subsubsection{恋爱、家庭婚姻}
    \subsubsection{职业}
    \subsubsection{生活、居住、交通}
    \subsubsection{青少年}
    \subsubsection{中、老年}
    \subsubsection{妇女}
    \subsubsection{残疾人}
    \subsubsection{社会福利}
    \subsubsection{社会病态与社会犯罪}
    \subsubsection{其他}

\subsection{地方政治概况}
    \subsubsection{台湾省}
    \subsubsection{香港特别行政区}
    \subsubsection{澳门特别行政区}


\subsection{政治制度史:清及清以前政治}
    \subsubsection{政治制度、国家机构}
    \subsubsection{选举制度}
    \subsubsection{人事制度 (职官 )}
        \paragraph{官制}
        \paragraph{考试}
        \paragraph{监察、监督}
    \subsubsection{政书}
    \subsubsection{警政}
    \subsubsection{阶级结构、社会结构}
        \paragraph{各阶级状况分析}
        \paragraph{民族问题}
        \paragraph{宗教问题}
        \paragraph{政党与政治团体}
        \paragraph{社会调查与社会分析}
    \subsubsection{社会生活与社会问题}


\subsection{政治制度史:民国时代政治}
    \subsubsection{民主革命理论}
        \paragraph{政策}
        \paragraph{政论}
        \paragraph{报刊社论}
        \paragraph{其他国家与地区的评论}

    \subsubsection{政治概况}
    \subsubsection{政治制度、国家机构}
        \paragraph{旧政协}
        \paragraph{议会、国会}
        \paragraph{“五权”制度}
        \paragraph{选举}

    \subsubsection{政治宣传与教育}
    \subsubsection{国家行政}
        \paragraph{中央行政}
        \paragraph{地方行政}
        \paragraph{人事制度、人事管理}
        \paragraph{警察}
        \paragraph{民政工作}
        \paragraph{监督、监察}

    \subsubsection{阶级结构、社会结构}
        \paragraph{各阶级状况分析}
        \paragraph{民族问题}
        \paragraph{华侨问题}
        \paragraph{宗教问题}
        \paragraph{政党和政治团体}
        \paragraph{封建行会组织}
        \paragraph{社会调查与社会分析}
    \subsubsection{社会生活与社会问题}


\subsection{新民主主义政治}



\chapter{世界各国政治}
\subsection{亚洲各国政治}
    \subsubsection{日本政治}
\subsection{非洲各国政治}
\subsection{欧洲各国政治}
    \subsubsection{苏联政治}
    \subsubsection{德国政治}
    \subsubsection{英国政治}
    \subsubsection{法国政治}
\subsection{大洋洲各国政治}
\subsection{美洲各国政治}
    \subsubsection{美国政治}






\chapter{外交、国际关系}
\section{外交、国际关系理论}
    \subsection{外交政策}
    \subsection{外交行政}
        \subsubsection{外交礼节}
        \subsubsection{外交机构}
        \subsubsection{外交事务}
    \subsection{外交特权}
    \subsection{引渡与驱逐问题}


\section{国际关系}
    \subsection{世界人民的友好往来与互相合作}
    \subsection{国际组织与会议}
        \subsubsection{国际联合会 (国际联盟 )}
        \subsubsection{联合国}
            \paragraph{大会}
            \paragraph{各种组织}
            \paragraph{各专门机构}

        \subsubsection{区域性组织和会议}
        \subsubsection{世界和平运动与组织}
            \paragraph{世界和平大会}
            \paragraph{世界和平理事会}
            \paragraph{亚洲及太平洋地区和平联络委员会}
            \paragraph{亚非人民团结组织}
            \paragraph{不结盟国家会议}
        \subsubsection{其他}

    \subsection{国际问题}
        \subsubsection{裁军问题}
        \subsubsection{禁止和销毁核武器问题}
        \subsubsection{领土争端和边界问题}
        \subsubsection{中东及巴勒斯坦问题}
        \subsubsection{国际安全问题,国际反恐怖、缉毒活动}
        \subsubsection{难民问题}
        \subsubsection{人权问题}
        \subsubsection{国籍问题}
        \subsubsection{其他国际争端问题}

    \subsection{国际条约汇编}
    \subsection{世界外交史、国际关系史}


\section{中国外交}
    \subsection{方针、政策及其阐述}
    \subsection{外交行政}
    \subsection{对外关系}
        \subsubsection{与各国人民的友好往来}
        \subsubsection{与各国政府的关系}
        \subsubsection{反对霸权主义、反对侵略扩张}
    \subsection{边界问题}
    \subsection{外侨问题}
    \subsection{引渡、驱逐问题}
    \subsection{条约}
    \subsection{地方对外关系}
    \subsection{中国外交史、对外关系史}
        \subsubsection{专题研究}
        \subsubsection{政策}
        \subsubsection{外交行政}
        \subsubsection{对外关系问题}
        \subsubsection{边界问题}
        \subsubsection{外侨问题}
        \subsubsection{引渡、驱逐问题}
        \subsubsection{条约、协定}
        \subsubsection{与各国外交关系史}

\section{亚洲外交}
\section{非洲外交}
\section{欧洲外交}
\section{大洋洲外交}
\section{美洲外交}



\end{document}

