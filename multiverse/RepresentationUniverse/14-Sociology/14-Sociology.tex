%%============
%%  ** Author: Shepherd Qirong
%%  ** Date: 2022-06-05 14:55:53
%%  ** Github: https://github.com/ShepherdQR
%%  ** LastEditors: Qirong ZHANG
%%  ** LastEditTime: 2024-12-01 21:59:58
%%  ** Copyright (c) 2019--20xx Shepherd Qirong. All rights reserved.
%%============


\documentclass[UTF8]{../RepresentationUniverse}
\begin{document}

\title{14-Sociology}
\date{Created on 20220605.\\   Last modified on \today.}
\maketitle
\tableofcontents


\chapter{Introduction}

孔德在《实证哲学》中提出社会学。

孟汉(Karl Mannheim)的知识社会学,Joachim Wach的宗教社会学,叶林(Eugen Ehrlich)的法律社会学,甚至人类学家斐司(Raymond Firth)称他We the Tikopia的调查报告作亲属社会学


拖着个“社会学”的名词表示是“以科学方法研究该项制度”的意思。


世界书局曾出过一套 “社会学丛书”,其中主要的是:社会的地理基础、心理基础、生物基础、文化基础等的题目。


孔德早已指出宇宙现象的级层,凡是在上级的必然以下级为基础,因之也可以用下级来“解释”上级。


在社会学里引成了许多派别:机械学派、生物学派、地理学派、文化学派。苏洛金 (Sorokin)曾写了一本《当代社会学学说》来介绍这许多派别。


从各制度的关系上去探讨


现代社会学的一个趋势就是社区研究,也称作社区分析。步骤:1)在一定时空坐落中去描画出一地方人民所赖以生活的社会结构;2)比较研究。在比较不同社区的社会结构时,常发现了每个社会结构有它配合的原则,原则不同,表现出来结构的形式也不一样。于是产生了“格式”的概念。

在社区分析这方面,现代社会学却和人类学的一部分通了家。


林德(Lynd) 的Middletown和马凌诺斯基(Malinowski)在Trobriad岛上的调查报告

嗣后人类学者开始研究文明人的社区,如槐南(Warner)的Yankee City Series.艾勃里(Embree)的《须惠村》(日本农村)以及拙作Peasant Life in China和Earthbound China,更不易分辨是人类学或社会学的作品了。


生理心理学,社会心理学,不属于同一层次。

\section{著作}

\begin{lstlisting}
    1.托马斯·霍布斯 主要著作《利维坦》等。
    2.约翰·洛克 主要著作《政府论》等。
    3.孟德斯鸠 主要著作《波斯人信札》、《论法的精神》等。
    4.让·雅克·卢梭 主要著作《论人类不平等的起源和基础》、《爱弥尔》、《社会契约论》等。
    5.亚当·斯密 主要著作《道德情操论》、《国富论》等。
    6.约翰·斯图亚特·穆勒 主要著作《论自由》、《功利主义》、《论妇女的从属地位》等。
    7.卡尔·马克思 主要著作《巴黎手稿》、《德意志意识形态》、《资本论》等。
    8.弗里德里希·尼采 主要著作《论道德的谱系》、《悲剧的诞
    生》、《查拉图斯特拉如是说》等。
    durkheim french sociologist
    durkheim french sociologist
    9.西格蒙德·弗洛伊德 主要著作《论梦》、《自我与本我》、《文明及其不满》等。
    10.马克斯·韦伯 主要著作《新教伦理与资本主义精神》、《经济与社会》等。
    11.埃米尔·杜尔凯姆 主要著作《社会分工》、《社会学方法论》、《自杀论》等。
    12.吉登斯,主要著作《社会学》、《社会学方法的新规则》、《社会的构成》等
    13.彼得·路德维希·柏格,主要著作《社会实体的构建》等。

    滕尼斯
    曼海姆
    帕累托
    社会静力学
    社会静力学
    马林诺夫斯基
    斯金纳
    威廉·萨姆纳(William Sumner)
    威廉·托马斯(William Thomas)
    弗洛里安·兹南尼基(Florian Zrannecki)
    帕森斯
    默顿
    斯梅尔瑟
    甘斯
    米尔斯
    米德
    刘易斯·科塞
    霍曼斯
    布劳
    爱默生
    戈夫曼
    舒茨
    哈罗德·加芬克尔
    阿多诺
    埃里希·弗洛姆
    赫伯特·马尔库塞(Herbert Marcuse)
    于尔根·哈贝玛斯(Jurgen Habermas)
    米歇尔·福柯
    利奥塔
    吕西安·斯费兹
    布迪厄
    安东尼·吉登斯
    埃利亚斯
    克利福德·格尔茨(文化人类学者,Clifford Geertz)
    费孝通
    林耀华
    冯钢
    吉登斯
\end{lstlisting}

\section{实证主义社会学}
\subsection{有机进化论}
\subsection{机械论}


\section{马克思主义社会学}


\section{社会发展理论}
探讨社会变迁规律性及其具体表现形式的学说。


\section{其他流派}

\subsection{结构功能主义}
\subsection{冲突理论}
\subsection{社会交换论}
\subsection{符号互动论}
\subsection{新功能主义}
\subsection{民俗学方法论}
\subsection{社会行为主义}
\subsection{现象学社会学}
\subsection{形式社会学}
\subsection{芝加哥学派}



\chapter{社会科学理论与方法论}
\section{科学研究的方针、政策及其阐述}
\section{科学的哲学思想}
\section{科学的方法论}
\section{术语规范及交流}
\section{与其他科学的关系}
\section{学派及其学说}
\section{资产阶级理论及其评论研究}
\section{社会科学史}
\section{社会科学现状、概况}
    \subsection{专利}
    \subsection{创造发明、先进经验}



\chapter{机关、团体、会议}
\subsubsection{国际组织}
\subsubsection{社会团体}
\subsubsection{研究机构}
\subsubsection{学术团体、学会、协会}
\subsubsection{学术会议、专业会议}
\subsubsection{展览会、展览馆、博物馆}
\subsubsection{图书馆、信息服务机构、咨询机构}
\subsubsection{企业}


\chapter{社会科学研究方法}
\subsubsection{调查方法、工作方法}
\subsubsection{统计方法、计算方法}
\subsubsection{试验方法与试验设备}
\subsubsection{分析、研究与鉴定}
\subsubsection{技术条件}
\subsubsection{组织管理}
\subsubsection{数据处理}
\subsubsection{新技术的应用}


\chapter{社会科学教育与普及}
\subsubsection{教育组织、学校}
\subsubsection{教学计划、教学大纲、课程研究}
\subsubsection{教学方法、教学参考书}
\subsubsection{教材、课本}
\subsubsection{习题、试题与题解}
\subsubsection{教学实验、实习}
\subsubsection{社会科学普及读物}



\chapter{社会科学书}

\section{社会科学丛书、文集、连续性出版物}
    \subsubsection{丛书 (汇刻书 )、文库}
    \subsubsection{全集、选集}
    \subsubsection{文集、会议录}
    \subsubsection{会议录}
    \subsubsection{学位论文、毕业论文}
    \subsubsection{杂著}
    \subsubsection{年鉴、年刊}
    \subsubsection{连续出版物}
    \subsubsection{政府出版物、团体出版物}

\section{社会科学参考工具书}
    \subsubsection{名词术语、辞典、百科全书 (类书 )}
    \subsubsection{手册、指南、一览表、年表}
    \subsubsection{目录、样本、说明书}
    \subsubsection{表解、图解、图册、公式、数据、地图}
    \subsubsection{条例、规程、标准}
    \subsubsection{统计资料}
    \subsubsection{参考资料}

\section{社会科学文献检索工具书}



\chapter{统计学}
\section{统计方法}
    \subsubsection{统计调查}
    \subsubsection{统计资料的分析和整理}
    \subsubsection{统计指数}
    \subsubsection{统计平均数}
    \subsubsection{统计图示法}
    \subsubsection{统计资料管理}
    \subsubsection{统计技术设备}
\section{专类统计学}
    \subsection{世界各国统计工作}
    \subsection{中国统计工作}
        \subsubsection{方针、政策}
        \subsubsection{统计制度}
        \subsubsection{统计机构、组织与管理}
        \subsubsection{统计事业史}
\section{统计资料}
    \subsection{世界}
    \subsection{中国}
      * C832.1/.7 各地区统计资料





\chapter{社会学}
\section{社会学理论与方法论}
    \subsubsection{社会学方法论}
    \subsubsection{学派及其研究}
    \subsubsection{社会学史、社会思想史}
\section{社会发展和变迁}
\section{社会结构和社会关系}
    \subsection{个人 (社会人 )}
    \subsection{社会团体}
    \subsection{社会关系、社会制约}
    \subsection{文化人类学、社会人类学}
    \subsection{民族学}
    \subsection{社会心理、社会行为}
        \subsubsection{阶级心理}
        \subsubsection{社会舆论}
        \subsubsection{群众心理}
        \subsubsection{宗教心理}
        \subsubsection{社会思潮}
        \subsubsection{社会行为}
        \subsubsection{其他}
    \subsection{地区社会学}
        \subsubsection{城市社会学}
        \subsubsection{农村社会学}
\section{社会生活和社会问题}
    \subsection{恋爱、家庭、婚姻}
        \subsubsection{家庭、家族}
        \subsubsection{婚姻}
        \subsubsection{两性问题}
        \subsubsection{生育}
    \subsection{职业}
    \subsection{生活与消费}
        \subsubsection{居住}
        \subsubsection{交通}
        \subsubsection{生活日用品供应与消费}
    \subsection{文教卫生}
    \subsection{青少年问题}
    \subsection{老年人问题}
    \subsection{妇女问题}
    \subsection{残疾人问题}
    \subsection{社会福利、救济、社会保障}
    \subsection{社会病态}
    \subsection{其他社会问题}
\section{社会利益}
\section{社会调查和分析}




\chapter{人口学}
\section{人口学理论与方法论}
    \subsubsection{人口学方法论}
    \subsubsection{与其他学科的关系}
    \subsubsection{人口学史}
\section{人口统计学}
    \subsubsection{人口调查法}
    \subsubsection{户籍登记法}
\section{人口地理学}
\section{人口与计划生育}
\section{世界各国人口调查及其研究}
    \subsection{世界人口}
    \subsection{中国人口}
        \subsubsection{人口政策与制度}
        \subsubsection{人口规划}
        \subsubsection{人口问题研究}
        \subsubsection{人口调查}





\chapter{管理学}
\section{管理学理论与方法论}
    \subsubsection{管理学的哲学基础}
    \subsubsection{管理学方法论}
    \subsubsection{与其他学科的关系}
    \subsubsection{学派及其研究}
    \subsubsection{管理学史}
\section{管理技术与方法}
    \subsection{管理数学}
    \subsection{管理的方式方法}
    \subsection{管理工作、管理人员}
    \subsection{办公室工作}
        \subsubsection{文书工作}
        \subsubsection{会议组织与管理}
    \subsection{管理信息系统}
    \subsection{管理工作自动化}
\section{咨询学}
    \subsubsection{咨询方法与咨询技术}
    \subsubsection{咨询管理}
    \subsubsection{咨询服务}
    \subsubsection{世界各国咨询业}
    \subsubsection{专科咨询学}
\section{领导学}
    \subsubsection{领导体制}
    \subsubsection{领导方法}
    \subsubsection{领导权}
    \subsubsection{领导组织建设}
\section{决策学}
\section{管理计划和控制}
\section{管理组织学}
\

{应用管理学}



\chapter{系统科学}



\chapter{民族学}
\section{民族学理论与方法论}
    \subsubsection{民族学与其他学科关系}
    \subsubsection{民族学学派}
\section{民族起源、发展、变迁}
\section{民族史志、民族地理}
\section{民俗学}
\section{民族社会形态、社会制度}
\section{民族性、民族心理}
\section{民族融合、民族同化}
\section{民族工作、民族问题}



\chapter{人才学}
\subsubsection{人才培养与人才选拔}
\subsubsection{人才预测与人才规划}
\subsubsection{人才管理}
\subsubsection{人才智力开发}
\subsubsection{世界各国人才调查及研究}
\subsubsection{人才市场}
\subsubsection{专门人才学}





\chapter{劳动科学}
\section{劳动科学基础理论}
    \subsubsection{劳动哲学}
    \subsubsection{劳动心理学}
    \subsubsection{劳动生理学}
    \subsubsection{劳动科学学}
\section{劳动经济学}
\section{劳动法学}
\section{劳动关系学}
\section{劳动管理学}
\section{职业培训}
    \subsubsection{劳动社会学}
    \subsubsection{劳动安全、劳动卫生}
    \subsubsection{劳动计量学}
    \subsubsection{劳动统计学}
\section{社会保障学}









\end{document}

