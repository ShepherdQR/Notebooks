%%============
%%  ** Author: Shepherd Qirong
%%  ** Date: 2022-06-05 14:55:53
%%  ** Github: https://github.com/ShepherdQR
%%  ** LastEditors: Shepherd Qirong
%%  ** LastEditTime: 2023-03-30 20:53:36
%%  ** Copyright (c) 2019--20xx Shepherd Qirong. All rights reserved.
%%============


\documentclass[UTF8]{../RepresentationUniverse}
\begin{document}

\title{14-Sociology}
\date{Created on 20220605.\\   Last modified on \today.}
\maketitle
\tableofcontents


\chapter{Introduction}

孔德在《实证哲学》中提出社会学。

孟汉(Karl Mannheim)的知识社会学,Joachim Wach的宗教社会学,叶林(Eugen Ehrlich)的法律社会学,甚至人类学家斐司(Raymond Firth)称他We the Tikopia的调查报告作亲属社会学


拖着个“社会学”的名词表示是“以科学方法研究该项制度”的意思。


世界书局曾出过一套 “社会学丛书”,其中主要的是:社会的地理基础、心理基础、生物基础、文化基础等的题目。


孔德早已指出宇宙现象的级层,凡是在上级的必然以下级为基础,因之也可以用下级来“解释”上级。


在社会学里引成了许多派别:机械学派、生物学派、地理学派、文化学派。苏洛金 (Sorokin)曾写了一本《当代社会学学说》来介绍这许多派别。


从各制度的关系上去探讨


现代社会学的一个趋势就是社区研究,也称作社区分析。步骤:1)在一定时空坐落中去描画出一地方人民所赖以生活的社会结构;2)比较研究。在比较不同社区的社会结构时,常发现了每个社会结构有它配合的原则,原则不同,表现出来结构的形式也不一样。于是产生了“格式”的概念。

在社区分析这方面,现代社会学却和人类学的一部分通了家。


林德(Lynd) 的Middletown和马凌诺斯基(Malinowski)在Trobriad岛上的调查报告

嗣后人类学者开始研究文明人的社区,如槐南(Warner)的Yankee City Series.艾勃里(Embree)的《须惠村》(日本农村)以及拙作Peasant Life in China和Earthbound China,更不易分辨是人类学或社会学的作品了。


生理心理学,社会心理学,不属于同一层次。

\section{著作}

\begin{lstlisting}
    1.托马斯·霍布斯 主要著作《利维坦》等。
    2.约翰·洛克 主要著作《政府论》等。
    3.孟德斯鸠 主要著作《波斯人信札》、《论法的精神》等。
    4.让·雅克·卢梭 主要著作《论人类不平等的起源和基础》、《爱弥尔》、《社会契约论》等。
    5.亚当·斯密 主要著作《道德情操论》、《国富论》等。
    6.约翰·斯图亚特·穆勒 主要著作《论自由》、《功利主义》、《论妇女的从属地位》等。
    7.卡尔·马克思 主要著作《巴黎手稿》、《德意志意识形态》、《资本论》等。
    8.弗里德里希·尼采 主要著作《论道德的谱系》、《悲剧的诞
    生》、《查拉图斯特拉如是说》等。
    durkheim french sociologist
    durkheim french sociologist
    9.西格蒙德·弗洛伊德 主要著作《论梦》、《自我与本我》、《文明及其不满》等。
    10.马克斯·韦伯 主要著作《新教伦理与资本主义精神》、《经济与社会》等。
    11.埃米尔·杜尔凯姆 主要著作《社会分工》、《社会学方法论》、《自杀论》等。
    12.吉登斯,主要著作《社会学》、《社会学方法的新规则》、《社会的构成》等
    13.彼得·路德维希·柏格,主要著作《社会实体的构建》等。

    滕尼斯
    曼海姆
    帕累托
    社会静力学
    社会静力学
    马林诺夫斯基
    斯金纳
    威廉·萨姆纳(William Sumner)
    威廉·托马斯(William Thomas)
    弗洛里安·兹南尼基(Florian Zrannecki)
    帕森斯
    默顿
    斯梅尔瑟
    甘斯
    米尔斯
    米德
    刘易斯·科塞
    霍曼斯
    布劳
    爱默生
    戈夫曼
    舒茨
    哈罗德·加芬克尔
    阿多诺
    埃里希·弗洛姆
    赫伯特·马尔库塞(Herbert Marcuse)
    于尔根·哈贝玛斯(Jurgen Habermas)
    米歇尔·福柯
    利奥塔
    吕西安·斯费兹
    布迪厄
    安东尼·吉登斯
    埃利亚斯
    克利福德·格尔茨(文化人类学者,Clifford Geertz)
    费孝通
    林耀华
    冯钢
    吉登斯
\end{lstlisting}

\section{实证主义社会学}
\subsection{有机进化论}
\subsection{机械论}


\section{马克思主义社会学}


\section{社会发展理论}
探讨社会变迁规律性及其具体表现形式的学说。


\section{其他流派}

\subsection{结构功能主义}
\subsection{冲突理论}
\subsection{社会交换论}
\subsection{符号互动论}
\subsection{新功能主义}
\subsection{民俗学方法论}
\subsection{社会行为主义}
\subsection{现象学社会学}
\subsection{形式社会学}
\subsection{芝加哥学派}



\end{document}

