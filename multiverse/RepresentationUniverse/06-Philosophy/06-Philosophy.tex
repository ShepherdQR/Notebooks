%%============
%%  ** Author: Shepherd Qirong
%%  ** Date: 2022-05-08 20:30:50
%%  ** Github: https://github.com/ShepherdQR
%%  ** LastEditors: Shepherd Qirong
%%  ** LastEditTime: 2023-03-18 22:00:39
%%  ** Copyright (c) 2019--20xx Shepherd Qirong. All rights reserved.
%%============

\documentclass[UTF8]{../RepresentationUniverse}
\begin{document}

\title{06-Philosophy}
\date{Created on 20220605.\\   Last modified on \today.}
\maketitle
\tableofcontents


\chapter{Introduction}

哲学是爱智慧。哲学伴随着哲学史。

\section{什么是哲学}

\begin{lstlisting}
1. 人类、社会、自然的一般规律的总结
2. 哲学是哲学史————黑格尔
3. 哲学让人聪明,让人糊涂。很多冲突的道理。
4. 讲道理,大到无法验证。
5. 意识形态的工具,总是有理。
6. 研究的对象是理想的事物。如画鬼,像不像不是评价标准,技法可以。
7. 无用。为了自由而追求学问。关注人的精神问题。工具理性和价值理性的平衡问题。
8. 爱智慧。柏拉图认为有限的人只能追求智慧,即热爱无限境界。
9. 有可能解决的问题就不是哲学了。
10. 靠理性,关注终极关怀(另一世界)。宗教靠信仰。
11. 无标准答案的难题。
\end{lstlisting}


\section{如何研究哲学}

哲学家的书难度,因为要用日常语言描述无限的东西,要强说不可说的东西。要猜测文字背后所指。

重点关注:要解决什么问题,怎么解决的。
了解哲学家的问题;
熟悉术语、特有概念;
了解逻辑。


\section{句子}


\begin{lstlisting}
   人有了意识,有了对生死的恐惧。根源问题是从生死,到永恒的问题。
   泰勒斯:人生有限,却思考无限的问题。

   哲学自我边缘化。哲学著作没人读,影响何在?
   哲学是一种思维生活方式。

\end{lstlisting}

\chapter{笔记-待整理区域}


西方哲学的特点:希腊:1)关注普遍性;2)为事物的存在找原因,即拯救现象。
亚里士多德:求知是人类的本性。


人参与社会,需要对话,语言的对话功能,需要有共同的对话平台,平台包括2方面:逻辑和共同的前提。最早的对话平台是一堆人构成的集团之间的神话。古希腊城邦,有一系列神话。为了城邦联合对抗波斯人,需要更为通用的对话平台,产生了哲学。哲学把人的理性作为共同的前提。而神话的共同前提设定是最高的神秘存在,就是神,内涵是人的一系列基本价值。

实用理性,不做终极追溯,有用就行。纯粹理性的最高价值是普世真理。

不同人的感性经验不同,抽象出共同的东西,一层层抽象,最后得到超验的东西,达到了神话的程度,自在之物,但不是神话,因为神话是凭空设定的。

哲学是把人类包括在内的,对终极存在的不停的追问、思考。

\section{Knowledge}

\subsection{Materials}

\subsubsection{ 陈宣良 哲学十讲}
陈宣良,1947年生于北京。1978年入武汉大学哲学系读硕士学位,1984年入中国人民大学哲学系攻读博士学位,1987年到中国青年政治学院任教。1989年到法国定居至今。著有《法国本体论哲学的演进》《理性主义》《死与道德》等。

神话,集团中的对话平台。









\section{命题}

\subsection{我思故我在}

笛卡尔论证上帝存在,因为经院哲学是主流的对话平台,启蒙思想作为新生的,从生存角度还是想融合进主流的。通过理性追溯到信仰,达到理性和信仰的和解。

理性无限,连上帝存在都可以证明了。人的理性建立在个人有限的不准确的感觉基础上,只能拉低理性的水平。尽量去掉感性,纯粹的理性更多一些来提高理性的水平。

我思故我在。思,所以是。思是思。作为人自身存在的是思想。思想是独立存在的,是它自己。自在的存在,自在之物,自己就是自己的东西。“存在”也叫作“实体”,它本身不是什么,可以什么都是也可以什么都不是,还没有本质的。

哲学追溯最原始的存在,建立一个终极平台。

上帝是当时哲学的终点。古希腊是无数神话,基督教是1个神话,基督教分裂后,陷入危机。每个人的上帝是不一样的,都想证明上帝的存在。当需要证明上帝存在的时候,就不是最高权威了,其实是人的理性来代替上帝的理性。笛卡尔与经院哲学的显著区别,就在于证明上帝存在的方式,脱离了神话本身。基督教的上帝时时在场,笛卡尔的上帝创世之后,世界就自己运转了。用逻辑的方式确立了一个最高的存在,这个存在其实就是人的思想,思想是自己存在的,思想就是思想。

引出了二元论的问题,目前无解。

理性主义最核心原则:怀疑。
首先回忆,然后自我意识的提升的第一步是怀疑。没有怀疑过的东西都不是真理。首先要确定1个不可怀疑的东西,一个起点,这个起点就是“我在怀疑”这件事。

科学不是理性主义,只有在反省科学的时候,才进入理性主义。

\subsection{存在就是被感知}

理性限制在有限范围内,信仰在理性之外,这样达到理性和信仰的和解。

洛克的白板说,通过不断获得对象的观念组合形成逻辑,产生对世界的解释。

因果关系是人的联想。休谟的观念,习惯是人生的伟大指南。如果直接关注未来是事情,目前的经验不能准确知道未来的事情,只能占卜了。

经验主义,按照观察到的事件之间的联系,人为创造现象发生的条件,如果获得了预测到的结论,建立了可实践的科学。

“认识”本身就是对对象的干扰。如果要设定是白板,所观察的对象就要是主动的发送观念,有问题。主体和对象对立之后,主观和客观对立起来之后,是无法解决的问题。

\subsection{人为自然界立法}



真理:主观认识和客观存在一致,且逻辑推演无矛盾。

康德想要找到和现实一致的且合乎逻辑的哲学,首先批判人的理性,找到理性的极限。康德认为,逻辑先天就有,先天的认识框架是时间和空间,时间和空间不是客观存在的(与牛顿的时间和空间是客观存在的不一致),内容从经验中来。人用时间和空间把散乱的现象、经验组织起来形成知识。纯粹理性指的是个人的纯粹的时间和空间概念是无用的。

如何判断出先天的清楚的又有内容的命题呢?例如1=1这样的命题收集起来没有意义。用时空框架把握感知内涵。

用时空框架看到世界时,怎么断定看到的世界就是那个世界本身呢?没法断定。看到的是现象。没法突破现象界的真理。或者是杂乱无章的观念,或者是纯粹的没有内涵的逻辑。

现象之后是什么呢?是自在之物,是存在本身,要说出它是什么的时候,它已经是什么而不是它了,它是什么呢,它是它自己。人没有可能认识那个东西。

人本身也是现象界的存在。掌握现象界的真理就够了。

存在是个问题,自在之物的设定,或对于存在存而不论,相当于上帝的设定,是用不着的,是多余的。


唯物主义,断定的本原的存在是物质;唯心主义任务是精神性的。自然界在精神内还是精神外,有问题。

P4 20:05















\chapter{哲学史}

    \section{东方哲学}
        \subsection{中国哲学}
        \subsection{印度哲学}
        \subsection{伊斯兰哲学}
        \subsection{日本哲学}


    \section{西方哲学}



    \section{古代哲学}

        \subsection{古希腊哲学}
        BC6世纪到公元后6世纪。希腊学校不存在,靠基督教统治。

        \subsubsection{城邦}

        罗马集成希腊的版图。

        分离的个人聚集,靠传统。希腊产生城邦,传统的影响小。繁荣在雅典。城邦民主制。

        人之间的影响靠语言。

        马其顿打败雅典,导致衰落。


        \paragraph{哲学产生的条件}
    
        有终极关怀问题;有时间思考;思想自由。

        \paragraph{特点}

        百家争鸣。

        人类的共同话语平台:1)神话,秘密,命运与规律。2)哲学,神话止步的地方继续追问。

        给自然以合理的解释,诞生哲学。


        \subsubsection{早期}

        宇宙存在的合理说明。有生有死,世界循环还存在,什么是不变的?构成世界的东西最根本的是什么?
        【回答】是本原。本原是万物之来,之归,有开端、主宰的意思。原因产生了宇宙,在宇宙中,宇宙无了还在。


        \paragraph{希腊东部}

        米立都学派,泰勒斯:大地浮在水上。
        【解读】水是万物的本原。水是比喻、象征,以水的性质来体会本原。

        赫拉克里特:宇宙是永恒的活火。
        【解读】更强调“变”。混沌、循环。

        Logus:活火的尺度、规律。其学生:人一次也踏不入。


        \paragraph{希腊南部}

        本体论转向。

        毕达哥拉斯:数是万物的本原。

        巴门尼德:认识的对象是不动不变的。异见之路和真理之路的区分,引导出“存在”。提出“存在”概念。思维与存在的统一性。哲学命题和逻辑。
        巴门尼德之前:桌子是由木头做的,不解决桌子是桌子的问题,是从时间上、构成材料上研究。
        巴门尼德之后:问桌子的本质。

        如何对自然形成知识。


        \paragraph{存在}

        核心概念。

        存在:可思想。
        非存在:流变中的现象,不能被叙说,不能被思想。

        词根、词源是印欧语系,S是P。being。汉语的存在有“空间性”。

        语言中什么是不变的。先存在再成为存在物。

        形而上学,与本体论是同义语。本体论:研究存在的。

        
        \subsubsection{雅典时期}

        智者:意思是圣贤,职业教师,教修辞学、辩论、诡辩。启蒙,用的怀疑主义。

        同一条河不能面对2个人。

        人是万物的尺度。
        【解读】尺度在每个人那里。一切理论都有对立的。


        \paragraph{苏格拉底}

        苏格拉底时期,只讲雄辩不论对错,否定知识的可能性,导致民主制度的衰落。

        知识。与道德相关的论题。

        用构成万物的材料解决自然本原,不对。应追问使万物称为万物的目的。称这目的为“善”。万物有向善的目标。

        认识你自己,试图证明别人比自己有知。让人们意识到自己是无知的,得罪很多人。

        70岁判死刑,2项罪名:教唆犯;引进外邦精神。

        政治理想与雅典制度产生了冲突。治理国家不需要治国的知识吗?救国靠知识上,但衰落不可避免。

        认识人自己。认识人心中的“善”,称为德性。车能跑是车的德性。人的德性,需要自觉,才能发出来。德性即知识,无知即罪恶。

        从知识角度考虑伦理学。

        一个东西是什么。认识事物的类本质。

        美是什么?具体的事物是相对的美。一切事物的美,美的事物是什么?美之为美,美本身是什么?

        通过思想把握存在。追问本质,用具体的事物回答有问题,从而逐渐用抽象的事物回答。

        辩证法:我无知但能引导你有知。

        黑格尔辩证法:由工人对话,推导出自身的矛盾运动。

        苏格拉底要论出真理。

        反击、诱导、归纳(经验观察从归纳开始。范围内有效,影响实验科学)、结论。

        知识是否可教?知识是先天的,学习是回忆起已经忘记的东西。


        \paragraph{柏拉图}

        理念化。

        《理想国》应译作《国家篇》。哲学家做王的国家好。

        灵魂理性————统治者国家
        灵魂激情————保卫者国家
        灵魂欲望————生产者国家

        4主德:智慧、勇敢、节制、正义。

        现象世界:信以为真的,可感的,不能形成知识。
        理念世界:宇宙的目的,本质世界。

        洞穴:影像,走出洞穴。

        理念,只有灵魂能看到。

        Idea







        \subsection{重要人物}


        \subsubsection{亚里士多德}
        \subsubsection{柏拉图}

    \section{中世纪哲学}
    \subsection{人文主义}
    \subsection{经院哲学}

    \section{文艺复兴时期哲学}

    \section{德国古典哲学}
    \section{俄国哲学}
    

    
    \section{当代哲学}

\section{哲学学派}
    \subsection{非洲哲学}
    \subsection{欧陆哲学}
    \subsection{东方哲学}
    \subsection{女权主义哲学}
    \subsection{无政府主义}
    \subsection{女权主义哲学}
    \subsection{自由主义}
    \subsection{马克思主义}




\chapter{现代哲学}
    \subsection{分析哲学}
    \subsection{生存哲学}
    \subsection{人文哲学}
    \subsection{解释学}
    \subsection{符号学}
    \subsection{实用主义哲学}
    




\chapter{认识论}
    \subsection{理由}
    \subsection{推理错误}

\chapter{逻辑学}
    \subsection{数理逻辑}
    \subsection{哲学逻辑}       

\chapter{伦理学}
    \subsection{应用伦理}
        \subsubsection{动物权益}
        \subsubsection{环境伦理}
        \subsubsection{医学伦理}
        \subsubsection{教育伦理}
        \subsubsection{政治伦理}
        \subsubsection{家庭伦理}
        \subsubsection{生命伦理学}
        \subsubsection{生态伦理}

    \subsection{元伦理学}
    \subsection{描述伦理,价值理论}
    \subsection{规范伦理}
    

\chapter{元哲学}
\chapter{过程哲学}
\chapter{死亡哲学}
\chapter{人生哲学}
\chapter{法律哲学}
\chapter{心灵哲学}
\chapter{比较哲学}
\chapter{知识论}




\chapter{美学}
    \subsection{美学史}
    \subsection{艺术美学}
    \subsection{技术美学}



\chapter{形而上学}
    \subsection{行动哲学}
    \subsection{决定论和自由意志}
    \subsection{本体}
    \subsection{时空哲学}
    \subsection{目的论}
    \subsection{有神论和无神论}


\chapter{现象学}


\chapter{马克思主义哲学}
    \section{辩证唯物主义}
    \section{历史唯物主义}
    \section{马克思主义哲学史}




\chapter{应用哲学}
    \subsection{经济学哲学}
    \subsection{教育理念}
    \subsection{工程哲学}
    \subsection{科学哲学}
    \subsection{政治哲学}
    \subsection{历史哲学}
    \subsection{语言哲学}
    \subsection{法哲学}
    \subsection{数学哲学}
    \subsection{音乐哲学}
    \subsection{心理学哲学}
    \subsection{宗教哲学}
    \subsection{自然科学哲学}
    \subsubsection{生物学哲学}
    \subsubsection{化学哲学}
    \subsubsection{物理学哲学}
    \subsection{社会科学哲学}
    \subsection{技术哲学}
    \subsection{系统理念}











\end{document}
