%%============
%%  ** Author: Qirong ZHANG
%%  ** Date: 2022-05-08 20:30:50
%%  ** Github: https://github.com/ShepherdQR
%%  ** LastEditors: Qirong ZHANG
%%  ** LastEditTime: 2024-12-07 23:21:16
%%  ** Copyright (c) 2019 Qirong ZHANG. All rights reserved.
%%  ** SPDX-License-Identifier: LGPL-3.0-or-later.
%%============

\documentclass[UTF8]{../RepresentationUniverse}
\begin{document}

\title{06-Philosophy}
\date{Created on 20220605.\\   Last modified on \today.}
\maketitle
\tableofcontents


\chapter{Introduction}

哲学是爱智慧。哲学伴随着哲学史。

\section{哲学教育、学习、普及}


\section{什么是哲学}

\begin{lstlisting}
1. 人类、社会、自然的一般规律的总结
2. 哲学是哲学史————黑格尔
3. 哲学让人聪明,让人糊涂。很多冲突的道理。
4. 讲道理,大到无法验证。
5. 意识形态的工具,总是有理。
6. 研究的对象是理想的事物。如画鬼,像不像不是评价标准,技法可以。
7. 无用。为了自由而追求学问。关注人的精神问题。工具理性和价值理性的平衡问题。
8. 爱智慧。柏拉图认为有限的人只能追求智慧,即热爱无限境界。
9. 有可能解决的问题就不是哲学了。
10. 靠理性,关注终极关怀 (另一世界 )。宗教靠信仰。
11. 无标准答案的难题。
\end{lstlisting}


\section{如何研究哲学}

哲学家的书难度,因为要用日常语言描述无限的东西,要强说不可说的东西。要猜测文字背后所指。

重点关注:要解决什么问题,怎么解决的。
了解哲学家的问题;
熟悉术语、特有概念;
了解逻辑。
传播和发展。

\section{从2大方面研究一位哲学家}

\subsection{著作、理论的学习和研究}
每个人的。

\begin{lstlisting}
    \subsubsection{选集、文集}
    \subsubsection{单行著作}
    \subsubsection{书信集、日记、函电、谈话}
    \subsubsection{诗词}
    \subsubsection{手迹}
    \subsubsection{专题汇编}
    \subsubsection{语录}
\end{lstlisting}


    

\subsection{生平和传记}
    每个人的。
    \begin{lstlisting}
        \subsubsection{传记}
        \subsubsection{生平事迹、回忆录}
        \subsubsection{年谱、年表}
        \subsubsection{纪念文集}
        \subsubsection{阐述、研究}
        \subsubsection{肖像、照片、画传、像章}
        \subsubsection{纪念地、故居、遗物}
    \end{lstlisting}





\section{句子}


\begin{lstlisting}
   人有了意识,有了对生死的恐惧。根源问题是从生死,到永恒的问题。
   泰勒斯:人生有限,却思考无限的问题。

   哲学自我边缘化。哲学著作没人读,影响何在?
   哲学是一种思维生活方式。

\end{lstlisting}

\chapter{笔记-待整理区域}


西方哲学的特点:希腊:1 )关注普遍性;2 )为事物的存在找原因,即拯救现象。
亚里士多德:求知是人类的本性。


人参与社会,需要对话,语言的对话功能,需要有共同的对话平台,平台包括2方面:逻辑和共同的前提。最早的对话平台是一堆人构成的集团之间的神话。古希腊城邦,有一系列神话。为了城邦联合对抗波斯人,需要更为通用的对话平台,产生了哲学。哲学把人的理性作为共同的前提。而神话的共同前提设定是最高的神秘存在,就是神,内涵是人的一系列基本价值。

实用理性,不做终极追溯,有用就行。纯粹理性的最高价值是普世真理。

不同人的感性经验不同,抽象出共同的东西,一层层抽象,最后得到超验的东西,达到了神话的程度,自在之物,但不是神话,因为神话是凭空设定的。

哲学是把人类包括在内的,对终极存在的不停的追问、思考。

\section{Knowledge}

\subsection{Materials}

\subsubsection{ 陈宣良 哲学十讲}
陈宣良,1947年生于北京。1978年入武汉大学哲学系读硕士学位,1984年入中国人民大学哲学系攻读博士学位,1987年到中国青年政治学院任教。1989年到法国定居至今。著有《法国本体论哲学的演进》《理性主义》《死与道德》等。

神话,集团中的对话平台。


\subsection{书籍}

\begin{lstlisting}
    西方哲学智慧
    智慧的探险
    《名家通识讲座书系》西方哲学十五讲
 
 \end{lstlisting}





\section{命题}

\subsection{我思故我在}

笛卡尔论证上帝存在,因为经院哲学是主流的对话平台,启蒙思想作为新生的,从生存角度还是想融合进主流的。通过理性追溯到信仰,达到理性和信仰的和解。

理性无限,连上帝存在都可以证明了。人的理性建立在个人有限的不准确的感觉基础上,只能拉低理性的水平。尽量去掉感性,纯粹的理性更多一些来提高理性的水平。

我思故我在。思,所以是。思是思。作为人自身存在的是思想。思想是独立存在的,是它自己。自在的存在,自在之物,自己就是自己的东西。“存在”也叫作“实体”,它本身不是什么,可以什么都是也可以什么都不是,还没有本质的。

哲学追溯最原始的存在,建立一个终极平台。

上帝是当时哲学的终点。古希腊是无数神话,基督教是1个神话,基督教分裂后,陷入危机。每个人的上帝是不一样的,都想证明上帝的存在。当需要证明上帝存在的时候,就不是最高权威了,其实是人的理性来代替上帝的理性。笛卡尔与经院哲学的显著区别,就在于证明上帝存在的方式,脱离了神话本身。基督教的上帝时时在场,笛卡尔的上帝创世之后,世界就自己运转了。用逻辑的方式确立了一个最高的存在,这个存在其实就是人的思想,思想是自己存在的,思想就是思想。

引出了二元论的问题,目前无解。

理性主义最核心原则:怀疑。
首先回忆,然后自我意识的提升的第一步是怀疑。没有怀疑过的东西都不是真理。首先要确定1个不可怀疑的东西,一个起点,这个起点就是“我在怀疑”这件事。

科学不是理性主义,只有在反省科学的时候,才进入理性主义。

\subsection{存在就是被感知}

理性限制在有限范围内,信仰在理性之外,这样达到理性和信仰的和解。

洛克的白板说,通过不断获得对象的观念组合形成逻辑,产生对世界的解释。

因果关系是人的联想。休谟的观念,习惯是人生的伟大指南。如果直接关注未来是事情,目前的经验不能准确知道未来的事情,只能占卜了。

经验主义,按照观察到的事件之间的联系,人为创造现象发生的条件,如果获得了预测到的结论,建立了可实践的科学。

“认识”本身就是对对象的干扰。如果要设定是白板,所观察的对象就要是主动的发送观念,有问题。主体和对象对立之后,主观和客观对立起来之后,是无法解决的问题。

\subsection{人为自然界立法}



真理:主观认识和客观存在一致,且逻辑推演无矛盾。

康德想要找到和现实一致的且合乎逻辑的哲学,首先批判人的理性,找到理性的极限。康德认为,逻辑先天就有,先天的认识框架是时间和空间,时间和空间不是客观存在的 (与牛顿的时间和空间是客观存在的不一致 ),内容从经验中来。人用时间和空间把散乱的现象、经验组织起来形成知识。纯粹理性指的是个人的纯粹的时间和空间概念是无用的。

如何判断出先天的清楚的又有内容的命题呢?例如1=1这样的命题收集起来没有意义。用时空框架把握感知内涵。

用时空框架看到世界时,怎么断定看到的世界就是那个世界本身呢?没法断定。看到的是现象。没法突破现象界的真理。或者是杂乱无章的观念,或者是纯粹的没有内涵的逻辑。

现象之后是什么呢?是自在之物,是存在本身,要说出它是什么的时候,它已经是什么而不是它了,它是什么呢,它是它自己。人没有可能认识那个东西。

人本身也是现象界的存在。掌握现象界的真理就够了。

存在是个问题,自在之物的设定,或对于存在存而不论,相当于上帝的设定,是用不着的,是多余的。


唯物主义,断定的本原的存在是物质;唯心主义任务是精神性的。自然界在精神内还是精神外,有问题。

P4 20:05















\chapter{哲学基本问题}

\section{哲学的对象、目的与方法}


\section{哲学的性质}
    \subsection{阶级性}
    \subsection{实践性}


\section{本体论}
    \subsection{宇宙论}
    \subsection{时空论}


\section{认识论}

    \subsection{理由}
    \subsection{推理错误}

    \subsection{决定论与非决定论}
    \subsection{自我论}


\section{价值论}
    \subsection{唯物主义}
        \subsubsection{朴素唯物主义}
        \subsubsection{形而上学唯物主义}
        \subsubsection{辩证唯物主义}
    \subsection{唯心主义}


    




\chapter{哲学史}





\section{基于历史发展的分类}
按照历史的发展,哲学主要包括4大阶段

\begin{lstlisting}
\subsection{古代哲学}
\subsection{中世纪哲学}
\subsection{近代哲学}
    \subsubsection{十七世纪哲学}
    \subsubsection{十八世纪哲学}
    \subsubsection{十九世纪哲学}
\subsection{现代哲学}
    \subsubsection{二十世纪哲学}
    \subsubsection{二十一世纪哲学}
\end{lstlisting}





\section{中国哲学}
        
\subsection{古代哲学}

\subsection{先秦哲学 (~公元前221年 )}
    \subsubsection{诸子前哲学}
    
    \subsubsection{儒家}
        \paragraph{四书}
        \paragraph{孔子 (孔丘,公元前551~前479年 )}
        \paragraph{孔子弟子}
        \paragraph{子思 (孔伋,公元前483~前402年 )}
        \paragraph{孟子 (孟轲,公元前390~前305年 )}
        \paragraph{荀子 (荀况,公元前313~前238年 )}
        \paragraph{其他}

    \subsubsection{道家}
        \paragraph{老子 (李耳 )}
        \paragraph{列子 (列御寇 )}
        \paragraph{杨朱 (公元前395~前335年 )}
        \paragraph{关尹子}
        \paragraph{庄子 (庄周,公元前369~前286年 )}
        \paragraph{其他}

    \subsubsection{墨家}
    \subsubsection{名家}
        \paragraph{邓析 (公元前?~前501年 )}
        \paragraph{宋尹学派 (宋钘、尹文 )}
        \paragraph{惠施 (公元前370~前310年 )}
        \paragraph{公孙龙 (公元前320~前250年 )}
        \paragraph{其他}

    \subsubsection{法家}
        \paragraph{管子 (管仲,公元前?~前645年 )}
        \paragraph{商鞅 (公孙鞅,公元前?~前338年 )}
        \paragraph{慎到 (公元前395~前315年 )}
        \paragraph{申不害 (公元前385~前337年 )}
        \paragraph{韩非 (公元前280~前233年 )}
        \paragraph{李斯 (公元前?~前208年 )}
        \paragraph{其他}

    \subsubsection{阴阳家}
    \subsubsection{纵横家}
        \paragraph{苏秦}
        \paragraph{张仪}

    \subsubsection{杂家}
        \paragraph{尸子 (尸佼,公元前390~前330年 )}
        \paragraph{吕不韦 (公元前?~前235年 )}
        \paragraph{孔鲋 (孔甲,公元前264~前208年 )}
        \paragraph{其他}


\subsection{秦汉、三国、晋、南北朝哲学 (公元前221年-公元589年 )}
    \subsubsection{秦汉哲学 (总论 ) (公元前221~公元220年 )}
    \subsubsection{秦代哲学 (公元前221~前207年 )}
    \subsubsection{汉代哲学 (公元前206~公元220年 )}

        \paragraph{陆贾}
        \paragraph{贾谊 (公元前200~前168年 )}
        \paragraph{淮南子 (刘安,公元前179~前122年 )}
        \paragraph{董仲舒 (公元前179~前104年 )}
        \paragraph{桓谭 (?-56年 )}
        \paragraph{王充 (27-97年 )}
        \paragraph{王符}
        \paragraph{荀悦 (148-209年 )}
        \paragraph{其他}

    \subsubsection{三国、晋、南北朝哲学 (220-589年 )}
        \paragraph{何晏 (190-249年 )}
        \paragraph{王弼 (226-249年 )}
        \paragraph{嵇康 (224-263年 )}
        \paragraph{杨泉}
        \paragraph{裴頠 (267-300年 )}
        \paragraph{郭象 (252-312年 )}
        \paragraph{抱朴子 (葛洪,约284-364年 )}
        \paragraph{范缜 (约450-510年 )}
        \paragraph{其他}

\subsection{隋、唐、宋、元、明、清哲学 (589年-1840年 )}
    \subsubsection{隋、唐、五代哲学 (581-960年 )}
        \paragraph{文中子 (王通,584-617年 )}
        \paragraph{吕才 (600-665年 )}
        \paragraph{李翱 (772-841年 )}
        \paragraph{刘禹锡 (772-842年 )}
        \paragraph{谭峭}
        \paragraph{其他}

    \subsubsection{宋、元哲学 (960-1368年 )}
        \paragraph{李觏 (1009-1059年 )}
        \paragraph{周敦颐 (濂溪,1017-1073年 )}
        \paragraph{邵雍 (康节,1011-1077年 )}
        \paragraph{张载 (横渠,1020-1077年 )}
        \paragraph{王安石 (1021-1086年 )}
        \paragraph{程颢 (明道,1032-1085年 )、程颐 (伊川,1033-1107年 )及程朱理学}
        \paragraph{朱熹 (1130-1200年 )及考亭学派}
        \paragraph{陆九渊 (象山,1139-1193年 )及其学派}
        \paragraph{陈亮 (1143-1194年 )及永康学派}
        \paragraph{叶适 (1150-1223年 )及永嘉学派}
        \paragraph{其他}

    \subsubsection{明代哲学 (1368-1644年 )}
        \paragraph{陈献章 (1428-1500年 )}
        \paragraph{王守仁 (王阳明,1472-1528年 )及陆王学派}
        \paragraph{王艮 (1483-1541年 )及泰州学派}
        \paragraph{王廷相 (1474-1544年 )}
        \paragraph{罗钦顺 (1465-1547年 )}
        \paragraph{黄绾 (1477-1551年 )}
        \paragraph{何心隐 (1517-1579年 )}
        \paragraph{李贽 (1527-1602年 )}
        \paragraph{吕坤 (1536-1618年 )}
        \paragraph{方以智 (1611-1671年 )}
        \paragraph{其他}

    \subsubsection{清代哲学(1644~1840年)}
        \paragraph{顾炎武 (1613-1682年 )}
        \paragraph{王夫之 (船山,1619-1692年 )}
        \paragraph{黄宗羲 (梨州,1610-1695年 )}
        \paragraph{唐甄 (1630-1704年 )}
        \paragraph{颜元 (习斋,1635-1704年 )、李塨 (恕谷,1659-1733年 )及颜李学派}
        \paragraph{戴震 (东原,1723-1777年 )}
        \paragraph{章学诚 (实斋,1738-1801年 )}
        \paragraph{焦循 (理堂,1763-1820年 )}
    
\subsection{近代哲学 (1840年-1916年 )}
    \paragraph{龚自珍 (1792-1841年 )}
    \paragraph{魏源 (1794-1857年 )}
    \paragraph{谭嗣同 (1865-1898年 )}
    \paragraph{严复 (1853-1921年 )}
    \paragraph{康有为 (1858-1927年 )}
    \paragraph{梁启超 (1873-1929年 )}
    \paragraph{章炳麟 (1869-1936年 )}

\subsection{现代哲学 (1919年- )}
\subsection{二十世纪哲学}
\subsection{二十一世纪哲学}



\section{亚洲}
    
\subsection{朝鲜哲学}
\subsection{日本哲学}
    \subsubsection{古代哲学}
    \subsubsection{封建时代哲学}
    \subsubsection{明治时代哲学}
    \subsubsection{现代哲学}

    
\subsection{印度哲学}
\subsection{越南哲学}
    
\subsubsection{阿拉伯哲学 (总论 )}
\subsubsection{伊朗哲学 (波斯哲学 )}
\subsubsection{以色列哲学}


\subsection{伊斯兰哲学}




\section{欧洲哲学}
\subsection{古代哲学}
    \subsubsection{古希腊哲学 (公元前6~6世纪 )}
    BC6世纪到公元后6世纪。希腊学校不存在,靠基督教统治。

    \paragraph{城邦}
    罗马集成希腊的版图。\\
    分离的个人聚集,靠传统。希腊产生城邦,传统的影响小。繁荣在雅典。城邦民主制。\\
    人之间的影响靠语言。\\
    马其顿打败雅典,导致衰落。\\


    \paragraph{哲学产生的条件}
    有终极关怀问题;有时间思考;思想自由。

    \paragraph{特点}
    百家争鸣。
    人类的共同话语平台:1 )神话,秘密,命运与规律。2 )哲学,神话止步的地方继续追问。
    给自然以合理的解释,诞生哲学。


    \paragraph{早期}
    宇宙存在的合理说明。有生有死,世界循环还存在,什么是不变的?构成世界的东西最根本的是什么?
    【回答】是本原。本原是万物之来,之归,有开端、主宰的意思。原因产生了宇宙,在宇宙中,宇宙无了还在。


    \subparagraph{希腊东部}
    米立都学派,泰勒斯:大地浮在水上。
    【解读】水是万物的本原。水是比喻、象征,以水的性质来体会本原。
    赫拉克里特:宇宙是永恒的活火。
    【解读】更强调“变”。混沌、循环。

    Logus:活火的尺度、规律。其学生:人一次也踏不入。


    \subparagraph{希腊南部}

    本体论转向。
    毕达哥拉斯:数是万物的本原。

    巴门尼德:认识的对象是不动不变的。异见之路和真理之路的区分,引导出“存在”。提出“存在”概念。思维与存在的统一性。哲学命题和逻辑。
    巴门尼德之前:桌子是由木头做的,不解决桌子是桌子的问题,是从时间上、构成材料上研究。
    巴门尼德之后:问桌子的本质。

    如何对自然形成知识。

    \subparagraph{存在}

    核心概念。

    存在:可思想。
    非存在:流变中的现象,不能被叙说,不能被思想。

    词根、词源是印欧语系,S是P。being。汉语的存在有“空间性”。

    语言中什么是不变的。先存在再成为存在物。

    形而上学,与本体论是同义语。本体论:研究存在的。

    
    \paragraph{雅典时期}

    智者:意思是圣贤,职业教师,教修辞学、辩论、诡辩。启蒙,用的怀疑主义。

    同一条河不能面对2个人。

    人是万物的尺度。
    【解读】尺度在每个人那里。一切理论都有对立的。


    \subsubsection{希腊奴隶制形成时期 (公元前7~前6世纪 )}
        \paragraph{爱奥尼亚学派}
        \paragraph{米利都学派}
            \subparagraph{泰勒斯 (Thales,约公元前624~前547年 )}
            \subparagraph{阿那克西曼德 (Anaxi-mandros,约公元前610~前546年 )}
            \subparagraph{阿那克西美尼 (Anaxime-nes,公元前588~前525年 )}
        \paragraph{爱非斯学派}
        \paragraph{毕达哥拉斯学派}
        \paragraph{埃利亚学派}
            \subparagraph{色诺芬尼 (Xenophanes,公元前565~前473年 )}
            \subparagraph{巴门尼德 (Parmenides,公元前6世纪末 )}
            \subparagraph{芝诺 (Zenon,公元前490~前436年 )}
        \paragraph{其他}

    \subsubsection{希腊奴隶主民主制繁荣和衰落时期 (公元前5~前4世纪 )}
        \paragraph{古希腊唯物论哲学学派}
            \subparagraph{阿那克萨哥拉 (Anaxagoras,公元前500~前428年 )}
            \subparagraph{恩培多克勒 (Empedok-les,公元前490~前430年 )}
            \subparagraph{德谟克利特 (Demokritos,约公元前460~前370年 )}
        \paragraph{智者派 (诡辩派 )}
        \paragraph{唯心论哲学学派}
            \subparagraph{苏格拉底 (Sokrates,公元前469~前399年 )}
            苏格拉底时期,只讲雄辩不论对错,否定知识的可能性,导致民主制度的衰落。

            知识。与道德相关的论题。
    
            用构成万物的材料解决自然本原,不对。应追问使万物称为万物的目的。称这目的为“善”。万物有向善的目标。
    
            认识你自己,试图证明别人比自己有知。让人们意识到自己是无知的,得罪很多人。
    
            70岁判死刑,2项罪名:教唆犯;引进外邦精神。
    
            政治理想与雅典制度产生了冲突。治理国家不需要治国的知识吗?救国靠知识上,但衰落不可避免。
    
            认识人自己。认识人心中的“善”,称为德性。车能跑是车的德性。人的德性,需要自觉,才能发出来。德性即知识,无知即罪恶。
    
            从知识角度考虑伦理学。
    
            一个东西是什么。认识事物的类本质。
    
            美是什么?具体的事物是相对的美。一切事物的美,美的事物是什么?美之为美,美本身是什么?
    
            通过思想把握存在。追问本质,用具体的事物回答有问题,从而逐渐用抽象的事物回答。
    
            辩证法:我无知但能引导你有知。
    
            黑格尔辩证法:由工人对话,推导出自身的矛盾运动。
    
            苏格拉底要论出真理。
    
            反击、诱导、归纳 (经验观察从归纳开始。范围内有效,影响实验科学 )、结论。
    
            知识是否可教?知识是先天的,学习是回忆起已经忘记的东西。


            \subparagraph{柏拉图 (Platon,公元前427~前347年 )}

            理念化。

            《理想国》应译作《国家篇》。哲学家做王的国家好。
    
            灵魂理性————统治者国家
            灵魂激情————保卫者国家
            灵魂欲望————生产者国家
    
            4主德:智慧、勇敢、节制、正义。
    
            现象世界:信以为真的,可感的,不能形成知识。
            理念世界:宇宙的目的,本质世界。
    
            洞穴:影像,走出洞穴。
    
            理念,只有灵魂能看到。
    
            Idea

            \subparagraph{亚里士多德 (Aristoteles,公元前384~前322年 )}
        \paragraph{麦加拉学派}
        \paragraph{犬儒学派}
        \paragraph{昔勒尼学派 (克兰尼学派 )}
        \paragraph{其他}


    \subsubsection{希腊奴隶制危机和衰落时期 (公元前336~前30年 )}
        \paragraph{伊壁鸠鲁及其学派}
        \paragraph{斯多亚派 (画廊派 )}
        \paragraph{怀疑论派}
        \paragraph{其他}

    \subsubsection{古罗马哲学}
        \paragraph{唯物论}
        \paragraph{折衷主义}
        \paragraph{新斯多亚派}
        \paragraph{新柏拉图主义}
        \paragraph{其他}


\subsection{中世纪哲学}
\subsubsection{教父哲学}
\subsubsection{经院哲学}
    \paragraph{托马斯·阿奎那 (Thomas,Aquinas,1225-1274年 )}
    \paragraph{安瑟伦 (Anselm,1033-1109年 )}
    \paragraph{邓斯·司各脱 (DunsScotus,Jahahnes,1265-1308年 )}
    \paragraph{奥卡姆 (Occam,Williamof,1300-1350年 )}
\subsubsection{神秘主义}
    \paragraph{爱克哈特 (Eckhart,M.J.约1260-1327年 )}
    \paragraph{亚克利巴 (Agrippa,1486-1535年 )}
    \paragraph{魏格尔 (Weigel,V.1533-1588年 )}
\subsubsection{资本主义产生时期 (文艺复兴时期,14-16世纪 )哲学}
\subsubsection{人文主义}
    \paragraph{佩脱拉克 (Petrurca,F.1304-1374年 )}
    \paragraph{薄伽丘 (Boccaccio,Gio-vanni,1313-1375年 )}
    \paragraph{彭波那齐 (Pomponnazzi,Pietro,1462-1525年 )}
    \paragraph{爱拉斯谟 (Erasmus,Desi-derius,1465-1536年 )}
    \paragraph{蒙台涅 (Montaigne,M.E.de,1533-1592年 )}
    \paragraph{斐未斯 (Vives,Louis,1492-1540年 )}
\subsubsection{科学和自然哲学}
    \paragraph{尼古拉 (库萨的 ) (Nicolaus,Cusanus,1401-1464年 )}
    \paragraph{伽利略 (Galileo,Galilei,1564-1642年 )}
    \paragraph{布鲁诺 (Bruno,Giordano,1548-1600年 )}
    \paragraph{其他}

    

\section{文艺复兴时期哲学:14–16世纪}

\subsection{十七~十九世纪前期哲学}
\subsection{十九世纪后期~二十世纪哲学}
\subsection{二十一世纪哲学}

\subsection{俄罗斯、俄国及苏联 (1917-1991年 )哲学}
\subsubsection{十八世纪及其以前哲学}
    \paragraph{罗蒙诺索夫 (Ломоносов,М.В.1711-1765年 )}
    \paragraph{拉吉舍夫 (Раднщев,А.Н.1749-1802年 )}
    \paragraph{其他}
\subsubsection{十九世纪哲学}
    \paragraph{别林斯基 (Белинcкий,В.Г.1811-1848年 )}
    \paragraph{赫尔岑 (Герцен,А.И.1812-1870年 )}
    \paragraph{奥格辽夫 (Огарёв,Н.П.1813-1877年 )}
    \paragraph{车尔尼雪夫斯基 (Черныщевский,Н.Г.1828-1889年 )}
    \paragraph{杜勃罗留勃夫 (Доброл-юбов,Н.А.1836-1861年 )}
    \paragraph{皮萨列夫 (Писарев,Д.И.1840-1868年 )}
    \paragraph{其他}

\subsubsection{十九世纪后期至二十世纪哲学}
    \paragraph{巴枯宁 (Бакунин,М.А.1814-1876年 )}
    \paragraph{拉甫罗夫 (Лавров,П.Л.1823-1900年 )}
    \paragraph{特卡切夫 (Ткачев,П.Н.1844-1885年 )}
    \paragraph{普列汉诺夫 (Плеханов,Г.В.1856-1918年 )}
    \paragraph{波格丹诺夫 (Богданов,А.А.1873-1928年 )}
    \paragraph{其他}
\subsubsection{二十一世纪哲学}


\subsection{德国哲学}
\subsubsection{十七世纪哲学}
    \paragraph{伯麦 (Bohme,Jakob.1575-1624年 )}
    \paragraph{莱布尼兹 (Leibniz,G.W.1646-1716年 )}
    \paragraph{其他}
\subsubsection{十八世纪~十九世纪前期哲学:德国古典哲学}
    \paragraph{康德 (Kant,I.1724-1804年 )}
    \paragraph{福尔斯特 (Forster,G.1754-1794年 )}
    \paragraph{费希特 (Fichte,J.G.1762-1814年 )}
    \paragraph{谢林 (Schelling,F.W.J.1775-1854年 )}
    \paragraph{黑格尔 (Hegel,G.W.F.1770-1831年 )}
    \paragraph{费尔巴哈 (Feuerbach,L.A.1804-1872年 )}
    \paragraph{其他}

\subsubsection{十九世纪后期哲学}
    \paragraph{叔本华 (Schopenhauer,A.1788-1860年 )}
    \paragraph{赫尔巴特 (Herbart,J.F.1776-1841年 )}
    \paragraph{洛兹 (Lotze,R.H.1817-1881年 )}
    \paragraph{海克尔 (Haeckel,E.H.1834-1919年 )}
    \paragraph{冯德 (Wundt,W.1832-1920年 )}
    \paragraph{哈特曼 (Hartmann,E.1842-1906年 )}
    \paragraph{尼采 (Nietzsche,F.1844-1900年 )}
    \paragraph{其他}

\subsubsection{二十世纪哲学}
    \paragraph{倭铿 (Eucken,R.C.1846-1926年 )}
    \paragraph{胡塞尔 (Husserl,E.1859-1938年 )}
    \paragraph{雅斯贝尔斯 (Jaspers,K.1883-1969年 )}
    \paragraph{海德格尔 (Heidegger,M.1889-1976年 )}
    \paragraph{其他}
\subsubsection{二十一世纪哲学}



\subsection{德意志民主共和国哲学 (1945-1990年 )}
\subsection{德意志联邦共和国哲学 (1945-1990年 )}
\subsection{奥地利哲学}
\subsection{丹麦哲学}
\subsection{阿尔巴尼亚哲学}
\subsection{意大利哲学}
\subsection{西班牙哲学}


\subsection{英国哲学}
\subsubsection{十七、十八世纪哲学}
    \paragraph{培根 (Bacon,F.1561-1626年 )}
    \paragraph{霍布斯 (Hobbes,T.1588-1679年 )}
    \paragraph{克德沃斯 (Cudworth,R.1617-1688年 )}
    \paragraph{洛克 (Locke,J.1632-1704年 )}
    \paragraph{托兰德 (Toland,John,1670-1722年 )}
    \paragraph{柯林斯 (Collins,Anthony,1676-1729年 )}
    \paragraph{贝克莱 (Berkeley,G.1684-1753年 )}
    \paragraph{李德 (Reid,T.1710-1796年 )}
    \paragraph{休谟 (Hume,D.1711-1776年 )}
    \paragraph{其他}

\subsubsection{十九世纪哲学}
    \paragraph{边沁 (Bentham,J.1748-1832年 )}
    \paragraph{约翰·斯图尔特·密尔 (Mill,J.S.1806-1873年 )}
    \paragraph{赫胥黎 (Huxley,T.H.1825-1895年 )}
    \paragraph{格林 (Green,T.H.1836-1882年 )}
    \paragraph{赫伯特·斯宾塞 (Spencer,H.1820-1903年 )}
    \paragraph{布拉德莱 (Bradley,F.H.1846-1924年 )}
    \paragraph{鲍桑葵 (Bosanguet,B.1848-1923年 )}
    \paragraph{其他}

\subsubsection{二十世纪哲学}
    \paragraph{亚历山大 (Alexander,S.1859-1938年 )}
    \paragraph{怀特海 (Whitehead,A.N.1861-1947年 )}
    \paragraph{麦克特 (Mctaggart,J.M.E.1866-1925年 )}
    \paragraph{伯特兰·罗素 (Russell,B.1872-1970年 )}
    \paragraph{穆尔 (Moore,G.E.1873-1958年 )}
    \paragraph{其他}

\subsubsection{二十一世纪哲学}
    \paragraph{马克思主义哲学在英国的传播与发展}
    \paragraph{荷兰哲学}
    \paragraph{斯宾诺莎 (Spinoza,B.1632-1677年 )}
    \paragraph{古林克斯 (Geulincx,A.1625-1669年 )}
    \paragraph{其他}



\subsubsection{法国哲学}
\subsubsection{十七、十八世纪哲学}
    \paragraph{笛卡儿 (Descartes,R.1596-1650年 )}
    \paragraph{伽桑狄 (Gassendi,P.1592-1655年 )}
    \paragraph{帕斯卡 (Pascal,B.1623-1662年 )}
    \paragraph{孟德斯鸠 (Montesquieu,C.L.de S.1689-1755年 )}
    \paragraph{伏尔泰 (Voltaire,1694-1778年 )}
    \paragraph{卢梭 (Rousseau,J.J.1712-1778年 )}
    \paragraph{拉·美特利 (La Mettrie,J.O.de1709-1751年 )}
    \paragraph{狄德罗 (Diderot,D.1713-1784年 )}
    \paragraph{爱尔维修 (Helvétius,C.A.1715-1771年 )}
    \paragraph{孔狄亚克 (Condillac,E.B.de1715-1780年 )}
    \paragraph{达兰贝尔 (D'Alembert,J.LeR.1717-1783年 )}
    \paragraph{霍尔巴赫 (Holbach,P.H.D.1723-1789年 )}
    \paragraph{其他}

\subsubsection{十九世纪哲学}
    \paragraph{库然 (Cousin,V.1792-1867年 )}
    \paragraph{奥古斯特·孔德 (Comte,A.1798-1857年 )}
    \paragraph{其他}
\subsubsection{二十世纪哲学}
    \paragraph{柏格森 (Bergson,H.1859-1941年 )}
    \paragraph{马利丹 (Maritain,J.1882-1973年 )}
    \paragraph{萨特尔 (Sartre,J.P.1905-1980年 )}
    \paragraph{其他}
\subsubsection{二十一世纪哲学}


\section{非洲哲学}
\subsection{北非哲学}


\section{大洋洲哲学}

\section{北美洲哲学}
\subsection{美国哲学}
\subsubsection{十八世纪~十九世纪前期哲学}
    \paragraph{富兰克林 (Franklin,B.1706-1790年 )}
    \paragraph{普利斯特莱 (Priestley,J.1733-1804年 )}
    \paragraph{库柏 (Cooper,T.1759-1840年 )}
    \paragraph{其他}
\subsubsection{十九世纪后期哲学}
    \paragraph{爱默生 (Emerson,R.W.1803-1882年 )}
    \paragraph{哈利斯 (Harris,W.T.1835-1909年 )}
    \paragraph{皮尔斯 (Peirce,C.S.1839-1914年 )}
    \paragraph{詹姆斯 (James,W.1842-1910年 )}
    \paragraph{波温 (Bowne,B.P.1847-1910年 )}
    \paragraph{罗伊斯 (Royce,J.1855-1916年 )}
    \paragraph{其他}
\subsubsection{二十世纪哲学}
    \paragraph{杜威 (Dewey,J.1859-1952年 )}
    \paragraph{桑塔亚那 (Santayana,G.1863-1952年 )}
    \paragraph{其他}
\subsubsection{二十一世纪哲学}

    
\section{当代哲学}
    \subsection{欧陆哲学}
    \subsection{东方哲学}




\chapter{哲学学派}
  

    \section{女权主义哲学}
    \section{无政府主义}
    \section{自由主义}


    \section{唯心主义}
        \subsubsection{形而上学}
        \subsubsection{唯心主义认识论、先验论}

    \section{实证论、经验批判主义 (马赫主义 )}
    \section{唯意志论、生命哲学}
    \section{新康德主义、新黑格尔主义}
    \section{新实在论、逻辑实证论 (新实证论、逻辑经验论 )}



\section{存在主义 (生存主义 )}

    \subsection{海德格尔,1889-1976,德国}




    \section{实用主义}
    \section{新托马斯主义 (新经院哲学 )}
    \section{其他哲学流派}
        \subsubsection{西方马克思主义}
        \subsubsection{哲学解释学}
        \subsubsection{哲学人类学}
    

\chapter{现代哲学}
    \subsection{分析哲学}




    \subsection{生存哲学}
    \subsection{人文哲学}
    \subsection{解释学}
    \subsection{符号学}
    \subsection{实用主义哲学}
    

    \chapter{哲学类别1}

    \section{现象学}
    \section{元哲学}
    \section{过程哲学}
    \section{死亡哲学}
    \section{人生哲学}
    \section{法律哲学}
    \section{心灵哲学}
    \section{比较哲学}
    \section{知识论}


    
\chapter{思维科学}
\section{方法论}
    \subsection{思维科学理论与方法论}
    \subsection{思维科学与其他科学的关系}
\section{思维规律}
\section{思维方式}
    \subsection{抽象思维}
    \subsection{形象思维}
    \subsection{灵感思维}
    \subsection{创造性思维}



\chapter{逻辑学}
* B81-0 逻辑学理论与方法论
* B81-05 逻辑学与其他学科的关系
* B81-06 逻辑学流派及其研究
* B81-09 逻辑学史、逻辑思想史
  * B81-092 中国逻辑学史
      * B81-093.51 印度因明学史
      * B81-093.71 阿拉伯逻辑学史
  * B81-095 西方逻辑学史


\section{辩证逻辑}
    \subsection{基础理论}
        \subsubsection{黑格尔的逻辑}
        \subsubsection{马克思主义的辩证逻辑}
        \subsubsection{客观与主观逻辑}
        \subsubsection{知性与理性逻辑}
        \subsubsection{辩证思维}
    \subsection{基本规律}
    \subsection{思维形式与方法}
        \subsubsection{概念、范畴}
        \subsubsection{辩证判断}
        \subsubsection{辩证推理}
        \subsubsection{辩证思维方法}

\section{形式逻辑 (名学、辩学 )}
    \subsection{基本规律}
    \subsection{思维形式和方法}
        \subsubsection{概念、范畴}
        \subsubsection{判断、命题}
        \subsubsection{演绎推理、三段论推理}
    \subsection{归纳推理 (归纳法 )}
    \subsection{论证与反驳}
    \subsection{谬误与诡辩}
\section{数理逻辑 (符号逻辑 )}
\section{概率逻辑}
\section{哲理逻辑 (非经典逻辑 )}
    \subsubsection{模态逻辑}
    \subsubsection{多值逻辑}
    \subsubsection{认知逻辑}
    \subsubsection{价值逻辑}
    \subsubsection{时态逻辑}
    \subsubsection{模糊逻辑}
    \subsubsection{相干与衍推逻辑}
    \subsubsection{自由逻辑}
    \subsubsection{其他}
\section{应用逻辑}





\chapter{伦理学}
\section{伦理学理论与方法论}
    \subsection{伦理学的哲学基础}
    \subsection{伦理学与其他科学的关系}
\begin{lstlisting}
    \subsubsection{道德与政治、道德与法制}
    \subsubsection{道德与社会}
    \subsubsection{道德与经济}
    \subsubsection{道德与心理}
    \subsubsection{道德与美学}
    \subsubsection{道德与宗教}
    \subsubsection{道德与语言}
    \subsubsection{道德与文艺}
    \subsubsection{道德与科学技术}
    \subsubsection{道德与环境}
    \subsubsection{其他}
\end{lstlisting}

    \subsection{伦理学流派及其研究}
        \subsubsection{人道主义}
        \subsubsection{享乐主义、利己主义}
        \subsubsection{人格主义}
        \subsubsection{功利主义、实用主义}
        \subsubsection{直观主义 (直觉主义 )}
        \subsubsection{元伦理学 (分析伦理学 )}
        \subsubsection{描述伦理,价值理论}
        \subsubsection{规范伦理学}
        \subsubsection{其他}
    \subsubsection{伦理学史}

    \subsection{应用伦理}
        \subsubsection{动物权益}
        \subsubsection{环境伦理}
        \subsubsection{医学伦理}
        \subsubsection{教育伦理}
        \subsubsection{政治伦理}
        \subsubsection{家庭伦理}
        \subsubsection{生命伦理学}
        \subsubsection{生态伦理}


\section{人生观、人生哲学}
    特定主义的人生观。
    
\section{道德}
    \subsection{国民公德}
    \subsection{集体公德}
    \subsection{职业道德 (工作道德 )}
        \subsubsection{专业工作道德}

    \subsection{家庭道德}
    \subsection{婚姻道德}
    \subsection{恋爱道德}
    \subsection{性道德、生育道德}

\section{社会公德}
    \subsection{友谊与同志关系}
    \subsection{公共秩序及纪律}
    \subsection{社会风尚}

\section{个人修养}
\section{其他伦理规范}





\chapter{美学}

\section{美学理论}
\subsection{美学哲学基础}
\subsection{美学与其他学科的关系}
\subsection{美学流派及其研究}
    \subsubsection{抽象主义美学、印象主义美学、形式主义美学}
    \subsubsection{唯美主义美学、相对主义美学}
    \subsubsection{结构主义美学}
    \subsubsection{自然主义美学}
    \subsubsection{其他}

\section{美学史}

\section{美学与社会生产}
    \subsubsection{技术美学}
    \subsubsection{生产场所美学}
    \subsubsection{产品美学}
    \subsubsection{劳动美学}
\section{美学与现实社会生活}
    \subsubsection{城市美学}
    \subsubsection{生活美学}
    \subsubsection{社会美学}

\section{艺术美学}




\chapter{形而上学}
    \subsection{行动哲学}
    \subsection{决定论和自由意志}
    \subsection{本体}
    \subsection{时空哲学}
    \subsection{目的论}
    \subsection{有神论和无神论}






%%马克思主义哲学
\chapter{马克思主义哲学}

\section{辩证唯物主义}

\subsection{物质论}
    \subsubsection{运动论}
    \subsubsection{时空论}
    \subsubsection{物质运动的规律性}

\subsection{意识论}
    \subsubsection{客观规律性与主观能动性}

\subsection{认识论、反映论}
    \subsubsection{认识的辩证过程}
    \subsubsection{真理论}

\subsection{唯物辩证法}
    \subsubsection{内因与外因}
    \subsubsection{矛盾的普遍性和特殊性}
    \subsubsection{矛盾发展的不平衡性}
    \subsubsection{矛盾的同一性和斗争性}
    \subsubsection{对抗性矛盾与非对抗性矛盾}
    \subsubsection{质变与量变}
    \subsubsection{否定之否定}

\subsection{唯物辩证法诸范畴}
    \subsubsection{现象与本质}
    \subsubsection{形式与内容}
    \subsubsection{全局与局部}
    \subsubsection{分析与综合}
    \subsubsection{原因与结果 (因果论 )}
    \subsubsection{必然性与偶然性}
    \subsubsection{可能性与现实性}
    \subsubsection{其他哲学范畴}

\subsection{方法论}
\subsection{辩证唯物主义的应用}
\subsection{自然哲学}
\subsection{自然辩证法}





\section{历史唯物主义}

\subsection{社会物质生活条件}

\subsection{社会基本矛盾}
    \subsubsection{生产力和生产关系}
    \subsubsection{经济基础和上层建筑}

\subsubsection{阶级理论}
\subsubsection{革命理论}
\subsubsection{国家理论}
\subsubsection{社会存在与社会意识}
\subsubsection{人民内部矛盾}
\subsubsection{人民在历史发展中的作用}




\section{马克思主义哲学史}





\subsection{马克思、恩格斯}
    \subsubsection{马克思主义形成时期 (1848年以前 )}
    \subsubsection{革命风暴的高涨与低落时期 (1848-1863年 )}
    \subsubsection{第一国际和巴黎公社时期 (1864-1872年 )}
    \subsubsection{马克思主义广泛传播和各国建立社会主义政党时期 (1873-1889年6月 )}
    \subsubsection{第二国际时期 (1889年7月-1895年 )}




\subsection{列宁}
    \subsubsection{俄国社会民主工党形成和布尔什维克派出现时期 (1904年以前 )}
    \subsubsection{第一次俄国革命时期 (1905-1907年 )}
    \subsubsection{斯托雷平反动时期和布尔什维克形成独立政党时期 (1908-1912年3月 )}
    \subsubsection{第一次世界大战以前工人运动的高涨及大战时期 (1912年4月-1916年 )}
    \subsubsection{第二次俄国革命和社会主义革命时期 (1917年 )}
    \subsubsection{帝国主义武装干涉和国内战争时期 (1918-1920年 )}
    \subsubsection{国民经济恢复时期 (1921-1924年 )}


\subsection{斯大林}
    \subsubsection{十月社会主义革命及其以前 (1917年及其以前 )}
    \subsubsection{帝国主义武装干涉和国内战争时期 (1918-1920年 )}
    \subsubsection{国民经济恢复时期 (1921-1925年 )}
    \subsubsection{为实现国家工业化而斗争时期 (1926-1929年 )}
    \subsubsection{为实现农业集体化而斗争时期 (1930-1934年 )}
    \subsubsection{社会主义建设时期 (1935-1941年5月 )}
    \subsubsection{苏联卫国战争时期 (1941年6月-1945年 )}
    \subsubsection{战后恢复和发展社会主义经济时期 (1946-1953年 )}


\subsection{斯大林}
    \subsubsection{第一次国内革命战争以前 (1924年以前 )}
    \subsubsection{第一次国内革命战争时期 (1924-1927年7月 )}
    \subsubsection{第二次国内革命战争时期 (1927年8月-1937年6月 )}
    \subsubsection{抗日战争时期 (1937年7月-1945年8月 )}
    \subsubsection{第三次国内革命战争时期 (1945年9月-1949年9月 )}
    \subsubsection{社会主义革命和社会主义建设时期 (1949年10月-1976年 )}



\subsection{毛泽东}
    \subsubsection{第一次国内革命战争以前 (1924年以前 )}
    \subsubsection{第一次国内革命战争时期 (1924-1927年7月 )}
    \subsubsection{第二次国内革命战争时期 (1927年8月-1937年6月 )}
    \subsubsection{抗日战争时期 (1937年7月-1945年8月 )}
    \subsubsection{第三次国内革命战争时期 (1945年9月-1949年9月 )}
    \subsubsection{社会主义革命和社会主义建设时期 (1949年10月-1976年 )}



\subsection{邓小平}







\chapter{应用哲学}
    \subsection{经济学哲学}
    \subsection{教育理念}
    \subsection{工程哲学}
    \subsection{科学哲学}

    做1个科学问题,要结合行业、学术和个人的现状,分析、判断、研讨该问题的前途、生命力、是否有一定质量和难度。如结构优化在于减重和提高性能,可以一直做下去。原则是从工程中来,到工程中去。可以是从0到1的开拓,但更多的是从1到0的机理研究,即从应用、现象中抽象出理论问题,然后探究科学规律。最终的成果需要是定量、定位、是基础研究 (有机理的探究 )。

    多场:热力、热化。

    制造时考虑变形。

    均匀性:压力分布->树脂流动。




    \subsection{政治哲学}
    \subsection{历史哲学}
    \subsection{语言哲学}
    \subsection{法哲学}
    \subsection{数学哲学}
    \subsection{音乐哲学}
    \subsection{心理学哲学}
    \subsection{宗教哲学}
    \subsection{自然科学哲学}
    \subsubsection{生物学哲学}
    \subsubsection{化学哲学}
    \subsubsection{物理学哲学}
    \subsection{社会科学哲学}
    \subsection{技术哲学}
    \subsection{系统理念}











\end{document}
