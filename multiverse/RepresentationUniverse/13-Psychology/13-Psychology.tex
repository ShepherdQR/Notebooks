%%============
%%  ** Author: Shepherd Qirong
%%  ** Date: 2022-06-05 14:55:53
%%  ** Github: https://github.com/ShepherdQR
%%  ** LastEditors: Qirong ZHANG
%%  ** LastEditTime: 2024-12-01 21:25:47
%%  ** Copyright (c) 2019--20xx Shepherd Qirong. All rights reserved.
%%============


\documentclass[UTF8]{../RepresentationUniverse}
\begin{document}

\title{13-Psychology}
\date{Created on 20220605.\\   Last modified on \today.}
\maketitle
\tableofcontents


\chapter{Introduction}


\chapter{心理学理论}
\section{心理学与其他学科的关系}
\section{心理学派别及其研究}
    \subsubsection{构造学派}
    \subsubsection{机能主义派}
    \subsubsection{行为主义派}
    \subsubsection{格式塔心理学派 (完形派 )}
    \subsubsection{精神分析学派}
    \subsubsection{存在主义心理学派}
    \subsubsection{人本主义学派}
    \subsubsection{其他}

\section{心理学史}

\chapter{心理研究方法}
\subsubsection{电生理技术}
\subsubsection{数理心理学、心理统计法}
\subsubsection{条件反射研究法}
\subsubsection{实验法}
\subsubsection{观察法}
\subsubsection{心理测验}

\chapter{心理过程与心理状态}
\subsubsection{认知}
\subsubsection{感觉与知觉}
\subsubsection{学习与记忆}
\subsubsection{表象与想象}
\subsubsection{言语与思维}
\subsubsection{情绪与情感}
\subsubsection{意识与潜意识}

\chapter{发生心理学}
\subsubsection{比较心理学}
\subsubsection{动物心理学}
\subsubsection{原始人类心理学}
\subsubsection{心理与遗传}
\subsubsection{其他}

\chapter{发展心理学 (人类心理学 )}
\section{儿童心理学}
    \subsubsection{胎儿、新生儿心理学}
    \subsubsection{幼儿心理学}
    \subsubsection{智力超常儿童心理学}
    \subsubsection{变态儿童心理学}
\section{青少年心理学}
\section{成年人心理学}
\section{老年人心理学}
\section{女性心理学}
\section{男性心理学}
\section{种族心理学}


\chapter{生理心理学}
    \section{神经心理}
    \section{感官生理心理}
    \section{内分泌与心理}
    \section{精神药物与心理}
    \section{神经化学与心理}
    \section{环境与生理心理}
        \subsubsection{居住环境与心理}
        \subsubsection{建筑、音响、照明与心理}
        \subsubsection{生态环境与心理}
        \subsubsection{特殊环境与心理}
        \subsubsection{灾害、事故、伤害与心理}
    \section{其他}

\chapter{变态、病态、超意识心理学}

\chapter{个性心理 (人格心理学 )}
\subsubsection{神经类型与气质}
\subsubsection{能力与才能}
\subsubsection{兴趣、态度}
\subsubsection{信念、意志、行为}
\subsubsection{智力、智慧}
\subsubsection{性格}
\subsubsection{个别差异}
\subsubsection{其他}
\chapter{应用心理学}

\end{document}

