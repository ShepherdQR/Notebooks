%%============
%%  ** Author: Qirong ZHANG
%%  ** Date: 2023-04-12 21:40:10
%%  ** Github: https://github.com/ShepherdQR
%%  ** LastEditors: Qirong ZHANG
%%  ** LastEditTime: 2024-12-28 20:19:42
%%  ** Copyright (c) 2019 Qirong ZHANG. All rights reserved.
%%  ** SPDX-License-Identifier: LGPL-3.0-or-later.
%%============



\documentclass[UTF8]{../../RepresentationUniverse}
\begin{document}

\title{04-01-ChineseHistory}
\date{Created on 20230412.\\   Last modified on \today.}
\maketitle
\tableofcontents


\chapter{Introduction}

中国历史

\section{书籍}

\begin{lstlisting}

入门:
    中华史纲
    简明中国历史读本
    商务印书馆的《中国历代大事年表》
    社科院的《简明中国历史知识手册》《中国历史年表》
    2016年央视制作那版《中国通史》的纪录片,对应书:中国通史【大字版五册套装】
    叶龙整理的钱穆先生讲中国史的讲稿,也叫《中国通史》
    台湾傅乐成的那版《中国通史》

高分:
《国史大纲》,钱穆,豆瓣9.3分
《中国历代政治得失》,钱穆,豆瓣9.2分
    吕思勉的《中国通史》
    
丛书:
    吕思勉先生的断代史五本(先秦史、秦汉史、魏晋南北朝史、隋唐五代史、三国史话)
    
    讲谈社《中国的历史》一共十卷(神话时代与夏、商周与春秋战国、秦汉、后汉三国、魏晋南北朝、隋唐、宋、辽夏金元、明清、末代王朝与近代中国)

    《哈佛中国史》六本(秦汉、南北朝、唐、宋、元明、大清)

    《剑桥中国史》中文版目前11本(秦汉、隋唐上、西夏金元、明【上下】、晚清【上下】、民国【上下】、新中国【两册】)

夏商周
《西周史》,杨宽,豆瓣9.0分
《先秦史》,吕思勉,豆瓣8.8分
《商文明》,张光直,豆瓣9.0分
《夏商西周的社会变迁》,晁福林,豆瓣8.1分
《中国上古史研究讲义》,顾颉刚,豆瓣9.1分


春秋战国
《春秋史》,童书业,豆瓣9.2分
《战国史》,杨宽,豆瓣9.1分


秦汉魏晋南北朝
《波峰与波谷》,阎步克,豆瓣8.8分,副标题“秦汉魏晋南北朝的政治文明”;
《魏晋南北朝史讲演录》,陈寅恪,豆瓣9.0分
《两晋南北朝史》,王仲荦,豆瓣9.3分
《北魏平城时代》,李凭,豆瓣8.7分


隋唐五代
《隋唐五代史》,王仲荤,豆瓣9.2分
《绚烂的世界帝国:隋唐时代》,豆瓣8.2分


宋元
《细说宋朝》,虞云国,豆瓣9.2分
《东京梦华录》,孟元老,豆瓣9.1分
《经略幽燕-宋辽军事灾难的战略分析》,曾瑞龙,豆瓣8.8分
《草原帝国》,格鲁塞,豆瓣8.7分


明清
《明朝那些事儿》,当年明月,豆瓣9.2分
《万历十五年》,黄仁宇,豆瓣8.9分
《天朝的崩溃:鸦片战争再研究》,茅海建,豆瓣9.1分
《叫魂:1768年中国妖术大恐慌》,孔飞力,豆瓣9.0分


中国近代史
《近代中国社会的新陈代谢》,陈旭麓,豆瓣9.1分
《中国近代史》,蒋延黻,豆瓣8.8分。经典。
《开放中的变迁》,金观涛,豆瓣9.2分
郭廷以《近代中国史纲》


\end{lstlisting}





\chapter{通史}
\section{革命史}
\section{文化史}
\section{古代史籍}
    \subsubsection{史记}
\section{历史事件}
\section{史料}
    \subsubsection{诏令 、奏议}
    \subsubsection{笔记、掌故、回忆录}
\section{历史研究、考订、评论}
\section{年表}
\section{中国历史普及读物}


\chapter{原始社会}
\chapter{奴隶社会 (公元前21世纪-公元前475年 )}
\chapter{封建社会 (公元前475年-公元581年 )}
\chapter{隋唐至清前朝 (581年-1840年 )}
    \subsubsection{隋朝}
    \subsubsection{唐朝}
    \subsubsection{五代、十国}
    \subsubsection{北宋}
    \subsubsection{南宋}
    \subsubsection{辽、金}
    \subsubsection{元朝}
    \subsubsection{明朝}
    \subsubsection{清朝}
\chapter{半殖民地半封建社会 (1840年-1949年 )}
    \subsubsection{中华民国早期}
\chapter{新民主主义革命时期 (1919年-1949年 )}
    \subsubsection{五四运动和中国共产党成立后 (1919年-1924年 )}
    \subsubsection{第一次国内革命战争时期 (1924年-1927年 )}
    \subsubsection{第二次国内革命战争 (土地革命战争 )时期 (1927年-1937年 )}
    \subsubsection{日本帝国主义入侵及全国抗日民主运动}
    \subsubsection{抗日战争时期}
    \subsubsection{第三次国内革命 (国共内战 )时期 (1945年-1949年 )}
    \subsubsection{解放区的革命建设和发展}
\chapter{中华人民共和国:社会主义革命和社会主义建设时期 (1949年- )}
\chapter{民族史志}
\chapter{地方史志}
    \subsubsection{香港}
    \subsubsection{澳门}
    \subsubsection{台湾}







\end{document}

