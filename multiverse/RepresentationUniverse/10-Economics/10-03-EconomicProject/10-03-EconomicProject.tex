%%============
%%  ** Author: Shepherd Qirong
%%  ** Date: 2022-06-05 14:55:53
%%  ** Github: https://github.com/ShepherdQR
%%  ** LastEditors: Qirong ZHANG
%%  ** LastEditTime: 2024-12-22 14:47:32
%%  ** Copyright (c) 2019--20xx Shepherd Qirong. All rights reserved.
%%============


\documentclass[UTF8]{../../RepresentationUniverse}
\begin{document}

\title{10-03-EconomicProject}
\date{Created on 20241222.\\   Last modified on \today.}
\maketitle
\tableofcontents


\chapter{Introduction}

经济计划与管理


\chapter{国民经济管理}
\subsubsection{经济预测}
\subsubsection{经济决策}
\subsubsection{生产行业管理}
\subsubsection{工商行政管理}
\subsubsection{科学技术管理}
\subsubsection{资源、环境和生态管理}
\subsubsection{能源管理}
\subsubsection{生产布局和区域经济管理}
\subsubsection{经济信息管理}



\chapter{经济计划}
\section{国民经济计划原理}
\section{国民经济计划体系}
\section{各种专门计划}
    \subsubsection{综合生产计划}
    \subsubsection{基本建设计划}
    \subsubsection{价格成本与流通费用计划}
    \subsubsection{国民收入计划}
    \subsubsection{部门经济计划}



\chapter{经济计算、经济数学方法}
\section{经济核算}
\section{经济统计学}
    \subsection{经济统计方法}
    \subsection{专门经济统计}
        \subsubsection{部门经济统计}
        \subsubsection{劳动统计}
        \subsubsection{国民经济计算体系}
        \subsubsection{人民生活统计}
        \subsubsection{其他专门经济统计}
    \subsection{国际经济统计}
    \subsection{经济统计组织与工作}

\section{投入产出分析}
\section{经济数学方法}
    \subsection{经济研究方法、工作方法}
        \subsubsection{电子计算机的应用}
    \subsection{数量经济学}
    \subsection{经济控制论、系统论、信息论}
        \subsubsection{经济控制论}
        \subsubsection{经济系统分析}
        \subsubsection{经济信息理论}
    \subsection{运筹学在经济中的应用}
        \subsubsection{线性规划}
        \subsubsection{博弈论}
        \subsubsection{网络理论、统筹法}
        \subsubsection{排队论}
    \subsection{费用效益分析 (成本-效益分析 )}
    \subsection{概率论与数理统计在经济中的应用}
    \subsection{经济数学方法的应用}




\chapter{会计}
\section{会计学}
    \subsubsection{会计数学}
\section{会计簿记方法}
    \subsubsection{资产负债表}
    \subsubsection{复式记账和账户}
    \subsubsection{会计凭证和财产清查}
    \subsubsection{帐簿和记帐技术}
    \subsubsection{会计报表}
    \subsubsection{会计检查和监督}
\section{会计设备}
\section{会计工作组织与制度}
\section{各种会计和簿记}
    \subsubsection{社会会计}
    \subsubsection{成本会计}
    \subsubsection{管理会计}
    \subsubsection{财务会计}
    \subsubsection{国际会计}
\section{各部门会计和簿记}
    \subsection{国家机关会计 (政府会计 )、预算会计}
    \subsection{企业会计}
    \subsection{金融、保险业会计}
    \subsection{基本建设、物资、施工企业会计}
    \subsection{农业会计}
    \subsection{工业会计}
    \subsection{交通运输业会计}
    \subsection{旅游业会计}
    \subsection{邮电业会计}
    \subsection{商业会计、外贸会计}
    \subsection{房地产开发企业会计}
    \subsection{文化、电影、新闻出版企业会计}


\chapter{审计}
\subsection{审计学}
\subsection{审计方法与技术}
\subsection{审计工作组织与制度}
    \subsubsection{世界}
    \subsubsection{中国}
        \paragraph{政策}
        \paragraph{制度}
        \paragraph{组织机构}
        \paragraph{地方审计业务}
\subsection{各类审计}
    \subsubsection{财政财务审计、财经法纪审计}
    \subsubsection{经济效益审计}
    \subsubsection{民间审计}
    \subsubsection{国家审计}
    \subsubsection{内部审计}
    \subsubsection{经济责任审计}
\subsection{专业审计}
    \subsubsection{农业审计}
    \subsubsection{工业、交通、邮电、旅游审计}
    \subsubsection{基本建设审计、物资审计、水利审计}
    \subsubsection{商业、外贸审计}
    \subsubsection{财政、金融审计}
    \subsubsection{行政事业审计}
    \subsubsection{外资审计}






\chapter{劳动经济}
\section{劳动经济理论}
\section{劳动力}
    \subsubsection{劳动力计划}
    \subsubsection{劳动力市场}
    \subsubsection{劳动力需求}
    \subsubsection{劳动力市场类型}
    \subsubsection{劳动制度}
    \subsubsection{劳动体制}
    \subsubsection{用工制度}
    \subsubsection{培训制度}
    \subsubsection{退休制度}
    \subsubsection{劳动就业与失业}
\section{劳动生产率}
\section{劳动组织和管理}
    \subsubsection{劳动组织}
    \subsubsection{劳动分工与协作}
    \subsubsection{劳动定额}
    \subsubsection{劳动竞赛}
    \subsubsection{劳动纪律、生产责任制}
\section{劳动工资、劳动报酬}
    \subsubsection{工资形式}
    \subsubsection{工资制度}
    \subsubsection{奖励制度}
\section{劳动工时}
\section{劳动关系}
\section{劳动保护}
\section{世界各国劳动经济概况}
    \subsection{世界}
        \subsubsection{发展中国家}
        \subsubsection{发达国家}
        \subsubsection{社会主义国家}
        \subsubsection{资本主义国家}
        \subsubsection{劳动经济史}
    \subsection{中国}
        \subsubsection{政策}
        \subsubsection{劳动力}
        \subsubsection{劳动生产率}
        \subsubsection{劳动组织}
        \subsubsection{劳动工资}
        \subsubsection{劳动工时}
        \subsubsection{劳动关系}
        \subsubsection{地方劳动经济}
        \subsubsection{劳动经济史} 





\chapter{物资经济}
\section{物资经济理论}
\section{物资管理}
    \subsubsection{物资管理体制}
    \subsubsection{物资计划}
    \subsubsection{物资统计}
\section{物资流通}
    \subsection{物资流通体制}
    \subsection{物资市场}
        \subsubsection{物资供应与需求}
        \subsubsection{生产资料贸易组织}
        \subsubsection{城乡交流}
        \subsubsection{物资流通专业化}
    \subsection{物资流通费用、资金和利润}
    \subsection{物资价格}
    \subsection{物资流通经济效益}
    \subsection{各类物资流通}
        \subsubsection{原材料流通}
        \subsubsection{设备流通}

\section{物资企业经营与管理}
    \subsubsection{计划管理}
    \subsubsection{采购管理}
    \subsubsection{质量管理}
    \subsubsection{库存、储备及调运管理}
    \subsubsection{定额消耗与节约}
    \subsubsection{财务管理}
    \subsubsection{物资经济现代化管理}

\section{世界各国物资经济}
    \subsection{世界}
    \subsection{中国}
        \subsubsection{物资计划与管理体制}
        \subsubsection{物资流通}
        \subsubsection{物资企业经营与管理}
        \subsubsection{地方物资经济}
        \subsubsection{物资经济史}






\chapter{企业经济}
\section{企业经济理论和方法}
    \subsubsection{企业经济与其他科学的关系}
    \subsubsection{企业经济效益}
    \subsubsection{企业管理的数学方法}
    \subsubsection{企业现代化管理}
\section{企业体制}
\section{企业计划与经营决策}
    \subsection{预测}
    \subsection{计划}
    \subsection{经营决策}
    \subsection{经济评价}
    \subsection{企业行政管理}
        \subsubsection{企业领导}
        \subsubsection{人事管理}
        \subsubsection{民主管理}
\section{企业生产管理}
    \subsubsection{企业技术管理}
    \subsubsection{产品管理}
    \subsubsection{企业资产管理}
    \subsubsection{企业 (行业 )间联系}
\section{企业供销管理}
\section{企业财务管理}
    \subsubsection{企业资金管理}
    \subsubsection{企业会计核算}
    \subsubsection{企业成本管理}
    \subsubsection{利润与收入分配}
    \subsubsection{经济活动分析}
\section{各种企业经济}
    \subsection{国有企业经济}
    \subsection{合作经济、合作社}
    \subsection{中小型企业、乡镇企业}
    \subsection{联合企业经济}
        \subsubsection{部门间经济综合体}
        \subsubsection{科研、生产联合企业}
        \subsubsection{合资经营企业}
        \subsubsection{高新技术企业}
    \subsection{私营企业}
    \subsection{公司}
    \subsection{跨国公司}
    \subsection{垄断组织}
\subsubsection{世界各国企业经济}
\subsubsection{世界}
\subsubsection{发展中国家}
\subsubsection{发达国家}
\subsubsection{社会主义国家}
\subsubsection{资本主义国家}
\subsubsection{中国}
\subsubsection{企业组织与体制}
\subsubsection{企业管理}
\subsubsection{各种企业经济}
\subsubsection{个别企业经济}
\subsubsection{地方企业经济}
\subsubsection{企业史}
      * F279.3/.7 各国






\chapter{基本建设经济}
    \section{基本建设理论与方法}
    \section{基本建设计划与管理}
    \section{基本建设研究与决策}
    \section{基本建设投资与经济效益}
    \section{基本建设组织与管理}
    \section{基本建设财务 (建设单位财务 )}
    \section{各部门基本建设经济}
        \subsubsection{农林建设}
        \subsubsection{工业建设}
        \subsubsection{交通运输建设}
        \subsubsection{商业、服务业建设}
        \subsubsection{文教事业建设}
        \subsubsection{科研事业建设}
        \subsubsection{公用事业建设}
        \subsubsection{住宅建设}




\chapter{城市与市政经济}
\section{城市经济理论}
\section{城镇的形成与发展}
    \subsubsection{城市发展道路、城市化}
    \subsubsection{城乡经济联系}
\section{城镇规划与建设}
\section{城市经济管理}
    \subsection{城市经济结构}
    \subsection{城市土地开发与利用}
    \subsection{房地产经济}
        \subsubsection{房地产经济理论}
        \subsubsection{房地产制度}
        \subsubsection{房地产管理}
        \subsubsection{房地产市场}
\section{基础设施、公用事业建设与管理}
    \subsubsection{公用事业}
    \subsubsection{动力设施}
    \subsubsection{交通运输与邮电设施}
    \subsubsection{其他市政设施}
\section{世界各国城市市政经济概况}
    \subsection{世界}
        \subsubsection{发展中国家}
        \subsubsection{发达国家}
        \subsubsection{社会主义国家}
        \subsubsection{资本主义国家}
    \subsection{中国}
        \subsubsection{城镇形成与发展}
        \subsubsection{城镇规划与建设}
        \subsubsection{城市经济管理}
        \subsubsection{基础设施与公用事业}
        \subsubsection{地方城市经济}
        \subsubsection{城市经济史} 




\chapter{END}




\end{document}

