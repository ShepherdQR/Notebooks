%%============
%%  ** Author: Qirong ZHANG
%%  ** Date: 2024-12-22 12:45:14
%%  ** Github: https://github.com/ShepherdQR
%%  ** LastEditors: Qirong ZHANG
%%  ** LastEditTime: 2024-12-22 13:01:11
%%  ** Copyright (c) 2019 Qirong ZHANG. All rights reserved.
%%  ** SPDX-License-Identifier: LGPL-3.0-or-later.
%%============



\documentclass[UTF8]{../../RepresentationUniverse}
\begin{document}

\title{10-01-EconomicsTheory}
\date{Created on 20241222.\\   Last modified on \today.}
\maketitle
\tableofcontents


\chapter{Introduction}

 

\chapter{马克思主义政治经济学 (总论 )}


\chapter{经济学基本问题}
\section{经济学的对象和方法}
\section{经济规律}
\section{经济范畴}
    \subsection{生产、生产力、生产关系、生产方式}
    \subsection{劳动、劳动生产率、劳动分工}
    \subsection{商品生产与交换}
        \subsubsection{价格理论}
        \subsubsection{需求理论、供给理论}
        \subsubsection{经济效益}
        \subsubsection{经济机制}
        \subsubsection{资本和剩余价值}
    \subsection{国民收入与分配}
    \subsection{消费与积累}
    \subsection{社会再生产}
    \subsection{其他经济范畴}
\section{宏观经济学}
\section{微观经济学}
\section{其他经济理论}
    \subsubsection{均衡理论}
    \subsubsection{静态经济学、动态经济学}
    \subsubsection{规范经济学、实证经济学}
    \subsubsection{合理预期}
    \subsubsection{经济政策理论}




\chapter{前资本主义社会生产方式}
    \subsubsection{原始社会}
    \subsubsection{奴隶社会}
    \subsubsection{封建社会}




\chapter{资本主义社会生产方式}
\section{生产关系、所有制}
\section{商品生产与交换}
    \subsubsection{商品、商品生产}
    \subsubsection{货币}
    \subsubsection{价值、价值规律}
    \subsubsection{价格}
    \subsubsection{经济效益}
    \subsubsection{经济机制}
\section{资本和剩余价值}
    \subsubsection{资本}
    \subsubsection{剩余价值及其分配}
\section{雇佣劳动和工资}
\section{资本积累和无产阶级贫困化}
\section{社会资本再生产}
    \subsubsection{社会生产、再生产}
    \subsubsection{部门间的关系}
    \subsubsection{社会总产品、国民生产总产值}
    \subsubsection{计划和市场}
\section{国民收入和分配}
    \subsubsection{国民财富}
    \subsubsection{消费与积累}
    \subsubsection{生活方式}
\section{经济循环}
    \subsubsection{经济周期与经济波动}
    \subsubsection{经济停滞、衰退与复苏}
    \subsubsection{景气预测}
\section{垄断资本主义-帝国主义}
    \subsubsection{金融资本和金融寡头}
    \subsubsection{垄断与竞争}
    \subsubsection{资本输出}
    \subsubsection{国家垄断资本}
    \subsubsection{经济军事化}
    \subsubsection{帝国主义对殖民地的经济掠夺}
    \subsubsection{“后工业社会”论}
\section{资本主义经济危机}




\chapter{社会主义社会生产方式}
\section{从资本主义到社会主义的过渡}
    \subsubsection{生产资料社会主义国有化}
    \subsubsection{国民经济社会主义改造}
    \subsubsection{社会主义工业化}
    \subsubsection{由前资本主义生产方式向社会主义过渡问题}
\section{社会主义社会生产力与生产关系}
    \subsubsection{社会主义物质技术基础}
    \subsubsection{生产力、生产关系、所有制}
\section{社会主义经济规律}
\section{社会主义劳动}
\section{商品生产与交换}
    \subsection{商品、商品生产}
    \subsection{货币}
    \subsection{价值、价格}
        \subsubsection{价值、价值规律}
        \subsubsection{价格}
        \subsubsection{成本利润}
    \subsection{经济效益}
    \subsection{社会主义计划与市场}
        \subsubsection{国家的经济职能}
    \subsection{经济机制}
    \subsection{资本和剩余价值}
\section{社会主义分配制度}
    \subsection{按劳分配原则}
    \subsection{物质鼓励}
    \subsection{分配方式}
        \subsubsection{全民所有制经济}
        \subsubsection{集体所有制经济}
        \subsubsection{私有制经济}
    \subsection{工资制度}
\section{国民收入与分配}
    \subsubsection{国民财富}
    \subsubsection{社会总产品和国民收入}
    \subsubsection{积累和消费}
    \subsubsection{生活方式}
\section{社会主义再生产}
    \subsubsection{简单再生产、扩大再生产}
    \subsubsection{部门间的关系}
\section{经济循环}



\chapter{共产主义社会生产方式}



\chapter{经济学分支科学}
    \subsubsection{生产力经济学}
    \subsubsection{增长经济学}
    \subsubsection{发展经济学}
    \subsubsection{福利经济学}
    \subsubsection{区域经济学}
    \subsubsection{国土经济学}
    \subsubsection{资源经济学}
    \subsubsection{生态经济学}
    \subsubsection{科学经济学、知识经济学}
    \subsubsection{技术经济学}
    \subsubsection{信息经济学}
    \subsubsection{公共经济学}
    \subsubsection{产业经济学}
    \subsubsection{非生产领域经济学}
    \subsubsection{消费经济学}
    \subsubsection{国防经济学}
    \subsubsection{家庭经济学}
    \subsubsection{民族经济学}
    \subsubsection{计量经济学}
    \subsubsection{比较经济学}
    \subsubsection{短缺经济学}
    \subsubsection{其他}


\chapter{各科经济学}


\chapter{经济思想史:世界}

\section{古代经济思想 (公元前约3500~公元476年 )}
\section{中世纪经济思想 (476-1640年 )}
\section{近现代经济思想 (1640年- )}
    \subsection{重商主义}
    \subsection{重农主义}
    \subsection{古典经济学}
    \subsection{庸俗经济学}
        \subsubsection{马尔萨斯主义}
        \subsubsection{历史学派、新历史学派}
        \subsubsection{奥地利学派 (心理学派 )}
        \subsubsection{社会学派}
        \subsubsection{数理经济学派 (洛桑学派 )}
        \subsubsection{北欧学派 (瑞典学派、斯德哥尔摩学派 )}
        \subsubsection{剑桥学派 (新古典学派 )}
        \subsubsection{凯恩斯学派、凯恩斯主义}
            \paragraph{新古典综合派、后凯恩斯主义}
            \paragraph{新剑桥学派}
        \subsubsection{制度学派、新制度学派 (现代制度学派 )}
    \subsection{合作主义}
    \subsection{新自由主义}
        \subsubsection{伦敦学派}
        \subsubsection{弗赖堡学派}
        \subsubsection{合理预期学派 (理性预期学派 )}
        \subsubsection{供给学派 (弹性学派 )}
    \subsection{货币主义}
    \subsection{熊彼得 (Schumpeter,J.A. )经济思想}
    \subsection{罗斯托 (Rostow,W.W. )经济思想}

\section{小资产阶级经济学派}
\section{空想社会主义经济思想}
\section{马克思列宁主义经济思想}
    \subsubsection{马克思、恩格斯经济思想}
    \subsubsection{列宁、斯大林经济思想}
    \subsubsection{毛泽东、邓小平经济思想}
    \subsubsection{西方马克思主义经济思想、右翼社会主义经济思想}



\chapter{经济思想史:中国世界}
\subsubsection{古代}
\subsubsection{近代}
\subsubsection{现代}





\chapter{End}




\end{document}

