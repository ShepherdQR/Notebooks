%%============
%%  ** Author: Qirong ZHANG
%%  ** Date: 2024-12-22 14:51:38
%%  ** Github: https://github.com/ShepherdQR
%%  ** LastEditors: Qirong ZHANG
%%  ** LastEditTime: 2024-12-22 15:55:03
%%  ** Copyright (c) 2019 Qirong ZHANG. All rights reserved.
%%  ** SPDX-License-Identifier: LGPL-3.0-or-later.
%%============



\documentclass[UTF8]{../../RepresentationUniverse}
\begin{document}

\title{10-06-TransportationAndPostEconomy}
\date{Created on 20221222.\\   Last modified on \today.}
\maketitle
\tableofcontents


\chapter{Introduction}



交通运输经济


邮电经济
 

\chapter{交通运输经济理论}
    \subsubsection{运输业计划和管理体制}
    \subsubsection{运输业建设与发展}
    \subsubsection{运输价格、成本与利润}
    \subsubsection{运输企业组织与经营管理}

\chapter{交通运输概况:世界}
\section{交通运输政策}
\section{运输业建设与发展}
    \subsubsection{运输业基本建设与布局}
    \subsubsection{国际合作与协调运输}
\section{国际联合运输}
    \subsubsection{货物运输}
    \subsubsection{旅客运输}
\section{国际组织和会议}
\section{条约、协定}
\section{交通史}
\section{交通经济地理}

\chapter{交通运输概况:中国}
\section{方针政策及其阐述}
\section{运输业计划与管理体制}
\section{交通运输建设和发展}
\section{联合运输}
    \subsubsection{国内}
    \subsubsection{国际}
\section{运输价格、成本与利润}
\section{运输企业组织与管理}
\section{地方交通运输概况}
\section{中国交通史}
\section{中国交通地理}



\chapter{铁路运输经济}

\section{铁路运输经济理论}
    \subsection{铁路计划与管理体制}
    \subsection{铁路建设与发展}
        \subsubsection{铁路基本建设与投资}
        \subsubsection{铁路配置}
        \subsubsection{铁路技术改造与革新}
        \subsubsection{铁路统计学}
        \subsubsection{铁路选线经济}
        \subsubsection{铁路施工经济}
    \subsection{铁路工业经济}
    \subsection{铁路运输成本、运价、票价}
        \subsubsection{运输成本}
        \subsubsection{运价、票价}
        \subsubsection{运输经济效益}
    \subsection{铁路企业组织和管理}
        \subsubsection{管理机构与组织系统}
        \subsubsection{技术管理、定额管理}
        \subsubsection{劳动组织}
        \subsubsection{生产责任制}
        \subsubsection{劳动竞赛、增产节约}
        \subsubsection{固定资产管理}
        \subsubsection{财务管理}
        \subsubsection{劳动保护、安全生产}
    \subsection{各种铁路的经营与管理}
    \subsection{运输业务}
        \subsubsection{行车组织}
        \subsubsection{货运工作}
        \subsubsection{客运工作}
        \subsubsection{铁路联运}
        \subsubsection{铁路国际联运}
        \subsubsection{运营}
    \subsubsection{列车员}


\section{世界铁路运输经济}
    \subsubsection{政策}
    \subsubsection{铁路运输建设与发展}
    \subsubsection{国际联合运输}
    \subsubsection{国际组织与会议}
    \subsubsection{条约、协定}
    \subsubsection{铁路运输史}
    \subsubsection{各国铁路运输经济}
    

\section{中国铁路运输经济}
    \subsection{方针政策及其阐述}
    \subsection{铁路计划与管理体制}
    \subsection{铁路建设与发展}
    \subsection{铁路联运}
        \subsubsection{国内}
        \subsubsection{国际}
    \subsection{铁路成本、运价、票价}
    \subsection{铁路运输企业与管理}
    \subsection{各线路概况}
    \subsection{地方铁路运输经济}
    \subsection{中国铁路史}



\chapter{陆路、公路运输经济}

\section{陆路、公路运输经济理论}
    \subsection{公路建设与发展}
        \subsubsection{建设经济调查分析与预测}
        \subsubsection{建设规划与布局}
        \subsubsection{技术发展与革新}
        \subsubsection{建设筹资与投资}
        \subsubsection{公路统计学}
    \subsection{公路运输成本、运价、票价}
    \subsection{公路运输企业组织与管理}
        \subsubsection{国营运输企业}
        \subsubsection{集体和个体运输企业}
        \subsubsection{合营运输企业}
        \subsubsection{劳动组织与生产责任制}
        \subsubsection{固定资产管理}
        \subsubsection{财务管理}
    \subsection{其他陆路运输}
    \subsection{公路运输业务}
        \subsubsection{站务工作}
        \subsubsection{货运工作}
        \subsubsection{客运工作}
        \subsubsection{汽车联运}
        \subsubsection{列车化运输}

\section{世界陆路、公路运输经济}
\section{中国陆路、公路运输经济}




\chapter{水路运输经济}
\section{水路运输经济理论}
    \subsection{水路运输建设与发展}
    \subsection{水路运输成本、运价、票价}
    \subsection{水路运输组织与管理}
        \subsubsection{运输组织}
        \subsubsection{业务计划}
        \subsubsection{劳动组织}
        \subsubsection{财务管理、经济核算}
    \subsection{各种水路运输}
        \subsubsection{内河运输}
        \subsubsection{海洋运输}
        \subsubsection{近海运输}
        \subsubsection{远洋运输}
    \subsection{水路运输业务}
        \subsubsection{货运工作}
        \subsubsection{客运工作}
        \subsubsection{水路联运}
        \subsubsection{国际联运}

\section{世界水路运输经济}

\section{中国水路运输经济}
    \subsection{方针政策及其阐述}
    \subsection{管理体制}
    \subsection{建设与发展}
    \subsection{联合运输}
        \subsubsection{国内}
        \subsubsection{国际}
    \subsection{运输成本、运价和票价}
    \subsection{企业组织和经营管理}
        \subsubsection{国营运输企业}
        \subsubsection{集体和个体运输企业}
        \subsubsection{合营运输企业}
        \subsubsection{劳动组织}
    \subsection{地方水路运输经济}
    \subsection{中国水路交通史}




\chapter{航空运输经济}
\section{航空运输经济理论}
    \subsection{航运计划工作}
    \subsection{航线开辟与航运基本建设}
    \subsection{运价与客运票价、运输成本与利润}
    \subsection{航运企业组织与经营管理}
    \subsection{专业飞行}
    \subsection{运输业务}
        \subsubsection{机场管理、站务}
        \subsubsection{机上管理}
        \subsubsection{客运}
        \subsubsection{货运}
        \subsubsection{国际联运、空运国际合作}
    \subsection{服务员}

\section{世界航空运输}
\section{中国航空运输}
    \subsection{方针政策及其阐述}
    \subsection{航运管理体制}
    \subsection{航运建设与发展}
    \subsection{运输成本、运价、票价及经济核算}
    \subsection{航运企业组织与经营管理}
    \subsection{地方航空运输经济}
    \subsection{航空运输与民航事业史}



\chapter{城市交通运输经济}
\section{城市交通运输经济理论}
    \subsection{城市交通运输建设}
    \subsection{运输成本、运价、票价与经济核算}
    \subsection{企业组织与管理}
    \subsection{各种车辆运输}
        \subsubsection{公共汽车、出租汽车}
        \subsubsection{电车}
        \subsubsection{地下铁路运输}
        \subsubsection{其他运输方式}
    \subsection{运输业务}
        \subsubsection{货运}
        \subsubsection{客运}
\section{世界城市交通运输经济}
\section{中国城市交通运输经济}
    \subsubsection{地方城市交通运输经济}
    \subsubsection{城市交通史}



\chapter{旅游经济}
\section{旅游经济理论与方法}
    \subsection{旅游规划与管理体制}
    \subsection{旅游事业建设与发展}
    \subsection{旅游企业组织与管理}
        \subsubsection{旅游服务业务}
        \subsubsection{旅游企业}
        \subsubsection{旅游财务管理}
    \subsection{各类型旅游}
    \subsection{旅游市场}

\section{世界旅游事业}
    \subsubsection{旅游事业史}
    \subsubsection{旅游经济地理}

\section{中国旅游事业}
    \subsubsection{方针政策及其阐述}
    \subsubsection{规划与管理体制}
    \subsubsection{旅游事业建设与发展}
    \subsubsection{旅游企业组织与管理}
    \subsubsection{地方旅游事业}
    \subsubsection{旅游事业史}
    \subsubsection{旅游经济地理}









\chapter{邮电经济理论}
\section{邮电业计划与管理体制}
\section{邮电业建设与发展}
\section{邮电企业组织与经营管理}
    \subsubsection{组织机构}
    \subsubsection{邮电业务}
    \subsubsection{财务管理、会计核算}


\chapter{邮政}
    \section{邮政事业计划和管理}
    \section{邮政建设与发展}
    \section{邮政业务收入、成本与利润}
    \section{邮政企业组织与经营管理}
        \subsubsection{组织机构}
        \subsubsection{现代化管理}
        \subsubsection{固定资产管理}
        \subsubsection{财务管理和经济核算}
    \section{邮政业务}
        \subsubsection{邮件收受、处理、运送、投递}
        \subsubsection{邮政其他业务}
        \subsubsection{邮费和邮票}
    \section{邮务员、分拣员}



\chapter{电信}
\section{电信事业计划和管理体制}
\section{电信建设与发展}
\section{电信企业组织和经营管理}
    \subsection{电话企业}
        \subsubsection{企业组织和管理}
            \paragraph{劳动组织、劳动效率}
            \paragraph{技术设备管理、材料管理}
            \paragraph{财务管理、经济核算}
            \paragraph{事业收入、成本和利润}
        \subsubsection{电话业务}
        \subsubsection{话务员}
    \subsection{电报业}
        \subsubsection{企业组织和管理}
        \subsubsection{电报业务}
        \subsubsection{报务员}
    \subsection{无线电通信企业}
    \subsection{通信网络企业}
\section{国际电信}



\chapter{世界各国邮电事业}
\section{世界}
    \subsubsection{政策}
    \subsubsection{国际邮电事业}
    \subsubsection{邮电事业史}
    \subsubsection{邮电地理}
\section{中国}
    \subsubsection{方针政策及其阐述}
    \subsubsection{邮电计划和管理体制}
    \subsubsection{邮电事业建设和发展}
    \subsubsection{对外邮电业务关系}
    \subsubsection{地方邮电事业}
    \subsubsection{邮电事业史}
    \subsubsection{邮电地理}







\chapter{end}



\end{document}

