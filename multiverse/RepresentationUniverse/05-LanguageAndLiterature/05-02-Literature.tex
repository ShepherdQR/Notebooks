%%============
%%  ** Author: Shepherd Qirong
%%  ** Date: 2023-05-01 22:28:48
%%  ** Github: https://github.com/ShepherdQR
%%  ** LastEditors: Shepherd Qirong
%%  ** LastEditTime: 2023-05-02 17:23:07
%%  ** Copyright (c) 2019--20xx Shepherd Qirong. All rights reserved.
%%============


\documentclass[UTF8]{../RepresentationUniverse}
\begin{document}

\title{05-02-Literature}
\date{Created on 20230501.\\   Last modified on \today.}
\maketitle
\tableofcontents


\chapter{Introduction}


\section{书籍}





\subsection{世界文学史}

\begin{lstlisting}
聂珍钊《外国文学史》
李斌宁《欧洲文学史》
郑克鲁《外国文学史》
唐建清《欧美文学研究引导》
徐葆耕《西方文学十五讲》
朱维之《外国文学史》
王忠祥《外国文学教程》
王佐良《英国散文的流变》
陈琨《西方现代派文学研究》
威尔逊,特里林, 诺顿六讲。

朱光潜《西方美学史》
丹纳《艺术哲学》
罗素《西方哲学史》

《神话与民族精神》
\end{lstlisting}



\subsection{文学名著}


\begin{lstlisting}


    《希腊的神话和传说》(斯威布编)
    荷马史诗:《伊利亚特》、《奥德赛》
    索福克勒斯:《俄狄浦斯王》【重要】
    欧里庇得斯:《美狄亚》【重要】
    但丁:《神曲》
    卜伽丘:《十日谈》
    拉伯雷:《巨人传》
    塞万提斯:《堂吉诃德》
    莎士比亚;《哈姆莱特【重要】》《奥瑟罗【重要】》《威尼斯商人》
    莫里哀:《伪君子》
    笛福:《鲁滨逊漂流记》
    斯威夫特:《格列佛游记》
    卢梭:《忏悔录》
    博马舍:《费加罗的婚姻》
    席勒:《阴谋与爱情》
    歌德:《浮士德》、《少年维特的烦恼【重要】》
    拜伦:《唐璜》
    雪莱;《西风颂》
    雨果:《巴黎圣母院》《悲惨世界》



\end{lstlisting}





\chapter{总览}



\begin{lstlisting}

    东方文学:到AC500是古代文学。AC500-19世纪中叶是中古时期,19世纪中叶到20世纪初是近代文学,20世纪以来是现当代文学。

    西方文学:
    古代:古希腊古罗马是源头。古希腊的神话、戏剧、史诗。
    古希腊古罗马:BC1200-AC500
    
    古希腊:BC3000-BC1200,克里特-迈锡尼文明;
    BC1200-BC800,荷马时代,英雄时代。
    BC800-BC600,大移民时代。
    BC500-BC400,古典时代。全盛时期。
    BC338-BC146,希腊化时期。BC146,马其顿被罗马征服。
    自由民主。理性主义与狂欢精神。
    
    
    到中古,意大利但丁《神曲》,14世纪初。
    14世纪文艺复兴到20世纪,称为近代。20世纪上是现代,20世纪下至今是当代。
    17世纪,古典主义,法国古典主义莫里哀的喜剧
    18世纪,启蒙主义,启蒙运动法国卢梭,德国歌德
    19世纪,浪漫主义,批判现实,唯美主义(王尔德),左拉(自然主义),象征主义(前期现代主义),
    20世纪,现代主义(各种思潮,意识流,荒诞派戏剧,黑色幽默,存在主义)
    
    
\end{lstlisting}




\chapter{古代,BC3000-AC500}

\section{远古,BC3000-BC1200}

古希腊:BC3000-BC1200,克里特-迈锡尼文明


\section{古希腊}

\begin{proposition}
    柏拉图与亚里士多德异同

    1)文艺与现实的关系。文艺模仿现实现实模仿理念;亚里士多德认为文艺模仿现实,承认文艺的现实性。
    
    2)文艺的社会功能。柏拉图认为政治是文艺价值的标准,而非审美性,是阶级的观点,为理想国的贵族服务;亚里士多德看到了文艺的认识作用、教育作用、审美作用,塔塔西斯(陶冶、作用)。
    
    3)文艺创作的源动力:柏拉图认为是灵感、迷狂,神灵给的;亚里士多德强调天才,即理性和智慧。
\end{proposition}



\subsection{古希腊神话}

\subsubsection{产生原因}

1)认识与解释自然现象、社会现象,表现出认识自然、征服自然的愿望和斗争精神。

2)泛灵论(古老的宗教观)和比喻类推(原始的思维方式)的产物。

神话是用想象和借助想象征服、支配、形象化自然力。————马克思


\subsubsection{内容}

\begin{lstlisting}
创世:前奥林匹斯神系,奥林匹斯神系

人类起源:抟土造人,说明生产力当时陶器,认识水平与生产力水平关联。

灾难:大洪水。伴水而生,水的灾难。

创造之父:各种能力的源头。

英雄:半人半神。

赫拉克勒斯,12件功劳。
伊阿宋
俄狄浦斯王
特罗亚
忒休斯

强调人的自然性、个体性。

后羿射日,夸父逐日,精卫填海等。注重人的社会责任。
\end{lstlisting}


\begin{proposition}
    东西方创世的差异

    1)宇宙观方面:
    西方:创造者与被创造者分开,二元对立;
    东方:天地自然、人,同源同构同体,一元,天人合一,物我一体。

    2)哲学观和审美观:
    西方:存在至高无上的神主宰世界,追问世界的本原、本质,探索美的本质;
    东方:天人合一。审美是直觉了悟、体验式的,整体意识。

    3)自然观:
    西方:对立,征服,奴役。拟人、移情。
    东方:热爱自然,亲近,融合。比兴,拟物。
\end{proposition}




\subsubsection{特点}
1)主神单一,统一。各神分工细致,分工和职责明确。神谱关系清晰。

2)神人同形同姓,神的形象高度人格化,体现人本主义,肯定人的力量。

3)有哲理和文学表现力。


\subsubsection{意义}

\begin{lstlisting}
希腊艺术的前提、土壤;
后代艺术家的创作素材;
启迪思想、审美教育;
历史文化价值
\end{lstlisting}


\subsection{古希腊戏剧}

\subsubsection{伦理禁忌与俄狄浦斯悲剧}

\subsubsection{美狄亚}

\begin{lstlisting}

一、美狄亚的性格有何特点?
二、如何看待美狄亚的杀子复仇?
三、如何看待伊阿宋形象?
四、美狄亚的悲剧是否为女性择婚的偶然失败?
五、美狄亚形象在中西文学史上有何原型意义?
六、作品是否体现出欧里庇得斯的怀疑论思想?
七、为什么作品被称为是“令人震惊的妇女心灵的悲剧?”

\end{lstlisting}


\subsection{古希腊诗歌}

泰奥克雷特,田园牧歌。


\section{荷马时代, BC1200-BC800}
BC1200-BC800,荷马时代,英雄时代。

\subsection{荷马史诗BC12-BC8}






BC2形成定本。特罗亚战争为背景。

《伊利亚特》特洛伊战争最后几十天发生的事情。围绕阿喀琉斯的愤怒和息怒,歌颂英勇善战的英雄。

《奥德赛》俄底修斯在特洛伊战争后返乡的故事,歌颂人与自然斗争的坚毅精神,反映奴隶制形成时期争夺财富和地位的斗争。




\begin{lstlisting}

    问:塑造了哪几类人物?代表?

英雄:
阿喀琉斯:忘我的战斗精神;温厚善良;捍卫个人尊严。阿喀琉斯的脚踝(唯一的弱点)。
赫克托耳:集体主义精神
俄底修斯:智者

女性:
安德洛玛克
潘奈洛佩

天神
奴隶
魔怪


找出所运用的比喻、拟人、反复等修辞手法。
    
\end{lstlisting}


\subsubsection{特点}
\begin{lstlisting}
    以诗叙事,戏剧性鲜明。
    多种叙述手法。倒序与顺序结合;多条叙事线。
    人物、环境描写
    比喻、拟人、反复等修辞手法。
\end{lstlisting}






《荷马史诗的英雄伦理观》龚群。
《荷马史诗与希腊帝国》蹇昌槐。
《被缚的女人————浅析《荷马史诗》中的女人群像》李权华


\section{大移民时代, BC800-BC600}

\section{古典时代, BC510-BC323}
全盛时期。

雅典,从BC800称为希腊的核心,到BC431年雅典和斯巴达之间的伯罗奔尼撒战争以雅典失败而结束,整个希腊开始由盛转衰。

雅典长官关注:渎神和无神论的文字、诽谤中伤的文字。

\section{希腊化时期, BC338-BC146}


\section{古罗马, BC146-AC500}






\chapter{中古,5世纪-14世纪}
\section{中世纪文学}


\subsection{但丁}



\chapter{文艺复兴,14世纪-17世纪}



\subsection{莎士比亚的诗歌与戏剧}





\chapter{17世纪,古典}

\section{古典主义}

\subsection{莫里哀}

\section{巴洛克文学与英国清教徒文学}


\section{法国-矫饰派}


矫饰派的“畅销书”:《阿斯特蕾》
1607年至1628年间,这部多达五千页的五卷本长河小说分期出版(最后一卷出版于作者逝世后)。《阿斯特蕾》是17世纪上半叶最为畅销的作品。


\chapter{18世纪,启蒙}


\section{英国}

\section{法国}

\section{德国}



\chapter{19世纪}

洛可可伤感主义。
浪漫主义,批判现实,唯美主义(王尔德),左拉(自然主义),象征主义(前期现代主义)


\section{浪漫主义}


\subsection{英国}

\subsection{法国}

\subsection{德国}

\subsection{主要作品}

\subsubsection{司汤达《红与黑》}
\subsubsection{巴尔扎克《人间喜剧》}
\subsubsection{福楼拜《包法利夫人》}
\subsubsection{狄更斯《双城记》}
\subsubsection{普希金《叶甫盖尼·奥涅金》}
\subsubsection{果戈里《钦差大臣》}
\subsubsection{托尔斯泰}
\subsubsection{海明威《老人与海》}
\subsubsection{劳伦斯}



\section{批判现实主义文学}

\subsection{托马斯·哈代}


\section{唯美主义}

\subsection{《道林格雷的画像》和《分成两半的子爵》}




\chapter{20世纪}


20世纪,现代主义(各种思潮,意识流,荒诞派戏剧,黑色幽默,存在主义)



\section{存在主义文学}


\section{荒诞派戏剧}



\section{表现主义文学}

\subsection{奥尼尔和《悲悼》三部曲}



\section{象征主义诗歌}
\subsection{艾略特}




\section{意识流}




\chapter{外国文学史书目}

\begin{lstlisting}

BC5000,埃及的成体系的神话
BC3000,埃及《亡灵书》

BC2100, 世上最早的史诗,巴比伦的《吉尔伽美什》

《格萨尔王》,中国藏族英雄史诗,长。

BC600, 文字的《荷马史诗》


希伯来的旧约,
印度,吠陀经


古希腊
希腊神话:奥维德《变形记》;维吉尔《伊尼德》
荷马史诗
伊索寓言【拉封丹包装伊索寓言】
叙事诗:赫西俄德《工作与时日》《神谱》
抒情诗:萨福(女诗人,中世纪焚毁了,拜伦称其为火焰般炽热);阿拉克瑞翁;品达。
历史作品:希罗多德《历史》;修昔底德《伯罗奔尼撒战争史》;色诺芬《远征记》。
演说词:伊索格拉底《泛希腊集会辞》;狄摩西尼《第三篇反腓力辞》。
柏拉图《柏拉图文艺对话集》
亚里士多德《诗学》


古典时期(6bc-4bc)
戏剧:埃斯库罗斯《俄瑞斯忒亚》《被缚的普罗米修斯》、索福克勒斯《俄狄浦斯王》、欧里庇得斯《美狄亚》;阿里斯托芬《鸟》
散文:希罗多德、修昔底德
文艺理论:柏拉图《理想国》、亚里士多德《诗学》

古罗马文学
共和时期(bc240-bc30)
利维乌斯·安德罗尼库斯《奥德修纪》
普劳图斯 改编喜剧
黄金时期(bc100-17)
卢克莱修《论自然》
奥维德《变形记》史诗光怪陆离故事
维吉尔《埃涅阿斯纪》第一部文人史诗 与罗马帝国有关
白银时代(17-130)
尤维纳利斯 讽刺作家


中世纪文学
教会文学
史诗、歌谣
《贝奥武普》欧洲最完整,北欧氏族阶段在大陆的生活
《罗兰之歌》法国,查理的大帝时代爱国英雄罗兰
骑士文学
亚瑟王与圆桌骑士
市民文学/城市文学
叙事诗《列那狐传奇》《玫瑰传奇》
抒情诗:吕特勃夫、维庸
戏剧《巴特兰律师》
但丁《神曲》


文艺复兴时期文学
意大利文学
但丁
彼特拉克
薄伽丘
法国:贵族、平民两种倾向
龙沙
弗朗索瓦·拉伯雷《巨人传》
蒙田《随笔集》怀疑论、欧洲近代散文创始人
西班牙
流浪汉小说《小癞子》
塞万提斯:“现代小说之父”《堂吉诃德》、《训诫小说集》
戏剧:洛卜·德·维加《羊泉村》
英国
杰弗利·乔叟《特洛伊拉斯和克莱西德》仿薄伽丘
托马斯·莫尔《乌托邦》对话体幻想小说
埃德蒙·斯宾塞《仙后》长诗
戏剧“大学才子派”
莎士比亚


17世纪文学
古典主义文学
始于法国
勒内·笛卡儿:唯理主义《方法论》、《心灵情感论》
弗朗瓦索·德·马莱布《劝慰杜佩里埃先生》
古典主义悲剧
皮埃尔·高乃依:古典主义悲剧创始人《熙德》、《贺拉斯》
让·拉辛《安德洛玛克》因私废公
让·德·拉封丹《寓言诗》寓言大成者
尼古拉·布瓦洛:理论家《诗的艺术》
莫里哀:戏剧《唐璜》贵族腐朽、《伪君子》宗教骗子
巴洛克文学
(16c-17c)精雕细刻;起源于意大利、西班牙,兴盛于法国
马里诺《阿多尼斯》维纳斯与美少年的爱情纠葛
阿尔戈特:长诗
《莱尔玛公爵颂》《孤独》
佩特罗·卡尔德隆《人生如梦》
法国小说:奥诺雷·德·于尓菲《阿丝特蕾》牧羊男女的故事
清教徒文学
约翰·弥尔顿:《失乐园》史诗 赞美撒旦的反抗、《复乐园》耶稣拒绝撒旦诱惑、《力士参孙》参孙憾大柱复仇
约翰·班扬《天路历程》梦境寓意,复辟时期腐败


18世纪文学
现实主义小说
丹尼尔·笛福《鲁宾逊漂流记》
约拿旦·斯威夫特《格列佛游记》
理查生:英国家庭小说开创者《帕米拉》《克拉丽莎》婚姻自主同中产温和的道德说教
托比亚·斯摩莱特《兰豋传》流浪汉小说,讽刺社会现实

伤感主义文学
劳伦斯·斯泰恩《伤感的旅行》
奥立维·哥尔德斯密斯《威克菲尔德的牧师》
法国
孟德斯鸠《法的精神》、《波斯人信札》第一部启蒙哲理小说
伏尔泰:启蒙哲理小说《老实人》讽刺盲目乐观,否定悲观,回到苦难现实、《如此世界》、《查第格》、《天真汉》
德尼·狄德罗:《百科全书》、《当好人还是坏人?》奠定现代话剧基础、《拉摩的侄儿》对话体小说、《宿命论者雅克和他的主人》
德国
高特舍德
高·埃·莱辛:戏剧《拉奥孔》、《汉堡剧评》、《艾米丽亚·迦洛蒂》
维兰德:小说家
狂飙突进运动,古典文学时期
歌德:《少年维特的烦恼》、《普罗米修斯》长诗、《浮士德》、《赫尔曼与窦绿苔》总结;《亲和力》、《诗与真》自传性作品、《威廉·迈斯特》教育小说、《西东合集》诗集
席勒
卢梭:《社会契约论》、《新爱洛依丝》、《爱弥儿》教育小说、《忏悔录》发掘自我


19世纪

19世纪浪漫主义文学
18世纪末产生,19世纪上半叶繁荣;个人独立和极端自由
施莱格尔兄弟
斯塔尔夫人
夏多布里昂《基督教真谛》提供散文、小说典范
拜伦、雪莱、济慈
雨果
英国====
先驱: 罗伯特·彭斯、威廉·布莱克
开创:湖畔派:华兹华斯《抒情歌谣集》、柯勒律治、罗伯特·骚塞
拜伦:《唐璜》叙事诗冒险、讽刺、自由
雪莱
德国====
克莱斯特
霍夫曼《金罐》
沙米索
法国====
雨果:(对比)诗歌《静观集》、戏剧、小说《巴黎圣母院》、《悲惨世界》、《笑面人》、《九三年》
大仲马:《三个火枪手》人物、《基督山伯爵》
乔治·桑:妇女问题、社会问题、田园小说《魔沼》
缪塞
美国====
爱默生
梭罗
华盛顿·欧文
詹姆斯·费尼莫·库柏:边疆传奇小说《最后一个莫西干人》
埃德加·爱伦·坡:最早的推理小说
霍桑:隐秘的恶《红字》
惠特曼:诗人《草叶集》一生的诗歌
麦尔维尔

19世纪现实主义文学
法国发源====
斯丹达尔:《拉辛与莎士比亚》、《红与黑》形成,《帕尔马修道院》战争描写;心理
巴尔扎克:《人间喜剧》最高成就
福楼拜:《包法利夫人》语言
梅里美:《高龙巴》、《嘉尔曼》
小仲马:《茶花女》
都德:《小东西》、《最后一课》、《柏林之围》

巴黎公社文学====
鲍狄埃:《国际歌》
米雪尔:《红石竹花》
瓦莱斯
英国====
狄更斯:《大卫·科波菲尔》、《远大前程》对上流社会幻想的破灭、《双城记》
萨克雷:《名利场》两个女人反应出的社会现实
夏绿蒂·勃兰特:《简·爱》;《呼啸山庄》、《谢利》、《维莱特》
盖斯凯尔夫人:《玛丽·巴顿》最早触及劳资矛盾
哈代:《德伯家的苔丝》
萧伯纳
高尔斯华绥
诗歌:厄内斯特·琼斯、威廉·林顿
德国====
海涅:民批专,诗人《德国,一个冬天的童话》诗
毕希纳:戏剧《丹东之死》
维尔特:诗人
北欧====
安徒生童话
勃兰兑斯:《19世纪文学主流》
易卜生:社会问题剧《青年同盟》、《社会支柱》、《人民公敌》、《玩偶之家》、《群鬼》
比昂松:戏剧《破产》、《挑战的手套》
俄国====
普希金:《叶普盖尼·奥涅金》多余人、《驿站长》小人物
莱蒙托夫:《当代英雄》
果戈里:戏剧《钦差大臣》、《死魂灵》
冈察洛夫《奥勃洛摩夫》
屠格涅夫《前夜》、《父与子》新人
车尔尼雪夫斯基:《怎么办?》
奥斯特洛夫斯基:戏剧《大雷雨》
涅克拉索夫《在俄罗斯谁能过好日子》
陀思妥耶夫斯基:《白夜》幻想家,《被欺凌与被凌辱的》、《死屋手记》监狱、《地下室手记》地下人、《罪与罚》、《白痴》、《群魔》、《少年》、《卡拉马佐夫兄弟》、
谢德林《戈洛夫略夫一家》
托尔斯泰:《童年》三部曲、《战争与和平》、《安娜·卡列尼娜》、《复活》;克服环境,心灵辩证法
契诃夫:《一个文官的死》、《胖子和瘦子》、《变色龙》简洁、《苦恼》隔膜、《万卡》、《草原》、《神经错乱》、《第六病室》、《跳来跳去的女人》、《带小狗的女人》、《套中人》、《樱桃园》
美国====
废奴文学
希尔德烈斯《白奴》
斯托夫人《汤姆叔叔的小屋》
乡土小说
哈特《咆哮营的幸运儿》
马克吐温:《竞选州长》、《镀金时代》南北战争后、《汤姆·索亚历险记》、《哈克贝利·费恩历险记》《王子与贫儿》、《百万英镑》;《神秘的陌生人》、《什么是人?》
亨利·詹姆斯:心理分析先河
诺里思:《章鱼》
欧·亨利:意料之外的结局
杰克·伦敦:《荒野的呼唤》、《白牙》、《铁蹄》、《马丁·伊登》、


19世纪自然主义和其他文学流派
产生于法国;照片式印象====
龚古尔兄弟
盖尔哈德·霍普特曼
斯特林堡
唯美主义====
泰奥菲尔·戈蒂埃
约翰·罗斯金
瓦尔特·佩特
爱伦·坡
王尔德:《谎言的衰朽》、《道林·格雷的画像》、《莎乐美》、《快乐王子集》
前期象征派====
让·莫雷亚斯:《象征主义宣言》
左拉:《卢贡-马卡尔家族》第二帝国一个家庭史、《萌芽》劳资矛盾
马拉美
波德莱尔:以丑为美,通感《恶之花》,语言精粹
保尔·魏尔伦
阿瑟·兰波
莫泊桑:《羊脂球》、《漂亮朋友》、《项链》语言简洁

20世纪文学
20世纪欧美现实主义文学====
劳伦斯:《虹》
萧伯纳:《芭芭拉少尉》
约翰·高尔斯华绥:《福尔赛世家》
毛姆:《人性的枷锁》
格雷厄姆·格林:《沉静的美国人》
愤怒的青年====
金斯莱·艾米斯:《幸运的吉姆》
约翰·奥斯本:《愤怒的回顾》
约翰·布莱思:《向上爬》
约翰·福尔斯:《法国中尉的女人》
多丽丝·莱辛:《金色笔记》
奈保尔:《毕司沃斯先生的房子》
阿加莎·克里斯蒂:《东方快车谋杀案》、《尼罗河惨案》
罗琳:《哈利波特》
法国
阿纳托尔·法朗士:《克兰克比尔》、《企鹅岛》、《诸神渴了》
罗曼·罗兰:《巨人传》、《约翰·克利斯朵夫》、《母与子》、《内心旅程》
马丁·杜伽尔:《蒂博一家》
安德烈·纪德:《背德者》、《窄门》、《伪币制造者》小说套小说
莫里亚克:《苔蕾丝·德盖鲁》、《蝮蛇结》
塞利纳:《茫茫黑夜漫游、《小王子》
马尔罗:《人的状况》上海工人起义
季奥诺
柯莱特
玛格丽特·杜拉斯:淡化情节,截取片段《情人》
玛格丽特·尤瑟纳尔:历史小说
德国
布莱希特
亨利希·曼:《臣仆》
托马斯·曼:《布登勃洛克一家》
君特·格拉斯:《铁皮鼓》

斯蒂芬·茨威格:《一个女人一生中的二十四个小时》
埃里希·马利亚·雷马克:战争题材《西线无故事》、《凯旋门》
亨利希·伯尔:《莱尼和他们》
美国
德莱塞
海明威:《太阳照样升起》、《永别了,武器》、《乞力马扎罗山上的雪》、《丧钟为谁而鸣》、《老人与海》
厄普顿·辛克莱:《屠场》揭发黑幕运动
辛克莱·刘易斯:《巴比特》庸俗市侩者
约翰·斯坦贝克:《愤怒的葡萄》
司各特·菲兹杰拉德:迷茫一代《了不起的盖茨比》
理查德·赖特:《土生子》犯罪心理
杰罗姆·大卫·塞林格:《麦田的守望者》
艾萨克·巴什维斯·辛格:《卢布林的魔法师》
索尔·贝楼:《奥吉马奇历险记》、《洪堡的礼物》
托妮·莫瑞森:《所罗门之歌》黑人受苦、《最蓝的眼睛》、《宠儿》
约翰·厄普代克:兔子四部曲
乔伊斯·卡洛尔·欧茨:《他们》30年代美下层人民的命运
纳博科夫:《洛丽塔》
华裔作家:汤亭亭、赵建秀
戏剧:田纳西·威廉斯《欲望号街车》、阿瑟·米勒《推销员之死》
玛格丽特·米切尔:《飘》
丹·布朗《达·芬奇密码》
其他欧
捷克
雅罗斯拉夫·哈谢克《好兵帅克》政治讽刺
米兰·昆德拉《生命中不能承受之轻》知识分子的感情生活和人生选择;《小说的艺术》、《生活在别处》、《不朽》、《被背叛的遗嘱》
挪威
克努特·哈姆生:《大地硕果》
西格里德·温赛特《克里斯汀》
意大利
格拉齐亚·黛莱达《风中芦苇》古老家族解体
皮兰德娄《六个寻找作者的剧中人》
姜尼·罗大里《洋葱头历险记》儿童作品
伊泰洛·卡尔维诺《分成两半的子爵》寓言小说、《寒冬夜行人》
苏联
高尔基:《随笔与短篇》、《童年》自传三部曲
肖洛霍夫:《静静的顿河》
布宁:《旧金山来的绅士》、《阿尔谢尼耶夫的一生》
库普林《决斗》
奥斯特洛夫斯基《钢铁是怎样炼成的》
阿列克赛·托尔斯泰《苦难的历程》
布尔加科夫《大师和玛格丽特》
爱伦堡:《解冻》 
艾特玛托夫
索尔仁尼琴:《癌病房》
拉丁美洲
韦拉:《旋涡》
博尔赫斯
巴尔加斯·略萨《城市与狗》

现代主义和后现代主义文学
后象征主义:诗歌
T.S.艾略特:《荒原》
叶芝
瓦雷里:《海滨墓园》
里尔克
梅特林克《青鸟》
庞德
表现主义
卡夫卡:《变形记》、《城堡》
奥尼尔
斯特林堡
恰佩克
未来主义
马里奈蒂:《他们来了》几百字,无人物
阿波利奈尔
马雅可夫斯基
超现实主义
布勒东:《娜佳》
阿拉贡
艾吕雅
意识流小说
詹姆斯·乔伊斯:《尤利西斯》
威廉·福克纳:《喧哗与骚动》
马塞尔·普鲁斯特:《追忆似水年华》意识流开山之作
伍尔夫:《墙上的斑点》
魔幻现实主义
加西亚·马尔克斯:《百年孤独》
阿斯图里亚斯
卡彭特尔
胡安·鲁尔福

后现代主义
存在主义文学
萨特:《禁闭》
加缪
波伏瓦
索尔·贝娄
戈尔丁
荒诞派戏剧
欧仁·尤奈斯库
贝克特:《等待戈多》
品特
阿尔比
新小说
萨洛特《怀疑的时代》、《橡皮》
黑色幽默
海勒:《第二十二条军规》
冯尼古特
品钦:《万有引力之虹》
巴思《烟草经纪人》


亚洲文学
古代
古埃及:《亡灵书》宗教诗歌集
古希伯来人:《旧约》
印度:《吠陀》《摩诃婆罗多》《罗摩衍那》
古巴比伦:《吉尔伽美什》
《圣经》
中古
日本:《万叶集》、《绯谐七部集》、《源氏物语》
阿拉伯:《一千零一夜》
朝鲜:金万重《谢氏南征记》《龙云梦》;《春香传》《沈清传》《兴夫传》
越南:
泰国:
印度:迦梨陀娑
近代
日本
森鸥外:《舞姬》
夏目簌石:《我是猫》、《哥儿》、《旅宿》
川端康成:《伊豆舞女》、《雪国》、《千只鹤》
芥川龙之介
小林多喜二:无阶作家
大江健三郎
印度
泰戈尔:《新月》《园丁集》《飞鸟集》《吉檀迦利》《戈拉》


\end{lstlisting}







\end{document}
