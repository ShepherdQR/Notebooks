%%============
%%  ** Author: Shepherd Qirong
%%  ** Date: 2022-06-05 14:55:53
%%  ** Github: https://github.com/ShepherdQR
%%  ** LastEditors: Shepherd Qirong
%%  ** LastEditTime: 2023-04-02 23:10:27
%%  ** Copyright (c) 2019--20xx Shepherd Qirong. All rights reserved.
%%============


\documentclass[UTF8]{../RepresentationUniverse}
\begin{document}

\title{05-LanguageAndLiterature}
\date{Created on 20220605.\\   Last modified on \today.}
\maketitle
\tableofcontents


\chapter{Introduction}




\section{保存的书单}

% \begin{lstlisting}
% 11
% \end{lstlisting}




\subsection{诺奖梯队}

\begin{lstlisting}
未获奖清单
    托尔斯泰
    契诃夫
    易卜生
    乔伊斯
    普鲁斯特
    卡夫卡
    里尔克
    亨利·詹姆斯
    博尔赫斯
    卡尔维诺

    哈代
    布莱希特
    斯特林堡
    康拉德
    艾斯拉庞德
    马克吐温
    左拉
    洛尔加
    策兰
    纳博科夫
    
    伍尔夫
    尤瑟纳尔
    阿赫马托娃
    茨维塔耶娃

90
    福克纳
    托马斯曼
    贝克特
    TS艾略特

80
    叶芝
    纪德
    海明威
    马尔克斯
    索尔贝娄
    萨拉马戈
    君特格拉斯
    奈保尔:待确认

70
    汉姆生
    萧伯纳:待确认
    尤金奥尼尔
    萨特:待确认
    加缪
    川端康成
    辛格
    米沃什
    库切
    门罗
    卡内蒂
    布罗茨基(美国诗人)
    特朗斯特罗姆
    肖洛霍夫 
    斯坦贝克:待确认

 60   
    高行健
    石黑一雄
    泰戈尔
    黑塞
    莫里亚克
    大江健三郎
    帕慕克
    莱辛
    略萨
    汉德克
    聂鲁达
    罗曼罗兰
    帕斯捷尔纳克
    辛波斯卡

50
    显克微支
    莫言
    耶利内克
    莫迪亚诺
    辛克莱刘易斯
    蒲宁
    赛珍珠
    赫塔米勒
    勒克莱齐奥
    亚力塞维奇
    吉卜林
    索尔仁尼琴
    
40
    蒙森
    罗素
    丘吉尔
    柏格森
    鲍勃·迪伦

\end{lstlisting}


\subsection{没有异议的伟大作家}
\begin{lstlisting}
    莎士比亚,英国;1564~1616;《哈姆雷特》:指导现代英语的发展
    但丁, 意大利;1265~1321;《神曲》:指导现代意大利语发展,揭开文艺复兴序幕,封建主义与资本主义的分界
    荷马, 希腊;约前九世纪~约前八世纪;《荷马史诗》:与圣经一起,西方宗教源头
    托尔斯泰, 俄罗斯;1828~1910;《战争与和平》:批判现实主义的顶峰
    陀思妥耶夫斯基, 俄罗斯;1821~1881;《卡拉马佐夫兄弟》:思想深度之最
    歌德, 德国;1749~1832;《浮士德》:启蒙主义文学的压卷之作
    乔伊斯, 爱尔兰;1882~1941;《尤利西斯》:最伟大的英语小说
    狄更斯, 英国;1812~1870;《远大前程》
    塞万提斯西班牙;1547~1616;《堂吉诃德》:现代小说之父
    曹雪芹, 中国;约1715~约1763;《红楼梦》:中国古代文学巅峰
    乔叟, 英国;1343~1400;《坎特伯雷故事集》:英语文学之父
    维吉尔, 意大利;前70~前19;《埃涅阿斯纪》:最伟大的拉丁语诗人
    杜甫, 中国;712~770;《杜工部集》:现实主义诗歌最高峰
    李白, 中国;701~762;《李太白集》:浪漫主义诗歌最高峰
    屈原, 中国;约前340~前278;《离骚》
    古印度两大史诗:蚁垤, 印度;约前4世纪~约前3世纪;《罗摩衍那》。毗耶娑, 印度;约前4世纪~约前3世纪;《摩诃婆罗多》
    弥尔顿, 英国;1608~1674;《失乐园》:启蒙文学的导师
    普鲁斯特, 法国;1871~1922;《追忆似水年华》:法语文学巅峰
    紫式部, 日本;约973~?;《源氏物语》:奠定了日本“物哀”的文学传统
    索福克勒斯, 希腊;前496~前406;《俄狄浦斯王》:古希腊戏剧巅峰
    普希金, 俄罗斯;1799~1837;《普希金诗集》:奠定现代标准俄语
    福克纳, 美国;1897~1962;《押沙龙,押沙龙!》:挖掘了意识流深度,诺奖天花板
    卡夫卡, 奥地利;1883~1924;《审判》:开辟了现代主义文学
    T.S.艾略特, 英国;1888~1965;《荒原》:最伟大的英语诗人
    福楼拜, 法国;1821~1880;《包法利夫人》:此部小说之后,小说可以和诗歌相提评论
    雨果, 法国;1802~1885;《悲惨世界》
    简·奥斯汀, 英国;1775~1817;《傲慢与偏见》
    乔治·艾略特, 英国;1819~1880;《米德尔马契》
    叶芝, 爱尔兰;1865~1939;《叶芝诗集》:浪漫主义过渡到现代主义
    契诃夫, 俄罗斯;1860~1904;《契诃夫短篇小说集》:现代主义戏剧奠基人
    萨迪, 伊朗;1208~1291;《果园》:波斯诗歌黄金时代最伟大的诗人
    欧里庇得斯, 希腊;前480~前406;《美狄亚》:历史上首次系统地展现妇女问题
    奥维德, 意大利;前43~17;《变形记》:古希腊古罗马神话的总集
    梅尔维尔, 美国;1819~1891;《白鲸》:美国最伟大的文学作品
    埃斯库罗斯, 希腊;前525~前456;《被缚的普罗米修斯》
    莫里哀, 法国;1622~1973;《伪君子》:法国古典喜剧之父
    托马斯·曼, 德国;1875~1955;《魔山》
    易卜生, 挪威;1828~1906;《玩偶之家》:现代戏剧之父
    巴尔扎克,法国;1799~1850;《高老头》。现代法国小说之父。
    阿里斯托芬(希腊;约前446~前385;《鸟》:古希腊喜剧之父。
    惠特曼(美国;1819~1892;《草叶集》。美国诗歌之父。
    庄周(中国;约前369~约前286;《庄子》。道家学说创始人之一。
    济慈(英国;1795~1821;《济慈诗集》。英国浪漫主义诗歌
    威廉·布莱克(英国;1757~1827;《天堂与地狱的婚姻》。英国第一位重要的浪漫主义诗人
    纳博科夫(1899~1977;美国;《洛丽塔》。在美国被誉为“当代小说之王”
    司汤达(法国;1783~1842;《红与黑》。西欧批判现实主义文学的奠基人
    托马斯·哈代(英国;1840~1928;《德伯家的苔丝》。英国维多利亚时代代表作家
    拉伯雷(法国;1494~1553;《巨人传》。欧洲第一部长篇小说
    彼特拉克(意大利;1304~1374;《歌集》。被誉为“人文主义之父”
    迦梨陀娑(印度;约4世纪~约5世纪;《沙恭达罗》:印度的莎士比亚
    萧伯纳(爱尔兰;1856~1950;《伤心之家》。杰出的现实主义剧作家
    波德莱尔(法国;1821~1867;《恶之花》。催生了象征主义、超现实主义
    海明威(美国;1899~1961;《老人与海》。美国“迷惘的一代”的代表作家
    D.H.劳伦斯(英国;1885~1930;《查泰莱夫人的情人》
    贝克特(爱尔兰;1906~1989;《等待戈多》。荒诞派戏剧的代表
    阿贝尔·加缪(法国;1913~1960;《局外人》。“存在主义文学”代表
    乔纳森·斯威夫特(英国;1667~1745;《格列佛游记》
    伍尔芙(英国;1882~1941;《到灯塔去》。意识流文学代表
    爱伦·坡(美国;1809~1849;《爱伦·坡短篇小说集》
    艾米莉·狄金森(美国;1830~1886;《狄金森诗集》
    艾米丽·勃朗特(英国;1818~1848;《呼啸山庄》
    马克·吐温(美国;1835~1910;《哈克贝利·费恩历险记》。美国批判现实主义文学的奠基人
    博尔赫斯(阿根廷;1899~1986;《博尔赫斯短篇小说集》。拉丁美洲小说之父
    夏洛蒂·勃朗特(英国;1816~1855;《简·爱》
    爱弥儿·左拉(法国;1840~1902;《萌芽》。自然主义文学创始人
    薄伽丘(意大利;1313~1375;《十日谈》
    丹尼尔·笛福(英国;1640~1731;《鲁滨逊漂流记》。英国散文之父
    狄德罗(法国;1713~1784;《宿命论者雅克和他的主人》。打破悲剧和喜剧的界限,奠定近代正剧理论。
    斯特林堡(瑞典;1849~1912;《朱丽小姐》。瑞典现代文学之父
    加西亚·马尔克斯(哥伦比亚;1927/1928~2014;《百年孤独》。魔幻现实主义文学最杰出的代表
    劳伦斯·斯特恩(爱尔兰;1713~1768;《项狄传》。《项狄传》可视为意识流文学的嚆矢
    雪莱(英国;1792~1822;《雪莱诗集》。第一位社会主义诗人
    菲尔多西(伊朗;940~1020;《列王纪》。被誉为“东方的荷马”,与萨迪、鲁米和哈菲兹并称为“波斯诗坛四柱”
    拜伦(英国;1788~1824;《唐璜》。欧洲浪漫主义文学的代表
    司马迁(中国;前145/前135~?;《史记》
    让·拉辛(法国;1639~1699;《昂朵马格》。欧洲古典悲剧的典范
    亨利·菲尔丁(英国;1707~1754;《弃儿汤姆·琼斯的个人历史》。英国小说之父
    萨克雷(英国;1811~1863;《名利场》
    华兹华斯(英国;1770~1850;《抒情歌谣集》
    罗伯特·穆齐尔(奥地利;1880~1942;《没有个性的人》。现代主义文学最为重要的作品之一
    聂鲁达(智利;1904~1973;《聂鲁达诗集》
    加西亚·洛尔迦(西班牙;1898~1936;《吉普赛谣曲集》。最伟大的西班牙诗人,完成了西班牙诗歌从古典到现代的转型
    泰戈尔(印度;1861~1941;《吉檀迦利》
    鲁迅(中国;1881~1936;《阿Q正传》。中国现代文学奠基人
    阿契贝(尼日利亚;1930~2013;《瓦解》。非洲文学之父
    司各特(英国;1771~1832;《艾凡赫》欧洲历史文学之父。他的去世标志着浪漫主义文学在英国的结束。
    鲁米(伊朗;1207~1273;《玛斯纳维启示录》。将宗教教义改编为故事诗的苏菲派领袖
    霍桑(美国;1804~1864;《红字》。心理小说之父
    苏轼(中国;1037~1101;《东坡全集》
    F·S·菲茨杰拉德(美国;1896~1940;《了不起的盖茨比》
    纪伯伦(黎巴嫩;1883~1931;《先知》。近代阿拉伯文学奠基人
    果戈里(俄罗斯;1809~1852;《死魂灵》。俄国散文之父。讽刺文学大师。引领批判现实主义文学成为19世纪俄国文坛的主流
    佩索阿(葡萄牙;1888~1935;《惶然录》。被誉为“欧洲现代主义的核心人物”
    夏目漱石(日本;1867~1916;《我是猫》。日本明治维新以来影响最大的作家
    胡安·鲁尔福(墨西哥;1917~1986;《佩德罗·巴拉莫》。魔幻现实主义文学之父
    关汉卿(中国;约1234~约1300;《窦娥冤》。中国古代戏曲的奠基人
    谷崎润一郎(日本;1886~1965;《细雪》。《细雪》被视作日本现代长篇小说的巅峰之作
    世阿弥(日本;1363~1443;《风姿花传》。最杰出的能剧作家和理论家
    安徒生(丹麦;1805~1875;《安徒生童话》。世界童话大王
\end{lstlisting}



\subsection{名著Z-持续收集整理的}

按国别。每个国家按绝对高度排名,后面的可能不准确,前几位较准确。先看各国前几位的。

\subsubsection{南欧}
包括塞尔维亚、科索沃、黑山、克罗地亚、斯洛文尼亚、波斯尼亚和黑塞哥维那、马其顿、罗马尼亚、保加利亚、阿尔巴尼亚、希腊、意大利、梵蒂冈、圣马力诺、马耳他、西班牙、葡萄牙和安道尔。
\begin{lstlisting}
   
\end{lstlisting}
【古罗马】
奥古斯丁《忏悔录》
卢克来修《悟性论》
奥维德《变形记》

【希腊】
荷马:《荷马史诗》(包括《奥德赛》和《伊利亚特》)
古希腊三大悲剧作家:埃斯库罗斯《被缚的普罗米修斯》,索福克勒斯《俄狄浦斯王》,欧里庇得斯《美狄亚》
卡赞扎基斯《希腊佐巴的故事》
苏格拉底
柏拉图
阿里斯多芬


【意大利】
但丁:《神曲》【奥登、乔治·赫伯特,威廉·巴恩斯,博尔赫兹】
维吉尔:《埃涅阿斯纪》
薄伽丘:《十日谈》
莱奥帕尔迪:《诗集》
斯维沃:《泽诺的意识》
莫兰黛:《历史》
皮兰德娄(诺奖)
卡尔维诺:《看不见的城市》
意大利诗人:蒙塔莱(诺奖)
安伯托·艾柯:《玫瑰的名字》



\subsubsection{北欧}
又称北欧五国,是丹麦、挪威、瑞典、芬兰、冰岛
\begin{lstlisting}
    【丹麦】
    安徒生:《安徒生童话故事集》
    克尔凯郭尔

    【挪威】
    哈姆生:《饥饿》
    易卜生:《玩偶之家》《海达·高步乐》
    
    【瑞典】1 部
    林格伦:《长袜子皮皮》
    
    【冰岛】2 部
    《尼雅尔萨迦》
    拉克斯内斯:《独立的人们》
\end{lstlisting}

\subsubsection{西欧}
英国、爱尔兰、荷兰、比利时、卢森堡、法国和摩纳哥。
\begin{lstlisting}

    【英国】
    莎士比亚:《哈姆雷特》《李尔王》《奥赛罗》
    乔治·艾略特:《米德尔马契》《亚当·彼得》【尤朵拉·威尔帝】
    乔叟:《坎特伯雷故事集》【玛丽安·摩尔】
    弥尔顿【失乐园】【哈德雷·布鲁姆,奥克塔维奥·帕斯】
    狄更斯:《远大前程》《大卫·科波菲尔》【本·穷森,菲利普·罗斯】《双城记》
    简·奥斯丁:《傲慢与偏见》《爱玛》
    勃朗特三姐妹:艾米莉·勃朗特(最强,39岁去世)《呼啸山庄》、夏洛蒂·勃朗特(次之,30岁去世)《简·爱》、安妮·勃朗特(29岁去世)《艾格妮丝·格雷》
    康拉德:《诺斯特罗莫》
    劳伦斯:《儿子与情人》《查泰莱夫人的情人》《虹》【赫胥黎】
    奥威尔:《1984》《动物农场-ok》
    弗吉尼亚·伍尔夫:《达洛维夫人》《到灯塔去-ok》
    多丽丝·莱辛:《金色笔记》
    乔纳森·斯威夫特:《格列佛游记》
    萨尔曼·鲁西迪(印度 / 英国):《午夜之子》
    爱丽丝·默多克《钟》
    沃尔特·佩特:《享乐主义者马里乌斯》
    亚历山大·蒲柏
    塞缪尔·约翰逊
    詹姆斯·鲍斯韦尔(传记作家)
    赫胥黎:《美丽新世界》
    丹尼尔·笛福:《鲁滨逊漂流记》
    玛丽·雪莱:《科学怪人》(1818,文学史上第一部科幻小说)
    约翰·罗纳德·瑞尔·托尔金:《霍比特人》《魔戒》
    阿加莎·克里斯蒂:《东方快车谋杀案》《尼罗河上的惨案》; 《无人生还》(首创“童谣杀人”“孤岛形式”)
    伊恩·麦克尤恩:《赎罪》
    威廉·戈尔丁(诺奖):《蝇王》
    道格拉斯·亚当斯:《银河系漫游指南》
    E·M·福斯特:《看得见风景的房间》《印度之行》
    约瑟夫·康拉德:《黑暗的心》
    达芙妮·杜穆里埃:《蝴蝶梦》
    肯尼斯·格雷厄姆:《柳林风声》
    安东尼·伯吉斯:《发条橙》
    C.S 刘易斯:《纳尼亚传奇》《狮子·女巫·魔衣橱》
    理查德·亚当斯:《兔子共和国》
    爱丽丝·沃克:《紫颜色》
    A.A 米尔恩:《小熊维尼》
    托马斯·哈代:《德伯家的苔丝》
    Malcolm Lowry:《火山下》
    伊芙琳·沃:《旧地重游》
    简奥斯汀:《傲慢与偏见》《艾玛》【《曼斯菲尔德花园》,福斯特,特雷弗,狄更斯,艾略特】
    诗人英国:济慈,雪莱,华兹华斯(《序曲》),丁尼生(《悼念》),TS艾略特(诺奖),哈特·克莱恩,乔金娜·罗塞蒂,加百利·罗塞蒂,斯温伯恩,威廉·布莱克,罗伯特·勃朗宁,John Donne(没有人是一座孤岛)
    
    【爱尔兰】
    乔伊斯:《尤利西斯》《一位女士的画像》《一个青年艺术家的肖像》《芬尼根的守灵夜》
    劳伦斯·斯特恩:《项狄传》
    塞缪尔·贝克特(诺奖):戏剧三部曲 《马洛伊》《马洛伊之死》《无名的人》
    奥斯卡·王尔德:《道林·格雷的画像》《童话故事集》
    爱尔兰诗人:叶芝

    【法国】
    普鲁斯特:《追忆逝水年华》
    福楼拜:《包法利夫人》【托马斯曼的《布登勃洛克一家》,阿诺德贝内特的《老妇人的故事》,《嘉莉妹妹》】、《情感教育》
    司汤达:《红与黑》《帕尔马修道院》
    巴尔扎克:《高老头》
    蒙田:《随笔集》
    狄德罗:《宿命论者雅克和他的主人》
    拉伯雷:《巨人传》
    莫泊桑
    左拉
    塞利纳:《茫茫黑夜漫游》
    玛格丽特·尤瑟纳尔:《哈德良回忆录》
    莫里哀:《唐璜》
    萨德侯爵(情色小说鼻祖)
    夏多布里昂
    卢梭
    伏尔泰
    加缪:《局外人》
    萨特
    塞利纳
    纪德
    但格里耶
    西蒙
    图尼埃尔:《少女与死》
    莫迪亚诺:《暗店街》
    勒克莱齐奥
    杜拉斯:《情人》
    大仲马:《基督山伯爵》
    德·圣埃克苏佩里:《小王子》
    雨果《悲惨世界》《巴黎圣母院》【契诃夫,果戈里,普希金,杜康日】
    法国诗人:奈瓦尔,波德莱尔(象征主义)、兰波、魏尔伦,策兰、艾吕雅、阿波利奈尔
\end{lstlisting}

\subsubsection{中欧}
波兰、捷克、斯洛伐克、匈牙利、德国、奥地利、瑞士、列支敦士登
\begin{lstlisting}
    【德国】
    歌德:《浮士德》《威廉·麦斯特》【席勒,卡莱尔】
    托马斯·曼:《布登勃洛克一家》《魔山》【哈罗德·布鲁姆,乔伊斯的尤利西斯,普鲁斯特的追忆似水年华】
    德布林:《柏林,亚历山大广场》
    格拉斯:《铁皮鼓》
    尼采(日神、酒神)
    安妮·弗兰克:《安妮日记》


    【奥地利】
    卡夫卡:《故事全集》《审判》《城堡》《变形记》【罗伯格里耶,格罗斯曼,马尔克斯,鲁尔福,萨特,加缪,福克纳,波德莱尔,爱伦坡,奥尼尔,斯特林堡】
    穆齐尔:《没有个性的人》
    弗洛伊德
    奥地利诗人:赖内·玛利亚·里尔克,霍夫曼斯塔尔

    【中欧其他】
    迪伦马特:《法官和他的刽子手》
\end{lstlisting}



\subsubsection{东欧}
爱沙尼亚、拉脱维亚、立陶宛、白俄罗斯、乌克兰、摩尔多瓦、俄罗斯
\begin{lstlisting}
    【俄罗斯】9 部 
    陀思妥耶夫斯基:《卡拉马佐夫兄弟》《罪与罚》《白痴》《群魔》
    托尔斯泰:《战争与和平》《安娜·卡列尼娜》《伊凡·伊里奇之死及其他》【斯坦尼斯拉夫斯基】
    契诃夫:《小说集》《带狗的女人》
    果戈里:《死魂灵》
    巴别尔《我的第一只鹅》

    一,黄金时代(1815-1900)
    普希金《叶甫盖尼·奥涅金》
    2. 果戈里《死魂灵》
    3. 莱蒙托夫《当代英雄》
    4. 屠格涅夫《猎人笔记》
    5. 托尔斯泰《战争与和平》《安娜·卡列尼娜》
    6. 陀斯妥耶夫斯基《罪与罚》《卡拉马佐夫兄弟》
    7. 契诃夫《万尼亚舅舅》《海鸥》《三姐妹》《樱桃园》
    
    二,白银时代(1900-1917)
    8. 蒲宁《阿尔谢尼耶夫的一生》
    9. 别雷《彼得堡》
    10. 帕斯捷尔纳克《日瓦格医生》
    
    三,青铜时代(1917-1991)
    11. 布尔加科夫《大师与玛格丽特》
    12. 纳博科夫《洛丽塔》《微暗的火》《爱达或爱欲》《说吧,记忆》
    14. 格罗斯曼《生活与命运》
    15. 沙拉莫夫《科雷马故事》
    16. 索尔仁尼琴《古拉格群岛》
    17. 茨普金《巴登夏日》
    18. 索科洛夫《愚人学校》




\end{lstlisting}


\subsubsection{亚洲}
    阿富汗、阿联酋、阿曼、阿塞拜疆、巴基斯坦、巴勒斯坦、巴林、不丹、朝鲜、东帝汶、菲律宾、格鲁吉亚、哈萨克斯坦、韩国、吉尔吉斯斯坦、柬埔寨、卡塔尔、科威特、老挝、黎巴嫩、马尔代夫、马来西亚、蒙古、孟加拉国、缅甸、尼泊尔、日本、沙特阿拉伯、斯里兰卡、塔吉克斯坦、泰国、土耳其、土库曼斯坦、文莱、乌兹别克斯坦、新加坡、叙利亚、亚美尼亚、也门、伊拉克、伊朗、以色列、印度、印度尼西亚、约旦、越南、中国。
\begin{lstlisting}

    【中国】
    曹雪芹:《红楼梦》
    鲁迅:《狂人日记及其他》
    张爱玲(苏青,梅娘,潘柳黛)
    沈从文
    老舍
    茅盾
    贾平凹
    巴金
    曹禺
    钱钟书
    余华
    汪曾祺
    徐志摩
    莫言
    王安忆
    金庸
    周作人
    朱自清
    郁达夫
    戴望舒
    史铁生
    北岛
    孙犁
    王蒙
    艾青
    余光中
    白先勇
    萧红
    路遥
    闻一多
    林语堂
    赵树理
    梁实秋
    郭沫若
    陈忠实《白鹿原》
    张恨水
    苏童
    冰心
    穆旦
    丁玲
    顾城
    舒婷
    张承志
    王朔
    刘震云
    韩少功
    阿城
    张洁
    三毛
    铁凝
    张炜
    李劼人
    宗璞
    郭小川
    柳青
    施蛰存
    张贤亮
    刘恒
    高晓声
    李锐
    徐bai訏

    杨绛
    蒋光慈
    胡兰成
    王小波
    李敖
    余秋雨《文化苦旅》


    【日本】
    紫氏部:《源氏物语》
    夏目漱石:《我是猫》《少爷》《心》
    芥川龙之介
    三岛由纪夫
    村上春树
    东野圭吾
    太宰治
    川端康成:《山音》
    渡边淳一
    岛田庄司
    伊坂幸太郎
    野口英世
    樋口一叶

    【印度】
    《摩诃婆罗多》
    蚁垤:《罗摩衍那》
    迦梨陀娑:《沙恭达罗》
    萨曼·鲁西迪:《午夜之子》

    【亚洲其他地区】
    《吉尔伽美什史诗》(美索不达米亚)
    《约伯记》(以色列)
    萨迪(伊朗):《果园》
    鲁米(伊朗):《玛斯纳维》
    《一千零一夜》(印度/伊朗/伊拉克/埃及)
\end{lstlisting}



\subsubsection{美洲-美国和加拿大}
美洲共35个国家和地区,北边的加拿大和美国,美国以南的33个国家和地区称为拉丁美洲
    【北美洲】阿根廷,巴哈马,伯利兹,美国,玻利维亚,巴西,巴巴多斯,加拿大,哥伦比亚,智利,哥斯达黎加,古巴,委内瑞拉,萨尔瓦多,厄瓜多尔,格林纳达,危地马拉。
    【南美洲】圭亚那,洪都拉斯,海地,牙买加,圣卢西亚,墨西哥,尼加拉瓜,巴拿马,秘鲁,乌拉圭,巴拉圭,苏里南,多米尼加,多米尼克,圣文森特和格林纳丁斯,特立尼达和多巴哥,安提瓜和巴布达,圣基茨和尼维斯。
\begin{lstlisting}
    【美国】
    福克纳:《押沙龙,押沙龙》《喧哗与骚动》《我弥留之际》
    爱伦·坡:《故事全集》
    惠特曼:《草叶集》
    海明威:《老人与海》《太阳照常升起》
    麦尔维尔:《白鲸》
    马克·吐温:《哈克贝利·费恩历险记》《百万英镑》《汤姆索亚历险记》
    纳博科夫(俄罗斯 / 美国):《洛丽塔》
    艾里森:《看不见的人》
    莫里森:《宠儿》
    刘易斯·卡罗尔《爱丽丝梦游仙境》【亨利·米勒】
    菲茨杰拉德《了不起的盖茨比》【田纳西·威廉斯】
    杰克·伦敦:《野性的呼唤》
    薇拉凯瑟
    伊迪斯·华顿《纯真年代》
    霍桑:《红字》【海明威,哈克·菲恩】
    弗兰纳里·奥康纳:《暴力者夺走它》【麦卡锡 《血红色子午线》】
    约翰·斯坦贝克:《愤怒的葡萄》
    哈珀·李:《杀死一只知更鸟》
    塞林格:《麦田守望者》
    路易莎·梅·奥尔科特:《小妇人》
    约翰·欧文:《为欧文·梅尼祈祷》
    约瑟夫·海勒:《第二十二条军规》
    玛格丽特·米切尔:《飘(又译为:乱世佳人》
    托妮·莫里森:《宠儿》
    约翰·斯坦贝克(诺奖1962):《人鼠之间》《愤怒的葡萄》
    杰克·凯鲁亚克(垮掉的一代):《在路上》
    库尔特·冯内古特(黑色幽默):《第五号屠宰场》
    E.B White:《夏洛特的网》
    马里奥·普佐:《教父》
    理查德·赖特:《土生子》
    佐拉·尼尔·赫斯顿:《他们眼望上苍》
    丹·布朗:《达芬奇密码》
    弗兰克·赫伯特:《沙丘》
    卡勒德·胡赛尼:《追风筝的人》
    马德琳·英格:《时间的皱纹》
    肯·克西:《飞越疯人院》
    杰夫.金尼:《小屁孩日记》
    雷·布拉德伯里《华氏451 度》
    美国20世纪三大戏剧家:田纳西·威廉斯《欲望号街车》、尤金·奥尼尔、阿瑟·米勒
    美国诗人:斯蒂文森,罗伯特·弗罗斯特,艾默生;艾米莉·狄金森;伊丽莎白·毕肖普;西尔维娅·普拉斯

    【加拿大】
    玛格丽特·阿特伍德:《使女的故事》
    露西·莫德·蒙哥马利:《绿山墙的安妮》
    
\end{lstlisting}




\subsubsection{拉丁美洲}
\begin{lstlisting}
    【西班牙】
    塞万提斯:《堂吉诃德》【影响了司汤达、福楼拜,梅尔维尔,马克吐温,陀思妥耶夫斯基,屠格涅夫,托马斯曼。巴尔扎克。理查孙的克拉利萨,希尼埃的加米叶,提拔拉斯的台莉,阿瑠斯托的安日丽各,但丁的珐琅彩斯卡,莫里哀的奥塞斯的,普玛西的费加罗,华尔特·司各特的丽贝卡。】
    罗卡:《吉普赛故事诗》
    卡尔德隆,戏剧家
    西班牙诗人:费德里科·加西亚·洛尔卡,路易斯·塞尔努达
    
    【葡萄牙】
    费尔南多·佩索阿 :《惶然录》又译为《不安之书》
    萨拉马戈:《失明症漫记》
    克罗兹《阿马罗神父的罪恶》
    葡萄牙诗人:卡蒙斯,费尔南多·佩索阿

    【墨西哥】
    胡安·鲁尔福(墨西哥):《佩德罗·巴拉莫》

    【巴西】
    罗萨(巴西):《广阔的腹地:条条小路》
    马查多·德·阿西斯:《幻灭三部曲》【苏珊·桑塔格,本杰明·莫泽】
    保罗·科埃略:《牧羊少年奇幻之旅》

    【拉美其他】
    博尔赫斯(阿根廷):《小说集》《沙之书》【戈迪默,拉费里埃】
    马尔克斯(哥伦比亚):《百年孤独》《霍乱时期的爱情》
    卡彭铁尔(古巴)

    拉美诗人:奥克塔维奥(墨西哥,诺奖1990)
\end{lstlisting}


\subsubsection{非洲}
有54个国家:阿尔及利亚、埃及、埃塞俄比亚、安哥拉、贝宁、博茨瓦纳、布基纳法索、布隆迪、赤道几内亚、多哥、厄立特里亚、佛得角、冈比亚、刚果(布)、刚果(金)、吉布提、几内亚、几内亚比绍、加纳、加蓬、津巴布韦、喀麦隆、科摩罗、科特迪瓦、肯尼亚、莱索托、利比里亚、利比亚、卢旺达、马达加斯加、马拉维、马里、毛里求斯、毛里塔尼亚、摩洛哥、莫桑比克、纳米比亚、南非、南苏丹、尼日尔、尼日利亚、塞拉利昂、塞内加尔、塞舌尔、圣多美和普林西比、斯威士兰、苏丹、索马里、坦桑尼亚、突尼斯、乌干达、赞比亚、乍得、中非。
\begin{lstlisting}
    【非洲其他】
    纳吉布·马哈富兹(埃及):《街魂》
    塔伊布·萨利赫(苏丹):《移居北方的时期》
    阿契贝(尼日利亚):《瓦解》
    
\end{lstlisting}


\subsubsection{无国别的-如宗教等}
\begin{lstlisting}

    【宗教系别】
    《旧约》【托马斯·福斯特,王尔德的《莎乐美》】
    《新约》
    默罕默德《古兰经》
    圣经:詹姆士王版本
\end{lstlisting}




% \subsubsection{AAAAA}
% \begin{lstlisting}

% \end{lstlisting}






\subsection{诺贝尔文学奖}

\begin{lstlisting}
    1、1901年苏利·普吕多姆bai(Sully Prudhomme)《孤独与沉思》 法国

    2、1902年特奥多尔·蒙森(daoChristian Theodor Mommsen)《罗马风云》 德国
    
    3、1903年比昂斯滕·比昂松(Bjørnstjerne Martinius Bjørnson)《挑战的手套》 挪威
    
    4、1904年弗雷德里克·米斯塔尔(Frédéric Mistral)《金岛》 法国
    
    何塞·埃切加赖(José Echegaray y Eizaguirre)《伟大的牵线人》 西班牙
    
    5、1905年亨利克·显克维支(Henryk Pius Sienkiewicz)《第三个女人》、《你往何处去》 波兰
    
    6、1906年乔祖埃·卡尔杜齐(Giosuè Alessandro Carducci)《青春诗》 意大利
    
    7、1907年约瑟夫·鲁德亚德·吉卜林(Joseph Rudyard Kipling)《老虎!老虎!【ok】》 英国
    
    8、1908年鲁道尔夫·欧肯(Rudolf Christoph Eucken)《精神生活漫笔》 德国
    
    9、1909年西尔玛·拉格洛夫(Selma Ottilia Lovisa Lagerlöf)《尼尔斯骑鹅旅行记》 瑞典
    
    10、1910年保尔·约翰·路德维希·冯·海塞(Paul Johann Ludwig von Heyse)《特雷庇姑娘》 德国
    
    11、1911年莫里斯·梅特林克(Maurice Polydore Marie Maeterlinck)《青鸟【ok】》、《花的智慧》 比利时
    
    12、1912年盖哈特·霍普特曼(Gerhart Johann Hauptmann)《群鼠》 德国
    
    13、1913年罗宾德拉纳特·泰戈尔(Rabindranath Tagore)《吉檀枷利》、《飞鸟集》 印度
    
    14、1914年未颁奖
    
    15、1915年罗曼·罗兰(Romain Rolland)《约翰·克利斯朵夫》、《名人传》 法国
    
    16、1916年魏尔纳·海顿斯坦姆(Gustaf Verner von Heidenstam)《朝圣年代》 瑞典
    
    17、1917年卡尔·耶勒鲁普(Henrik Pontoppidan)《磨坊血案》 丹麦
    
    亨利克·彭托皮丹(Karl Adolph Gjellerup)《天国》 丹麦
    
    18、1918年未颁奖
    
    19、1919年 卡尔·施皮特勒(Carl Friedrich Spitteler)《奥林比亚的春天》 瑞士
    
    20、1920年克努特·汉姆生(Knut Hamsun)《土地的成长》 挪威
    
    21、1921年阿纳托尔·法郎士(Anatole France)《苔依丝》 法国
    
    22、1922年哈辛特·贝纳文特·伊·马丁内斯(Jacinto Benavente y Martínez)《不吉利的姑娘》 西班牙
    
    23、1923年威廉·勃特勒·叶芝(William Butler Yeats)《丽达与天鹅》、《凯尔特的薄雾》 爱尔兰
    
    24、1924年弗拉迪斯拉夫·莱蒙特(Władysław Stanisław Reymont)《福地》 波兰
    
    25、1925年乔治·萧伯纳(George Bernard Shaw)《皮格马利翁》、《圣女贞德【ok】》 爱尔兰
    
    26、1926年格拉齐亚·黛莱达(Grazia Maria Deledda)《邪恶之路》 意大利
    
    27、1927年亨利·柏格森(Henri-Louis Bergson)《创造进化论》 法国
    
    28、1928年西格里德·温塞特(Sigrid Undset)《新娘—主人—十字架》 挪威
    
    29、1929年保尔·托马斯·曼(Paul Thomas Mann)《布登勃洛克一家》、《魔山》 德国
    
    30、1930年辛克莱·刘易斯(Harry Sinclair Lewis)《大街》、《巴比特》 美国
    
    31、1931年埃利克·阿克塞尔·卡尔费尔德(Erik Axel Karlfeldt)《荒原和爱情》 瑞典
    
    32、1932年约翰·高尔斯华绥(John Galsworthy)《福尔赛世家》 英国
    
    33、1933年伊凡·亚历克塞维奇·蒲宁(Ivan Alekseyevich Bunin)《耶利哥的玫瑰》、《米佳的爱》 俄国
    
    34、1934年路伊吉·皮兰德娄(Luigi Pirandello)《寻找自我》、《六个寻找剧作家的角色》 意大利
    
    35、1935年未颁奖
    
    36、1936年尤金·奥尼尔(Eugene Gladstone O'Neill)《天边外》 美国
    
    37、1937年罗杰·马丁·杜·加尔(Roger Martin du Gard)《蒂伯一家》 法国
    
    38、1938年赛珍珠(Pearl Sydenstricker Buck)《大地》 美国
    
    39、1939年弗兰斯·埃米尔·西兰帕(Frans Eemil Sillanpää)《少女西丽亚》 芬兰
    
    40、1940年到1943年未颁奖
    
    41、1944年约翰内斯·威廉·扬森(Johannes Vilhelm Jensen)《漫长的旅行》 丹麦
    
    42、1945年加夫列拉·米斯特拉尔(Gabriela Mistral)《柔情》 智利
    
    43、1946年赫尔曼·黑塞(Hermann Karl Hesse)《荒原狼》 德国
    
    44、1947年安德烈·纪德(André Paul Guillaume Gide)《田园交响曲》、《背德者》 法国
    
    45、1948年托马斯·斯特恩斯·艾略特(Thomas Stearns Eliot)《荒原》、《四个四重奏》 英国
    
    46、1949年威廉·福克纳(William Cuthbert Faulkner)《八月之光》、《我弥留之际》、《喧哗与骚动》 美国
    
    47、1950年 帕特兰·亚瑟·威廉·罗素(Bertrand Arthur William Russell)《哲学—数学—文学》 英国
    
    48、1951年帕尔·费比安·拉格奎斯特(Pär Fabian Lagerkvist)《大盗巴拉巴》 瑞典
    
    49、1952年弗朗索瓦·莫里亚克(François Charles Mauriac)《给麻风病人的吻》、《爱的荒漠》 法国
    
    50、1953年温斯顿·丘吉尔(Winston Leonard Spencer-Churchill)《不需要的战争》 英国
    
    51、1954年欧内斯特·海明威(Ernest Miller Hemingway)《老人与海【ok】》 美国
    
    52、1955年 赫尔多尔·奇里扬·拉克斯内斯(Halldór Kiljan Laxness)《冰岛之钟》 冰岛
    
    53、1956年胡安·拉蒙·希梅内斯(Juan Ramón Jiménez Mantecón)《小毛驴和我【ok】》、《悲哀的咏叹调》 西班牙
    
    54、1957年阿尔贝·加缪(Albert Camus)《局外人【ok】》、《鼠疫》 法国
    
    55、1958年鲍里斯·列昂尼多维奇·帕斯捷尔纳克(Boris Leonidovich Pasternak)《日瓦戈医生》 苏联
    
    56、1959年萨瓦多尔·夸西莫多(Salvatore Quasimodo)《水与土》 意大利
    
    57、1960年圣琼·佩斯(Saint-John Perse)《蓝色恋歌》 法国
    
    58、1961年伊沃·安德里奇(Ivo Andrić)《桥·小姐》 南斯拉夫
    
    59、1962年约翰·斯坦贝克(John Ernst Steinbeck)《人鼠之间》、《愤怒的葡萄》、《月亮下去了》 美国
    
    60、1963年乔治·塞菲里斯(Giorgos Seferis)《“画眉鸟”号》 希腊
    
    61、1964年让·保罗·萨特(Jean-Paul Charles Aymard Sartre)《词语》 法国
    
    62、1965年米哈伊尔·亚历山大罗维奇·肖洛霍夫(Mikhail Aleksandrovich Sholokhov)《静静的顿河》 苏联
    
    63、1966年萨缪尔·约瑟夫·阿格农(Shmuel Yosef Agnon)《行为之书》 以色列
    
    奈莉·萨克斯(Nelly Sachs)《逃亡》 德国
    
    64、1967年安赫尔·阿斯图里亚斯(Miguel Ángel Asturias Rosales)《玉米人》、《总统先生》 危地马拉
    
    65、1968年川端康成(Yasunari Kawabata)《雪国·千只鹤·古都【ok】》 日本
    
    66、1969年萨缪尔·贝克特(Samuel Barclay Beckett)《等待戈多【ok】》 爱尔兰
    
    67、1970年亚历山大·索尔仁尼琴(Aleksandr Isayevich Solzhenitsyn)《癌病房》、《古拉格群岛》 苏联
    
    68、1971年巴勃鲁·聂鲁达(Pablo Neruda)《情诗·哀诗·赞诗》 智利
    
    69、1972年亨利希·伯尔(Heinrich Theodor Böll)《女士及众生相》 西德
    
    70、1973年帕特里克·怀特(Patrick Victor Martindale White)《风暴眼》 澳大利亚
    
    71、1974年埃温特·约翰逊(Eyvind Johnson)《乌洛夫的故事》 瑞典
    
    72、哈里·埃德蒙·马丁逊(Harry Martinson)《露珠里的世界》 瑞典
    
    73、1975年埃乌杰尼奥·蒙塔莱(Eugenio Montale)《生活之恶》 意大利
    
    74、1976年索尔·贝娄(Saul Bellow)《赫索格》 美国
    
    75、1977年阿莱克桑德雷·梅洛(Vicente Pío Aleixandre y Merlo)《天堂的影子》 西班牙
    
    76、1978年艾萨克·巴什维斯·辛格(Isaac Bashevis Singer)《魔术师·原野王》 美国
    
    77、1979年奥德修斯·埃里蒂斯(Odysseas Elytis)《英雄挽歌》 希腊
    
    78、1980年切斯拉夫·米沃什(Czesław Miłosz)《拆散的笔记簿》、《米沃什词典》 美国/波兰,诗人
    
    79、1981年埃利亚斯·卡内蒂(Elias Canetti)《迷茫》 英国
    
    80、1982年加夫列尔·加西亚·马尔克斯(Gabriel García Márquez)《百年孤独【ok】》、《霍乱时期的爱情》 哥伦比亚
    
    81、1983年威廉·戈尔丁(William Gerald Golding)《蝇王》、《金字塔》 英国
    
    82、1984年雅罗斯拉夫·塞弗尔特(Jaroslav Seifert)《紫罗兰》 捷克斯洛伐克
    
    83、1985年克洛德·西蒙(Claude Simon)《弗兰德公路·农事诗》 法国
    
    84、1986年沃莱·索因卡(Akinwande Wole Soyinka)《雄狮与宝石》 尼日利亚
    
    85、1987年约瑟夫·布罗茨基(Iosif Aleksandrovich Brodsky)《从彼得堡到斯德哥尔摩》 美国,诗人
    
    86、1988年纳吉布·马哈富兹(Naguib Mahfouz)《街魂》 埃及
    
    87、1989年卡米洛·何塞·塞拉(Camilo José Cela y Trulock)《为亡灵弹奏》 西班牙
    
    88、1990年奥克塔维奥·帕斯(Octavio Paz Lozano)《太阳石》 墨西哥
    
    89、1991年内丁·戈迪默(Nadine Gordimer)《七月的人民》 南非
    
    90、1992年德里克·沃尔科特(Derek Alton Walcott)《西印度群岛》 圣卢西亚
    
    91、1993年托尼·莫里森(Toni Morrison)《天堂》、《宠儿》、《所罗门之歌》、《最蓝的眼睛》 美国
    
    92、1994年大江健三郎(Kenzaburō Ōe)《个人的体验》 日本
    
    93、1995年谢默斯·希尼(Seamus Justin Heaney)《一位自然主义者之死》、《通向黑暗之门》、《在外过冬》、《北方》、《野外作业》、《苦路岛》、《山楂灯》 爱尔兰,诗人
    
    94、1996年维斯瓦娃·辛波丝卡(Maria Wisława Anna Szymborska)《我们为此活着》、《向自己提出问题》、《呼唤雪人》、《盐》、《一百种乐趣》、《桥上的历史》、《结束与开始》 波兰
    
    95、1997年达里奥·福(Dario Fo)《喜剧的神秘》、《我们不能也不愿意付钱》、《大胸魔鬼》 意大利
    
    96、1998年若泽·萨拉马戈(José de Sousa Saramago)《里斯本围困史》、《修道院纪事》 葡萄牙
    
    97、1999年君特·格拉斯(Günter Wilhelm Grass)《铁皮鼓》 德国
    
    98、2000年高行健(Gao Xingjian)《灵山》 法国
    
    99、2001年维·苏·奈保尔(Vidiadhar Surajprasad Naipaul)《神秘的按摩师》、《米格尔街》、《大河湾》、《岛上的旗帜》、《超越信仰》、《神秘的新来者》 英国
    
    100、2002年凯尔泰斯·伊姆雷(Imre Kertész)《无形的命运》、《英国旗》、《船夫日记》、《惨败》 匈牙利
    
    101、2003年约翰·马克斯韦尔·库切(John Maxwell Coetzee)《耻》、《彼得堡的大师》 南非
    
    102、2004年埃尔弗里德·耶利内克(Elfriede Jelinek)《利莎的影子》、《钢琴教师》 奥地利
    
    103、2005年哈罗德·品特(Harold Pinter)《看房者》、《生日晚会》、《归家》 英国
    
    104、2006年奥罕·帕慕克(Ferit Orhan Pamuk)《白色城堡》、《我的名字叫红》、《伊斯坦布尔》 土耳其
    
    105、2007年多丽丝·莱辛(Doris May Lessing)《金色笔记》、《幸存者回忆录》、《黑暗前的夏天》 英国
    
    106、2008年勒·克莱齐奥(Jean-Marie Gustave Le Clézio)《战争》、《逃之书》、《墨西哥之梦》、《非洲人》、《诉讼笔录》 法国/毛里求斯
    
    107、2009年赫塔·米勒(Herta Müller)《河水奔流》、《行走界线》、《狐狸那时已是猎人》、《呼吸秋千》、《心兽》 德国
    
    108、2010年马里奥·巴尔加斯·略萨(Jorge Mario Pedro Vargas Llosa)《绿房子》、《世界末日之战》《城市与狗》、《酒吧长谈》、《谁是杀人犯》 西班牙/秘鲁
    
    109、2011年托马斯·特朗斯特罗姆(Tomas Gösta Tranströmer)《17首诗》、《途中的秘密》、《半完成的天空》、《看见黑暗》、《为生者和死者》、《悲哀贡多拉》 瑞典
    
    110、2012年莫言(Mo Yan)《红高粱》、《酒国》、《丰乳肥臀》、《生死疲劳》、《蛙【ok】》 中国
    
    111、2013年爱丽丝·门罗(Alice Ann Munro)《逃离》、《快乐影子之舞》、《爱的进程》、《女孩和女人们的生活》 加拿大
    
    112、2014年帕特里克·莫迪亚诺(Jean Patrick Modiano)《星形广场》、《暗店街》、《青春咖啡馆》 法国
    
    113、2015年斯维特拉娜·阿列克谢耶维奇(Svetlana Alexandrovna Alexievich)《切尔诺贝利的回忆:核灾难口述史》、《最后的见证:失去童年的孩子们》、《战争的非女性面孔》、《最后一个证人》 白俄罗斯
    
    114、2016年鲍勃·迪伦(Bob Dylan)
    
    115、2017年石黑一雄(Kazuo Ishiguro)英国《长日将尽》
    
    116、2018年奥尔嘉·朵卡萩(Olga Tokarczuk)波兰《太古和其他的时间》
    
    117、2019年彼得·汉德克(Peter Handke)奥地利《骂观众》
    
    118、2020年露易丝·格丽克(Louise Glück)美国《直到世界反映了灵魂最深层的需要》《月光的合金》
    
    119、2021年阿卜杜勒拉扎克•格尔纳(Abdulrazak Gurnah) 坦桑尼亚 《天堂》(1994 年)、《荒漠》(2005 年)和海边(2001 年)

    120、2022年安妮•埃尔诺(Annie Ernaux)法国《悠悠岁月【ok】》
\end{lstlisting}





\section{其他的参考使用过的书单}


\subsection{名著C-ok已参考完成}

\begin{lstlisting}
【梯队1】crown
莎士比亚《哈姆雷特》
维吉尔《埃涅阿斯纪》
但丁《神曲》【奥登、乔治·赫伯特,威廉·巴恩斯,博尔赫兹】
塞万提斯《堂吉诃德》【影响了司汤达、福楼拜,梅尔维尔,马克吐温,陀思妥耶夫斯基,屠格涅夫,托马斯曼。巴尔扎克。理查孙的克拉利萨,希尼埃的加米叶,提拔拉斯的台莉,阿瑠斯托的安日丽各,但丁的珐琅彩斯卡,莫里哀的奥塞斯的,普玛西的费加罗,华尔特·司各特的丽贝卡。】
乔叟《坎特伯雷故事集》【玛丽安·摩尔】
蒙田《蒙田随笔》【】
弥尔顿【失乐园】【哈德雷·布鲁姆,奥克塔维奥·帕斯】
托尔斯泰《战争与和平》《安娜卡列尼娜》【斯坦尼斯拉夫斯基】
卢克来修《悟性论》
奥古斯丁《忏悔录》

【梯队2】wisdom
《旧约》【托马斯·福斯特,王尔德的《莎乐美》】
歌德《浮士德》《威廉·麦斯特》【席勒,卡莱尔】
托马斯曼《魔山》【哈罗德·布鲁姆,乔伊斯的尤利西斯,普鲁斯特的追忆似水年华】
《新约》
默罕默德《古兰经》
鲍斯威尔
塞缪尔·约翰逊
弗洛伊德
苏格拉底和柏拉图

【梯队3】understanding
契诃夫《带狗的女人》
莫里哀《唐璜》
易卜生《玩偶之家》《海达·高步乐》
卡夫卡《变形记》【罗伯格里耶,格罗斯曼,马尔克斯,鲁尔福,萨特,加缪,福克纳,波德莱尔,爱伦坡,奥尼尔,斯特林堡】
王尔德
贝克特(诺奖)
皮兰德娄(诺奖)
尼采(日神、酒神)
克尔凯郭尔

【梯队4】love
梅尔维尔《白鲸》
勃朗特《呼啸山庄》
霍桑《红字》【海明威,哈克·菲恩】
简奥斯汀《傲慢与偏见》《艾玛》【《曼斯菲尔德花园》,福斯特,特雷弗,狄更斯,艾略特】
乔纳森·斯威夫特《格列佛游记》
夏洛蒂·勃朗特《简爱》
紫式部《源氏物语》
弗吉尼亚·伍尔夫《到灯塔去》
约翰·唐恩
亚历山大·蒲柏

【梯队5】【诗人】severtity
意大利:莱奥帕尔迪
美国:斯蒂文森,罗伯特·弗罗斯特,艾米丽·狄金森,艾默生
英国:济慈,雪莱,华兹华斯(《序曲》),丁尼生(《悼念》),艾略特(诺奖)

【梯队6】victory
荷马《伊利亚特》
乔伊斯《尤利西斯》
司汤达《红与黑》《帕尔马修道院》
海明威《太阳照常升起》
福克纳《我弥留之际》《喧哗与躁动》《押沙龙,押沙龙》
奥康纳《暴力者夺走它》【麦卡锡 《血红色子午线》】
马克吐温《百万英镑》《汤姆索亚历险记》
卡彭铁尔
卡蒙斯(葡萄牙诗人)
奥克塔维奥(诺奖)

【梯队7】bearuty
雨果《悲惨世界》《巴黎圣母院》【契诃夫,果戈里,普希金,杜康日】
沃尔特·佩特《享乐主义者马里乌斯》
奈瓦尔(法国诗人)
波德莱尔(法国象征派诗人)
霍夫曼斯塔尔,奥地利诗人
斯温伯恩,英国诗人
乔金娜·罗塞蒂和加百利·罗塞蒂,英国诗人

【梯队8】spendor
艾略特《米德尔马契》《亚当·彼得》【尤朵拉·威尔帝】
菲茨杰拉德《了不起的盖茨比》【田纳西·威廉斯】
伊迪斯·华顿《纯真年代》
薇拉凯瑟,
爱丽丝·默多克《钟》
费尔南多·佩索阿,葡萄牙诗人,
费德里科·加西亚·洛尔卡,西班牙诗人;
路易斯·塞尔努达,西班牙诗人;
哈特·克莱恩,英国诗人
惠特曼

【梯队9】fuondation
福楼拜《包法利夫人》【托马斯曼的《布登勃洛克一家》,阿诺德贝内特的《老妇人的故事》,《嘉莉妹妹》】
博尔赫斯《沙之书》【戈迪默,拉费里埃】
卡尔维诺《看不见的城市》
劳伦斯《查泰莱夫人的情人》《虹》【赫胥黎】
马查多·德·阿西斯《幻灭三部曲》【苏珊·桑塔格,本杰明·莫泽】
克罗兹《阿马罗神父的罪恶》
威廉姆斯《欲望号街车》
威廉·布莱克,英国诗人
蒙塔莱(诺奖)
赖内·玛利亚·里尔克,奥地利诗人。

【维度10】kingdom
陀思妥耶夫斯基《卡拉马佐夫兄弟》《罪与罚》
刘易斯·卡罗尔《爱丽丝梦游仙境》【亨利·米勒】
巴尔扎克《高老头》
狄更斯《远大前程》《大卫科波菲尔》【本·穷森,菲利普·罗斯,】
亨利·詹姆斯《一位女士的画像》
巴别尔《我的第一只鹅》
拉尔夫·艾莉森《看不见的人》
叶芝
勃朗宁
策兰

其他:阿里斯多芬,卡尔德隆,欧里庇得斯。
\end{lstlisting}


\subsection{名著B-ok已参考完成}

\begin{lstlisting}

    【希腊】4 部
    荷马:《奥德赛》
    索福克里斯:《俄狄浦斯王》
    欧里庇得斯:《美狄亚》
    卡赞扎基斯《希腊佐巴的故事》
    
    【意大利】7 部
    维吉尔:《埃涅阿斯纪》
    奥维德:《变形记》
    但丁:《神曲》
    薄伽丘:《十日谈》
    莱奥帕尔迪:《诗集》
    斯维沃:《泽诺的意识》
    莫兰黛:《历史》
    
    【英国】16 部 
    乔叟:《坎特伯雷故事集》
    乔纳森·斯威夫特:《格列佛游记》
    莎士比亚:《哈姆雷特》《李尔王》《奥赛罗》
    简·奥斯丁:《傲慢与偏见》
    艾米莉·勃朗特《呼啸山庄》
    乔治·艾略特:《米德尔马契》
    狄更斯:《远大前程》
    康拉德:《诺斯特罗莫》
    劳伦斯:《儿子与情人》
    奥威尔:《1984》
    伍尔夫:《达洛维夫人》《到灯塔去》
    多丽丝·莱辛:《金色笔记》
    萨尔曼·鲁西迪(印度 / 英国):《午夜之子》
    
    【爱尔兰】5 部 
    劳伦斯·斯特恩:《项狄传》
    乔伊斯:《尤利西斯》
    贝克特:戏剧三部曲 《马洛伊》《马洛伊之死》《无名的人》
    
    【法国】12 部 
    拉伯雷:《巨人传》
    蒙田:《随笔集》
    狄德罗:《宿命论者雅克和他的主人》
    司汤达:《红与黑》
    巴尔扎克:《高老头》
    福楼拜:《包法利夫人》、《情感教育》
    普鲁斯特:《追忆逝水年华》
    塞利纳:《茫茫黑夜漫游》
    加缪:《局外人》
    策兰(罗马尼亚 / 法国):《诗集》
    玛格丽特·尤瑟纳尔:《哈德良回忆录》
    
    【德国】5 部
    歌德:《浮士德》
    德布林:《柏林,亚历山大广场》
    托马斯·曼:《布登勃洛克一家》《魔山》
    格拉斯:《铁皮鼓》
    
    【奥地利】4 部 
    穆齐尔:《没有个性的人》
    卡夫卡:《故事全集》《审判》《城堡》
    
    【丹麦】1 部 
    安徒生:《安徒生童话故事集》
    
    【挪威】2 部 
    哈姆生:《饥饿》
    易卜生:《玩偶之家》
    
    【瑞典】1 部
    林格伦:《长袜子皮皮》
    
    【冰岛】2 部
    《尼雅尔萨迦》
    拉克斯内斯:《独立的人们》
    
    【西班牙】2 部 
    塞万提斯:《堂吉诃德》
    罗卡:《吉普赛故事诗》
    
    【葡萄牙】2 部
    费尔南多·佩索阿 :《惶然录》又译为《不安之书》
    萨拉马戈:《失明症漫记》
    
    【俄罗斯】9 部 
    果戈里:《死魂灵》
    陀思妥耶夫斯基:《罪与罚》《白痴》《群魔》《卡拉马佐夫兄弟》
    托尔斯泰:《战争与和平》《安娜·卡列尼娜》《伊凡·伊里奇之死及其他》
    契诃夫:《小说集》
    
    【美国】10 部 
    爱伦·坡:《故事全集》
    惠特曼:《草叶集》
    麦尔维尔:《白鲸》
    马克·吐温:《哈克贝利·费恩历险记》
    福克纳:《押沙龙,押沙龙》《喧哗与骚动》
    海明威:《老人与海》
    纳博科夫(俄罗斯 / 美国):《洛丽塔》
    艾里森:《看不见的人》
    莫里森:《宠儿》
    
    【拉丁美洲地区】5 部 
    罗萨(巴西):《广阔的腹地:条条小路》
    胡安·鲁尔福(墨西哥):《佩德罗·巴拉莫》
    博尔赫斯(阿根廷):《小说集》
    马尔克斯(哥伦比亚):《百年孤独》《霍乱时期的爱情》
    
    【非洲地区】3 部 
    纳吉布·马哈富兹(埃及):《街魂》
    塔伊布·萨利赫(苏丹):《移居北方的时期》
    阿契贝(尼日利亚):《瓦解》
    
    【亚洲地区】11 部 
    《吉尔伽美什史诗》(美索不达米亚)
    《约伯记》(以色列)
    萨迪(伊朗):《果园》
    鲁米(伊朗):《玛斯纳维》
    《摩诃婆罗多》(印度)
    蚁垤(印度):《罗摩衍那》
    迦梨陀娑(印度):《沙恭达罗》
    《一千零一夜》(印度/伊朗/伊拉克/埃及)
    鲁迅(中国):《狂人日记及其他》
    紫氏部(日本):《源氏物语》
    川端康成(日本):《山音》
\end{lstlisting}



\subsection{名著A-ok已参考完成}

1. 了不起的盖茨比(F.斯科特·菲茨杰拉德)

2. 1984(乔治·奥威尔)

3. 杀死一只知更鸟(哈珀·李)

4. 洛丽塔(弗拉基米尔·弗拉基米尔罗维奇·纳博科夫)

5. 麦田守望者(J.D 塞林格)

6. 指环王(J.R.R.托尔金)

7. 22条军规(约瑟夫·海勒)

8. 愤怒的葡萄(约翰·斯坦贝克)

9. 傲慢与偏见(简·奥斯丁)

10. 尤利西斯(詹姆斯·乔伊斯)

11. 百年孤独(加西亚.马尔克斯)

12. 战争与和平(列夫·托尔斯泰)

13. 美丽新世界(奥尔德斯 ·赫胥黎)

14. 简爱(夏洛特·勃朗特)

15. 乱世佳人(玛格丽特·米切尔)

16. 安娜·卡列尼娜(列夫·托尔斯泰)

17. 呼啸山庄(艾米莉·勃朗特)

18. 蝇王(威廉·戈尔丁)

19. 宠儿(托妮·莫里森)

20. 喧哗与骚动(威廉·福克纳)

21. 唐吉可德(塞万提斯)

22. 霍比特人(J.R.R.托尔金)

23. 动物农场(乔治·奥威尔)

24. 哈克贝利·费恩历险记(马克·吐温)

25. 在路上(杰克·凯鲁亚克)

26. 米德尔马契(乔治·艾略特)

27. 看不见的人(拉尔夫·艾里森)

28. 白鲸(赫尔曼 麦尔维尔)

29. 罪与罚(费奥多尔·陀思妥耶夫斯基)

30. 到灯塔去(弗吉尼亚伍尔夫)

31. 五号屠场(库尔特·冯内古特)

32. 小妇人(路易莎·梅·奥尔科特)

33. 远大前程(查尔斯·狄更斯)

34. 银河系搭车客指南(道格拉斯·亚当斯)

35. 包法利夫人(福楼拜)

36. 爱丽丝梦游仙境(刘易斯·卡罗尔)

37. 太阳照常升起(欧内斯特·海明威)

38. 印度之行(E M 福斯特)

39. 双城记(查尔斯·狄更斯)

40. 爱玛(简·奥斯丁)

41. 达洛维夫人(弗吉尼亚伍尔夫)

42. 黑暗的心(约瑟夫·康拉德)

43. 这个世界土崩瓦解了(钦努阿·阿契贝)

44. 卡拉马佐夫兄弟(费奥多尔·陀思妥耶夫斯基)

45. 夏洛特的网(E.B White)

46. 科学怪人(玛丽·雪莱)

47. 蝴蝶梦(达芙妮·杜穆里埃)

48. 人鼠之间(约翰·斯坦贝克)

49. 柳林风声(肯尼斯·格雷厄姆)

50. 发条橙(安东尼·伯吉斯)

51. 基督山伯爵(大仲马)

52. 小王子(德·圣埃克苏佩里)

53. 纳尼亚传奇(C.S 刘易斯)

54. 追忆逝水年华(马塞尔·普鲁斯特)

55. 大卫·科波菲尔(查尔斯·狄更斯)

56. 午夜之子(萨曼·鲁西迪)

57. 审判(弗朗兹·卡夫卡)

58. 安妮日记(安妮·弗兰克)

59. 格列佛游记(乔纳森·斯威夫特)

60. 道林·格雷的画像(奥斯卡·王尔德)

61. 哈姆雷特(威廉·莎士比亚)

62. 土生子(理查德·赖特)

63. 野性的呼唤(杰克·伦敦)

64. 使女的故事(玛格丽特·阿特伍德)

65. 紫颜色(爱丽丝·沃克)

66. 局外人(阿尔伯特·加缪)

67. 他们眼望上苍 (佐拉·尼尔·赫斯顿)

68. 丧钟为谁而鸣(欧内斯特·海明威)

69. 红字(霍桑)

70. 一个青年艺术家的肖像 (詹姆斯·乔伊斯)

71. 圣经:詹姆士王版本

72. 狮子、女巫与魔衣柜(C.S 刘易斯)

73. 悲惨世界(维克多·雨果)

74. 绿山墙的安妮(L.M 蒙哥马利)

75. 钟罩(西尔维娅.普拉斯)

76. 达芬奇密码(丹·布朗)

77. 玫瑰的名字(翁贝托·埃科)

78. 鲁滨逊漂流记(丹尼尔·笛福)

79. 沙丘(弗兰克·赫伯特)

80. 追风筝的人(卡勒德·胡赛尼)

81. 时间的皱纹(英格)

82. 小熊维尼(A.A 米尔恩)

83. 飞越疯人院(肯·克西)

84. 苔丝(托马斯·哈代)

85. 火山下(Malcolm Lowry)

86. 奥德赛(荷马)

87. 我弥留之际(威廉·福克纳)

88. 小屁孩日记(杰夫.金尼)

89. 旧在重游:查尔斯·赖德上尉神圣的渎神回忆(伊芙琳·沃)

90. 华氏451(雷·布拉德伯里)

91. 牧羊少年奇幻之旅(保罗柯艾略)

92. 兔子共和国(理查德·亚当斯)

93. 老人与海(欧内斯特·海明威)

94. 赎罪(伊恩·麦克尤恩)

95. 教父(马里奥·普佐)

96. 无人生还(阿加莎·克里斯蒂)

97. 为欧文·梅尼祈祷(约翰·欧文)

98. 坎特伯雷故事(杰弗里·乔叟)

99. 纯真年代 (伊迪丝.华顿)

100. 一位女士的画像(亨利·詹姆斯)

\end{document}

