%%============
%%  ** Author: Shepherd Qirong
%%  ** Date: 2022-06-05 14:55:53
%%  ** Github: https://github.com/ShepherdQR
%%  ** LastEditors: Shepherd Qirong
%%  ** LastEditTime: 2023-04-29 22:33:16
%%  ** Copyright (c) 2019--20xx Shepherd Qirong. All rights reserved.
%%============


\documentclass[UTF8]{../RepresentationUniverse}
\begin{document}

\title{05-LanguageAndLiterature}
\date{Created on 20220605.\\   Last modified on \today.}
\maketitle
\tableofcontents


\chapter{Introduction}




\section{保存的书单}

% \begin{lstlisting}
% 11
% \end{lstlisting}




\subsection{诺奖梯队}

\begin{lstlisting}
未获奖清单
    托尔斯泰
    契诃夫
    易卜生
    乔伊斯
    普鲁斯特
    卡夫卡
    里尔克
    亨利·詹姆斯
    博尔赫斯
    卡尔维诺

    哈代
    布莱希特
    斯特林堡
    康拉德
    艾斯拉庞德
    马克吐温
    左拉
    洛尔加
    策兰
    纳博科夫
    
    伍尔夫
    尤瑟纳尔
    阿赫马托娃
    茨维塔耶娃

90
    福克纳
    托马斯曼
    贝克特
    TS艾略特

80
    叶芝
    纪德
    海明威
    马尔克斯
    索尔贝娄
    萨拉马戈
    君特格拉斯
    奈保尔:待确认

70
    汉姆生
    萧伯纳:待确认
    尤金奥尼尔
    萨特:待确认
    加缪
    川端康成
    辛格
    米沃什
    库切
    门罗
    卡内蒂
    布罗茨基(美国诗人) 
    特朗斯特罗姆
    肖洛霍夫 
    斯坦贝克:待确认

 60   
    高行健
    石黑一雄
    泰戈尔
    黑塞
    莫里亚克
    大江健三郎
    帕慕克
    莱辛
    略萨
    汉德克
    聂鲁达
    罗曼罗兰
    帕斯捷尔纳克
    辛波斯卡

50
    显克微支
    莫言
    耶利内克
    莫迪亚诺
    辛克莱刘易斯
    蒲宁
    赛珍珠
    赫塔米勒
    勒克莱齐奥
    亚力塞维奇
    吉卜林
    索尔仁尼琴
    
40
    蒙森
    罗素
    丘吉尔
    柏格森
    鲍勃·迪伦

\end{lstlisting}


\subsection{没有异议的伟大作家}
\begin{lstlisting}
    莎士比亚,英国;1564~1616;《哈姆雷特》:指导现代英语的发展
    但丁, 意大利;1265~1321;《神曲》:指导现代意大利语发展,揭开文艺复兴序幕,封建主义与资本主义的分界
    荷马, 希腊;约前九世纪~约前八世纪;《荷马史诗》:与圣经一起,西方宗教源头
    托尔斯泰, 俄罗斯;1828~1910;《战争与和平》:批判现实主义的顶峰
    乔叟, 英国;1343~1400;《坎特伯雷故事集》:英语文学之父
    狄更斯, 英国;1812~1870;《远大前程》《大卫·科波菲尔》
    乔伊斯, 爱尔兰;1882~1941;《尤利西斯》:最伟大的英语小说
    弥尔顿, 英国;1608~1674;《失乐园》:启蒙文学的导师
    维吉尔, 意大利;前70~前19;《埃涅阿斯纪》:最伟大的拉丁语诗人
    歌德, 德国;1749~1832;《浮士德》:启蒙主义文学的压卷之作
    塞万提斯,西班牙;1547~1616;《堂吉诃德》:现代小说之父
    紫式部, 日本;约973~?;《源氏物语》:奠定了日本“物哀”的文学传统
    索福克勒斯, 希腊;前496~前406;《俄狄浦斯王》:古希腊戏剧巅峰
    福克纳, 美国;1897~1962;《押沙龙,押沙龙!》:挖掘了意识流深度,诺奖天花板
    陀思妥耶夫斯基, 俄罗斯;1821~1881;《卡拉马佐夫兄弟》:思想深度之最
    简·奥斯汀, 英国;1775~1817;《傲慢与偏见》
    乔治·艾略特, 英国;1819~1880;《米德尔马契》
    -叶芝, 爱尔兰;1865~1939;《叶芝诗集》:浪漫主义过渡到现代主义
    普希金, 俄罗斯;1799~1837;《普希金诗集》:奠定现代标准俄语
    欧里庇得斯, 希腊;前480~前406;《美狄亚》:历史上首次系统地展现妇女问题
    约翰·多恩(John Donne); 1572~1631; 《日出》《歌谣与十四行诗》《歌与短歌》;玄学派诗人
    梅尔维尔, 美国;1819~1891;《白鲸》:美国最伟大的文学作品
    济慈(英国;1795~1821;《济慈诗集》。英国浪漫主义诗歌
    奥维德, 意大利;前43~17;《变形记》:古希腊古罗马神话的总集
    杜甫, 中国;712~770;《杜工部集》:现实主义诗歌最高峰
    埃斯库罗斯, 希腊;前525~前456;《被缚的普罗米修斯》
    福楼拜, 法国;1821~1880;《包法利夫人》:此部小说之后,小说可以和诗歌相提评论
    卡夫卡, 奥地利;1883~1924;《审判》《城堡》:开辟了现代主义文学
    莫里哀, 法国;1622~1973;《伪君子》:法国古典喜剧之父
    华兹华斯(英国;1770~1850;《抒情歌谣集》
    阿里斯托芬(希腊;约前446~前385;《鸟》《和平》:古希腊喜剧之父。
    托马斯·曼, 德国;1875~1955;《魔山》《布登勃洛克一家》
    易卜生, 挪威;1828~1906;《玩偶之家》:现代戏剧之父
    契诃夫, 俄罗斯;1860~1904;《契诃夫短篇小说集》:现代主义戏剧奠基人
    亨利·詹姆斯, 英国, 1843~1916; 《一位女士的画像》
    纳博科夫(1899~1977;美国;《洛丽塔【ok】》。在美国被誉为“当代小说之王”
    惠特曼(美国;1819~1892;《草叶集》。美国诗歌之父。
    巴尔扎克,法国;1799~1850;《高老头》。现代法国小说之父。
    乔纳森·斯威夫特(英国;1667~1745;《格列佛游记》
    司汤达(法国;1783~1842;《红与黑》。西欧批判现实主义文学的奠基人
    托马斯·哈代(英国;1840~1928;《德伯家的苔丝》。英国维多利亚时代代表作家
    萧伯纳(爱尔兰;1856~1950;《伤心之家》《华伦夫人的职业》。杰出的现实主义剧作家
    -海明威(美国;1899~1961;《老人与海》《永别了,武器》。美国“迷惘的一代”的代表作家
    D.H.劳伦斯(英国;1885~1930;《查泰莱夫人的情人》《儿子与情人》
    波德莱尔(法国;1821~1867;《恶之花》。催生了象征主义、超现实主义
    -贝克特(爱尔兰;1906~1989;《等待戈多》。荒诞派戏剧的代表
    -伍尔芙(英国;1882~1941;《到灯塔去》。意识流文学代表
    亚历山大·蒲柏 (Alexander Pope), 英国; 1688~1744;《夺发记》。为牛顿写墓志铭。
    拉伯雷(法国;1494~1553;《巨人传》。欧洲第一部长篇小说
    彼特拉克(意大利;1304~1374;《歌集》。被誉为“人文主义之父”
    艾米莉·狄金森(美国;1830~1886;《狄金森诗集》
    爱伦·坡(美国;1809~1849;《爱伦·坡短篇小说集》
    亨利·菲尔丁(英国;1707~1754;《弃儿汤姆·琼斯的个人历史》。英国小说之父
    约瑟夫·康拉德, 英国; 1857~1924; 《吉姆爷》《黑暗的心》,海洋小说大师
    罗伯特·布朗宁, 英国; 1812~1889;维多利亚时代第二大诗人.[维多利亚时期的三大诗人阿尔弗莱德·丁尼生、罗伯特·勃朗宁、马修·阿诺德]
    -阿贝尔·加缪(法国;1913~1960;《局外人》。“存在主义文学”代表
    夏洛蒂·勃朗特(英国;1816~1855;《简·爱》
    艾米丽·勃朗特(英国;1818~1848;《呼啸山庄》
    让·拉辛(法国;1639~1699;《昂朵马格》《费德尔》。欧洲古典悲剧的典范
    马克·吐温(美国;1835~1910;《哈克贝利·费恩历险记》《汤姆索亚历险记》。美国批判现实主义文学的奠基人
    斯特林堡(瑞典;1849~1912;《朱丽小姐》《被放逐者》。瑞典现代文学之父
    爱弥儿·左拉(法国;1840~1902;《萌芽》《娜娜》。自然主义文学创始人
    博尔赫斯(阿根廷;1899~1986;《博尔赫斯短篇小说集》。拉丁美洲小说之父
    曹雪芹, 中国;约1715~约1763;《红楼梦》:中国古代文学巅峰
    薄伽丘(意大利;1313~1375;《十日谈》【TTTT】
    伏尔泰, 法国;1694~约1778;《哲学通信》
    劳伦斯·斯特恩(爱尔兰;1713~1768;《项狄传》。《项狄传》可视为意识流文学的嚆矢
    萨克雷(英国;1811~1863;《名利场》
    雪莱(英国;1792~1822;《雪莱诗集》。第一位社会主义诗人
    尤金·奥尼尔,美国;1888~1953;《天边外》
    斯蒂文森,美国; 1879~1955;《秋天的极光》
    拜伦(英国;1788~1824;《唐璜》。欧洲浪漫主义文学的代表
    -加西亚·马尔克斯(哥伦比亚;1927/1928~2014;《百年孤独》。魔幻现实主义文学最杰出的代表
    司各特(英国;1771~1832;《艾凡赫》欧洲历史文学之父。他的去世标志着浪漫主义文学在英国的结束。
    聂鲁达(智利;1904~1973;《聂鲁达诗集》
    罗伯特·穆齐尔(奥地利;1880~1942;《没有个性的人》。现代主义文学最为重要的作品之一
    阿尔弗雷德·丁尼生, 英国, 1809~1892, 维多利亚时期代表诗人。
    弗兰纳里·奥康纳, 美国, 1925~1964;《暴力者夺走它》《短篇小说集》
    卡图卢斯, 古罗马, 约公元前87~约公元前54,诗人。
    加西亚·洛尔迦(西班牙;1898~1936;《吉普赛谣曲集》。最伟大的西班牙诗人,完成了西班牙诗歌从古典到现代的转型
    霍桑(美国;1804~1864;《红字》。心理小说之父
    西奥多·德莱塞, 美国, 1871~1945, 现实主义;《嘉莉妹妹》《美国的悲剧》
    拉尔夫·埃里森(Ralph Ellison),美国, 1914-1994,黑人作家,《隐形人》
    安东尼·特罗洛普, 英国, 1815-1882,《巴塞特郡纪事》
    F·S·菲茨杰拉德(美国;1896~1940;《了不起的盖茨比》
    雨果, 法国;1802~1885;《悲惨世界》
    泰戈尔(印度;1861~1941;《吉檀迦利》
    丹尼尔·笛福(英国;1640~1731;《鲁滨逊漂流记》。英国散文之父
    格拉斯, 德国, 1927-2015, 《铁皮鼓》
    鲁迅(中国;1881~1936;《阿Q正传》。中国现代文学奠基人
    爱德华·福斯特, 英国,1879-1970 《最漫长的旅程》
    艾萨克·辛格, 波裔美籍, 1904-1991《傻瓜吉姆佩尔》
    谷崎润一郎(日本;1886~1965;《细雪》。《细雪》被视作日本现代长篇小说的巅峰之作
    理查德·赖特, 美国, 1908-1960,《土生子》
    斯泰因(女) ,美国, 1874-1946, 《三个女人的一生》
    世阿弥, 日本, 1363-1443,《高砂》
    王尔德, 英国, 1854-1900,《夜莺与玫瑰》《莎乐美》《瑞丁监狱之歌》


==【其他】
    普鲁斯特, 法国;1871~1922;《追忆似水年华》:法语文学巅峰
    威廉·布莱克(英国;1757~1827;《天堂与地狱的婚姻》。英国第一位重要的浪漫主义诗人
    狄德罗(法国;1713~1784;《宿命论者雅克和他的主人》。打破悲剧和喜剧的界限,奠定近代正剧理论。
    菲尔多西(伊朗;940~1020;《列王纪》。被誉为“东方的荷马”,与萨迪、鲁米和哈菲兹并称为“波斯诗坛四柱”
    阿契贝(尼日利亚;1930~2013;《瓦解》。非洲文学之父
    鲁米(伊朗;1207~1273;《玛斯纳维启示录》。将宗教教义改编为故事诗的苏菲派领袖
    纪伯伦(黎巴嫩;1883~1931;《先知》。近代阿拉伯文学奠基人
    果戈里(俄罗斯;1809~1852;《死魂灵》。俄国散文之父。讽刺文学大师。引领批判现实主义文学成为19世纪俄国文坛的主流
    佩索阿(葡萄牙;1888~1935;《惶然录》。被誉为“欧洲现代主义的核心人物”
    夏目漱石(日本;1867~1916;《我是猫》。日本明治维新以来影响最大的作家
    胡安·鲁尔福(墨西哥;1917~1986;《佩德罗·巴拉莫》。魔幻现实主义文学之父
    世阿弥(日本;1363~1443;《风姿花传》。最杰出的能剧作家和理论家
    安徒生(丹麦;1805~1875;《安徒生童话》。世界童话大王
    迦梨陀娑(印度;约4世纪~约5世纪;《沙恭达罗》:印度的莎士比亚
    古印度两大史诗:蚁垤, 印度;约前4世纪~约前3世纪;《罗摩衍那》。毗耶娑, 印度;约前4世纪~约前3世纪;《摩诃婆罗多》
    -T.S.艾略特, 英国;1888~1965;《荒原》:最伟大的英语诗人
    萨迪, 伊朗;1208~1291;《果园》:波斯诗歌黄金时代最伟大的诗人
    
==【中国】
    庄周(中国;约前369~约前286;《庄子》。道家学说创始人之一。
    李白, 中国;701~762;《李太白集》:浪漫主义诗歌最高峰
    苏轼(中国;1037~1101;《东坡全集》
    司马迁(中国;前145/前135~?;《史记》
    -屈原, 中国;约前340~前278;《离骚》
    关汉卿(中国;约1234~约1300;《窦娥冤》。中国古代戏曲的奠基人


\end{lstlisting}


\subsection{中国文学}

底板为“中国-20世纪中国小说100部”。
\begin{lstlisting}

1 《呐喊》 鲁迅
2 《边城》 沈从文
3 《骆驼祥子》 老舍
4 《传奇》 张爱玲
5 《围城》 钱锺书
6 《子夜》 茅盾
7 《台北人》 白先勇
8 《家》 巴金
9 《呼兰河传》 萧红
10 《老残游记》 刘鹗
11 《寒夜》 巴金
12 《彷徨》 鲁迅
13 《官场现形记》 李伯元
14 《财主底儿女们》 路翎
15 《将军族》 陈映真
16 《沉沦》 郁达夫
17 《死水微澜》 李劼人
18 《红高粱》 莫言
19 《小二黑结婚》 赵树理
20 《棋王》 阿城


21 《家变》 王文兴
22 《马桥词典》 韩少功
23 《亚细亚的孤儿》 吴浊流
24 《半生缘》 张爱玲
25 《四世同堂》 老舍
26 《胡雪岩》 高阳
27 《啼笑因缘》 张恨水
28 《儿子的大玩偶》 黄春明
29 《射雕英雄传》 金庸
30 《莎菲女士的日记》 丁玲
31 《鹿鼎记》 金庸
32 《孽海花》 曾朴
33 《惹事》 赖和
34 《嫁妆一牛车》 王祯和
35 《异域》 邓克保(柏扬)
36 《曾国藩》 唐浩明
37 《原乡人》 锺理和
38 《白鹿原》 陈忠实
39 《长恨歌》 王安忆
40 《吉陵春秋》 李永平


41 《黄祸》 保密(王力雄)
42 《狂风沙》 司马中原
43 《艳阳天》 浩然
44 《公墓》 穆时英
45 《旧址》 李锐
46 《星星.月亮.太阳》 徐速
47 《台湾人叁部曲》 锺肇政
48 《洗澡》 杨绛
49 《旋风》 姜贵
50 《荷花淀》 孙犁 
51 《我城》 西西
52 《受戒》 汪曾祺
53 《铁浆》 朱西宁
54 《世纪末华丽》 朱天文
55 《蜀山剑侠传》 还珠楼主
56 《又见棕榈,又见棕榈》 於梨华
57 《浮躁》 贾平凹
58 《组织部新来的年轻人》 王蒙
59 《玉梨魂》 徐枕亚
60《 香港叁部曲》 施叔青


61 《京华烟云》 林语堂
62 《倪焕之》 叶圣陶
63 《春桃》 许地山
64 《桑青与桃红》 聂华苓
65 《蓝与黑》 王蓝
66 《二月》 柔石
67 《风萧萧》 徐言于
68 《芙蓉镇》 古华
69 《地之子》 台静农
70 《城南旧事》 林海音
71 《古船》 张炜
72 《酒徒》 刘以鬯
73 《未央歌》 鹿桥
74 《沉重的翅膀》 张洁
75 《果园城记》 师陀
76 《人啊,人!》 戴厚英
77 《黄金时代》 王小波
78 《狗日的粮食》 刘恒
79 《棋王》 张系国
80 《赖索》 黄凡


81 《妻妾成群》 苏童
82 《霸王别姬》 李碧华
83 《杀夫》 李昂
84 《楚留香》 古龙
85 《窗外》 琼瑶
86 《沉默之岛》 苏伟贞
87 《白发魔女传》 梁羽生
88 《古都》 朱天心
89 《尹县长》 陈若曦
90 《四喜忧国》 张大春
91 《喜宝》 亦舒
92 《男人一半是女人》 张贤亮
93 《将军底头》 施蛰存
94 《蓝血人》 倪匡
95 《二十年目睹之怪现状》 吴趼人
96 《活着》 余华
97 《冈底斯的诱惑》 马原
98 《十年十意》 林斤澜
99 《北极风情画》 无名氏
100 《雍正皇帝》 二月河

\end{lstlisting}



\subsection{名著Z-持续收集整理的}

按国别。每个国家按绝对高度排名,后面的可能不准确,前几位较准确。先看各国前几位的。

\subsubsection{南欧}
包括塞尔维亚、科索沃、黑山、克罗地亚、斯洛文尼亚、波斯尼亚和黑塞哥维那、马其顿、罗马尼亚、保加利亚、阿尔巴尼亚、希腊、意大利、梵蒂冈、圣马力诺、马耳他、西班牙、葡萄牙和安道尔。
\begin{lstlisting}
    【古罗马】
    奥古斯丁《忏悔录》
    卢克来修《悟性论》
    奥维德《变形记》
    
    【希腊】
    荷马:《荷马史诗》(包括《奥德赛》和《伊利亚特》) 
    古希腊三大悲剧作家:埃斯库罗斯《被缚的普罗米修斯》,索福克勒斯《俄狄浦斯王》,欧里庇得斯《美狄亚》
    卡赞扎基斯《希腊佐巴的故事》
    苏格拉底
    柏拉图
    阿里斯多芬
    
    
    【意大利】
    但丁:《神曲》【奥登、乔治·赫伯特,威廉·巴恩斯,博尔赫兹】
    维吉尔:《埃涅阿斯纪》
    薄伽丘:《十日谈》
    莱奥帕尔迪:《诗集》
    斯维沃:《泽诺的意识》
    莫兰黛:《历史》
    皮兰德娄(诺奖) 
    卡尔维诺:《看不见的城市》
    意大利诗人:蒙塔莱(诺奖) 
    安伯托·艾柯:《玫瑰的名字》
\end{lstlisting}




\subsubsection{北欧}
又称北欧五国,是丹麦、挪威、瑞典、芬兰、冰岛
\begin{lstlisting}
    【丹麦】
    安徒生:《安徒生童话故事集》
    克尔凯郭尔

    【挪威】
    哈姆生:《饥饿》
    易卜生:《玩偶之家》《海达·高步乐》
    
    【瑞典】1 部
    林格伦:《长袜子皮皮》
    
    【冰岛】2 部
    《尼雅尔萨迦》
    拉克斯内斯:《独立的人们》
\end{lstlisting}

\subsubsection{西欧}
英国、爱尔兰、荷兰、比利时、卢森堡、法国和摩纳哥。
\begin{lstlisting}

    【英国】
    莎士比亚:《哈姆雷特》《李尔王》《奥赛罗》
    乔治·艾略特:《米德尔马契》《亚当·彼得》【尤朵拉·威尔帝】
    乔叟:《坎特伯雷故事集》【玛丽安·摩尔】
    弥尔顿【失乐园】【哈德雷·布鲁姆,奥克塔维奥·帕斯】
    狄更斯:《远大前程》《大卫·科波菲尔》【本·穷森,菲利普·罗斯】《双城记》
    简·奥斯丁:《傲慢与偏见》《爱玛》
    勃朗特三姐妹:艾米莉·勃朗特(最强,39岁去世) 《呼啸山庄》、夏洛蒂·勃朗特(次之,30岁去世) 《简·爱》、安妮·勃朗特(29岁去世) 《艾格妮丝·格雷》
    劳伦斯:《儿子与情人》《查泰莱夫人的情人》《虹》【赫胥黎】;《恋爱中的女人》(Women in Love) 1920
    弗吉尼亚·伍尔夫:《达洛维夫人》《到灯塔去-ok》
    奥威尔:《1984》《动物农场-ok》
    多丽丝·莱辛:《金色笔记》
    乔纳森·斯威夫特:《格列佛游记》
    萨尔曼·鲁西迪(印度 / 英国) :《午夜之子》
    爱丽丝·默多克《钟》
    沃尔特·佩特:《享乐主义者马里乌斯》
    亚历山大·蒲柏
    赫胥黎:《美丽新世界》;《针锋相对》 (Point Counter Point) 1928
    玛丽·雪莱:《科学怪人》(1818,文学史上第一部科幻小说) 
    Malcolm Lowry:《火山下》(Under the Volcano) 1947
    巴特勒(Samuel Butler) 英国 《众生之路》(The Way of All Flesh) 1903
    库斯勒(Arthur Koestler) 英籍匈牙利 《中午的黑暗》(Darkness at Noon) 1941
    格雷夫斯(Robert Graves) 英国 《我,克劳迪亚斯》(I, Claudius) 1934
    丹尼尔·笛福:《鲁滨逊漂流记》
    约翰·罗纳德·瑞尔·托尔金:《霍比特人》《魔戒》
    阿加莎·克里斯蒂:《东方快车谋杀案》《尼罗河上的惨案》; 《无人生还》(首创“童谣杀人”“孤岛形式”)
    E·M·福斯特:《印度之行》;《霍华德庄园》 (Howards End) 1910;《看得见风景的房间》
    哈谢克(Ford Madox Ford) 英国 《好兵帅克》(The Good Soldier) 1915
    伊夫林•沃(Evelyn Waugh) 英国 《一把尘土》(A Handful of Dust) 1934;《独家新闻》(Scoop ) 1938;《旧地重游》(Brideshead Revisited) 1945
    格林(Graham Greene) 英国 《问题的核心》( The Heart of the Matter) 1948
    威廉·戈尔丁(诺奖) :《蝇王》
    康拉德:《诺斯特罗莫》 (Nostromo) 1904;《黑暗之心》( Heart of Darkness) 1902;《吉姆爷》(Lord Jim) 1900
    鲍威尔(Anthony Powell) 英国 《与时代合拍的舞蹈》(A Dance to the Music of Time) 1975
    福特(Ford Madox Ford) 英国 《行进的目的》 (Parade's End) 1928
    毕尔邦(Max Beerbohm) 英国 《朱莱卡·多卜生》(Zuleika Dobson) 1911
    安东尼·伯吉斯:《发条橙》
    毛姆(W. Somerset Maugham) 英国 《人性的枷锁》 (Of Human Bondage) 1915
    德雷尔(Lawrence Durrell) 英国 《亚历山大四部曲》(The Alexandraia Quartet) 1960
    休斯(Richard Hughes) 英国 《牙买加飓风》 (A Hig hWind in Jamaica) 1929
    奈保尔(V. S. Naipaul) 特里尼达 《毕司沃斯先生的房子》(A House for Mr. Biswas) 1961;《河湾》(A Bend in the River) 1979
    斯帕克(Muriel Spark) 英国 《布罗迪小姐的青春》(The Prime of Miss Jean Brodie) 1961
    吉卜林(Rudyard Kipling) 英国 《基姆》(Kim)  1901
    伊丽莎白•鲍温(Elizabeth Bowen) 英国 《心之死》(The Death of the Heart) 1938
    本涅特(Arnold Bennett) 英国 《老妇人的故事》(The Old Wives' Tale) 1908
    伦敦(Jack London) 英国 《野性的呼唤》(The Call of the Wild) 1903《白牙》《海狼》
    格林(Henry Green) 英国 《爱》(Loving) 1945
    拉什迪(Salman Rushdie) (印裔英籍)  《午夜之子》(Midnight's Children) 1981
    福尔斯(John Fowles) 英国 《巫术师》(The Magus) 1966
    里丝(Jean Rhys) 英国 《辽阔的藻海》(Wide Sargasso) 1966
    莫多克(Iris Murdoch) 英国 《在网下》(Under the Net) 1954
    塞缪尔·约翰逊
    詹姆斯·鲍斯韦尔(传记作家) 
    道格拉斯·亚当斯:《银河系漫游指南》
    约瑟夫·康拉德:《黑暗的心》
    达芙妮·杜穆里埃:《蝴蝶梦》
    肯尼斯·格雷厄姆:《柳林风声》
    C.S 刘易斯:《纳尼亚传奇》《狮子·女巫·魔衣橱》
    理查德·亚当斯:《兔子共和国》
    爱丽丝·沃克:《紫颜色》
    伊恩·麦克尤恩:《赎罪》
    A.A 米尔恩:《小熊维尼》
    托马斯·哈代:《德伯家的苔丝》
    伊芙琳·沃:《旧地重游》
    简奥斯汀:《傲慢与偏见》《艾玛》【《曼斯菲尔德花园》,福斯特,特雷弗,狄更斯,艾略特】
    诗人英国:济慈,雪莱,华兹华斯(《序曲》) ,丁尼生(《悼念》) ,TS艾略特(诺奖) ,哈特·克莱恩,乔金娜·罗塞蒂,加百利·罗塞蒂,斯温伯恩,威廉·布莱克,罗伯特·勃朗宁,John Donne(没有人是一座孤岛) 
    
    
    【爱尔兰】
    乔伊斯:《尤利西斯》《一位女士的画像》《一个青年艺术家的肖像》《芬尼根的守灵夜》
    劳伦斯·斯特恩:《项狄传》
    塞缪尔·贝克特(诺奖) :戏剧三部曲 《马洛伊》《马洛伊之死》《无名的人》
    奥斯卡·王尔德:《道林·格雷的画像》《童话故事集》
    爱尔兰诗人:叶芝

    【法国】
    普鲁斯特:《追忆逝水年华》
    福楼拜:《包法利夫人》【托马斯曼的《布登勃洛克一家》,阿诺德贝内特的《老妇人的故事》,《嘉莉妹妹》】、《情感教育》
    司汤达:《红与黑》《帕尔马修道院》
    巴尔扎克:《高老头》
    蒙田:《随笔集》
    狄德罗:《宿命论者雅克和他的主人》
    拉伯雷:《巨人传》
    莫泊桑
    左拉
    塞利纳:《茫茫黑夜漫游》
    玛格丽特·尤瑟纳尔:《哈德良回忆录》
    莫里哀:《唐璜》
    萨德侯爵(情色小说鼻祖) 
    夏多布里昂
    卢梭
    伏尔泰
    加缪:《局外人》
    萨特
    塞利纳
    纪德
    米歇尔·图尼埃,《礼拜五》
    但格里耶
    西蒙
    图尼埃尔:《少女与死》
    莫迪亚诺:《暗店街》
    勒克莱齐奥
    杜拉斯:《情人》
    大仲马:《基督山伯爵》
    德·圣埃克苏佩里:《小王子》
    雨果《悲惨世界》《巴黎圣母院》【契诃夫,果戈里,普希金,杜康日】
    法国诗人:奈瓦尔,波德莱尔(象征主义) 、兰波、魏尔伦,策兰、艾吕雅、阿波利奈尔
\end{lstlisting}

\subsubsection{中欧}
波兰、捷克、斯洛伐克、匈牙利、德国、奥地利、瑞士、列支敦士登
\begin{lstlisting}
    【德国】
    歌德:《浮士德》《威廉·麦斯特》【席勒,卡莱尔】
    托马斯·曼:《布登勃洛克一家》《魔山》【哈罗德·布鲁姆,乔伊斯的尤利西斯,普鲁斯特的追忆似水年华】
    德布林:《柏林,亚历山大广场》
    格拉斯:《铁皮鼓》
    尼采(日神、酒神) 
    安妮·弗兰克:《安妮日记》


    【奥地利】
    卡夫卡:《故事全集》《审判》《城堡》《变形记》【罗伯格里耶,格罗斯曼,马尔克斯,鲁尔福,萨特,加缪,福克纳,波德莱尔,爱伦坡,奥尼尔,斯特林堡】
    穆齐尔:《没有个性的人》
    弗洛伊德
    茨威格:《人类群星闪耀时》
    奥地利诗人:赖内·玛利亚·里尔克,霍夫曼斯塔尔

    【中欧其他】
    迪伦马特:《法官和他的刽子手》
\end{lstlisting}



\subsubsection{东欧}
爱沙尼亚、拉脱维亚、立陶宛、白俄罗斯、乌克兰、摩尔多瓦、俄罗斯
\begin{lstlisting}
    【俄罗斯】9 部 
    陀思妥耶夫斯基:【先从短篇开始:地下室手记,白夜】《卡拉马佐夫兄弟》《罪与罚》《白痴》《群魔》
    托尔斯泰:《战争与和平》《安娜·卡列尼娜》《伊凡·伊里奇之死及其他》【斯坦尼斯拉夫斯基】
    契诃夫:《小说集》《带狗的女人》
    果戈里:《死魂灵》
    巴别尔《我的第一只鹅》

    一,黄金时代(1815-1900) 
    普希金《叶甫盖尼·奥涅金》
    2. 果戈里《死魂灵》
    3. 莱蒙托夫《当代英雄》
    4. 屠格涅夫《猎人笔记》
    5. 托尔斯泰《战争与和平》《安娜·卡列尼娜》
    6. 陀斯妥耶夫斯基《罪与罚》《卡拉马佐夫兄弟》
    7. 契诃夫《万尼亚舅舅》《海鸥》《三姐妹》《樱桃园》
    
    二,白银时代(1900-1917) 
    8. 蒲宁《阿尔谢尼耶夫的一生》
    9. 别雷《彼得堡》
    10. 帕斯捷尔纳克《日瓦格医生》
    
    三,青铜时代(1917-1991) 
    11. 布尔加科夫《大师与玛格丽特》
    12. 纳博科夫《洛丽塔》《微暗的火》《爱达或爱欲》《说吧,记忆》
    14. 格罗斯曼《生活与命运》
    15. 沙拉莫夫《科雷马故事》
    16. 索尔仁尼琴《古拉格群岛》
    17. 茨普金《巴登夏日》
    18. 索科洛夫《愚人学校》




\end{lstlisting}


\subsubsection{亚洲}
    阿富汗、阿联酋、阿曼、阿塞拜疆、巴基斯坦、巴勒斯坦、巴林、不丹、朝鲜、东帝汶、菲律宾、格鲁吉亚、哈萨克斯坦、韩国、吉尔吉斯斯坦、柬埔寨、卡塔尔、科威特、老挝、黎巴嫩、马尔代夫、马来西亚、蒙古、孟加拉国、缅甸、尼泊尔、日本、沙特阿拉伯、斯里兰卡、塔吉克斯坦、泰国、土耳其、土库曼斯坦、文莱、乌兹别克斯坦、新加坡、叙利亚、亚美尼亚、也门、伊拉克、伊朗、以色列、印度、印度尼西亚、约旦、越南、中国。
\begin{lstlisting}

    【中国】
    曹雪芹:《红楼梦》
    鲁迅:《狂人日记及其他》
    张爱玲(苏青,梅娘,潘柳黛) 
    沈从文
    老舍
    茅盾
    贾平凹
    巴金
    曹禺
    钱钟书
    余华
    汪曾祺
    徐志摩
    莫言
    王安忆
    金庸
    周作人
    朱自清
    郁达夫
    戴望舒
    史铁生
    北岛
    孙犁
    王蒙
    艾青
    余光中
    白先勇
    萧红
    路遥
    闻一多
    林语堂
    赵树理
    梁实秋
    郭沫若
    陈忠实《白鹿原》
    张恨水
    苏童
    冰心
    穆旦
    丁玲
    顾城
    舒婷
    张承志
    王朔
    刘震云
    韩少功
    阿城
    张洁
    三毛
    铁凝
    张炜
    李劼人
    宗璞
    郭小川
    柳青
    施蛰存
    张贤亮
    刘恒
    高晓声
    李锐
    徐bai訏

    杨绛
    蒋光慈
    胡兰成
    王小波
    李敖
    余秋雨《文化苦旅》


    【日本】
    紫氏部:《源氏物语》
    夏目漱石:《我是猫》《少爷》《心》
    芥川龙之介
    三岛由纪夫《金阁寺》《潮骚》《假面的告白》《爱的饥渴》《禁色》
    村上春树
    东野圭吾
    太宰治
    川端康成:《山音》
    渡边淳一
    岛田庄司
    伊坂幸太郎
    野口英世
    樋口一叶

    【印度】
    《摩诃婆罗多》
    蚁垤:《罗摩衍那》
    迦梨陀娑:《沙恭达罗》
    萨曼·鲁西迪:《午夜之子》

    【亚洲其他地区】
    《吉尔伽美什史诗》(美索不达米亚) 
    《约伯记》(以色列) 
    萨迪(伊朗) :《果园》
    鲁米(伊朗) :《玛斯纳维》
    《一千零一夜》(印度/伊朗/伊拉克/埃及) 
\end{lstlisting}



\subsubsection{美洲-美国和加拿大}
美洲共35个国家和地区,北边的加拿大和美国,美国以南的33个国家和地区称为拉丁美洲
    【北美洲】阿根廷,巴哈马,伯利兹,美国,玻利维亚,巴西,巴巴多斯,加拿大,哥伦比亚,智利,哥斯达黎加,古巴,委内瑞拉,萨尔瓦多,厄瓜多尔,格林纳达,危地马拉。
    【南美洲】圭亚那,洪都拉斯,海地,牙买加,圣卢西亚,墨西哥,尼加拉瓜,巴拿马,秘鲁,乌拉圭,巴拉圭,苏里南,多米尼加,多米尼克,圣文森特和格林纳丁斯,特立尼达和多巴哥,安提瓜和巴布达,圣基茨和尼维斯。
\begin{lstlisting}
    【美国】
    福克纳(William Faulkner) 《押沙龙,押沙龙》《喧哗与骚动》《我弥留之际》;《八月之光》(Light in August) 1932
    爱伦·坡:《故事全集》
    惠特曼:《草叶集》
    麦尔维尔:《乌鸦》《白鲸》
    纳博科夫(俄罗斯 / 美国) :《洛丽塔》;《微暗的火》(Pale Fire) 1962
    海明威:《老人与海》《太阳照常升起》《永别了,武器》 (A Farewell to Arms) 1929
    马克·吐温:《哈克贝利·费恩历险记》《百万英镑》《汤姆索亚历险记》
    莫里森:《宠儿》
    刘易斯·卡罗尔《爱丽丝梦游仙境》【亨利·米勒】
    杰克·伦敦:《野性的呼唤》
    薇拉凯瑟

    霍桑:《红字》【海明威,哈克·菲恩】
    弗兰纳里·奥康纳:《暴力者夺走它》【麦卡锡 《血红色子午线》】
    哈珀·李:《杀死一只知更鸟》

    路易莎·梅·奥尔科特:《小妇人》
    约翰·欧文:《为欧文·梅尼祈祷》
    约瑟夫·海勒:《第二十二条军规》
    菲茨杰拉德《了不起的盖茨比》【田纳西·威廉斯】;《夜色温柔》 (Tender Is the Night) 1934
    约翰·斯坦贝克(诺奖1962) :《人鼠之间》《愤怒的葡萄》
    德莱赛(Theodore Dreiser) 美国 《美国的悲剧》(An American Tragedy) 1925
    麦卡勒斯(Carson McCullers) 美国 《心是孤独的猎手》(The Heart Is a Longly Heart) 1940
    库尔特·冯内古特(黑色幽默) :《第五号屠宰场》
    艾里森:《看不见的人》
    理查德·赖特:《土生子》
    贝娄(Saul Bellow) 美国 《雨王亨德森》( Henderson the Rain King) 1959《奥吉•马奇历险记》
    奥哈拉(John O’Hara) 美国 《相约萨马拉》 (Appointment in Samarra) 1934
    多斯•帕索斯(John Dos Passos) 美国 《美国》 三部曲(U.S.A.) 1936
    安德森(Sherwood Anderson) 美国 《小城畸人》 (Winesburg, Ohio) 1919
    詹姆斯(Henry James) 美国 《鸽翼》(The Wings of the Dove) 1902;《使节》(The Ambassadors) 1903;《金碗》(The Golden Bowl) 1904
    法雷尔(James T. Farrell) 美国 《斯塔兹• 朗尼根》三部曲(Studs Lonigan-trilogy) 1935
    德莱赛(Theodore Dreiser) 美国 《嘉莉妹妹》 (Sister Carrie) 1900
    华伦(Robert Penn Warren) 美国 《国王的人马》(All the King’s Men) 1946
    怀尔德(Thornton Wilder) 美国 《圣路易斯雷大桥》(The Bridge of San Luis Rey) 1927
    鲍德温(James Baldwin) 美国 《向苍天呼吁》 (Go Tell It on the Mountain) 1953
    迪基(James Dickey) 美国 《解救》(Deliverance ) 1970
    米勒(Henry Miller) 美国 《北回归线》 (Tropic of Cancer) 1934
    梅勒(Norman Mailer) 美国 《裸者和死者》 (The Naked and Dead) 1948
    罗斯(Philp Roth) 美国 《波特诺伊的怨诉》 (Portnoy's Complaint) 1969
    杰克·凯鲁亚克(垮掉的一代) :《在路上》
    哈米特(Dashiell Hammett) 美国 《马尔他之鹰》(The Maltese Falcon) 1930
    华顿(Edith Wharton) 美国 《纯真年代》(The Age of Innocence) 1920
    珀西(WalkerPercy) 美国 《看电影的人》 (The Moviegoer) 1961
    凯瑟(Willa Cather) 美国 《大主教之死》( Death Comes to Archbishop) 1927
    锺斯(James Ramon Jones) 美国《从这里到永恒》(From Here to Eternity) 1951
    契弗(John Cheever) 美国 《瓦卜肖特纪事》 (The Wapshot Chronicles) 1957
    塞林格(J. D. Salinger) 美国 《麦田的守望者》 (The Catcher in the Rye) 1951
    刘易斯(Sinclair Lewis) 美国 《大街》(Main Street) 1920
    华顿(Edith Wharton) 美国 《欢乐之家》(The House of Mirth) 1905
    韦斯特(Nathaniel West) 美国 《蝗灾之日》 (The Day of the Locust) 1939
    斯泰格纳(Wallace Stegner) 美国 《安息角》(Angle of Repose) 1971
    达可托罗(E. L. Doctorow) 美国 《爵士乐》( Ragtime) 1975
    考德威尔(Erskine Caldwell) 美国 《烟草路》 (Tobacco Road) 1932
    肯尼迪(William Kennedy) 美国 《铁草》( Ironweed) 1983
    斯泰隆(William Styron) 美国 《苏菲的选择》(Sophie's Choice) 1979
    鲍尔斯(Paul Bowles) 美国 《遮蔽的天空》( The Sheltering Sky) 1949
    凯恩(James M. Cain) 美国 《邮差总按两遍铃》(The Postman Always Rings Twice) 1934
    唐利维(J. P. Donleavy) 美国 《眼线》(The Ginger Man) 1955
    塔金顿(Booth Tarkington) 美国《了不起的安德森家族》(The Magnificent Ambersons) 1918
    玛格丽特·米切尔:《飘,又译为:乱世佳人》
    托妮·莫里森:《宠儿》
    E.B White:《夏洛特的网》
    马里奥·普佐:《教父》
    佐拉·尼尔·赫斯顿:《他们眼望上苍》
    丹·布朗:《达芬奇密码》
    弗兰克·赫伯特:《沙丘》
    卡勒德·胡赛尼:《追风筝的人》
    马德琳·英格:《时间的皱纹》
    肯·克西:《飞越疯人院》
    杰夫.金尼:《小屁孩日记》
    雷·布拉德伯里《华氏451 度》
    美国20世纪三大戏剧家:田纳西·威廉斯《欲望号街车》、尤金·奥尼尔、阿瑟·米勒
    美国诗人:斯蒂文森,罗伯特·弗罗斯特,艾默生;艾米莉·狄金森;伊丽莎白·毕肖普;西尔维娅·普拉斯

    【加拿大】
    玛格丽特·阿特伍德:《使女的故事》
    露西·莫德·蒙哥马利:《绿山墙的安妮》
    
\end{lstlisting}




\subsubsection{拉丁美洲}
\begin{lstlisting}
    【西班牙】
    塞万提斯:《堂吉诃德》【影响了司汤达、福楼拜,梅尔维尔,马克吐温,陀思妥耶夫斯基,屠格涅夫,托马斯曼。巴尔扎克。理查孙的克拉利萨,希尼埃的加米叶,提拔拉斯的台莉,阿瑠斯托的安日丽各,但丁的珐琅彩斯卡,莫里哀的奥塞斯的,普玛西的费加罗,华尔特·司各特的丽贝卡。】
    罗卡:《吉普赛故事诗》
    卡尔德隆,戏剧家
    西班牙诗人:费德里科·加西亚·洛尔卡,路易斯·塞尔努达
    
    【葡萄牙】
    费尔南多·佩索阿 :《惶然录》又译为《不安之书》
    萨拉马戈:《失明症漫记》
    克罗兹《阿马罗神父的罪恶》
    葡萄牙诗人:卡蒙斯,费尔南多·佩索阿

    【墨西哥】
    胡安·鲁尔福(墨西哥) :《佩德罗·巴拉莫》

    【巴西】
    罗萨(巴西) :《广阔的腹地:条条小路》
    马查多·德·阿西斯:《幻灭三部曲》【苏珊·桑塔格,本杰明·莫泽】
    保罗·科埃略:《牧羊少年奇幻之旅》

    【拉美其他】
    博尔赫斯(阿根廷) :《小说集》《沙之书》【戈迪默,拉费里埃】
    马尔克斯(哥伦比亚) :《百年孤独》《霍乱时期的爱情》
    卡彭铁尔(古巴) 


    拉美诗人:奥克塔维奥(墨西哥,诺奖1990) 
\end{lstlisting}


\subsubsection{非洲}
有54个国家:阿尔及利亚、埃及、埃塞俄比亚、安哥拉、贝宁、博茨瓦纳、布基纳法索、布隆迪、赤道几内亚、多哥、厄立特里亚、佛得角、冈比亚、刚果(布) 、刚果(金) 、吉布提、几内亚、几内亚比绍、加纳、加蓬、津巴布韦、喀麦隆、科摩罗、科特迪瓦、肯尼亚、莱索托、利比里亚、利比亚、卢旺达、马达加斯加、马拉维、马里、毛里求斯、毛里塔尼亚、摩洛哥、莫桑比克、纳米比亚、南非、南苏丹、尼日尔、尼日利亚、塞拉利昂、塞内加尔、塞舌尔、圣多美和普林西比、斯威士兰、苏丹、索马里、坦桑尼亚、突尼斯、乌干达、赞比亚、乍得、中非。
\begin{lstlisting}
    【非洲其他】
    纳吉布·马哈富兹(埃及) :《街魂》
    塔伊布·萨利赫(苏丹) :《移居北方的时期》
    阿契贝(尼日利亚) :《瓦解》
    
\end{lstlisting}


\subsubsection{无国别的-如宗教等}
\begin{lstlisting}

    【宗教系别】
    《旧约》【托马斯·福斯特,王尔德的《莎乐美》】
    《新约》
    默罕默德《古兰经》
    圣经:詹姆士王版本
\end{lstlisting}




% \subsubsection{AAAAA}
% \begin{lstlisting}

% \end{lstlisting}






\subsection{诺贝尔文学奖}

\begin{lstlisting}
    -1、1901年苏利·普吕多姆(Sully Prudhomme) 《孤独与沉思》 法国

    2、1902年特奥多尔·蒙森(daoChristian Theodor Mommsen) 《罗马风云》 德国
    
    3、1903年比昂斯滕·比昂松(Bjørnstjerne Martinius Bjørnson) 《挑战的手套》 挪威
    
    4、1904年弗雷德里克·米斯塔尔(Frédéric Mistral) 《金岛》 法国
    
        何塞·埃切加赖(José Echegaray y Eizaguirre) 《伟大的牵线人》 西班牙
    
    5、1905年亨利克·显克维支(Henryk Pius Sienkiewicz) 《第三个女人》、《你往何处去》 波兰
    
    6、1906年乔祖埃·卡尔杜齐(Giosuè Alessandro Carducci) 《青春诗》 意大利
    
    7、1907年约瑟夫·鲁德亚德·吉卜林(Joseph Rudyard Kipling) 《老虎!老虎!【ok】》 英国
    
    8、1908年鲁道尔夫·欧肯(Rudolf Christoph Eucken) 《精神生活漫笔》 德国
    
    9、1909年西尔玛·拉格洛夫(Selma Ottilia Lovisa Lagerlöf) 《尼尔斯骑鹅旅行记》 瑞典
    
    10、1910年保尔·约翰·路德维希·冯·海塞(Paul Johann Ludwig von Heyse) 《特雷庇姑娘》 德国
    
    -11、1911年莫里斯·梅特林克(Maurice Polydore Marie Maeterlinck) 《青鸟【ok】》、《花的智慧》 比利时
    
    12、1912年盖哈特·霍普特曼(Gerhart Johann Hauptmann) 《群鼠》 德国
    
    -13、1913年罗宾德拉纳特·泰戈尔(Rabindranath Tagore) 《吉檀枷利》、《飞鸟集》 印度
    
    14、1914年未颁奖
    
    15、1915年罗曼·罗兰(Romain Rolland) 《约翰·克利斯朵夫》、《名人传》 法国
    
    16、1916年魏尔纳·海顿斯坦姆(Gustaf Verner von Heidenstam) 《朝圣年代》 瑞典
    
    17、1917年卡尔·耶勒鲁普(Henrik Pontoppidan) 《明娜》《磨坊血案》 丹麦
    
        亨利克·彭托皮丹(Karl Adolph Gjellerup) 《天国》 丹麦
    
    18、1918年未颁奖
    
    19、1919年 卡尔·施皮特勒(Carl Friedrich Spitteler) 《奥林比亚的春天》 瑞士
    
    20、1920年克努特·汉姆生(Knut Hamsun) 《土地的成长》 挪威
    
    21、1921年阿纳托尔·法郎士(Anatole France) 《苔依丝》 法国
    
    22、1922年哈辛特·贝纳文特·伊·马丁内斯(Jacinto Benavente y Martínez) 《不吉利的姑娘》 西班牙
    
    -23、1923年威廉·勃特勒·叶芝(William Butler Yeats) 《丽达与天鹅》、《凯尔特的薄雾》 爱尔兰
    
    24、1924年弗拉迪斯拉夫·莱蒙特(Władysław Stanisław Reymont) 《福地》 波兰
    
    -25、1925年乔治·萧伯纳(George Bernard Shaw) 《皮格马利翁》、《圣女贞德【ok】》 爱尔兰
    
    26、1926年格拉齐亚·黛莱达(Grazia Maria Deledda) 《邪恶之路》 意大利
    
    27、1927年亨利·柏格森(Henri-Louis Bergson) 《创造进化论》 法国
    
    28、1928年西格里德·温塞特(Sigrid Undset) 《新娘—主人—十字架》 挪威
    
    29、1929年保尔·托马斯·曼(Paul Thomas Mann) 《布登勃洛克一家》、《魔山》 德国
    
    30、1930年辛克莱·刘易斯(Harry Sinclair Lewis) 《大街》、《巴比特》 美国
    
    31、1931年埃利克·阿克塞尔·卡尔费尔德(Erik Axel Karlfeldt) 《荒原和爱情》 瑞典
    
    32、1932年约翰·高尔斯华绥(John Galsworthy) "福尔赛世家"三部曲:《有产业的人》《骑虎》《出租》 英国
    
    33、1933年伊凡·亚历克塞维奇·蒲宁(Ivan Alekseyevich Bunin) 《耶利哥的玫瑰》、《米佳的爱》 俄国
    
    34、1934年路伊吉·皮兰德娄(Luigi Pirandello) 《寻找自我》、《六个寻找剧作家的角色》 意大利
    
    35、1935年未颁奖
    
    36、1936年尤金·奥尼尔(Eugene Gladstone O'Neill) 《天边外》 美国
    
    37、1937年罗杰·马丁·杜·加尔(Roger Martin du Gard) 《蒂伯一家》 法国
    
    38、1938年赛珍珠(Pearl Sydenstricker Buck) 《大地》 美国
    
    39、1939年弗兰斯·埃米尔·西兰帕(Frans Eemil Sillanpää) 《少女西丽亚》 芬兰
    
    40、1940年到1943年未颁奖
    
    41、1944年约翰内斯·威廉·扬森(Johannes Vilhelm Jensen) 《漫长的旅行》 丹麦
    
    42、1945年加夫列拉·米斯特拉尔(Gabriela Mistral) 《柔情》 智利
    
    43、1946年赫尔曼·黑塞(Hermann Karl Hesse) 《荒原狼》 德国
    
    44、1947年安德烈·纪德(André Paul Guillaume Gide) 《田园交响曲》、《背德者》 法国
    
    -45、1948年托马斯·斯特恩斯·艾略特(Thomas Stearns Eliot) 《荒原》、《四个四重奏》 英国
    
    46、1949年威廉·福克纳(William Cuthbert Faulkner) 《八月之光》、《我弥留之际》、《喧哗与骚动》 美国
    
    47、1950年 帕特兰·亚瑟·威廉·罗素(Bertrand Arthur William Russell) 《哲学—数学—文学》 英国
    
    48、1951年帕尔·费比安·拉格奎斯特(Pär Fabian Lagerkvist) 《大盗巴拉巴》 瑞典
    
    49、1952年弗朗索瓦·莫里亚克(François Charles Mauriac) 《给麻风病人的吻》、《爱的荒漠》 法国
    
    50、1953年温斯顿·丘吉尔(Winston Leonard Spencer-Churchill) 《不需要的战争》 英国
    
    -51、1954年欧内斯特·海明威(Ernest Miller Hemingway) 《老人与海【ok】》 美国
    
    52、1955年 赫尔多尔·奇里扬·拉克斯内斯(Halldór Kiljan Laxness) 《冰岛之钟》 冰岛
    
    -53、1956年胡安·拉蒙·希梅内斯(Juan Ramón Jiménez Mantecón) 《小毛驴和我【ok】》、《悲哀的咏叹调》 西班牙
    
    -54、1957年阿尔贝·加缪(Albert Camus) 《局外人【ok】》、《鼠疫》 法国
    
    55、1958年鲍里斯·列昂尼多维奇·帕斯捷尔纳克(Boris Leonidovich Pasternak) 《日瓦戈医生》 苏联
    
    56、1959年萨瓦多尔·夸西莫多(Salvatore Quasimodo) 《水与土》 意大利
    
    57、1960年圣琼·佩斯(Saint-John Perse) 《蓝色恋歌》 法国
    
    58、1961年伊沃·安德里奇(Ivo Andrić) 《桥·小姐》 南斯拉夫
    
    59、1962年约翰·斯坦贝克(John Ernst Steinbeck) 《人鼠之间》、《愤怒的葡萄》、《月亮下去了》 美国
    
    60、1963年乔治·塞菲里斯(Giorgos Seferis) 《“画眉鸟”号》 希腊
    
    61、1964年让·保罗·萨特(Jean-Paul Charles Aymard Sartre) 《词语》 法国
    
    62、1965年米哈伊尔·亚历山大罗维奇·肖洛霍夫(Mikhail Aleksandrovich Sholokhov) 《静静的顿河》 苏联
    
    63、1966年萨缪尔·约瑟夫·阿格农(Shmuel Yosef Agnon) 《行为之书》 以色列
    
    奈莉·萨克斯(Nelly Sachs) 《逃亡》 德国
    
    64、1967年安赫尔·阿斯图里亚斯(Miguel Ángel Asturias Rosales) 《玉米人》、《总统先生》 危地马拉
    
    -65、1968年川端康成(Yasunari Kawabata) 《雪国·千只鹤·古都【ok】》 日本
    
    66、1969年萨缪尔·贝克特(Samuel Barclay Beckett) 《等待戈多【ok】》 爱尔兰
    
    67、1970年亚历山大·索尔仁尼琴(Aleksandr Isayevich Solzhenitsyn) 《癌病房》、《古拉格群岛》 苏联
    
    -68、1971年巴勃鲁·聂鲁达(Pablo Neruda) 《情诗·哀诗·赞诗》 智利
    
    69、1972年亨利希·伯尔(Heinrich Theodor Böll) 《女士及众生相》 西德
    
    70、1973年帕特里克·怀特(Patrick Victor Martindale White) 《风暴眼》 澳大利亚
    
    71、1974年埃温特·约翰逊(Eyvind Johnson) 《乌洛夫的故事》 瑞典
    
    72、哈里·埃德蒙·马丁逊(Harry Martinson) 《露珠里的世界》 瑞典
    
    73、1975年埃乌杰尼奥·蒙塔莱(Eugenio Montale) 《生活之恶》 意大利
    
    74、1976年索尔·贝娄(Saul Bellow) 《赫索格》 美国
    
    75、1977年阿莱克桑德雷·梅洛(Vicente Pío Aleixandre y Merlo) 《天堂的影子》 西班牙
    
    76、1978年艾萨克·巴什维斯·辛格(Isaac Bashevis Singer) 《魔术师·原野王》 美国
    
    77、1979年奥德修斯·埃里蒂斯(Odysseas Elytis) 《英雄挽歌》 希腊
    
    -78、1980年切斯拉夫·米沃什(Czesław Miłosz) 《拆散的笔记簿》、《米沃什词典》 美国/波兰,诗人
    
    79、1981年埃利亚斯·卡内蒂(Elias Canetti) 《迷茫》 英国
    
    80、1982年加夫列尔·加西亚·马尔克斯(Gabriel García Márquez) 《百年孤独【ok】》、《霍乱时期的爱情》 哥伦比亚
    
    81、1983年威廉·戈尔丁(William Gerald Golding) 《蝇王》、《金字塔》 英国
    
    82、1984年雅罗斯拉夫·塞弗尔特(Jaroslav Seifert) 《紫罗兰》 捷克斯洛伐克
    
    83、1985年克洛德·西蒙(Claude Simon) 《弗兰德公路·农事诗》 法国
    
    84、1986年沃莱·索因卡(Akinwande Wole Soyinka) 《雄狮与宝石》 尼日利亚
    
    85、1987年约瑟夫·布罗茨基(Iosif Aleksandrovich Brodsky) 《从彼得堡到斯德哥尔摩》 美国,诗人
    
    86、1988年纳吉布·马哈富兹(Naguib Mahfouz) 《街魂》 埃及
    
    87、1989年卡米洛·何塞·塞拉(Camilo José Cela y Trulock) 《为亡灵弹奏》 西班牙
    
    88、1990年奥克塔维奥·帕斯(Octavio Paz Lozano) 《太阳石》 墨西哥
    
    89、1991年内丁·戈迪默(Nadine Gordimer) 《七月的人民》 南非
    
    90、1992年德里克·沃尔科特(Derek Alton Walcott) 《西印度群岛》 圣卢西亚
    
    91、1993年托尼·莫里森(Toni Morrison) 《天堂》、《宠儿》、《所罗门之歌》、《最蓝的眼睛》 美国
    
    92、1994年大江健三郎(Kenzaburō Ōe) 《个人的体验》 日本
    
    93、1995年谢默斯·希尼(Seamus Justin Heaney) 《一位自然主义者之死》、《通向黑暗之门》、《在外过冬》、《北方》、《野外作业》、《苦路岛》、《山楂灯》 爱尔兰,诗人
    
    94、1996年维斯瓦娃·辛波丝卡(Maria Wisława Anna Szymborska) 《我们为此活着》、《向自己提出问题》、《呼唤雪人》、《盐》、《一百种乐趣》、《桥上的历史》、《结束与开始》 波兰
    
    95、1997年达里奥·福(Dario Fo) 《喜剧的神秘》、《我们不能也不愿意付钱》、《大胸魔鬼》 意大利
    
    96、1998年若泽·萨拉马戈(José de Sousa Saramago) 《里斯本围困史》、《修道院纪事》 葡萄牙
    
    97、1999年君特·格拉斯(Günter Wilhelm Grass) 《铁皮鼓》 德国
    
    98、2000年高行健(Gao Xingjian) 《灵山》 法国
    
    99、2001年维·苏·奈保尔(Vidiadhar Surajprasad Naipaul) 《神秘的按摩师》、《米格尔街》、《大河湾》、《岛上的旗帜》、《超越信仰》、《神秘的新来者》 英国
    
    100、2002年凯尔泰斯·伊姆雷(Imre Kertész) 《无形的命运》、《英国旗》、《船夫日记》、《惨败》 匈牙利
    
    -101、2003年约翰·马克斯韦尔·库切(John Maxwell Coetzee) 《耻》、《彼得堡的大师》 南非
    
    102、2004年埃尔弗里德·耶利内克(Elfriede Jelinek) 《利莎的影子》、《钢琴教师》 奥地利
    
    103、2005年哈罗德·品特(Harold Pinter) 《看房者》、《生日晚会》、《归家》 英国
    
    104、2006年奥罕·帕慕克(Ferit Orhan Pamuk) 《白色城堡》、《我的名字叫红》、《伊斯坦布尔》 土耳其
    
    105、2007年多丽丝·莱辛(Doris May Lessing) 《金色笔记》、《幸存者回忆录》、《黑暗前的夏天》 英国
    
    106、2008年勒·克莱齐奥(Jean-Marie Gustave Le Clézio) 《战争》、《逃之书》、《墨西哥之梦》、《非洲人》、《诉讼笔录》 法国/毛里求斯
    
    107、2009年赫塔·米勒(Herta Müller) 《河水奔流》、《行走界线》、《狐狸那时已是猎人》、《呼吸秋千》、《心兽》 德国
    
    108、2010年马里奥·巴尔加斯·略萨(Jorge Mario Pedro Vargas Llosa) 《绿房子》、《世界末日之战》《城市与狗》、《酒吧长谈》、《谁是杀人犯》 西班牙/秘鲁
    
    -109、2011年托马斯·特朗斯特罗姆(Tomas Gösta Tranströmer) 《17首诗》、《途中的秘密》、《半完成的天空》、《看见黑暗》、《为生者和死者》、《悲哀贡多拉》 瑞典
    
    -110、2012年莫言(Mo Yan) 《红高粱》、《酒国》、《丰乳肥臀》、《生死疲劳》、《蛙【ok】》 中国
    
    111、2013年爱丽丝·门罗(Alice Ann Munro) 《逃离》、《快乐影子之舞》、《爱的进程》、《女孩和女人们的生活》 加拿大
    
    112、2014年帕特里克·莫迪亚诺(Jean Patrick Modiano) 《星形广场》、《暗店街》、《青春咖啡馆》 法国
    
    113、2015年斯维特拉娜·阿列克谢耶维奇(Svetlana Alexandrovna Alexievich) 《切尔诺贝利的回忆:核灾难口述史》、《最后的见证:失去童年的孩子们》、《战争的非女性面孔》、《最后一个证人》 白俄罗斯
    
    114、2016年鲍勃·迪伦(Bob Dylan) 
    
    115、2017年石黑一雄(Kazuo Ishiguro) 英国《长日将尽》
    
    116、2018年奥尔嘉·朵卡萩(Olga Tokarczuk) 波兰《太古和其他的时间》
    
    117、2019年彼得·汉德克(Peter Handke) 奥地利《骂观众》
    
    118、2020年露易丝·格丽克(Louise Glück)美国《直到世界反映了灵魂最深层的需要》《月光的合金》
    
    119、2021年阿卜杜勒拉扎克•格尔纳(Abdulrazak Gurnah) 坦桑尼亚 《天堂》(1994 年) 、《荒漠》(2005 年) 和海边(2001 年) 

    -120、2022年安妮•埃尔诺(Annie Ernaux) 法国《悠悠岁月【ok】》
\end{lstlisting}






\subsection{世界文学史}

聂珍钊《外国文学史》
李斌宁《欧洲文学史》
郑克鲁《外国文学史》
唐建清《欧美文学研究引导》
徐葆耕《西方文学十五讲》
朱维之《外国文学史》
王忠祥《外国文学教程》
王佐良《英国散文的流变》
陈琨《西方现代派文学研究》
威尔逊,特里林, 诺顿六讲。

朱光潜《西方美学史》
丹纳《艺术哲学》
罗素《西方哲学史》


\begin{lstlisting}


《希腊的神话和传说》(斯威布编)
荷马史诗:《伊利亚特》、《奥德赛》
索福克勒斯:《俄狄浦斯王》【重要】
欧里庇得斯:《美狄亚》【重要】
但丁:《神曲》
卜伽丘:《十日谈》
拉伯雷:《巨人传》
塞万提斯:《堂吉诃德》
莎士比亚;《哈姆莱特【重要】》《奥瑟罗【重要】》《威尼斯商人》
莫里哀:《伪君子》
笛福:《鲁滨逊漂流记》
斯威夫特:《格列佛游记》
卢梭:《忏悔录》
博马舍:《费加罗的婚姻》
席勒:《阴谋与爱情》
歌德:《浮士德》、《少年维特的烦恼【重要】》
拜伦:《唐璜》
雪莱;《西风颂》
雨果:《巴黎圣母院》《悲惨世界》





BC5000,埃及的成体系的神话
BC3000,埃及《亡灵书》

BC2100, 世上最早的史诗,巴比伦的《吉尔伽美什》

《格萨尔王》,中国藏族英雄史诗,长。

BC600, 文字的《荷马史诗》


希伯来的旧约,
印度,吠陀经

东方文学:到AC500是古代文学。AC500-19世纪中叶是中古时期,19世纪中叶到20世纪初是近代文学,20世纪以来是现当代文学。

西方文学:

古代:古希腊古罗马是源头。古希腊的神话、戏剧、史诗。
古希腊古罗马:BC1200-AC500

古希腊:BC3000-BC1200,克里特-迈锡尼文明;
BC1200-BC800,荷马时代,英雄时代。
BC800-BC600,大移民时代。
BC500-BC400,古典时代。全盛时期。
BC338-BC146,希腊化时期。BC146,马其顿被罗马征服。
自由民主。理性主义与狂欢精神。


到中古,意大利但丁《神曲》,14世纪初。
14世纪文艺复兴到20世纪,称为近代。20世纪上是现代,20世纪下至今是当代。
17世纪,古典主义,法国古典主义莫里哀的喜剧
18世纪,启蒙主义,启蒙运动法国卢梭,德国歌德
19世纪,浪漫主义,批判现实,唯美主义(王尔德),左拉(自然主义),象征主义(前期现代主义),
20世纪,现代主义(各种思潮,意识流,荒诞派戏剧,黑色幽默,存在主义)




古希腊
希腊神话
荷马史诗
伊索寓言

古典时期(6bc-4bc)
戏剧:埃斯库罗斯《俄瑞斯忒亚》《被缚的普罗米修斯》、索福克勒斯《俄狄浦斯王》、欧里庇得斯《美狄亚》;阿里斯托芬《鸟》
散文:希罗多德、修昔底德
文艺理论:柏拉图《理想国》、亚里士多德《诗学》

古罗马文学
共和时期(bc240-bc30)
利维乌斯·安德罗尼库斯《奥德修纪》
普劳图斯 改编喜剧
黄金时期(bc100-17)
卢克莱修《论自然》
奥维德《变形记》史诗光怪陆离故事
维吉尔《埃涅阿斯纪》第一部文人史诗 与罗马帝国有关
白银时代(17-130)
尤维纳利斯 讽刺作家


中世纪文学
教会文学
史诗、歌谣
《贝奥武普》欧洲最完整,北欧氏族阶段在大陆的生活
《罗兰之歌》法国,查理的大帝时代爱国英雄罗兰
骑士文学
亚瑟王与圆桌骑士
市民文学/城市文学
叙事诗《列那狐传奇》《玫瑰传奇》
抒情诗:吕特勃夫、维庸
戏剧《巴特兰律师》
但丁《神曲》


文艺复兴时期文学
意大利文学
但丁
彼特拉克
薄伽丘
法国:贵族、平民两种倾向
龙沙
弗朗索瓦·拉伯雷《巨人传》
蒙田《随笔集》怀疑论、欧洲近代散文创始人
西班牙
流浪汉小说《小癞子》
塞万提斯:“现代小说之父”《堂吉诃德》、《训诫小说集》
戏剧:洛卜·德·维加《羊泉村》
英国
杰弗利·乔叟《特洛伊拉斯和克莱西德》仿薄伽丘
托马斯·莫尔《乌托邦》对话体幻想小说
埃德蒙·斯宾塞《仙后》长诗
戏剧“大学才子派”
莎士比亚


17世纪文学
古典主义文学
始于法国
勒内·笛卡儿:唯理主义《方法论》、《心灵情感论》
弗朗瓦索·德·马莱布《劝慰杜佩里埃先生》
古典主义悲剧
皮埃尔·高乃依:古典主义悲剧创始人《熙德》、《贺拉斯》
让·拉辛《安德洛玛克》因私废公
让·德·拉封丹《寓言诗》寓言大成者
尼古拉·布瓦洛:理论家《诗的艺术》
莫里哀:戏剧《唐璜》贵族腐朽、《伪君子》宗教骗子
巴洛克文学
(16c-17c)精雕细刻;起源于意大利、西班牙,兴盛于法国
马里诺《阿多尼斯》维纳斯与美少年的爱情纠葛
阿尔戈特:长诗
《莱尔玛公爵颂》《孤独》
佩特罗·卡尔德隆《人生如梦》
法国小说:奥诺雷·德·于尓菲《阿丝特蕾》牧羊男女的故事
清教徒文学
约翰·弥尔顿:《失乐园》史诗 赞美撒旦的反抗、《复乐园》耶稣拒绝撒旦诱惑、《力士参孙》参孙憾大柱复仇
约翰·班扬《天路历程》梦境寓意,复辟时期腐败


18世纪文学
现实主义小说
丹尼尔·笛福《鲁宾逊漂流记》
约拿旦·斯威夫特《格列佛游记》
理查生:英国家庭小说开创者《帕米拉》《克拉丽莎》婚姻自主同中产温和的道德说教
托比亚·斯摩莱特《兰豋传》流浪汉小说,讽刺社会现实

伤感主义文学
劳伦斯·斯泰恩《伤感的旅行》
奥立维·哥尔德斯密斯《威克菲尔德的牧师》
法国
孟德斯鸠《法的精神》、《波斯人信札》第一部启蒙哲理小说
伏尔泰:启蒙哲理小说《老实人》讽刺盲目乐观,否定悲观,回到苦难现实、《如此世界》、《查第格》、《天真汉》
德尼·狄德罗:《百科全书》、《当好人还是坏人?》奠定现代话剧基础、《拉摩的侄儿》对话体小说、《宿命论者雅克和他的主人》
德国
高特舍德
高·埃·莱辛:戏剧《拉奥孔》、《汉堡剧评》、《艾米丽亚·迦洛蒂》
维兰德:小说家
狂飙突进运动,古典文学时期
歌德:《少年维特的烦恼》、《普罗米修斯》长诗、《浮士德》、《赫尔曼与窦绿苔》总结;《亲和力》、《诗与真》自传性作品、《威廉·迈斯特》教育小说、《西东合集》诗集
席勒
卢梭:《社会契约论》、《新爱洛依丝》、《爱弥儿》教育小说、《忏悔录》发掘自我


19世纪

19世纪浪漫主义文学
18世纪末产生,19世纪上半叶繁荣;个人独立和极端自由
施莱格尔兄弟
斯塔尔夫人
夏多布里昂《基督教真谛》提供散文、小说典范
拜伦、雪莱、济慈
雨果
英国====
先驱: 罗伯特·彭斯、威廉·布莱克
开创:湖畔派:华兹华斯《抒情歌谣集》、柯勒律治、罗伯特·骚塞
拜伦:《唐璜》叙事诗冒险、讽刺、自由
雪莱
德国====
克莱斯特
霍夫曼《金罐》
沙米索
法国====
雨果:(对比)诗歌《静观集》、戏剧、小说《巴黎圣母院》、《悲惨世界》、《笑面人》、《九三年》
大仲马:《三个火枪手》人物、《基督山伯爵》
乔治·桑:妇女问题、社会问题、田园小说《魔沼》
缪塞
美国====
爱默生
梭罗
华盛顿·欧文
詹姆斯·费尼莫·库柏:边疆传奇小说《最后一个莫西干人》
埃德加·爱伦·坡:最早的推理小说
霍桑:隐秘的恶《红字》
惠特曼:诗人《草叶集》一生的诗歌
麦尔维尔

19世纪现实主义文学
法国发源====
斯丹达尔:《拉辛与莎士比亚》、《红与黑》形成,《帕尔马修道院》战争描写;心理
巴尔扎克:《人间喜剧》最高成就
福楼拜:《包法利夫人》语言
梅里美:《高龙巴》、《嘉尔曼》
小仲马:《茶花女》
都德:《小东西》、《最后一课》、《柏林之围》

巴黎公社文学====
鲍狄埃:《国际歌》
米雪尔:《红石竹花》
瓦莱斯
英国====
狄更斯:《大卫·科波菲尔》、《远大前程》对上流社会幻想的破灭、《双城记》
萨克雷:《名利场》两个女人反应出的社会现实
夏绿蒂·勃兰特:《简·爱》;《呼啸山庄》、《谢利》、《维莱特》
盖斯凯尔夫人:《玛丽·巴顿》最早触及劳资矛盾
哈代:《德伯家的苔丝》
萧伯纳
高尔斯华绥
诗歌:厄内斯特·琼斯、威廉·林顿
德国====
海涅:民批专,诗人《德国,一个冬天的童话》诗
毕希纳:戏剧《丹东之死》
维尔特:诗人
北欧====
安徒生童话
勃兰兑斯:《19世纪文学主流》
易卜生:社会问题剧《青年同盟》、《社会支柱》、《人民公敌》、《玩偶之家》、《群鬼》
比昂松:戏剧《破产》、《挑战的手套》
俄国====
普希金:《叶普盖尼·奥涅金》多余人、《驿站长》小人物
莱蒙托夫:《当代英雄》
果戈里:戏剧《钦差大臣》、《死魂灵》
冈察洛夫《奥勃洛摩夫》
屠格涅夫《前夜》、《父与子》新人
车尔尼雪夫斯基:《怎么办?》
奥斯特洛夫斯基:戏剧《大雷雨》
涅克拉索夫《在俄罗斯谁能过好日子》
陀思妥耶夫斯基:《白夜》幻想家,《被欺凌与被凌辱的》、《死屋手记》监狱、《地下室手记》地下人、《罪与罚》、《白痴》、《群魔》、《少年》、《卡拉马佐夫兄弟》、
谢德林《戈洛夫略夫一家》
托尔斯泰:《童年》三部曲、《战争与和平》、《安娜·卡列尼娜》、《复活》;克服环境,心灵辩证法
契诃夫:《一个文官的死》、《胖子和瘦子》、《变色龙》简洁、《苦恼》隔膜、《万卡》、《草原》、《神经错乱》、《第六病室》、《跳来跳去的女人》、《带小狗的女人》、《套中人》、《樱桃园》
美国====
废奴文学
希尔德烈斯《白奴》
斯托夫人《汤姆叔叔的小屋》
乡土小说
哈特《咆哮营的幸运儿》
马克吐温:《竞选州长》、《镀金时代》南北战争后、《汤姆·索亚历险记》、《哈克贝利·费恩历险记》《王子与贫儿》、《百万英镑》;《神秘的陌生人》、《什么是人?》
亨利·詹姆斯:心理分析先河
诺里思:《章鱼》
欧·亨利:意料之外的结局
杰克·伦敦:《荒野的呼唤》、《白牙》、《铁蹄》、《马丁·伊登》、


19世纪自然主义和其他文学流派
产生于法国;照片式印象====
龚古尔兄弟
盖尔哈德·霍普特曼
斯特林堡
唯美主义====
泰奥菲尔·戈蒂埃
约翰·罗斯金
瓦尔特·佩特
爱伦·坡
王尔德:《谎言的衰朽》、《道林·格雷的画像》、《莎乐美》、《快乐王子集》
前期象征派====
让·莫雷亚斯:《象征主义宣言》
左拉:《卢贡-马卡尔家族》第二帝国一个家庭史、《萌芽》劳资矛盾
马拉美
波德莱尔:以丑为美,通感《恶之花》,语言精粹
保尔·魏尔伦
阿瑟·兰波
莫泊桑:《羊脂球》、《漂亮朋友》、《项链》语言简洁

20世纪文学
20世纪欧美现实主义文学====
劳伦斯:《虹》
萧伯纳:《芭芭拉少尉》
约翰·高尔斯华绥:《福尔赛世家》
毛姆:《人性的枷锁》
格雷厄姆·格林:《沉静的美国人》
愤怒的青年====
金斯莱·艾米斯:《幸运的吉姆》
约翰·奥斯本:《愤怒的回顾》
约翰·布莱思:《向上爬》
约翰·福尔斯:《法国中尉的女人》
多丽丝·莱辛:《金色笔记》
奈保尔:《毕司沃斯先生的房子》
阿加莎·克里斯蒂:《东方快车谋杀案》、《尼罗河惨案》
罗琳:《哈利波特》
法国
阿纳托尔·法朗士:《克兰克比尔》、《企鹅岛》、《诸神渴了》
罗曼·罗兰:《巨人传》、《约翰·克利斯朵夫》、《母与子》、《内心旅程》
马丁·杜伽尔:《蒂博一家》
安德烈·纪德:《背德者》、《窄门》、《伪币制造者》小说套小说
莫里亚克:《苔蕾丝·德盖鲁》、《蝮蛇结》
塞利纳:《茫茫黑夜漫游、《小王子》
马尔罗:《人的状况》上海工人起义
季奥诺
柯莱特
玛格丽特·杜拉斯:淡化情节,截取片段《情人》
玛格丽特·尤瑟纳尔:历史小说
德国
布莱希特
亨利希·曼:《臣仆》
托马斯·曼:《布登勃洛克一家》
君特·格拉斯:《铁皮鼓》

斯蒂芬·茨威格:《一个女人一生中的二十四个小时》
埃里希·马利亚·雷马克:战争题材《西线无故事》、《凯旋门》
亨利希·伯尔:《莱尼和他们》
美国
德莱塞
海明威:《太阳照样升起》、《永别了,武器》、《乞力马扎罗山上的雪》、《丧钟为谁而鸣》、《老人与海》
厄普顿·辛克莱:《屠场》揭发黑幕运动
辛克莱·刘易斯:《巴比特》庸俗市侩者
约翰·斯坦贝克:《愤怒的葡萄》
司各特·菲兹杰拉德:迷茫一代《了不起的盖茨比》
理查德·赖特:《土生子》犯罪心理
杰罗姆·大卫·塞林格:《麦田的守望者》
艾萨克·巴什维斯·辛格:《卢布林的魔法师》
索尔·贝楼:《奥吉马奇历险记》、《洪堡的礼物》
托妮·莫瑞森:《所罗门之歌》黑人受苦、《最蓝的眼睛》、《宠儿》
约翰·厄普代克:兔子四部曲
乔伊斯·卡洛尔·欧茨:《他们》30年代美下层人民的命运
纳博科夫:《洛丽塔》
华裔作家:汤亭亭、赵建秀
戏剧:田纳西·威廉斯《欲望号街车》、阿瑟·米勒《推销员之死》
玛格丽特·米切尔:《飘》
丹·布朗《达·芬奇密码》
其他欧
捷克
雅罗斯拉夫·哈谢克《好兵帅克》政治讽刺
米兰·昆德拉《生命中不能承受之轻》知识分子的感情生活和人生选择;《小说的艺术》、《生活在别处》、《不朽》、《被背叛的遗嘱》
挪威
克努特·哈姆生:《大地硕果》
西格里德·温赛特《克里斯汀》
意大利
格拉齐亚·黛莱达《风中芦苇》古老家族解体
皮兰德娄《六个寻找作者的剧中人》
姜尼·罗大里《洋葱头历险记》儿童作品
伊泰洛·卡尔维诺《分成两半的子爵》寓言小说、《寒冬夜行人》
苏联
高尔基:《随笔与短篇》、《童年》自传三部曲
肖洛霍夫:《静静的顿河》
布宁:《旧金山来的绅士》、《阿尔谢尼耶夫的一生》
库普林《决斗》
奥斯特洛夫斯基《钢铁是怎样炼成的》
阿列克赛·托尔斯泰《苦难的历程》
布尔加科夫《大师和玛格丽特》
爱伦堡:《解冻》 
艾特玛托夫
索尔仁尼琴:《癌病房》
拉丁美洲
韦拉:《旋涡》
博尔赫斯
巴尔加斯·略萨《城市与狗》

现代主义和后现代主义文学
后象征主义:诗歌
T.S.艾略特:《荒原》
叶芝
瓦雷里:《海滨墓园》
里尔克
梅特林克《青鸟》
庞德
表现主义
卡夫卡:《变形记》、《城堡》
奥尼尔
斯特林堡
恰佩克
未来主义
马里奈蒂:《他们来了》几百字,无人物
阿波利奈尔
马雅可夫斯基
超现实主义
布勒东:《娜佳》
阿拉贡
艾吕雅
意识流小说
詹姆斯·乔伊斯:《尤利西斯》
威廉·福克纳:《喧哗与骚动》
马塞尔·普鲁斯特:《追忆似水年华》意识流开山之作
伍尔夫:《墙上的斑点》
魔幻现实主义
加西亚·马尔克斯:《百年孤独》
阿斯图里亚斯
卡彭特尔
胡安·鲁尔福

后现代主义
存在主义文学
萨特:《禁闭》
加缪
波伏瓦
索尔·贝娄
戈尔丁
荒诞派戏剧
欧仁·尤奈斯库
贝克特:《等待戈多》
品特
阿尔比
新小说
萨洛特《怀疑的时代》、《橡皮》
黑色幽默
海勒:《第二十二条军规》
冯尼古特
品钦:《万有引力之虹》
巴思《烟草经纪人》


亚洲文学
古代
古埃及:《亡灵书》宗教诗歌集
古希伯来人:《旧约》
印度:《吠陀》《摩诃婆罗多》《罗摩衍那》
古巴比伦:《吉尔伽美什》
《圣经》
中古
日本:《万叶集》、《绯谐七部集》、《源氏物语》
阿拉伯:《一千零一夜》
朝鲜:金万重《谢氏南征记》《龙云梦》;《春香传》《沈清传》《兴夫传》
越南:
泰国:
印度:迦梨陀娑
近代
日本
森鸥外:《舞姬》
夏目簌石:《我是猫》、《哥儿》、《旅宿》
川端康成:《伊豆舞女》、《雪国》、《千只鹤》
芥川龙之介
小林多喜二:无阶作家
大江健三郎
印度
泰戈尔:《新月》《园丁集》《飞鸟集》《吉檀迦利》《戈拉》


\end{lstlisting}


\end{document}

