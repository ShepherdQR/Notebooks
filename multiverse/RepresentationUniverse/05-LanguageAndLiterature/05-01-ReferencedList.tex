

\documentclass[UTF8]{../RepresentationUniverse}
\begin{document}

\title{05-01-ReferencedList}
\date{Created on 20230420.\\   Last modified on \today.}
\maketitle
\tableofcontents


\chapter{Introduction}

收集参考过的书籍榜单




\section{其他的参考使用过的书单}


\subsection{名著E-OK已参考完成-20世纪100大英文小说-兰登书屋}
美国兰登书屋的《当代文库》编辑小组于1998年7月间选出了本世纪一百大英文小说。
\begin{lstlisting}
    1.乔伊斯(James Joyce) 爱尔兰 《尤利西斯》(Ulysses) 1922

    2.菲茨杰拉德(F. S. Fitzgerald) 美国 《了不起的盖茨比》(The Great Gatsby) 1925
    
    3.乔伊斯(James Joyce) 爱尔兰 《一个青年艺术家的画像》(A Portrait of the Artist as a Young Man) 1916
    
    4.纳博科夫(Vladimir Nabokov) 俄裔美籍 《洛丽塔》(Lolita) 1955
    
    5.赫胥黎(Aldous Huxley) 英国 《美丽新世界》(Brave New World) 1932
    
    6.福克纳(William Faulkner) 美国 《喧嚣与骚动》(The Sound and Fury) 1929
    
    7.海勒(Joseph Heller) 美国 《第二十二条军规》(Catch-22) 1961
    
    8.库斯勒(Arthur Koestler) 英籍匈牙利 《中午的黑暗》(Darkness at Noon) 1941
    
    9.劳伦斯(D. H. Lawrence) 英国 《儿子与情人》(Sons and Lovers) 1913
    
    10.斯坦贝克(John Steinbeck) 美国 《愤怒的葡萄》(The Grapes of Wrath) 1939
    
    11.劳瑞(Malcolm Lowry) 英国 《在火山下》(Under the Volcano) 1947
    
    12.巴特勒(Samuel Butler) 英国 《众生之路》(The Way of All Flesh) 1903
    
    13.奥威尔(George Orwell) 英国 《一九八四》(1984) 1949
    
    14.格雷夫斯(Robert Graves) 英国 《我,克劳迪亚斯》(I, Claudius) 1934
    
    15.伍尔芙(Virginia Woolf) 英国 《到灯塔去》(To the Light house) 1927
    
    16.德莱赛(Theodore Dreiser) 美国 《美国的悲剧》(An American Tragedy) 1925
    
    17.麦卡勒斯(Carson McCullers) 美国 《心是孤独的猎手》(The Heart Is a Longly Heart) 1940
    
    18.冯内古特(Kurt Vonnegut) 美国 《五号屠场》(Slaughterhouse-Five) 1969
    
    19.艾里森(Ralph Ellison) 美国 《看不见的人》(Invisible Man) 1952
    
    20.赖特(Richard Wright) 美国 《土生子》(Native Son) 1940
    
    21.贝娄(Saul Bellow) 美国 《雨王亨德森》( Henderson the Rain King) 1959
    
    22.奥哈拉(John O’Hara) 美国 《相约萨马拉》 (Appointment in Samarra) 1934
    
    23.多斯•帕索斯(John Dos Passos) 美国 《美国》 三部曲(U.S.A.) 1936
    
    24.安德森(Sherwood Anderson) 美国 《小城畸人》 (Winesburg, Ohio) 1919
    
    25.福斯特(E. M. Forster) 英国 《印度之行》(A Passage to India) 1924
    
    26.詹姆斯(Henry James) 美国 《鸽翼》(The Wings of the Dove) 1902
    
    27.詹姆斯(Henry James) 美国 《使节》(The Ambassadors) 1903
    
    28.菲茨杰拉德(F. S. Fitzgerald) 美国 《夜色温柔》 (Tender Is the Night) 1934
    
    29.法雷尔(James T. Farrell) 美国 《斯塔兹• 朗尼根》三部曲(Studs Lonigan-trilogy) 1935
    
    30.哈谢克(Ford Madox Ford) 英国 《好兵帅克》(The Good Soldier) 1915
    
    31.奥威尔(George Orwell) 英国 《动物农场》( Animal Farm) 1945
    
    32.詹姆斯(Henry James) 美国 《金碗》(The Golden Bowl) 1904
    
    33.德莱赛(Theodore Dreiser) 美国 《嘉莉妹妹》 (Sister Carrie) 1900
    
    34.伊夫林•沃(Evelyn Waugh) 英国 《一把尘土》(A Handful of Dust) 1934
    
    35.福克纳(William Faulkner) 美国 《我弥留之际》(As I Lay Dying) 1930
    
    36.华伦(Robert Penn Warren) 美国 《国王的人马》(All the King’s Men) 1946
    
    37.怀尔德(Thornton Wilder) 美国 《圣路易斯雷大桥》(The Bridge of San Luis Rey) 1927
    
    38.福斯特(E. M. Forster) 英国 《霍华德庄园》 (Howards End) 1910
    
    39.鲍德温(James Baldwin) 美国 《向苍天呼吁》 (Go Tell It on the Mountain) 1953
    
    40.格林(Graham Greene) 英国 《问题的核心》( The Heart of the Matter) 1948
    
    41.戈尔丁(William Golding) 英国 《蝇王》(Lord of the Flies) 1954
    
    42.迪基(James Dickey) 美国 《解救》(Deliverance ) 1970
    
    43.鲍威尔(Anthony Powell) 英国 《与时代合拍的舞蹈》(A Dance to the Music of Time) 1975
    
    44.赫胥黎(Aldous Huxley) 英国 《针锋相对》 (Point Counter Point) 1928
    
    45.海明威(Ernest Hemingway) 美国 《太阳照常升起》(The Sun Also Rise) 1926
    
    46.康拉德(Joseph Conrad) 英国 《特务》(The Secret Agent) 1907
    
    47.康拉德(Joseph Conrad) 英国 《诺斯特罗莫》 (Nostromo) 1904
    
    48.劳伦斯(D. H. Lawrence) 英国 《虹》( Rainbow) 1915
    
    49.劳伦斯(D. H. Lawrence) 英国 《恋爱中的女人》(Women in Love) 1920
    
    50.米勒(Henry Miller) 美国 《北回归线》 (Tropic of Cancer) 1934
    
    51.梅勒(Norman Mailer) 美国 《裸者和死者》 (The Naked and Dead) 1948
    
    52.罗斯(Philp Roth) 美国 《波特诺伊的怨诉》 (Portnoy's Complaint) 1969
    
    53.纳博科夫(Vladimir Nabokov) 俄裔美籍 《微暗的火》(Pale Fire) 1962
    
    54.福克纳(William Faulkner) 美国 《八月之光》(Light in August) 1932
    
    55.凯鲁亚克(Jack Kerouac) 美国 《在路上》(On the Road) 1957
    
    56.哈米特(Dashiell Hammett) 美国 《马尔他之鹰》(The Maltese Falcon) 1930
    
    57.福特(Ford Madox Ford) 英国 《行进的目的》 (Parade's End) 1928
    
    58.华顿(Edith Wharton) 美国 《纯真年代》(The Age of Innocence) 1920
    
    59.毕尔邦(Max Beerbohm) 英国 《朱莱卡·多卜生》(Zuleika Dobson) 1911
    
    60.珀西(WalkerPercy) 美国 《看电影的人》 (The Moviegoer) 1961
    
    61.凯瑟(Willa Cather) 美国 《大主教之死》( Death Comes to Archbishop) 1927
    
    62.锺斯(James Ramon Jones) 美国《从这里到永恒》(From Here to Eternity) 1951
    
    63.契弗(John Cheever) 美国 《瓦卜肖特纪事》 (The Wapshot Chronicles) 1957
    
    64.塞林格(J. D. Salinger) 美国 《麦田的守望者》 (The Catcher in the Rye) 1951
    
    65.伯吉斯(Anthony Burgess) 英国 《发条橙》(A Clockwork Orange) 1962
    
    66.毛姆(W. Somerset Maugham) 英国 《人性的枷锁》 (Of Human Bondage) 1915
    
    67.康拉德(Joseph Conrad) 英国 《黑暗之心》( Heart of Darkness) 1902
    
    68.刘易斯(Sinclair Lewis) 美国 《大街》(Main Street) 1920
    
    69.华顿(Edith Wharton) 美国 《欢乐之家》(The House of Mirth) 1905
    
    70.德雷尔(Lawrence Durrell) 英国 《亚历山大四部曲》(The Alexandraia Quartet) 1960
    
    71.休斯(Richard Hughes) 英国 《牙买加飓风》 (A Hig hWind in Jamaica) 1929
    
    72.奈保尔(V. S. Naipaul) 特里尼达 《毕司沃斯先生的房子》(A House for Mr. Biswas) 1961
    
    73.韦斯特(Nathaniel West) 美国 《蝗灾之日》 (The Day of the Locust) 1939
    
    74.海明威(Ernest Hemingway) 美国 《永别了,武器》 (A Farewell to Arms) 1929
    
    75.伊夫林•沃(Evelyn Waugh) 英国 《独家新闻》(Scoop ) 1938
    
    76.斯帕克(Muriel Spark) 英国 《布罗迪小姐的青春》(The Prime of Miss Jean Brodie) 1961
    
    77.乔伊斯(James Joyce) 爱尔兰 《芬尼根的守灵夜》 (Finnegans Wake) 1939
    
    78.吉卜林(Rudyard Kipling) 英国 《基姆》(Kim)  1901
    
    79.福斯特(E. M. Forster) 英国 《看得见风景的房间》 (A Room with a View) 1908
    
    80.伊夫林•沃(Evelyn Waugh) 英国 《旧地重游》(Brideshead Revisited) 1945
    
    81.贝娄(Saul Bellow) 美国 《奥吉•马奇历险记》(The Adventures of Augie March) 1971
    
    82.斯泰格纳(Wallace Stegner) 美国 《安息角》(Angle of Repose) 1971
    
    83.奈保尔(V.S.Naipaul) 特里尼达 《河湾》(A Bend in the River) 1979
    
    84.伊丽莎白•鲍温(Elizabeth Bowen) 英国 《心之死》(The Death of the Heart) 1938
    
    85.康拉德(Joseph Conrad) 英国 《吉姆爷》(Lord Jim) 1900
    
    86.达可托罗(E. L. Doctorow) 美国 《爵士乐》( Ragtime) 1975
    
    87.本涅特(Arnold Bennett) 英国 《老妇人的故事》(The Old Wives' Tale) 1908
    
    88.伦敦(Jack London) 英国 《野性的呼唤》(The Call of the Wild) 1903
    
    89.格林(Henry Green) 英国 《爱》(Loving) 1945
    
    90.拉什迪(Salman Rushdie) (印裔英籍)  《午夜之子》(Midnight's Children) 1981
    
    91.考德威尔(Erskine Caldwell) 美国 《烟草路》 (Tobacco Road) 1932
    
    92.肯尼迪(William Kennedy) 美国 《铁草》( Ironweed) 1983
    
    93.福尔斯(John Fowles) 英国 《巫术师》(The Magus) 1966
    
    94.里丝(Jean Rhys) 英国 《辽阔的藻海》(Wide Sargasso) 1966
    
    95.莫多克(Iris Murdoch) 英国 《在网下》(Under the Net) 1954
    
    96.斯泰隆(William Styron) 美国 《苏菲的选择》(Sophie's Choice) 1979
    
    97.鲍尔斯(Paul Bowles) 美国 《遮蔽的天空》( The Sheltering Sky) 1949
    
    98.凯恩(James M. Cain) 美国 《邮差总按两遍铃》(The Postman Always Rings Twice) 1934
    
    99.唐利维(J. P. Donleavy) 美国 《眼线》(The Ginger Man) 1955
    
    100.塔金顿(Booth Tarkington) 美国《了不起的安德森家族》(The Magnificent Ambersons) 1918
\end{lstlisting}

\subsection{名著D-OK已参考完成-中国-20世纪中国小说}
\begin{lstlisting}

1 《呐喊》 鲁迅
2 《边城》 沈从文
3 《骆驼祥子》 老舍
4 《传奇》 张爱玲
5 《围城》 钱锺书
6 《子夜》 茅盾
7 《台北人》 白先勇
8 《家》 巴金
9 《呼兰河传》 萧红
10 《老残游记》 刘鹗
11 《寒夜》 巴金
12 《彷徨》 鲁迅
13 《官场现形记》 李伯元
14 《财主底儿女们》 路翎
15 《将军族》 陈映真
16 《沉沦》 郁达夫
17 《死水微澜》 李劼人
18 《红高粱》 莫言
19 《小二黑结婚》 赵树理
20 《棋王》 阿城


21 《家变》 王文兴
22 《马桥词典》 韩少功
23 《亚细亚的孤儿》 吴浊流
24 《半生缘》 张爱玲
25 《四世同堂》 老舍
26 《胡雪岩》 高阳
27 《啼笑因缘》 张恨水
28 《儿子的大玩偶》 黄春明
29 《射雕英雄传》 金庸
30 《莎菲女士的日记》 丁玲
31 《鹿鼎记》 金庸
32 《孽海花》 曾朴
33 《惹事》 赖和
34 《嫁妆一牛车》 王祯和
35 《异域》 邓克保(柏扬)
36 《曾国藩》 唐浩明
37 《原乡人》 锺理和
38 《白鹿原》 陈忠实
39 《长恨歌》 王安忆
40 《吉陵春秋》 李永平


41 《黄祸》 保密(王力雄)
42 《狂风沙》 司马中原
43 《艳阳天》 浩然
44 《公墓》 穆时英
45 《旧址》 李锐
46 《星星.月亮.太阳》 徐速
47 《台湾人叁部曲》 锺肇政
48 《洗澡》 杨绛
49 《旋风》 姜贵
50 《荷花淀》 孙犁 
51 《我城》 西西
52 《受戒》 汪曾祺
53 《铁浆》 朱西宁
54 《世纪末华丽》 朱天文
55 《蜀山剑侠传》 还珠楼主
56 《又见棕榈,又见棕榈》 於梨华
57 《浮躁》 贾平凹
58 《组织部新来的年轻人》 王蒙
59 《玉梨魂》 徐枕亚
60《 香港叁部曲》 施叔青


61 《京华烟云》 林语堂
62 《倪焕之》 叶圣陶
63 《春桃》 许地山
64 《桑青与桃红》 聂华苓
65 《蓝与黑》 王蓝
66 《二月》 柔石
67 《风萧萧》 徐言于
68 《芙蓉镇》 古华
69 《地之子》 台静农
70 《城南旧事》 林海音
71 《古船》 张炜
72 《酒徒》 刘以鬯
73 《未央歌》 鹿桥
74 《沉重的翅膀》 张洁
75 《果园城记》 师陀
76 《人啊,人!》 戴厚英
77 《黄金时代》 王小波
78 《狗日的粮食》 刘恒
79 《棋王》 张系国
80 《赖索》 黄凡


81 《妻妾成群》 苏童
82 《霸王别姬》 李碧华
83 《杀夫》 李昂
84 《楚留香》 古龙
85 《窗外》 琼瑶
86 《沉默之岛》 苏伟贞
87 《白发魔女传》 梁羽生
88 《古都》 朱天心
89 《尹县长》 陈若曦
90 《四喜忧国》 张大春
91 《喜宝》 亦舒
92 《男人一半是女人》 张贤亮
93 《将军底头》 施蛰存
94 《蓝血人》 倪匡
95 《二十年目睹之怪现状》 吴趼人
96 《活着》 余华
97 《冈底斯的诱惑》 马原
98 《十年十意》 林斤澜
99 《北极风情画》 无名氏
100 《雍正皇帝》 二月河

\end{lstlisting}


\subsection{名著C-ok已参考完成}

\begin{lstlisting}
【梯队1】crown
莎士比亚《哈姆雷特》
维吉尔《埃涅阿斯纪》
但丁《神曲》【奥登、乔治·赫伯特,威廉·巴恩斯,博尔赫兹】
塞万提斯《堂吉诃德》【影响了司汤达、福楼拜,梅尔维尔,马克吐温,陀思妥耶夫斯基,屠格涅夫,托马斯曼。巴尔扎克。理查孙的克拉利萨,希尼埃的加米叶,提拔拉斯的台莉,阿瑠斯托的安日丽各,但丁的珐琅彩斯卡,莫里哀的奥塞斯的,普玛西的费加罗,华尔特·司各特的丽贝卡。】
乔叟《坎特伯雷故事集》【玛丽安·摩尔】
蒙田《蒙田随笔》【】
弥尔顿【失乐园】【哈德雷·布鲁姆,奥克塔维奥·帕斯】
托尔斯泰《战争与和平》《安娜卡列尼娜》【斯坦尼斯拉夫斯基】
卢克来修《悟性论》
奥古斯丁《忏悔录》

【梯队2】wisdom
《旧约》【托马斯·福斯特,王尔德的《莎乐美》】
歌德《浮士德》《威廉·麦斯特》【席勒,卡莱尔】
托马斯曼《魔山》【哈罗德·布鲁姆,乔伊斯的尤利西斯,普鲁斯特的追忆似水年华】
《新约》
默罕默德《古兰经》
鲍斯威尔
塞缪尔·约翰逊
弗洛伊德
苏格拉底和柏拉图

【梯队3】understanding
契诃夫《带狗的女人》
莫里哀《唐璜》
易卜生《玩偶之家》《海达·高步乐》
卡夫卡《变形记》【罗伯格里耶,格罗斯曼,马尔克斯,鲁尔福,萨特,加缪,福克纳,波德莱尔,爱伦坡,奥尼尔,斯特林堡】
王尔德
贝克特(诺奖) 
皮兰德娄(诺奖) 
尼采(日神、酒神) 
克尔凯郭尔

【梯队4】love
梅尔维尔《白鲸》
勃朗特《呼啸山庄》
霍桑《红字》【海明威,哈克·菲恩】
简奥斯汀《傲慢与偏见》《艾玛》【《曼斯菲尔德花园》,福斯特,特雷弗,狄更斯,艾略特】
乔纳森·斯威夫特《格列佛游记》
夏洛蒂·勃朗特《简爱》
紫式部《源氏物语》
弗吉尼亚·伍尔夫《到灯塔去》
约翰·唐恩
亚历山大·蒲柏

【梯队5】【诗人】severtity
意大利:莱奥帕尔迪
美国:斯蒂文森,罗伯特·弗罗斯特,艾米丽·狄金森,艾默生
英国:济慈,雪莱,华兹华斯(《序曲》) ,丁尼生(《悼念》) ,艾略特(诺奖) 

【梯队6】victory
荷马《伊利亚特》
乔伊斯《尤利西斯》
司汤达《红与黑》《帕尔马修道院》
海明威《太阳照常升起》
福克纳《我弥留之际》《喧哗与躁动》《押沙龙,押沙龙》
奥康纳《暴力者夺走它》【麦卡锡 《血红色子午线》】
马克吐温《百万英镑》《汤姆索亚历险记》
卡彭铁尔
卡蒙斯(葡萄牙诗人) 
奥克塔维奥(诺奖) 

【梯队7】bearuty
雨果《悲惨世界》《巴黎圣母院》【契诃夫,果戈里,普希金,杜康日】
沃尔特·佩特《享乐主义者马里乌斯》
奈瓦尔(法国诗人) 
波德莱尔(法国象征派诗人) 
霍夫曼斯塔尔,奥地利诗人
斯温伯恩,英国诗人
乔金娜·罗塞蒂和加百利·罗塞蒂,英国诗人

【梯队8】spendor
艾略特《米德尔马契》《亚当·彼得》【尤朵拉·威尔帝】
菲茨杰拉德《了不起的盖茨比》【田纳西·威廉斯】
伊迪斯·华顿《纯真年代》
薇拉凯瑟,
爱丽丝·默多克《钟》
费尔南多·佩索阿,葡萄牙诗人,
费德里科·加西亚·洛尔卡,西班牙诗人;
路易斯·塞尔努达,西班牙诗人;
哈特·克莱恩,英国诗人
惠特曼

【梯队9】fuondation
福楼拜《包法利夫人》【托马斯曼的《布登勃洛克一家》,阿诺德贝内特的《老妇人的故事》,《嘉莉妹妹》】
博尔赫斯《沙之书》【戈迪默,拉费里埃】
卡尔维诺《看不见的城市》
劳伦斯《查泰莱夫人的情人》《虹》【赫胥黎】
马查多·德·阿西斯《幻灭三部曲》【苏珊·桑塔格,本杰明·莫泽】
克罗兹《阿马罗神父的罪恶》
威廉姆斯《欲望号街车》
威廉·布莱克,英国诗人
蒙塔莱(诺奖) 
赖内·玛利亚·里尔克,奥地利诗人。

【维度10】kingdom
陀思妥耶夫斯基《卡拉马佐夫兄弟》《罪与罚》
刘易斯·卡罗尔《爱丽丝梦游仙境》【亨利·米勒】
巴尔扎克《高老头》
狄更斯《远大前程》《大卫科波菲尔》【本·穷森,菲利普·罗斯,】
亨利·詹姆斯《一位女士的画像》
巴别尔《我的第一只鹅》
拉尔夫·艾莉森《看不见的人》
叶芝
勃朗宁
策兰

其他:阿里斯多芬,卡尔德隆,欧里庇得斯。
\end{lstlisting}


\subsection{名著B-ok已参考完成}

\begin{lstlisting}

    【希腊】4 部
    荷马:《奥德赛》
    索福克里斯:《俄狄浦斯王》
    欧里庇得斯:《美狄亚》
    卡赞扎基斯《希腊佐巴的故事》
    
    【意大利】7 部
    维吉尔:《埃涅阿斯纪》
    奥维德:《变形记》
    但丁:《神曲》
    薄伽丘:《十日谈》
    莱奥帕尔迪:《诗集》
    斯维沃:《泽诺的意识》
    莫兰黛:《历史》
    
    【英国】16 部 
    乔叟:《坎特伯雷故事集》
    乔纳森·斯威夫特:《格列佛游记》
    莎士比亚:《哈姆雷特》《李尔王》《奥赛罗》
    简·奥斯丁:《傲慢与偏见》
    艾米莉·勃朗特《呼啸山庄》
    乔治·艾略特:《米德尔马契》
    狄更斯:《远大前程》
    康拉德:《诺斯特罗莫》
    劳伦斯:《儿子与情人》
    奥威尔:《1984》
    伍尔夫:《达洛维夫人》《到灯塔去》
    多丽丝·莱辛:《金色笔记》
    萨尔曼·鲁西迪(印度 / 英国) :《午夜之子》
    
    【爱尔兰】5 部 
    劳伦斯·斯特恩:《项狄传》
    乔伊斯:《尤利西斯》
    贝克特:戏剧三部曲 《马洛伊》《马洛伊之死》《无名的人》
    
    【法国】12 部 
    拉伯雷:《巨人传》
    蒙田:《随笔集》
    狄德罗:《宿命论者雅克和他的主人》
    司汤达:《红与黑》
    巴尔扎克:《高老头》
    福楼拜:《包法利夫人》、《情感教育》
    普鲁斯特:《追忆逝水年华》
    塞利纳:《茫茫黑夜漫游》
    加缪:《局外人》
    策兰(罗马尼亚 / 法国) :《诗集》
    玛格丽特·尤瑟纳尔:《哈德良回忆录》
    
    【德国】5 部
    歌德:《浮士德》
    德布林:《柏林,亚历山大广场》
    托马斯·曼:《布登勃洛克一家》《魔山》
    格拉斯:《铁皮鼓》
    
    【奥地利】4 部 
    穆齐尔:《没有个性的人》
    卡夫卡:《故事全集》《审判》《城堡》
    
    【丹麦】1 部 
    安徒生:《安徒生童话故事集》
    
    【挪威】2 部 
    哈姆生:《饥饿》
    易卜生:《玩偶之家》
    
    【瑞典】1 部
    林格伦:《长袜子皮皮》
    
    【冰岛】2 部
    《尼雅尔萨迦》
    拉克斯内斯:《独立的人们》
    
    【西班牙】2 部 
    塞万提斯:《堂吉诃德》
    罗卡:《吉普赛故事诗》
    
    【葡萄牙】2 部
    费尔南多·佩索阿 :《惶然录》又译为《不安之书》
    萨拉马戈:《失明症漫记》
    
    【俄罗斯】9 部 
    果戈里:《死魂灵》
    陀思妥耶夫斯基:《罪与罚》《白痴》《群魔》《卡拉马佐夫兄弟》
    托尔斯泰:《战争与和平》《安娜·卡列尼娜》《伊凡·伊里奇之死及其他》
    契诃夫:《小说集》
    
    【美国】10 部 
    爱伦·坡:《故事全集》
    惠特曼:《草叶集》
    麦尔维尔:《白鲸》
    马克·吐温:《哈克贝利·费恩历险记》
    福克纳:《押沙龙,押沙龙》《喧哗与骚动》
    海明威:《老人与海》
    纳博科夫(俄罗斯 / 美国) :《洛丽塔》
    艾里森:《看不见的人》
    莫里森:《宠儿》
    
    【拉丁美洲地区】5 部 
    罗萨(巴西) :《广阔的腹地:条条小路》
    胡安·鲁尔福(墨西哥) :《佩德罗·巴拉莫》
    博尔赫斯(阿根廷) :《小说集》
    马尔克斯(哥伦比亚) :《百年孤独》《霍乱时期的爱情》
    
    【非洲地区】3 部 
    纳吉布·马哈富兹(埃及) :《街魂》
    塔伊布·萨利赫(苏丹) :《移居北方的时期》
    阿契贝(尼日利亚) :《瓦解》
    
    【亚洲地区】11 部 
    《吉尔伽美什史诗》(美索不达米亚) 
    《约伯记》(以色列) 
    萨迪(伊朗) :《果园》
    鲁米(伊朗) :《玛斯纳维》
    《摩诃婆罗多》(印度) 
    蚁垤(印度) :《罗摩衍那》
    迦梨陀娑(印度) :《沙恭达罗》
    《一千零一夜》(印度/伊朗/伊拉克/埃及) 
    鲁迅(中国) :《狂人日记及其他》
    紫氏部(日本) :《源氏物语》
    川端康成(日本) :《山音》
\end{lstlisting}



\subsection{名著A-ok已参考完成}

1. 了不起的盖茨比(F.斯科特·菲茨杰拉德) 

2. 1984(乔治·奥威尔) 

3. 杀死一只知更鸟(哈珀·李) 

4. 洛丽塔(弗拉基米尔·弗拉基米尔罗维奇·纳博科夫) 

5. 麦田守望者(J.D 塞林格) 

6. 指环王(J.R.R.托尔金) 

7. 22条军规(约瑟夫·海勒) 

8. 愤怒的葡萄(约翰·斯坦贝克) 

9. 傲慢与偏见(简·奥斯丁) 

10. 尤利西斯(詹姆斯·乔伊斯) 

11. 百年孤独(加西亚.马尔克斯) 

12. 战争与和平(列夫·托尔斯泰) 

13. 美丽新世界(奥尔德斯 ·赫胥黎) 

14. 简爱(夏洛特·勃朗特) 

15. 乱世佳人(玛格丽特·米切尔) 

16. 安娜·卡列尼娜(列夫·托尔斯泰) 

17. 呼啸山庄(艾米莉·勃朗特) 

18. 蝇王(威廉·戈尔丁) 

19. 宠儿(托妮·莫里森) 

20. 喧哗与骚动(威廉·福克纳) 

21. 唐吉可德(塞万提斯) 

22. 霍比特人(J.R.R.托尔金) 

23. 动物农场(乔治·奥威尔) 

24. 哈克贝利·费恩历险记(马克·吐温) 

25. 在路上(杰克·凯鲁亚克) 

26. 米德尔马契(乔治·艾略特) 

27. 看不见的人(拉尔夫·艾里森) 

28. 白鲸(赫尔曼 麦尔维尔) 

29. 罪与罚(费奥多尔·陀思妥耶夫斯基) 

30. 到灯塔去(弗吉尼亚伍尔夫) 

31. 五号屠场(库尔特·冯内古特) 

32. 小妇人(路易莎·梅·奥尔科特) 

33. 远大前程(查尔斯·狄更斯) 

34. 银河系搭车客指南(道格拉斯·亚当斯) 

35. 包法利夫人(福楼拜) 

36. 爱丽丝梦游仙境(刘易斯·卡罗尔) 

37. 太阳照常升起(欧内斯特·海明威) 

38. 印度之行(E M 福斯特) 

39. 双城记(查尔斯·狄更斯) 

40. 爱玛(简·奥斯丁) 

41. 达洛维夫人(弗吉尼亚伍尔夫) 

42. 黑暗的心(约瑟夫·康拉德) 

43. 这个世界土崩瓦解了(钦努阿·阿契贝) 

44. 卡拉马佐夫兄弟(费奥多尔·陀思妥耶夫斯基) 

45. 夏洛特的网(E.B White) 

46. 科学怪人(玛丽·雪莱) 

47. 蝴蝶梦(达芙妮·杜穆里埃) 

48. 人鼠之间(约翰·斯坦贝克) 

49. 柳林风声(肯尼斯·格雷厄姆) 

50. 发条橙(安东尼·伯吉斯) 

51. 基督山伯爵(大仲马) 

52. 小王子(德·圣埃克苏佩里) 

53. 纳尼亚传奇(C.S 刘易斯) 

54. 追忆逝水年华(马塞尔·普鲁斯特) 

55. 大卫·科波菲尔(查尔斯·狄更斯) 

56. 午夜之子(萨曼·鲁西迪) 

57. 审判(弗朗兹·卡夫卡) 

58. 安妮日记(安妮·弗兰克) 

59. 格列佛游记(乔纳森·斯威夫特) 

60. 道林·格雷的画像(奥斯卡·王尔德) 

61. 哈姆雷特(威廉·莎士比亚) 

62. 土生子(理查德·赖特) 

63. 野性的呼唤(杰克·伦敦) 

64. 使女的故事(玛格丽特·阿特伍德) 

65. 紫颜色(爱丽丝·沃克) 

66. 局外人(阿尔伯特·加缪) 

67. 他们眼望上苍 (佐拉·尼尔·赫斯顿)

68. 丧钟为谁而鸣(欧内斯特·海明威) 

69. 红字(霍桑) 

70. 一个青年艺术家的肖像 (詹姆斯·乔伊斯) 

71. 圣经:詹姆士王版本

72. 狮子、女巫与魔衣柜(C.S 刘易斯) 

73. 悲惨世界(维克多·雨果) 

74. 绿山墙的安妮(L.M 蒙哥马利) 

75. 钟罩(西尔维娅.普拉斯) 

76. 达芬奇密码(丹·布朗) 

77. 玫瑰的名字(翁贝托·埃科) 

78. 鲁滨逊漂流记(丹尼尔·笛福) 

79. 沙丘(弗兰克·赫伯特) 

80. 追风筝的人(卡勒德·胡赛尼) 

81. 时间的皱纹(英格) 

82. 小熊维尼(A.A 米尔恩) 

83. 飞越疯人院(肯·克西) 

84. 苔丝(托马斯·哈代) 

85. 火山下(Malcolm Lowry) 

86. 奥德赛(荷马) 

87. 我弥留之际(威廉·福克纳) 

88. 小屁孩日记(杰夫.金尼) 

89. 旧在重游:查尔斯·赖德上尉神圣的渎神回忆(伊芙琳·沃) 

90. 华氏451(雷·布拉德伯里) 

91. 牧羊少年奇幻之旅(保罗柯艾略) 

92. 兔子共和国(理查德·亚当斯) 

93. 老人与海(欧内斯特·海明威) 

94. 赎罪(伊恩·麦克尤恩) 

95. 教父(马里奥·普佐) 

96. 无人生还(阿加莎·克里斯蒂) 

97. 为欧文·梅尼祈祷(约翰·欧文) 

98. 坎特伯雷故事(杰弗里·乔叟) 

99. 纯真年代 (伊迪丝.华顿)

100. 一位女士的画像(亨利·詹姆斯) 


\end{document}

