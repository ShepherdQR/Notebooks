%%============
%%  ** Author: Qirong ZHANG
%%  ** Date: 2022-05-08 20:33:12
%%  ** Github: https://github.com/ShepherdQR
%%  ** LastEditors: Qirong ZHANG
%%  ** LastEditTime: 2024-12-28 23:26:35
%%  ** Copyright (c) 2019 Qirong ZHANG. All rights reserved.
%%  ** SPDX-License-Identifier: LGPL-3.0-or-later.
%%============



\documentclass[UTF8]{RepresentationUniverse}

\includeonly{
    01-Introduction
}

\begin{document}
%%============
%%  ** Author: Shepherd Qirong
%%  ** Date: 2022-06-04 23:40:10
%%  ** Github: https://github.com/ShepherdQR
%%  ** LastEditors: Shepherd Qirong
%%  ** LastEditTime: 2022-06-04 23:40:55
%%  ** Copyright (c) 2019--20xx Shepherd Qirong. All rights reserved.
%%============

\chapter{Introduction}

20220604,进入计算的世界。







\chapter{艺术}%% empty here

\chapter{历史}%% empty here

\chapter{语言和文学}%% empty here
    \section{语言学}%% empty here
    \section{文学}%% empty here




\chapter{哲学}

\section{哲学史}
% \subsection{古代哲学}
% \subsection{当代哲学}
% \subsection{中世纪哲学}
%     \subsubsection{人文主义}
%     \subsubsection{经院哲学}
% \subsection{现代哲学}
\section{美学}
\section{应用哲学}
% \subsection{经济学哲学}
% \subsection{教育理念}
% \subsection{工程哲学}
% \subsection{历史哲学}
% \subsection{语言哲学}
% \subsection{法哲学}
% \subsection{数学哲学}
% \subsection{音乐哲学}
% \subsection{心理学哲学}
% \subsection{宗教哲学}
% \subsection{自然科学哲学}
% \subsubsection{生物学哲学}
% \subsubsection{化学哲学}
% \subsubsection{物理学哲学}
% \subsection{社会科学哲学}
% \subsection{技术哲学}
% \subsection{系统理念}

\section{认识论}
% \subsection{理由}
% \subsection{推理错误}

\section{伦理}
% \subsection{应用伦理}
%     \subsubsection{动物权益}
%     \subsubsection{生命伦理学}
%     \subsubsection{环境伦理}
% \subsection{元伦理学}
% \subsection{描述伦理,价值理论}
% \subsection{规范伦理}


\section{逻辑}
% \subsection{数理逻辑}
% \subsection{哲学逻辑}
\section{元哲学}
\section{形而上学}
% \subsection{行动哲学}
% \subsection{决定论和自由意志}
% \subsection{本体}
% \subsection{时空哲学}
% \subsection{目的论}
% \subsection{有神论和无神论}
\section{哲学传统}
% \subsection{非洲哲学}
% \subsection{分析哲学}
% \subsection{亚里士多德主义}
% \subsection{欧陆哲学}
% \subsection{东方哲学}
% \subsection{女权主义哲学}
% \subsection{柏拉图主义}
\section{社会哲学与政治哲学}
% \subsection{无政府主义}
% \subsection{女权主义哲学}
% \subsection{自由主义}
% \subsection{马克思主义}

\chapter{宗教学}
\section{圣经研究}
% \subsection{圣经希伯来语,通用希腊语,亚拉姆语}
\section{宗教研究}
\section{佛教神学}
% \subsection{巴利语研究}
\section{基督教神学}
% \subsection{英国国教神学}
% \subsection{浸信会神学}
% \subsection{天主教神学}
% \subsection{东正教神学}
% \subsection{新教神学}
\section{印度教神学}
% \subsection{梵文研究}
% \subsection{达罗毗荼研究}
\section{犹太神学}
\section{穆斯林神学}
% \subsection{阿拉伯研究}


\chapter{人类学}
\section{生物人类学}
\section{语言人类学}
\section{文化人类学}
\section{社会人类学}

\chapter{考古学}
\section{生物文化人类学}
\section{进化人类学}
\section{女权考古}
\section{法医人类学}
\section{海洋考古}
\section{古人类学}



\chapter{经济学}
\section{理论}
% \subsection{经济史与经济思想史}
% \subsection{西方经济学}
% \subsection{世界经济}
%     \subsubsection{国际经济学}
% \subsection{人口、资源与环境经济学}
% \subsection{经济理论}
%     \subsubsection{无政府主义经济学}
%     \subsubsection{复杂性经济学}
%     \subsubsection{人类发展理论}
%     \subsubsection{社会选择理论}
\section{应用}
% \subsection{区域经济学}
%     \subsubsection{公共财政}
% \subsection{财政学与税收学}
% \subsection{金融学与保险学}
%     \subsubsection{金融计量经济学}
%     \subsubsection{金融经济学}
% \subsection{产业经济学}
%     \subsubsection{产业组织}
% \subsection{国际贸易学}
% \subsection{劳动经济学}
% \subsection{数量经济学}
%     \subsubsection{计算经济学}
%     \subsubsection{计量经济学}
%     \subsubsection{数理经济学}
% \subsection{特定领域经济学}
%   \subsubsection{农业经济学}
% 	\subsubsection{行为经济学}
% 	\subsubsection{生物经济学}
% 	\subsubsection{消费经济学}
% 	\subsubsection{发展经济学}
% 	\subsubsection{生态经济学}
% 	\subsubsection{经济地理学}
% 	\subsubsection{经济社会学}
% 	\subsubsection{经济系统}
% 	\subsubsection{教育经济学}
% 	\subsubsection{能源经济学}
% 	\subsubsection{创业经济学}
% 	\subsubsection{环境经济学}
% 	\subsubsection{进化经济学}
% 	\subsubsection{实验经济学}
% 	\subsubsection{女权经济学}
% 	\subsubsection{绿色经济学}
% 	\subsubsection{增长经济学}
% 	\subsubsection{信息经济学}
% 	\subsubsection{制度经济学}
% 	\subsubsection{伊斯兰经济学}
% 	\subsubsection{法律经济学}
% 	\subsubsection{宏观经济学}
% 	\subsubsection{管理经济学}
% 	\subsubsection{马克思经济学}
% 	\subsubsection{微观经济学}
% 	\subsubsection{货币经济学}
% 	\subsubsection{神经经济学}
% 	\subsubsection{参与式经济学}
% 	\subsubsection{政治经济}
% 	\subsubsection{公共经济学}
% 	\subsubsection{房地产经济学}
% 	\subsubsection{资源经济学}
% 	\subsubsection{社会主义经济学}
% 	\subsubsection{社会经济学}
% 	\subsubsection{交通经济学}
% 	\subsubsection{福利经济学}


\chapter{地理}

\section{自然地理学}
% \subsection{大气}
% \subsection{生物地理学}
% \subsection{气候学}
% \subsection{沿海地理}
% \subsection{应急管理}
% \subsection{环境地理学}
% \subsection{地球生物学}
% \subsection{地球化学}
% \subsection{地质学}
% \subsection{地理信息学}
% \subsection{地貌学}
% \subsection{地球物理学}
% \subsection{冰川学}
% \subsection{水文学}
% \subsection{景观生态}
% \subsection{岩性}
% \subsection{气象}
% \subsection{矿物学}
% \subsection{海洋学}
% \subsection{古地理学}
% \subsection{古生物学}
% \subsection{岩石学}
% \subsection{第四纪科学}
% \subsection{土壤地理}

\section{人文地理学}
% \subsection{行为地理学}
% \subsection{认知地理学}
% \subsection{文化地理}
% \subsection{发展地理学}
% \subsection{经济地理学}
% \subsection{健康地理学}
% \subsection{历史地理}
% \subsection{语言地理学}
% \subsection{数理地理学}
% \subsection{营销地理}
% \subsection{军事地理}
% \subsection{政治地理学}
% \subsection{人口地理}
% \subsection{宗教地理学}
% \subsection{社会地理学}
% \subsection{战略地理}
% \subsection{时间地理}
% \subsection{旅游地理学}
% \subsection{交通地理}
% \subsection{城市地理}


\section{综合地理}
\section{制图}
% \subsection{天体制图}
% \subsection{行星制图}
% \subsection{地形}


\chapter{政治学}
\section{政治史}
\section{政治理论研究}
% \subsection{公民}
% \subsection{政策研究}
% \subsection{政治行为}
% \subsection{政治文化}
% \subsection{政治经济}
% \subsection{政治哲学}
% \subsection{公共法}
% \subsection{生理学}
% \subsection{社会选择理论}

\section{国际政治研究}
% \subsection{比较政治}
% \subsection{地缘政治(政治地理学)}
% \subsection{国际关系}
% \subsection{国际组织}
% \subsection{民族主义研究}
% \subsection{和平与冲突研究}
% \subsection{公共行政}

\section{不同国家的政治}
% \subsection{美国政治}
% \subsection{加拿大政治}
% \subsection{欧洲政治}
% \subsection{新加坡政治}
% \subsection{中国}


\chapter{心理学}

\section{基础心理学}
% \subsection{生物(生理)心理学}
% \subsection{发展心理学}
% \subsection{学习心理学}
% \subsection{人格心理学}
%     \subsubsection{变态心理}
%     \subsubsection{超个人心理学}
% \subsection{社会心理学}
%     \subsubsection{文化心理学}
%     \subsubsection{亚洲心理学}
%     \subsubsection{黑人心理学}
%     \subsubsection{社区心理学}
%     \subsubsection{环境心理学}

\section{应用心理学}
% \subsection{临床心理学}
%     \subsubsection{临床神经心理学}
% \subsection{认知心理学}
%     \subsubsection{精神分析}
% \subsection{教育心理学}
% \subsection{犯罪心理学}
% \subsection{产业心理学}
%     \subsubsection{比较心理学}
%     \subsubsection{保护心理学}
%     \subsubsection{消费心理}
%     \subsubsection{咨询心理学}
% \subsection{灾害心理学}
%     \subsubsection{康复心理学}
% \subsection{运动心理学}
\section{其他各种各样的心理学}
% \subsubsection{差异心理学}
% \subsubsection{生态心理学}
% \subsubsection{进化心理学}
% \subsubsection{实验心理学}
% \subsubsection{群体心理学}
% \subsubsection{家庭心理学}
% \subsubsection{女性心理学}
% \subsubsection{法医发展心理学}
% \subsubsection{法医心理学}
% \subsubsection{健康心理学}
% \subsubsection{人本主义心理学}
% \subsubsection{土著心理学}
% \subsubsection{法律心理学}
% \subsubsection{数学心理学}
% \subsubsection{媒体心理学}
% \subsubsection{医学心理学}
% \subsubsection{军事心理学}
% \subsubsection{道德心理学和描述性伦理学}
% \subsubsection{音乐心理学}
% \subsubsection{神经心理学}
% \subsubsection{职业健康心理学}
% \subsubsection{职业心理学}
% \subsubsection{组织心理学(又名,工业心理学)}
% \subsubsection{超心理学(大纲)}
% \subsubsection{儿科心理学}
% \subsubsection{土壤学(儿童研究)}
% \subsubsection{现象学}
% \subsubsection{政治心理学}
% \subsubsection{积极心理学}
% \subsubsection{心理生物学}
% \subsubsection{宗教心理学}
% \subsubsection{心理测量学}
% \subsubsection{精神病理学}
% \subsubsection{心理物理学}
% \subsubsection{数量心理学}
% \subsubsection{学校心理学}
% \subsubsection{交通心理学}



\chapter{社会学}


\section{理论研究}
% \subsection{分析社会学}
% \subsection{社会转型}
%     \subsubsection{计算社会学}
%     \subsubsection{经济社会学/社会经济学}
%     \subsubsection{经济发展}
%     \subsubsection{社会发展}
% \subsection{性学}
%     \subsubsection{异性恋}
%     \subsubsection{人类性行为}
%     \subsubsection{酷儿研究/酷儿理论}
%     \subsubsection{性教育}
% \subsection{社会变革}
% \subsection{社会冲突理论}
% \subsection{社会控制}
%     \subsubsection{社会控制论}
%     \subsubsection{纯社会学}
% \subsection{社会分层}
% \subsection{社会理论}
% \subsection{行为社会学}
% \subsection{集体行为}
%     \subsubsection{社会运动}
% \subsection{社区信息学}
%     \subsubsection{社交网络分析}
% \subsection{比较社会学}
% \subsection{冲突论}
% \subsection{犯罪学/刑事司法}
% \subsection{关键管理研究}
% \subsection{批判社会学}
% \subsection{文化社会学}
% \subsection{文化研究/民族研究}
%     \subsubsection{非洲研究}
%     \subsubsection{跨文化研究}
%     \subsubsection{文化学}
%     \subsubsection{聋哑研究}
%     \subsubsection{民族学}
%     \subsubsection{乌托邦研究}
%     \subsubsection{白度研究}
% \subsection{未来研究}
% \subsection{性别研究}
%     \subsubsection{男性研究}
%     \subsubsection{女性研究}
% \subsection{互动主义}
% \subsection{解释社会学}
%     \subsubsection{民族方法学}
%     \subsubsection{现象学}
%     \subsubsection{社会建构主义}
%     \subsubsection{象征互动主义}


\section{区域研究}
% \subsection{非洲研究}
% \subsection{美洲研究}
%     \subsubsection{美国研究}
%     \subsubsection{阿巴拉契亚研究}
%     \subsubsection{加拿大研究}
%     \subsubsection{拉丁美洲研究}
% \subsection{亚洲研究}
%     \subsubsection{中亚研究}
%     \subsubsection{东亚研究}
%     \subsubsection{印度学}
%     \subsubsection{伊朗研究}
%     \subsubsection{日本研究}
%     \subsubsection{韩国研究}
%     \subsubsection{巴基斯坦研究}
%     \subsubsection{佛教学}
%     \subsubsection{汉学(大纲)}
%     \subsubsection{东南亚研究}
%     \subsubsection{泰国研究}
% \section{澳大利亚研究}
% \section{欧洲研究}
%     \subsubsection{凯尔特人研究}
%     \subsubsection{德语研究}
%     \subsubsection{波兰社会学}
%     \subsubsection{斯堪的纳维亚研究}
%     \subsubsection{斯拉夫研究}
% \section{中东研究}
%     \subsubsection{阿拉伯研究}
%     \subsubsection{亚述学}
%     \subsubsection{埃及学}
%     \subsubsection{犹太研究}

\section{应用}
% \subsection{休闲研究}
% \subsection{政治社会学}
% \subsection{公共社会学}
% \subsection{社会工程学}
% \subsection{人口统计/人口}
% \subsection{数字社会学}
% \subsection{戏剧社会学}
% \subsection{经济社会学}
% \subsection{教育社会学}
% \subsection{经验社会学}
% \subsection{环境社会学}
% \subsection{进化社会学}
% \subsection{女性主义社会学}
% \subsection{形象社会学}
% \subsection{历史社会学}
% \subsection{人类生态学}
% \subsection{人文社会学}
% \subsection{工业社会学}
% \subsection{宏观社会学}
% \subsection{中社会学}
% \subsection{微观社会学}
% \subsection{组织研究}
% \subsection{精神分析社会学}
% \subsection{科学研究/科学和技术研究}
% \subsection{嫉妒社会学}
% \subsection{马克思主义社会学}
% \subsection{数学社会学}
% \subsection{医学社会学}
% \subsection{军事社会学}
% \subsection{自然资源社会学}
% \subsection{现象学社会学}
% \subsection{政策社会学}
% \subsection{社会资本}
% \subsection{社会经济}
% \subsection{社会哲学}
% \subsection{社会政策}
% \subsection{社会心理学}
% \subsection{社会生物学}
% \subsection{社会语言学}
% \subsection{老龄化社会学}
% \subsection{农业社会学}
% \subsection{艺术社会学}
% \subsection{自闭症社会学}
% \subsection{童年社会学}
% \subsection{冲突社会学}
% \subsection{文化社会学}
% \subsection{网络空间社会学}
% \subsection{发展社会学}
% \subsection{越轨社会学}
% \subsection{灾难社会学}
% \subsection{教育社会学}
% \subsection{情绪社会学}
% \subsection{父亲社会学}
% \subsection{金融社会学}
% \subsection{食物社会学}
% \subsection{性别社会学}
% \subsection{世代社会学}
% \subsection{全球化社会学}
% \subsection{政府社会学}
% \subsection{健康与疾病社会学}
% \subsection{人类意识社会学}
% \subsection{移民社会学}
% \subsection{知识社会学}
% \subsection{语言社会学}
% \subsection{法律社会学}
% \subsection{休闲社会学}
% \subsection{文学社会学}
% \subsection{市场社会学}
% \subsection{婚姻社会学}
% \subsection{母性社会学}
% \subsection{音乐社会学}
% \subsection{自然资源社会学}
% \subsection{组织社会学}
% \subsection{和平、战争和社会冲突的社会学}
% \subsection{惩罚社会学}
% \subsection{种族和民族关系社会学}
% \subsection{宗教社会学}
% \subsection{风险社会学}
% \subsection{科学社会学}
% \subsection{科学知识社会学}
% \subsection{社会变迁社会学}
% \subsection{社会运动社会学}
% \subsection{空间社会学}
% \subsection{体育社会学}
% \subsection{技术社会学}
% \subsection{恐怖主义社会学}
% \subsection{身体社会学}
% \subsection{家庭社会学}
% \subsection{科学史社会学}
% \subsection{互联网社会学}
% \subsection{工作社会学}
% \subsection{社会音乐学}
% \subsection{结构社会学}
% \subsection{理论社会学}
% \subsection{城市研究或城市社会学/农村社会学}
% \subsection{受害者学}
% \subsection{视觉社会学}
% \section{建筑社会学}









\end{document}
