%%============
%%  ** Author: Shepherd Qirong
%%  ** Date: 2019-08-10 19:42:00
%%  ** Github: https://github.com/ShepherdQR
%%  ** LastEditors: Shepherd Qirong
%%  ** LastEditTime: 2021-12-20 23:45:13
%%  ** Copyright (c) 2019--20xx Shepherd Qirong. All rights reserved.
%%============

\chapter{Topology}


a:点集拓扑学,b:代数拓扑学,c:同伦论,d:低维拓扑学,e:同调论,f:维数论,g:格上拓扑学,h:纤维丛论,i:几何拓扑学,j:奇点理论,k:微分拓扑学,l:拓扑学其他学科。


\section{AlgebraicTopology}
\subsection{摘要}
拓扑空间概念、性质、构造方法(如映射锥)
基本群的计算方法
奇异同调群3个定理:同伦不变性,正合序列,切除定理
奇异上同调(有环结构),泛系数定理,K\H unneth定理
代数拓扑通过寻找拓扑不变量给拓扑空间做分类。通过函子,把输入的拓扑空间变成群,把映射对应为同态,把同胚对应为同构。梦想是通过证明同构能够断言空间同胚。梦想还未实现,目前三维流形的分类为完成。
\subsection{拓扑空间}
通常研究连续映射、度量空间\\
性质:紧致性(任意开覆盖有子覆盖),连通性(不能表示成不相交的开子集之并),道路连通,分离性\\
同胚:对于对于拓扑空间X和Y,称$X\cong Y$,如果对于$ X \autorightleftharpoons{f}{g}Y $,有$ g\circ f =1_X,f\circ g=1_Y $\\
拓扑性质:同胚意义下不变的性质















 
\begin{comment}
    dsa 
    X \autorightleftharpoons{d969696}{3}Y
    \overset{f}{ \underset{g}{\rightleftharpoons} } 
    \xlongequal[d]{dfafdsf}
    \autorightleftharpoons{f}{g} Z

    \begin{equation}
    \label{homeomorphism}
    \begin{split}
        &\text{if: }X \autorightleftharpoons{f}{g}Y\\
        &\text{where: }g\circ f =1_X,f\circ g=1_Y\\
        &\text{then: }X\cong Y\\
    \end{split}
    \end{equation}

\end{comment}

