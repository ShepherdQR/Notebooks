

\chapter{Others}



\section{combinatorics}
组合数学,离散数学

广义的组合数学就是离散数学,狭义的组合数学是离散数学除图论、代数结构、数理逻辑等的部分。

狭义的组合数学主要研究满足一定条件的组态(也称组合模型)的存在、计数以及构造等方面的问题。 组合数学的主要内容有组合计数、组合设计、组合矩阵、组合优化(最佳组合)等。

\section{fuzzy mathematics}
模糊数学

由于模糊性概念已经找到了模糊集的描述方式,人们运用概念进行判断、评价、推理、决策和控制的过程也可以用模糊性数学的方法来描述。例如模糊聚类分析、模糊模式识别、模糊综合评判、模糊决策与模糊预测、模糊控制、模糊信息处理等。

\section{Quantum mathematics}

量子数学


\section{Applied mathematics}
应用数学
